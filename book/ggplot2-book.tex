\documentclass[graybox,envcountchap,sectrefs]{svmono}

\usepackage[scaled=0.92,varqu]{inconsolata}

\usepackage{float}
\usepackage{index}
% index functions separately
\newindex{code}{adx}{and}{R code index}
\newcommand{\indexf}[1]{\index[code]{#1@\texttt{#1()}}}
\newcommand{\indexc}[1]{\index[code]{#1@\texttt{#1}}}

% Taken from pandoc x.md -o test.tex --standalone
\usepackage{color}
\usepackage{fancyvrb}
\newcommand{\VerbBar}{|}
\newcommand{\VERB}{\Verb[commandchars=\\\{\}]}
\DefineVerbatimEnvironment{Highlighting}{Verbatim}{commandchars=\\\{\}}
\newenvironment{Shaded}{}{}
\newcommand{\KeywordTok} [1]{\textcolor[rgb]{0.00,0.44,0.13}{{#1}}}
\newcommand{\DataTypeTok}[1]{\textcolor[rgb]{0.56,0.13,0.00}{{#1}}}
\newcommand{\DecValTok}  [1]{\textcolor[rgb]{0.25,0.63,0.44}{{#1}}}
\newcommand{\BaseNTok}   [1]{\textcolor[rgb]{0.25,0.63,0.44}{{#1}}}
\newcommand{\FloatTok}   [1]{\textcolor[rgb]{0.25,0.63,0.44}{{#1}}}
\newcommand{\CharTok}    [1]{\textcolor[rgb]{0.25,0.44,0.63}{{#1}}}
\newcommand{\StringTok}  [1]{\textcolor[rgb]{0.25,0.44,0.63}{{#1}}}
\newcommand{\CommentTok} [1]{\textcolor[rgb]{0.38,0.63,0.69}{{#1}}}
\newcommand{\OtherTok}   [1]{\textcolor[rgb]{0.00,0.44,0.13}{{#1}}}
\newcommand{\AlertTok}   [1]{\textcolor[rgb]{1.00,0.00,0.00}{{#1}}}
\newcommand{\FunctionTok}[1]{\textcolor[rgb]{0.02,0.16,0.49}{{#1}}}
\newcommand{\ErrorTok}   [1]{\textcolor[rgb]{1.00,0.00,0.00}{{#1}}}
\newcommand{\NormalTok}  [1]{{#1}}

\newcommand{\OperatorTok}  [1]{{#1}}
\newcommand{\ControlFlowTok}  [1]{{#1}}
%
\usepackage{longtable}
\usepackage{booktabs}
\usepackage{graphicx}
\DeclareGraphicsExtensions{.pdf,.png}
\providecommand{\tightlist}{\setlength{\itemsep}{0pt}\setlength{\parskip}{0pt}}

\usepackage[hyphens]{url}
\usepackage{hyperref}

% Place links in parens
\renewcommand{\href}[2]{#2 (\url{#1})}
% Use auto ref for internal links
\let\oldhyperlink=\hyperlink
\renewcommand{\hyperlink}[2]{\autoref{#1}}
\def\chapterautorefname{Chapter}
\def\sectionautorefname{Section}
\def\subsectionautorefname{Section}
\def\subsubsectionautorefname{Section}

\setlength{\emergencystretch}{3em}  % prevent overfull lines
\vbadness=10000 % suppress underfull \vbox
\hbadness=10000 % suppress underfull \vbox
\hfuzz=10pt

\makeindex
\title{ggplot2}
\subtitle{Elegant Graphics for Data Analysis}
\author{Hadley Wickham}

\begin{document}

\frontmatter
\maketitle

\begin{dedication}
To my parents, Alison \& Brian Wickham. Without them, and their unconditional
love and support, none of this would have been possible.
\end{dedication}

\include{preface}

\tableofcontents

\mainmatter

\part{Getting started}

\providecommand{\setflag}{\newif \ifwhole \wholefalse}
\setflag
\ifwhole\else
  \documentclass[11pt,twoside,openright]{scrbook}
	\usepackage[utf8]{inputenc}
	\usepackage{fullpage}
	\usepackage[pdftex]{graphicx}
	\usepackage{hyperref}
	\usepackage{minitoc}
	
	\setcounter{secnumdepth}{0}

	\title{ggplot}
	\author{Hadley Wickham}

\renewcommand{\topfraction}{0.9}	% max fraction of floats at top
\renewcommand{\bottomfraction}{0.8}	% max fraction of floats at bottom
%   Parameters for TEXT pages (not float pages):
\setcounter{topnumber}{2}
\setcounter{bottomnumber}{2}
\renewcommand{\dbltopfraction}{0.9}	% fit big float above 2-col. text
\renewcommand{\textfraction}{0.07}	% allow minimal text w. figs
%   Parameters for FLOAT pages (not text pages):
\renewcommand{\floatpagefraction}{0.7}	% require fuller float pages
% N.B.: floatpagefraction MUST be less than topfraction !!
\renewcommand{\dblfloatpagefraction}{0.7}	% require fuller float pages


	\begin{document}
\fi

% \setchapterpreamble[u]{% 
% \dictum[Anonymous]{Forecasting is the art of saying 
% what is going to happen and then to explain 
% why it didn’t.}} 

% what is ggplot
% how to get ggplot
% how does it fit in with other r packages (including comparison table)
% what is the grammar of graphics etc? what the grammar doesn't do
% outline of book

\chapter{Introduction}

Ggplot is an R package for producing statistical graphics.  It builds on the grid graphics system [ref], and uses the philosophy outlined in the Grammar of Graphics [ref] to produce a powerful and flexible plotting system that is still easy to use.  

% This model of graphics is quite different to the existing base and lattice graphic system and so you'll need to learn a new way of thinking about graphics.  

This book provides helps you to do this as well as providing a practical introduction to ggplot with lots of example code and graphics.

This book assumes some familiarity with R, to the level described in Dalgaard’s Introductory Statistics with R.  , but knowledge of the grammar of graphics is not necessary, it will be explained as we go along.

What else should go in here?

% From words (scatterplot, pie chart) to sentences which describe completely the structure of a plot.
% 
% Objected oriented design
% 
% Revision and iteration

\section{What is the grammar of graphics?}

The grammar of graphics is an answer to a question: what is a statistical graphic?  To me, a graphic is a mapping  ({\bf scale})  from {\bf data} to {\bf aes}thetic attributes (colour, shape, size) of {\bf geom}etric objects (points, lines, bars).  The plot may also contain {\bf stat}istical transformations of the data, and is drawn on a specific {\bf coord}inate system.  {\bf Facet}ting can be used to generate the same plot for different subsets of the dataset.  It is the combination of these independent component that make up a graphic.  The components are described in more detail below.

\begin{itemize}
	\item Data is the most important thing, and the thing that you provide.
	\item Geometric objects (or geoms for short) represent what you actually see on the plot: points, lines, polygons, etc.
	\item Statistics transform the data in many useful ways.  They are optional.
	\item Scales map values in the data space to values in an aesthetic space, whether it be colour, or size, or shape.  Scales also provide a legend to make it possible to read the graph.
	\item A coordinate system describes how data coordinates are mapped to the plane of the graphic.  A coordinate system also provides axes and gridlines to make it possible to read the graph.
	\item Facetting, or conditioning, specification.  It is often useful to be able to reproduce the same plot for different subsets of the data.  The facetting specification describes those subsets and how the plot should be arranged.
\end{itemize}

It is also important to talk about what the grammar doesn't do:

\begin{itemize}
	\item It doesn't advise what graphics you should make to answer the questions you are interested in.  It describes what a meaningful plot is, but most meaningful plots are unrelated to your questions. While this book endeavours to promote a sensible process for producing plots of data, the focus of the book is largely on how to produce the plots you want, not knowing what plots to produce. For more advice on this topic, you may want to read: [Reference Cleveland, Chambers, Tukey? Naomi Robbins? etc]

	\item Ironically, it doesn't specify what a graphic should look like.  The finer points of display, for example, font size or background colour, are not specified by the grammar.  In practice to create a useful plotting system we will need to describe these in some way. Similarly, the grammar does not specify how to make an attractive graphic, and while the defaults in ggplot have been chosen with care, you may need to consult other references to create a beautiful plot: [Tufte, Carr papers, ]

	\item It does not describe interaction: the grammar of graphics describes only static graphics, and there is essentially no benefit to displaying on a computer screen as opposed to on a piece of paper.  ggplot can only create static graphics, so for dynamic and interactive graphics you will have to look elsewhere.  [Mondrian book] and [GGobi book] provide excellent introduction to two different interactive graphics packages: Mondrian and GGobi.

\end{itemize}

\section{How does ggplot fit in with other R graphics?}

There are a number of other graphics systems available in R: base graphics, grid graphics and trellis/lattice graphics.  How does ggplot differ from them?

\begin{itemize} 
	\item Base graphics has basically a pen on paper model: you can only draw on top, not modify or delete existing content or change the axes etc.  There is no (user accessible) representation of the graphics, apart from the appearance on the plot. Base graphics includes both tools for drawing primitives and entire plots. No other plotting system does this.  Base graphics functions are generally fast, but have limited scope.

	\item Grid graphics provides greatly improved drawing primitives. The graphical objects can be represented independently of the plot and modified later. A system of viewports (each containing its own coordinate system) makes it easier to layout complex graphics.  Grid provides drawing primitives, but no tools for producing statistical graphics.

	\item The lattice package uses grid graphics to implement the trellis graphics system of Cleveland, is a considerable improvement over base graphics.  You can easily produce conditioned plots, and some plotting details (eg.\ legends) are taken care of automatically.  However, lattice graphics lack a formal model of graphics, which can make it hard to extend.

	\item ggplot is an attempt to take the good things about base and lattice graphics and improve on them by having a model which supports the production of any kind of statistical graphic, based on principles outlined above.  The solid underlying model of ggplot makes it easy to describe a wide range of graphics with a compact syntax, and independent components make extension easy.  Like lattice, ggplot uses grid to draw the graphics, which means you can exercise much low level control over the appearance of the plot

	\item Many other R packages implement specialist graphics but no others provide a framework for producing statistical graphics. While this is fine if you just want to produce a one-off graphic, it is generally hard to combine these with other graphics you may be using.

\end{itemize}

\section{About this book}\label{sec:about_this_book}

% This book is available online for free at \url{http://had.co.nz/ggplot/book}.  However, if you want to support the development of ggplot (and save yourself the hassle of printing and binding a large pdf) you can also buy a printed version for \$US 40 (price may change).  

The book is structured to lead you from being a new user of ggplot to a developer creating new components for specialised plots:

\begin{itemize}
	\item Chapter One describes how you can quickly get started using {\tt qplot} to make graphics, just like you can using {\tt plot}.  This chapter introduces several important ggplot concepts: grob functions, aesthetic mappings and facetting.
	
	\item While {\tt qplot} is a quick way to get started, it doesn't give full control over all the  available options, so Chapter Two describes how to build up a plot piece by piece, exercising full control over the available options.  This chapter discusses the components of a plot, laying the ground for the following chapters which describe these components in detail, and teach you how to build your own.  You will also learn some techniques using the reshape package to get data into a convenient form for ggplot.

	\item The most crucial component of a plot are the geometric and statistic objects, and Chapters Three and Four describes what they do, how they work, and lists the most commonly used.  Mastery of this chapter will give you the ability to pick and choose the most appropriate tool for your visual display needs.  The chapter concludes by showing you how to build your own geom and stat objects that you can extend ggplot to meet your specific needs.

	\item Understanding how scales works is crucial for fine tuning the perceptual properties of your plot.  Customising scales gives fine control over the exact appearance of the plot, and helps to support the story that you are telling.  Chapter Five will show you what scales are available, how to adjust their parameters, and how to create your own.

	\item Non-cartesian coordinate systems are somewhat rare, but when you need them, its hard to go without.  In Chapter Six, the different coordinate systems are described and illustrated, and you will learn how to create you own.
	
	% \item Chapter Seven introduces my philosophy of data.  This chapter isn't crucial, but will help you understand the type of data {\tt ggplot} expects, and how to transform your data into that format.  It also discusses facetting in more detail, and discusses some ideas for combining modelling and graphics.
	
	\item Sometimes you need more control over the output than ggplot provides.  In this case, you will need to modify the low level grid output used to draw the graphics.  In Chapter Seven, you will how this output is constructed, how to control and modify it, and how to add additional annotations to the plot.

\end{itemize}

The ggplot website, \url{http://had.co.nz/ggplot}, provides updates to this book, information about features in the latest version of ggplot, and talks and papers related to ggplot.  All graphics used on the book are listed on the site, along with the code and data needed to reproduce them.  There is also a gallery of ggplot graphics used in real life.  If you would like your graphics to be included in the gallery, please send me reproducible code and a paragraph or two describing your plot.

\section{Installation}\label{sub:installation}

To get started using ggplot, the first thing you need to do is install it.  Make sure you have a recent version of R from \url{http://r-project.org}, and then follow the instructions below to download and install the ggplot package.  

There are usually two versions of ggplot available, one stable version and one development version. The stable version is well-tested and well-documented before release.  It is available on CRAN, and can be installed with the following R code:

\begin{verbatim}
	install.packages("ggplot", dep=TRUE)
\end{verbatim}

The development version is not so well tested or documented, but includes new features that I'm working on.  It may also contain bug fixes for recently discovered bugs.  It can be installed as follows:

\begin{verbatim}
	install.packages("ggplot", repos="http://www.ggobi.org/r", dep=TRUE)
\end{verbatim}

A changelog listing changes between versions is available on the ggplot website.  I will do my best to make sure that changes are backward compatible, so you shouldn't have to rewrite your old code.  However, from time to time, I may need to make bigger changes that do affect your code.  If you need to ensure that your old code will continue to run, I would recommend using use R's versioned installs:

\begin{verbatim}
	install.packages("ggplot", dep=TRUE, installWithVers=TRUE))
\end{verbatim}

Now installed packages will have a version number associated with them, and you can load a specific version like so:

\begin{verbatim}
	library(ggplot, version="0.5")
\end{verbatim}

{\tt ggplot} isn't perfect, so from time-to-time you may encounter something that doesn't way the way you think it should.  If this happens, please email me \href{mailto:h.wickham@gmail.com}{h.wickham@gmail.com} a reproducible example of your problem, as well as a description of what you think should happen.  The more information you provide, the more likely I am going to be able to help you.

% \section{Formatting conventions}
% 
% grammar of graphics vs The Grammar of Graphics

\section{Acknowledgements}\label{sec:acknolwedgements}

Many people have contributed to this book with high-level structural insights, and bug reports.  In particular, I would to thank: Lee Wilkinson, for discussions that cemented my understand of the grammar; Gabor Grothendienk for early helpful comments; Heike Hofmann and Di Cook for being great major professors; Charlotte Wickham; the students of stat480 and stat503 at ISU, for using it; Debby Swayne, for many helpful comments on targeting the book.

\ifwhole\else
  \end{document}
\fi

\chapter{Getting started with ggplot2}\label{cha:getting-started}

\section{Introduction}

The goal of this chapter is to teach you how to produce useful graphics
with ggplot2 as quickly as possible. You'll learn the basics of
\texttt{ggplot()} along with some useful ``recipes'' to make the most
important plots. \texttt{ggplot()} allows you to make complex plots with
just a few lines of code because it's based on a rich underlying theory,
the grammar of graphics. Here we'll skip the theory and focus on the
practice, and in later chapters you'll learn how to use the full
expressive power of the grammar.

In this chapter you'll learn:

\begin{itemize}
\item
  About the \texttt{mpg} dataset included with ggplot2,
  \hyperref[sec:fuel-economy-data]{mpg}.
\item
  The three key components of every plot: data, aesthetics and geoms,
  \hyperref[sec:basic-use]{key components}.
\item
  How to add additional variables to a plot with aesthetics,
  \hyperref[aesthetics]{aesthetics}.
\item
  How to display additional categorical variables in a plot using small
  multiples created by facetting,
  \hyperref[sec:qplot-facetting]{facetting}.
\item
  A variety of different geoms that you can use to create different
  types of plots, \hyperref[sec:plot-geoms]{geoms}.
\item
  How to modify the axes, \hyperref[sec:axes]{axes}.
\item
  Things you can do with a plot object other than display it, like save
  it to disk, \hyperref[sec:output]{output}.
\item
  \texttt{qplot()}, a handy shortcut for when you just want to quickly
  bang out a simple plot without thinking about the grammar at all,
  \hyperref[qplot]{qplot}.
\end{itemize}

\hyperdef{}{sec:fuel-economy-data}{\section{Fuel economy
data}\label{sec:fuel-economy-data}}

In this chapter, we'll mostly use one data set that's bundled with
ggplot2: \texttt{mpg}. It includes information about the fuel economy of
popular car models in 1999 and 2008, collected by the US Environmental
Protection Agency, \url{http://fueleconomy.gov}. You can access the data
by loading ggplot2: \index{Data!mpg@\texttt{mpg}}

\begin{Shaded}
\begin{Highlighting}[]
\KeywordTok{library}\NormalTok{(ggplot2)}
\NormalTok{mpg}
\CommentTok{#> Source: local data frame [234 x 11]}
\CommentTok{#> }
\CommentTok{#>    manufacturer model displ  year   cyl      trans   drv   cty   hwy}
\CommentTok{#>           (chr) (chr) (dbl) (int) (int)      (chr) (chr) (int) (int)}
\CommentTok{#> 1          audi    a4   1.8  1999     4   auto(l5)     f    18    29}
\CommentTok{#> 2          audi    a4   1.8  1999     4 manual(m5)     f    21    29}
\CommentTok{#> 3          audi    a4   2.0  2008     4 manual(m6)     f    20    31}
\CommentTok{#> 4          audi    a4   2.0  2008     4   auto(av)     f    21    30}
\CommentTok{#> 5          audi    a4   2.8  1999     6   auto(l5)     f    16    26}
\CommentTok{#> 6          audi    a4   2.8  1999     6 manual(m5)     f    18    26}
\CommentTok{#> ..          ...   ...   ...   ...   ...        ...   ...   ...   ...}
\CommentTok{#> Variables not shown: fl (chr), class (chr)}
\end{Highlighting}
\end{Shaded}

The variables are mostly self-explanatory:

\begin{itemize}
\item
  \texttt{cty} and \texttt{hwy} record miles per gallon (mpg) for city
  and highway driving.
\item
  \texttt{displ} is the engine displacement in litres.
\item
  \texttt{drv} is the drivetrain: front wheel (f), rear wheel (r) or
  four wheel (4).
\item
  \texttt{model} is the model of car. There are 38 models, selected
  because they had a new edition every year between 1999 and 2008.
\item
  \texttt{class} (not shown), is a categorical variable describing the
  ``type'' of car: two seater, SUV, compact, etc.
\end{itemize}

This dataset suggests many interesting questions. How are engine size
and fuel economy related? Do certain manufacturers care more about fuel
economy than others? Has fuel economy improved in the last ten years? We
will try to answer some of these questions, and in the process learn how
to create some basic plots with ggplot2.

\subsection{Exercises}

\begin{enumerate}
\def\labelenumi{\arabic{enumi}.}
\item
  List five functions that you could use to get more information about
  the \texttt{mpg} dataset.
\item
  How can you find out what other datasets are included with ggplot2?
\item
  Apart from the US, most countries use fuel consumption (fuel consumed
  over fixed distance) rather than fuel economy (distance travelled with
  fixed amount of fuel). How could you convert \texttt{cty} and
  \texttt{hwy} into the European standard of l/100km?
\item
  Which manufacturer has the most the models in this dataset? Which
  model has the most variations? Does your answer change if you remove
  the redundant specification of drive train (e.g. ``pathfinder 4wd'',
  ``a4 quattro'') from the model name?
\end{enumerate}

\hyperdef{}{sec:basic-use}{\section{Key
components}\label{sec:basic-use}}

Every ggplot2 plot has three key components:

\begin{enumerate}
\def\labelenumi{\arabic{enumi}.}
\item
  \textbf{data},
\item
  A set of \textbf{aesthetic mappings} between variables in the data and
  visual properties, and
\item
  At least one layer which describes how to render each observation.
  Layers are usually created with a \textbf{geom} function.
\end{enumerate}

Here's a simple example: \index{Scatterplot} \indexf{ggplot}

\begin{Shaded}
\begin{Highlighting}[]
\KeywordTok{ggplot}\NormalTok{(mpg, }\KeywordTok{aes}\NormalTok{(}\DataTypeTok{x =} \NormalTok{displ, }\DataTypeTok{y =} \NormalTok{hwy)) +}\StringTok{ }
\StringTok{  }\KeywordTok{geom_point}\NormalTok{()}
\end{Highlighting}
\end{Shaded}

\begin{figure}[H]
  \centering
  \includegraphics[width=0.65\linewidth]{_figures/ggplot/qscatter-1}
\end{figure}

This produces a scatterplot defined by:

\begin{enumerate}
\def\labelenumi{\arabic{enumi}.}
\tightlist
\item
  Data: \texttt{mpg}.
\item
  Aesthetic mapping: engine size mapped to x position, fuel economy to y
  position.
\item
  Layer: points.
\end{enumerate}

Pay attention to the structure of this function call: data and aesthetic
mappings are supplied in \texttt{ggplot()}, then layers are added on
with \texttt{+}. This is an important pattern, and as you learn more
about ggplot2 you'll construct increasingly sophisticated plots by
adding on more types of components.

Almost every plot maps a variable to \texttt{x} and \texttt{y}, so
naming these aesthetics is tedious, so the first two unnamed arguments
to \texttt{aes()} will be mapped to \texttt{x} and \texttt{y}. This
means that the following code is identical to the example above:

\begin{Shaded}
\begin{Highlighting}[]
\KeywordTok{ggplot}\NormalTok{(mpg, }\KeywordTok{aes}\NormalTok{(displ, hwy)) +}
\StringTok{  }\KeywordTok{geom_point}\NormalTok{()}
\end{Highlighting}
\end{Shaded}

I'll stick to that style throughout the book, so don't forget that the
first two arguments to \texttt{aes()} are \texttt{x} and \texttt{y}.
Note that I've put each command on a new line. I recommend doing this in
your own code, so it's easy to scan a plot specification and see exactly
what's there. In this chapter, I'll sometimes use just one line per
plot, because it makes it easier to see the differences between plot
variations.

The plot shows a strong correlation: as the engine size gets bigger, the
fuel economy gets worse. There are also some interesting outliers: some
cars with large engines get higher fuel economy than average. What sort
of cars do you think they are?

\subsection{Exercises}

\begin{enumerate}
\def\labelenumi{\arabic{enumi}.}
\item
  How would you describe the relationship between \texttt{cty} and
  \texttt{hwy}? Do you have any concerns about drawing conclusions from
  that plot?
\item
  What does
  \texttt{ggplot(mpg,\ aes(model,\ manufacturer))\ +\ geom\_point()}
  show? Is it useful? How could you modify the data to make it more
  informative?
\item
  Describe the data, aesthetic mappings and layers used for each of the
  following plots. You'll need to guess a little because you haven't
  seen all the datasets and functions yet, but use your common sense!
  See if you can predict what the plot will look like before running the
  code.

  \begin{enumerate}
  \def\labelenumii{\arabic{enumii}.}
  \tightlist
  \item
    \texttt{ggplot(mpg,\ aes(cty,\ hwy))\ +\ geom\_point()}
  \item
    \texttt{ggplot(diamonds,\ aes(carat,\ price))\ +\ geom\_point()}
  \item
    \texttt{ggplot(economics,\ aes(date,\ unemploy))\ +\ geom\_line()}
  \item
    \texttt{ggplot(mpg,\ aes(cty))\ +\ geom\_histogram()}
  \end{enumerate}
\end{enumerate}

\hyperdef{}{aesthetics}{\section{Colour, size, shape and other aesthetic
attributes}\label{aesthetics}}

To add additional variables to a plot, we can use other aesthetics like
colour, shape, and size (NB: while I use British spelling throughout
this book, ggplot2 also accepts American spellings). These work in the
same way as the \texttt{x} and \texttt{y} aesthetics, and are added into
the call to \texttt{aes()}: \index{Aesthetics} \indexf{aes}

\begin{itemize}
\tightlist
\item
  \texttt{aes(displ,\ hwy,\ colour\ =\ class)}
\item
  \texttt{aes(displ,\ hwy,\ shape\ =\ drv)}
\item
  \texttt{aes(displ,\ hwy,\ size\ =\ cyl)}
\end{itemize}

ggplot2 takes care of the details of converting data (e.g., `f', `r',
`4') into aesthetics (e.g., `red', `yellow', `green') with a
\textbf{scale}. There is one scale for each aesthetic mapping in a plot.
The scale is also responsible for creating a guide, an axis or legend,
that allows you to read the plot, converting aesthetic values back into
data values. For now, we'll stick with the default scales provided by
ggplot2. You'll learn how to override them in \hyperref[cha:scales]{the
scales chapter}.

To learn more about those outlying variables in the previous
scatterplot, we could map the class variable to colour:

\begin{Shaded}
\begin{Highlighting}[]
\KeywordTok{ggplot}\NormalTok{(mpg, }\KeywordTok{aes}\NormalTok{(displ, cty, }\DataTypeTok{colour =} \NormalTok{class)) +}\StringTok{ }
\StringTok{  }\KeywordTok{geom_point}\NormalTok{()}
\end{Highlighting}
\end{Shaded}

\begin{figure}[H]
  \centering
  \includegraphics[width=0.65\linewidth]{_figures/ggplot/qplot-aesthetics-1}
\end{figure}

This gives each point a unique colour corresponding to its class. The
legend allows us to read data values from the colour, showing us that
the group of cars with unusually high fuel economy for their engine size
are two seaters: cars with big engines, but lightweight bodies.

If you want to set an aesthetic to a fixed value, without scaling it, do
so in the individual layer outside of \texttt{aes()}. Compare the
following two plots: \index{Aesthetics!setting}

\begin{Shaded}
\begin{Highlighting}[]
\KeywordTok{ggplot}\NormalTok{(mpg, }\KeywordTok{aes}\NormalTok{(displ, hwy)) +}\StringTok{ }\KeywordTok{geom_point}\NormalTok{(}\KeywordTok{aes}\NormalTok{(}\DataTypeTok{colour =} \StringTok{"blue"}\NormalTok{))}
\KeywordTok{ggplot}\NormalTok{(mpg, }\KeywordTok{aes}\NormalTok{(displ, hwy)) +}\StringTok{ }\KeywordTok{geom_point}\NormalTok{(}\DataTypeTok{colour =} \StringTok{"blue"}\NormalTok{)}
\end{Highlighting}
\end{Shaded}

\begin{figure}[H]
  \includegraphics[width=0.5\linewidth]{_figures/ggplot/unnamed-chunk-4-1}%
  \includegraphics[width=0.5\linewidth]{_figures/ggplot/unnamed-chunk-4-2}
\end{figure}

In the first plot, the value ``blue'' is scaled to a pinkish colour, and
a legend is added. In the second plot, the points are given the R colour
blue. This is an important technique and you'll learn more about it in
\hyperref[sub:setting-mapping]{setting vs.~mapping}. See
\texttt{vignette("ggplot2-specs")} for the values needed for colour and
other aesthetics.

Different types of aesthetic attributes work better with different types
of variables. For example, colour and shape work well with categorical
variables, while size works well for continuous variables. The amount of
data also makes a difference: if there is a lot of data it can be hard
to distinguish different groups. An alternative solution is to use
facetting, as described next.

When using aesthetics in a plot, less is usually more. It's difficult to
see the simultaneous relationships among colour and shape and size, so
exercise restraint when using aesthetics. Instead of trying to make one
very complex plot that shows everything at once, see if you can create a
series of simple plots that tell a story, leading the reader from
ignorance to knowledge.

\subsection{Exercises}

\begin{enumerate}
\def\labelenumi{\arabic{enumi}.}
\item
  Experiment with the colour, shape and size aesthetics. What happens
  when you map them to continuous values? What about categorical values?
  What happens when you use more than one aesthetic in a plot?
\item
  What happens if you map a continuous variable to shape? Why? What
  happens if you map \texttt{trans} to shape? Why?
\item
  How is drive train related to fuel economy? How is drive train related
  to engine size and class?
\end{enumerate}

\hyperdef{}{sec:qplot-facetting}{\section{Facetting}\label{sec:qplot-facetting}}

Another technique for displaying additional categorical variables on a
plot is facetting. Facetting creates tables of graphics by splitting the
data into subsets and displaying the same graph for each subset. You'll
learn more about facetting in \hyperref[sec:facetting]{Facetting}, but
it's such a useful technique that you need to know it right away.
\index{Facetting}

There are two types of facetting: grid and wrapped. Wrapped is the most
useful, so we'll discuss it here, and you can learn about grid facetting
later. To facet a plot you simply add a facetting specification with
\texttt{facet\_wrap()}, which takes the name of a variable preceded by
\texttt{\textasciitilde{}}. \indexf{facet\_wrap}

\begin{Shaded}
\begin{Highlighting}[]
\KeywordTok{ggplot}\NormalTok{(mpg, }\KeywordTok{aes}\NormalTok{(displ, hwy)) +}\StringTok{ }
\StringTok{  }\KeywordTok{geom_point}\NormalTok{() +}\StringTok{ }
\StringTok{  }\KeywordTok{facet_wrap}\NormalTok{(~class)}
\end{Highlighting}
\end{Shaded}

\begin{figure}[H]
  \includegraphics[width=1\linewidth]{_figures/ggplot/facet-1}
\end{figure}

You might wonder when to use facetting and when to use aesthetics.
You'll learn more about the relative advantages and disadvantages of
each in \hyperref[sub:group-vs-facet]{grouping vs.~facetting}.

\subsection{Exercises}

\begin{enumerate}
\def\labelenumi{\arabic{enumi}.}
\item
  What happens if you try to facet by a continuous variable like
  \texttt{hwy}? What about \texttt{cyl}? What's the key difference?
\item
  Use facetting to explore the 3-way relationship between fuel economy,
  engine size, and number of cylinders. How does facetting by number of
  cylinders change your assessement of the relationship between engine
  size and fuel economy?
\item
  Read the documentation for \texttt{facet\_wrap()}. What arguments can
  you use to control how many rows and columns appear in the output?
\item
  What does the \texttt{scales} argument to \texttt{facet\_wrap()} do?
  When might you use it?
\end{enumerate}

\hyperdef{}{sec:plot-geoms}{\section{Plot geoms}\label{sec:plot-geoms}}

You might guess that by substituting \texttt{geom\_point()} for a
different geom function, you'd get a different type of plot. That's a
great guess! In the following sections, you'll learn about some of the
other important geoms provided in ggplot2. This isn't an exhaustive
list, but should cover the most commonly used plot types. You'll learn
more in \hyperref[cha:toolbox]{the toolbox}.

\begin{itemize}
\item
  \texttt{geom\_smooth()} fits a smoother to the data and displays the
  smooth and its standard error.
\item
  \texttt{geom\_boxplot()} produces a box-and-whisker plot to summarise
  the distribution of a set of points.
\item
  \texttt{geom\_histogram()} and \texttt{geom\_freqpoly()} show the
  distribution of continuous variables.
\item
  \texttt{geom\_bar()} shows the distribution of categorical variables.
\item
  \texttt{geom\_path()} and \texttt{geom\_line()} draw lines between the
  data points. A line plot is constrained to produce lines that travel
  from left to right, while paths can go in any direction. Lines are
  typically used to explore how things change over time.
\end{itemize}

\subsection{Adding a smoother to a plot}\label{sub:smooth}

If you have a scatterplot with a lot of noise, it can be hard to see the
dominant pattern. In this case it's useful to add a smoothed line to the
plot with \texttt{geom\_smooth()}: \index{Smoothing}
\indexf{geom\_smooth}

\begin{Shaded}
\begin{Highlighting}[]
\KeywordTok{ggplot}\NormalTok{(mpg, }\KeywordTok{aes}\NormalTok{(displ, hwy)) +}\StringTok{ }
\StringTok{  }\KeywordTok{geom_point}\NormalTok{() +}\StringTok{ }
\StringTok{  }\KeywordTok{geom_smooth}\NormalTok{()}
\end{Highlighting}
\end{Shaded}

\begin{figure}[H]
  \centering
  \includegraphics[width=0.65\linewidth]{_figures/ggplot/qplot-smooth-1}
\end{figure}

This overlays the scatterplot with a smooth curve, including an
assessment of uncertainty in the form of point-wise confidence intervals
shown in grey. If you're not interested in the confidence interval, turn
it off with \texttt{geom\_smooth(se\ =\ FALSE)}.

An important argument to \texttt{geom\_smooth()} is the \texttt{method},
which allows you to choose which type of model is used to fit the smooth
curve:

\begin{itemize}
\item
  \texttt{method\ =\ "loess"}, the default for small n, uses a smooth
  local regression (as described in \texttt{?loess}). The wiggliness of
  the line is controlled by the \texttt{span} parameter, which ranges
  from 0 (exceedingly wiggly) to 1 (not so wiggly).

\begin{Shaded}
\begin{Highlighting}[]
\KeywordTok{ggplot}\NormalTok{(mpg, }\KeywordTok{aes}\NormalTok{(displ, hwy)) +}\StringTok{ }
\StringTok{  }\KeywordTok{geom_point}\NormalTok{() +}\StringTok{ }
\StringTok{  }\KeywordTok{geom_smooth}\NormalTok{(}\DataTypeTok{span =} \FloatTok{0.2}\NormalTok{)}

\KeywordTok{ggplot}\NormalTok{(mpg, }\KeywordTok{aes}\NormalTok{(displ, hwy)) +}\StringTok{ }
\StringTok{  }\KeywordTok{geom_point}\NormalTok{() +}\StringTok{ }
\StringTok{  }\KeywordTok{geom_smooth}\NormalTok{(}\DataTypeTok{span =} \DecValTok{1}\NormalTok{)}
\end{Highlighting}
\end{Shaded}

  \begin{figure}[H]
    \includegraphics[width=0.5\linewidth]{_figures/ggplot/smooth-loess-1}%
    \includegraphics[width=0.5\linewidth]{_figures/ggplot/smooth-loess-2}
  \end{figure}

  Loess does not work well for large datasets (it's \(O(n^2)\) in
  memory), so an alternative smoothing algorithm is used when \(n\) is
  greater than 1,000.
\item
  \texttt{method\ =\ "gam"} fits a generalised additive model provided
  by the \textbf{mgcv} package. You need to first load mgcv, then use a
  formula like \texttt{formula\ =\ y\ \textasciitilde{}\ s(x)} or
  \texttt{y\ \textasciitilde{}\ s(x,\ bs\ =\ "cs")} (for large data).
  This is what ggplot2 uses when there are more than 1,000 points.
  \index{mgcv}

\begin{Shaded}
\begin{Highlighting}[]
\KeywordTok{library}\NormalTok{(mgcv)}
\KeywordTok{ggplot}\NormalTok{(mpg, }\KeywordTok{aes}\NormalTok{(displ, hwy)) +}\StringTok{ }
\StringTok{  }\KeywordTok{geom_point}\NormalTok{() +}\StringTok{ }
\StringTok{  }\KeywordTok{geom_smooth}\NormalTok{(}\DataTypeTok{method =} \StringTok{"gam"}\NormalTok{, }\DataTypeTok{formula =} \NormalTok{y ~}\StringTok{ }\KeywordTok{s}\NormalTok{(x))}
\end{Highlighting}
\end{Shaded}

  \begin{figure}[H]
    \includegraphics[width=0.5\linewidth]{_figures/ggplot/smooth-gam-1}
  \end{figure}
\item
  \texttt{method\ =\ "lm"} fits a linear model, giving the line of best
  fit.

\begin{Shaded}
\begin{Highlighting}[]
\KeywordTok{ggplot}\NormalTok{(mpg, }\KeywordTok{aes}\NormalTok{(displ, hwy)) +}\StringTok{ }
\StringTok{  }\KeywordTok{geom_point}\NormalTok{() +}\StringTok{ }
\StringTok{  }\KeywordTok{geom_smooth}\NormalTok{(}\DataTypeTok{method =} \StringTok{"lm"}\NormalTok{)}
\end{Highlighting}
\end{Shaded}

  \begin{figure}[H]
    \includegraphics[width=0.5\linewidth]{_figures/ggplot/smooth-lm-1}
  \end{figure}
\item
  \texttt{method\ =\ "rlm"} works like \texttt{lm()}, but uses a robust
  fitting algorithm so that outliers don't affect the fit as much. It's
  part of the \textbf{MASS} package, so remember to load that first.
  \index{MASS}
\end{itemize}

\subsection{Boxplots and jittered points}\label{sub:boxplot}

When a set of data includes a categorical variable and one or more
continuous variables, you will probably be interested to know how the
values of the continuous variables vary with the levels of the
categorical variable. Say we're interested in seeing how fuel economy
varies within car class. We might start with a scatterplot like this:

\begin{Shaded}
\begin{Highlighting}[]
\KeywordTok{ggplot}\NormalTok{(mpg, }\KeywordTok{aes}\NormalTok{(drv, hwy)) +}\StringTok{ }
\StringTok{  }\KeywordTok{geom_point}\NormalTok{()}
\end{Highlighting}
\end{Shaded}

\begin{figure}[H]
  \centering
  \includegraphics[width=0.5\linewidth]{_figures/ggplot/unnamed-chunk-5-1}
\end{figure}

Because there are few unique values of both class and hwy, there is a
lot of overplotting. Many points are plotted in the same location, and
it's difficult to see the distribution. There are three useful
techniques that help alleviate the problem:

\begin{itemize}
\item
  Jittering, \texttt{geom\_jitter()}, adds a little random noise to the
  data which can help avoid overplotting. \index{Jittering}
  \indexf{geom\_jitter}
\item
  Boxplots, \texttt{geom\_boxplot()}, summarise the shape of the
  distribution with a handful of summary statistics. \index{Boxplot}
  \indexf{geom\_boxplot}
\item
  Violin plots, \texttt{geom\_violin()}, show a compact representation
  of the ``density'' of the distribution, highlighting the areas where
  more points are found. \index{Violin plot} \indexf{geom\_violin}
\end{itemize}

These are illustrated below:

\begin{Shaded}
\begin{Highlighting}[]
\KeywordTok{ggplot}\NormalTok{(mpg, }\KeywordTok{aes}\NormalTok{(drv, hwy)) +}\StringTok{ }\KeywordTok{geom_jitter}\NormalTok{()}
\KeywordTok{ggplot}\NormalTok{(mpg, }\KeywordTok{aes}\NormalTok{(drv, hwy)) +}\StringTok{ }\KeywordTok{geom_boxplot}\NormalTok{()}
\KeywordTok{ggplot}\NormalTok{(mpg, }\KeywordTok{aes}\NormalTok{(drv, hwy)) +}\StringTok{ }\KeywordTok{geom_violin}\NormalTok{()}
\end{Highlighting}
\end{Shaded}

\begin{figure}[H]
  \includegraphics[width=0.333\linewidth]{_figures/ggplot/jitter-boxplot-1}%
  \includegraphics[width=0.333\linewidth]{_figures/ggplot/jitter-boxplot-2}%
  \includegraphics[width=0.333\linewidth]{_figures/ggplot/jitter-boxplot-3}
\end{figure}

Each method has its strengths and weaknesses. Boxplots summarise the
bulk of the distribution with only five numbers, while jittered plots
show every point but only work with relatively small datasets. Violin
plots give the richest display, but rely on the calculation of a density
estimate, which can be hard to interpret.

For jittered points, \texttt{geom\_jitter()} offers the same control
over aesthetics as \texttt{geom\_point()}: \texttt{size},
\texttt{colour}, and \texttt{shape}. For \texttt{geom\_boxplot()} and
\texttt{geom\_violin()}, you can control the outline \texttt{colour} or
the internal \texttt{fill} colour.

\subsection{Histograms and frequency polygons}\label{sub:distribution}

Histograms and frequency polygons show the distribution of a single
numeric variable. They provide more information about the distribution
of a single group than boxplots do, at the expense of needing more
space. \index{Histogram} \indexf{geom\_histogram}

\begin{Shaded}
\begin{Highlighting}[]
\KeywordTok{ggplot}\NormalTok{(mpg, }\KeywordTok{aes}\NormalTok{(hwy)) +}\StringTok{ }\KeywordTok{geom_histogram}\NormalTok{()}
\CommentTok{#> `stat_bin()` using `bins = 30`. Pick better value with `binwidth`.}
\KeywordTok{ggplot}\NormalTok{(mpg, }\KeywordTok{aes}\NormalTok{(hwy)) +}\StringTok{ }\KeywordTok{geom_freqpoly}\NormalTok{()}
\CommentTok{#> `stat_bin()` using `bins = 30`. Pick better value with `binwidth`.}
\end{Highlighting}
\end{Shaded}

\begin{figure}[H]
  \includegraphics[width=0.5\linewidth]{_figures/ggplot/dist-1}%
  \includegraphics[width=0.5\linewidth]{_figures/ggplot/dist-2}
\end{figure}

Both histograms and frequency polygons work in the same way: they bin
the data, then count the number of observations in each bin. The only
difference is the display: histograms use bars and frequency polygons
use lines.

You can control the width of the bins with the \texttt{binwidth}
argument (if you don't want evenly spaced bins you can use the
\texttt{breaks} argument). It is \textbf{very important} to experiment
with the bin width. The default just splits your data into 30 bins,
which is unlikely to be the best choice. You should always try many bin
widths, and you may find you need multiple bin widths to tell the full
story of your data.

\begin{Shaded}
\begin{Highlighting}[]
\KeywordTok{ggplot}\NormalTok{(mpg, }\KeywordTok{aes}\NormalTok{(hwy)) +}\StringTok{ }
\StringTok{  }\KeywordTok{geom_freqpoly}\NormalTok{(}\DataTypeTok{binwidth =} \FloatTok{2.5}\NormalTok{)}
\KeywordTok{ggplot}\NormalTok{(mpg, }\KeywordTok{aes}\NormalTok{(hwy)) +}\StringTok{ }
\StringTok{  }\KeywordTok{geom_freqpoly}\NormalTok{(}\DataTypeTok{binwidth =} \DecValTok{1}\NormalTok{)}
\end{Highlighting}
\end{Shaded}

\begin{figure}[H]
  \includegraphics[width=0.5\linewidth]{_figures/ggplot/unnamed-chunk-6-1}%
  \includegraphics[width=0.5\linewidth]{_figures/ggplot/unnamed-chunk-6-2}
\end{figure}

An alternative to the frequency polygon is the density plot,
\texttt{geom\_density()}. I'm not a fan of density plots because they
are harder to interpret since the underlying computations are more
complex. They also make assumptions that are not true for all data,
namely that the underlying distribution is continuous, unbounded, and
smooth.

To compare the distributions of different subgroups, you can map a
categorical variable to either fill (for \texttt{geom\_histogram()}) or
colour (for \texttt{geom\_freqpoly()}). It's easier to compare
distributions using the frequency polygon because the underlying
perceptual task is easier. You can also use facetting: this makes
comparisons a little harder, but it's easier to see the distribution of
each group.

\begin{Shaded}
\begin{Highlighting}[]
\KeywordTok{ggplot}\NormalTok{(mpg, }\KeywordTok{aes}\NormalTok{(displ, }\DataTypeTok{colour =} \NormalTok{drv)) +}\StringTok{ }
\StringTok{  }\KeywordTok{geom_freqpoly}\NormalTok{(}\DataTypeTok{binwidth =} \FloatTok{0.5}\NormalTok{)}
\KeywordTok{ggplot}\NormalTok{(mpg, }\KeywordTok{aes}\NormalTok{(displ, }\DataTypeTok{fill =} \NormalTok{drv)) +}\StringTok{ }
\StringTok{  }\KeywordTok{geom_histogram}\NormalTok{(}\DataTypeTok{binwidth =} \FloatTok{0.5}\NormalTok{) +}\StringTok{ }
\StringTok{  }\KeywordTok{facet_wrap}\NormalTok{(~drv, }\DataTypeTok{ncol =} \DecValTok{1}\NormalTok{)}
\end{Highlighting}
\end{Shaded}

\begin{figure}[H]
  \includegraphics[width=0.5\linewidth]{_figures/ggplot/dist-fill-1}%
  \includegraphics[width=0.5\linewidth]{_figures/ggplot/dist-fill-2}
\end{figure}

\subsection{Bar charts}\label{sub:bar}

The discrete analogue of the histogram is the bar chart,
\texttt{geom\_bar()}. It's easy to use: \index{Barchart}
\indexf{geom\_bar}

\begin{Shaded}
\begin{Highlighting}[]
\KeywordTok{ggplot}\NormalTok{(mpg, }\KeywordTok{aes}\NormalTok{(manufacturer)) +}\StringTok{ }
\StringTok{  }\KeywordTok{geom_bar}\NormalTok{()}
\end{Highlighting}
\end{Shaded}

\begin{figure}[H]
  \includegraphics[width=1\linewidth]{_figures/ggplot/dist-bar-1}
\end{figure}

(You'll learn how to fix the labels in \hyperref[sub:theme-axis]{axis
labels}).

Bar charts can be confusing because there are two rather different plots
that are both commonly called bar charts. The above form expects you to
have unsummarised data, and each observation contributes one unit to the
height of each bar. The other form of bar chart is used for
presummarised data. For example, you might have three drugs with their
average effect:

\begin{Shaded}
\begin{Highlighting}[]
\NormalTok{drugs <-}\StringTok{ }\KeywordTok{data.frame}\NormalTok{(}
  \DataTypeTok{drug =} \KeywordTok{c}\NormalTok{(}\StringTok{"a"}\NormalTok{, }\StringTok{"b"}\NormalTok{, }\StringTok{"c"}\NormalTok{),}
  \DataTypeTok{effect =} \KeywordTok{c}\NormalTok{(}\FloatTok{4.2}\NormalTok{, }\FloatTok{9.7}\NormalTok{, }\FloatTok{6.1}\NormalTok{)}
\NormalTok{)}
\end{Highlighting}
\end{Shaded}

To display this sort of data, you need to tell \texttt{geom\_bar()} to
not run the default stat which bins and counts the data. However, I
think it's even better to use \texttt{geom\_point()} because points take
up less space than bars, and don't require that the y axis includes 0.

\begin{Shaded}
\begin{Highlighting}[]
\KeywordTok{ggplot}\NormalTok{(drugs, }\KeywordTok{aes}\NormalTok{(drug, effect)) +}\StringTok{ }\KeywordTok{geom_bar}\NormalTok{(}\DataTypeTok{stat =} \StringTok{"identity"}\NormalTok{)}
\KeywordTok{ggplot}\NormalTok{(drugs, }\KeywordTok{aes}\NormalTok{(drug, effect)) +}\StringTok{ }\KeywordTok{geom_point}\NormalTok{()}
\end{Highlighting}
\end{Shaded}

\begin{figure}[H]
  \includegraphics[width=0.5\linewidth]{_figures/ggplot/unnamed-chunk-8-1}%
  \includegraphics[width=0.5\linewidth]{_figures/ggplot/unnamed-chunk-8-2}
\end{figure}

\subsection{Time series with line and path plots}\label{sub:line}

Line and path plots are typically used for time series data. Line plots
join the points from left to right, while path plots join them in the
order that they appear in the dataset (in other words, a line plot is a
path plot of the data sorted by x value). Line plots usually have time
on the x-axis, showing how a single variable has changed over time. Path
plots show how two variables have simultaneously changed over time, with
time encoded in the way that observations are connected.

Because the year variable in the \texttt{mpg} dataset only has two
values, we'll show some time series plots using the \texttt{economics}
dataset, which contains economic data on the US measured over the last
40 years. The figure below shows two plots of unemployment over time,
both produced using \texttt{geom\_line()}. The first shows the
unemployment rate while the second shows the median number of weeks
unemployed. We can already see some differences in these two variables,
particularly in the last peak, where the unemployment percentage is
lower than it was in the preceding peaks, but the length of unemployment
is high. \indexf{geom\_line} \indexf{geom\_path}
\index{Data!economics@\texttt{economics}}

\begin{Shaded}
\begin{Highlighting}[]
\KeywordTok{ggplot}\NormalTok{(economics, }\KeywordTok{aes}\NormalTok{(date, unemploy /}\StringTok{ }\NormalTok{pop)) +}
\StringTok{  }\KeywordTok{geom_line}\NormalTok{()}
\KeywordTok{ggplot}\NormalTok{(economics, }\KeywordTok{aes}\NormalTok{(date, uempmed)) +}
\StringTok{  }\KeywordTok{geom_line}\NormalTok{()}
\end{Highlighting}
\end{Shaded}

\begin{figure}[H]
  \includegraphics[width=0.5\linewidth]{_figures/ggplot/line-employment-1}%
  \includegraphics[width=0.5\linewidth]{_figures/ggplot/line-employment-2}
\end{figure}

To examine this relationship in greater detail, we would like to draw
both time series on the same plot. We could draw a scatterplot of
unemployment rate vs.~length of unemployment, but then we could no
longer see the evolution over time. The solution is to join points
adjacent in time with line segments, forming a \emph{path} plot.

Below we plot unemployment rate vs.~length of unemployment and join the
individual observations with a path. Because of the many line crossings,
the direction in which time flows isn't easy to see in the first plot.
In the second plot, we colour the points to make it easier to see the
direction of time.

\begin{Shaded}
\begin{Highlighting}[]
\KeywordTok{ggplot}\NormalTok{(economics, }\KeywordTok{aes}\NormalTok{(unemploy /}\StringTok{ }\NormalTok{pop, uempmed)) +}\StringTok{ }
\StringTok{  }\KeywordTok{geom_path}\NormalTok{() +}
\StringTok{  }\KeywordTok{geom_point}\NormalTok{()}

\NormalTok{year <-}\StringTok{ }\NormalTok{function(x) }\KeywordTok{as.POSIXlt}\NormalTok{(x)$year +}\StringTok{ }\DecValTok{1900}
\KeywordTok{ggplot}\NormalTok{(economics, }\KeywordTok{aes}\NormalTok{(unemploy /}\StringTok{ }\NormalTok{pop, uempmed)) +}\StringTok{ }
\StringTok{  }\KeywordTok{geom_path}\NormalTok{(}\DataTypeTok{colour =} \StringTok{"grey50"}\NormalTok{) +}
\StringTok{  }\KeywordTok{geom_point}\NormalTok{(}\KeywordTok{aes}\NormalTok{(}\DataTypeTok{colour =} \KeywordTok{year}\NormalTok{(date)))}
\end{Highlighting}
\end{Shaded}

\begin{figure}[H]
  \includegraphics[width=0.5\linewidth]{_figures/ggplot/path-employ-1}%
  \includegraphics[width=0.5\linewidth]{_figures/ggplot/path-employ-2}
\end{figure}

We can see that unemployment rate and length of unemployment are highly
correlated, but in recent years the length of unemployment has been
increasing relative to the unemployment rate.

With longitudinal data, you often want to display multiple time series
on each plot, each series representing one individual. To do this you
need to map the \texttt{group} aesthetic to a variable encoding the
group membership of each observation. This is explained in more depth in
\hyperref[sec:grouping]{grouping}.
\index{Longitudinal data|see{Data, longitudinal}}
\index{Data!longitudinal}

\subsection{Exercises}

\begin{enumerate}
\def\labelenumi{\arabic{enumi}.}
\item
  What's the problem with the plot created by
  \texttt{ggplot(mpg,\ aes(cty,\ hwy))\ +\ geom\_point()}? Which of the
  geoms described above is most effective at remedying the problem?
\item
  One challenge with
  \texttt{ggplot(mpg,\ aes(class,\ hwy))\ +\ geom\_boxplot()} is that
  the ordering of \texttt{class} is alphabetical, which is not terribly
  useful. How could you change the factor levels to be more informative?

  Rather than reordering the factor by hand, you can do it automatically
  based on the data:
  \texttt{ggplot(mpg,\ aes(reorder(class,\ hwy),\ hwy))\ +\ geom\_boxplot()}.
  What does \texttt{reorder()} do? Read the documentation.
\item
  Explore the distribution of the carat variable in the
  \texttt{diamonds} dataset. What binwidth reveals the most interesting
  patterns?
\item
  Explore the distribution of the price variable in the
  \texttt{diamonds} data. How does the distribution vary by cut?
\item
  You now know (at least) three ways to compare the distributions of
  subgroups: \texttt{geom\_violin()}, \texttt{geom\_freqpoly()} and the
  colour aesthetic, or \texttt{geom\_histogram()} and facetting. What
  are the strengths and weaknesses of each approach? What other
  approaches could you try?
\item
  Read the documentation for \texttt{geom\_bar()}. What does the
  \texttt{weight} aesthetic do?
\item
  Using the techniques already discussed in this chapter, come up with
  three ways to visualise a 2d categorical distribution. Try them out by
  visualising the distribution of \texttt{model} and
  \texttt{manufacturer}, \texttt{trans} and \texttt{class}, and
  \texttt{cyl} and \texttt{trans}.
\end{enumerate}

\hyperdef{}{sec:axes}{\section{Modifying the axes}\label{sec:axes}}

You'll learn the full range of options available in
\hyperref[cha:scales]{scales}, but two families of useful helpers let
you make the most common modifications. \texttt{xlab()} and
\texttt{ylab()} modify the x- and y-axis labels: \indexf{xlab}
\indexf{ylab}

\begin{Shaded}
\begin{Highlighting}[]
\KeywordTok{ggplot}\NormalTok{(mpg, }\KeywordTok{aes}\NormalTok{(cty, hwy)) +}
\StringTok{  }\KeywordTok{geom_point}\NormalTok{(}\DataTypeTok{alpha =} \DecValTok{1} \NormalTok{/}\StringTok{ }\DecValTok{3}\NormalTok{)}

\KeywordTok{ggplot}\NormalTok{(mpg, }\KeywordTok{aes}\NormalTok{(cty, hwy)) +}
\StringTok{  }\KeywordTok{geom_point}\NormalTok{(}\DataTypeTok{alpha =} \DecValTok{1} \NormalTok{/}\StringTok{ }\DecValTok{3}\NormalTok{) +}\StringTok{ }
\StringTok{  }\KeywordTok{xlab}\NormalTok{(}\StringTok{"city driving (mpg)"}\NormalTok{) +}\StringTok{ }
\StringTok{  }\KeywordTok{ylab}\NormalTok{(}\StringTok{"highway driving (mpg)"}\NormalTok{)}

\CommentTok{# Remove the axis labels with NULL}
\KeywordTok{ggplot}\NormalTok{(mpg, }\KeywordTok{aes}\NormalTok{(cty, hwy)) +}
\StringTok{  }\KeywordTok{geom_point}\NormalTok{(}\DataTypeTok{alpha =} \DecValTok{1} \NormalTok{/}\StringTok{ }\DecValTok{3}\NormalTok{) +}\StringTok{ }
\StringTok{  }\KeywordTok{xlab}\NormalTok{(}\OtherTok{NULL}\NormalTok{) +}\StringTok{ }
\StringTok{  }\KeywordTok{ylab}\NormalTok{(}\OtherTok{NULL}\NormalTok{)}
\end{Highlighting}
\end{Shaded}

\begin{figure}[H]
  \includegraphics[width=0.333\linewidth]{_figures/ggplot/unnamed-chunk-9-1}%
  \includegraphics[width=0.333\linewidth]{_figures/ggplot/unnamed-chunk-9-2}%
  \includegraphics[width=0.333\linewidth]{_figures/ggplot/unnamed-chunk-9-3}
\end{figure}

\texttt{xlim()} and \texttt{ylim()} modify the limits of axes:
\indexf{xlim} \indexf{ylim}

\begin{Shaded}
\begin{Highlighting}[]
\KeywordTok{ggplot}\NormalTok{(mpg, }\KeywordTok{aes}\NormalTok{(drv, hwy)) +}
\StringTok{  }\KeywordTok{geom_jitter}\NormalTok{(}\DataTypeTok{width =} \FloatTok{0.25}\NormalTok{)}

\KeywordTok{ggplot}\NormalTok{(mpg, }\KeywordTok{aes}\NormalTok{(drv, hwy)) +}
\StringTok{  }\KeywordTok{geom_jitter}\NormalTok{(}\DataTypeTok{width =} \FloatTok{0.25}\NormalTok{) +}\StringTok{ }
\StringTok{  }\KeywordTok{xlim}\NormalTok{(}\StringTok{"f"}\NormalTok{, }\StringTok{"r"}\NormalTok{) +}\StringTok{ }
\StringTok{  }\KeywordTok{ylim}\NormalTok{(}\DecValTok{20}\NormalTok{, }\DecValTok{30}\NormalTok{)}
\CommentTok{#> Warning: Removed 138 rows containing missing values (geom_point).}
  
\CommentTok{# For continuous scales, use NA to set only one limit}
\KeywordTok{ggplot}\NormalTok{(mpg, }\KeywordTok{aes}\NormalTok{(drv, hwy)) +}
\StringTok{  }\KeywordTok{geom_jitter}\NormalTok{(}\DataTypeTok{width =} \FloatTok{0.25}\NormalTok{, }\DataTypeTok{na.rm =} \OtherTok{TRUE}\NormalTok{) +}\StringTok{ }
\StringTok{  }\KeywordTok{ylim}\NormalTok{(}\OtherTok{NA}\NormalTok{, }\DecValTok{30}\NormalTok{)}
\end{Highlighting}
\end{Shaded}

\begin{figure}[H]
  \includegraphics[width=0.333\linewidth]{_figures/ggplot/unnamed-chunk-10-1}%
  \includegraphics[width=0.333\linewidth]{_figures/ggplot/unnamed-chunk-10-2}%
  \includegraphics[width=0.333\linewidth]{_figures/ggplot/unnamed-chunk-10-3}
\end{figure}

Changing the axes limits sets values outside the range to \texttt{NA}.
You can suppress the associated warning with \texttt{na.rm\ =\ TRUE}.

\hyperdef{}{sec:output}{\section{Output}\label{sec:output}}

Most of the time you create a plot object and immediately plot it, but
you can also save a plot to a variable and manipulate it:

\begin{Shaded}
\begin{Highlighting}[]
\NormalTok{p <-}\StringTok{ }\KeywordTok{ggplot}\NormalTok{(mpg, }\KeywordTok{aes}\NormalTok{(displ, hwy, }\DataTypeTok{colour =} \KeywordTok{factor}\NormalTok{(cyl))) +}
\StringTok{  }\KeywordTok{geom_point}\NormalTok{()}
\end{Highlighting}
\end{Shaded}

Once you have a plot object, there are a few things you can do with it:

\begin{itemize}
\item
  Render it on screen with \texttt{print()}. This happens automatically
  when running interactively, but inside a loop or function, you'll need
  to \texttt{print()} it yourself. \indexf{print}

\begin{Shaded}
\begin{Highlighting}[]
\KeywordTok{print}\NormalTok{(p)}
\end{Highlighting}
\end{Shaded}

  \begin{figure}[H]
    \centering
    \includegraphics[width=0.65\linewidth]{_figures/ggplot/unnamed-chunk-11-1}
  \end{figure}
\item
  Save it to disk with \texttt{ggsave()}, described in
  \hyperref[sec:saving]{saving your output}.

\begin{Shaded}
\begin{Highlighting}[]
\CommentTok{# Save png to disk}
\KeywordTok{ggsave}\NormalTok{(}\StringTok{"plot.png"}\NormalTok{, }\DataTypeTok{width =} \DecValTok{5}\NormalTok{, }\DataTypeTok{height =} \DecValTok{5}\NormalTok{)}
\end{Highlighting}
\end{Shaded}
\item
  Briefly describe its structure with \texttt{summary()}.
  \indexf{summary}

\begin{Shaded}
\begin{Highlighting}[]
\KeywordTok{summary}\NormalTok{(p)}
\CommentTok{#> data: manufacturer, model, displ, year, cyl, trans, drv, cty,}
\CommentTok{#>   hwy, fl, class [234x11]}
\CommentTok{#> mapping:  x = displ, y = hwy, colour = factor(cyl)}
\CommentTok{#> faceting: facet_null() }
\CommentTok{#> -----------------------------------}
\CommentTok{#> geom_point: na.rm = FALSE}
\CommentTok{#> stat_identity: na.rm = FALSE}
\CommentTok{#> position_identity}
\end{Highlighting}
\end{Shaded}
\item
  Save a cached copy of it to disk, with \texttt{saveRDS()}. This saves
  a complete copy of the plot object, so you can easily re-create it
  with \texttt{readRDS()}. \indexf{saveRDS} \indexf{readRDS}

\begin{Shaded}
\begin{Highlighting}[]
\KeywordTok{saveRDS}\NormalTok{(p, }\StringTok{"plot.rds"}\NormalTok{)}
\NormalTok{q <-}\StringTok{ }\KeywordTok{readRDS}\NormalTok{(}\StringTok{"plot.rds"}\NormalTok{)}
\end{Highlighting}
\end{Shaded}
\end{itemize}

You'll learn more about how to manipulate these objects in
\hyperref[cha:programming]{programming with ggplot2}.

\hyperdef{}{qplot}{\section{Quick plots}\label{qplot}}

In some cases, you will want to create a quick plot with a minimum of
typing. In these cases you may prefer to use \texttt{qplot()} over
\texttt{ggplot()}. \texttt{qplot()} lets you define a plot in a single
call, picking a geom by default if you don't supply one. To use it,
provide a set of aesthetics and a data set: \indexf{qplot}

\begin{Shaded}
\begin{Highlighting}[]
\KeywordTok{qplot}\NormalTok{(displ, hwy, }\DataTypeTok{data =} \NormalTok{mpg)}
\KeywordTok{qplot}\NormalTok{(displ, }\DataTypeTok{data =} \NormalTok{mpg)}
\CommentTok{#> `stat_bin()` using `bins = 30`. Pick better value with `binwidth`.}
\end{Highlighting}
\end{Shaded}

\begin{figure}[H]
  \includegraphics[width=0.5\linewidth]{_figures/ggplot/unnamed-chunk-15-1}%
  \includegraphics[width=0.5\linewidth]{_figures/ggplot/unnamed-chunk-15-2}
\end{figure}

Unless otherwise specified, \texttt{qplot()} tries to pick a sensible
geometry and statistic based on the arguments provided. For example, if
you give \texttt{qplot()} \texttt{x} and \texttt{y} variables, it'll
create a scatterplot. If you just give it an \texttt{x}, it'll create a
histogram or bar chart depending on the type of variable.

\texttt{qplot()} assumes that all variables should be scaled by default.
If you want to set an aesthetic to a constant, you need to use
\texttt{I()}: \indexf{I}

\begin{Shaded}
\begin{Highlighting}[]
\KeywordTok{qplot}\NormalTok{(displ, hwy, }\DataTypeTok{data =} \NormalTok{mpg, }\DataTypeTok{colour =} \StringTok{"blue"}\NormalTok{)}
\KeywordTok{qplot}\NormalTok{(displ, hwy, }\DataTypeTok{data =} \NormalTok{mpg, }\DataTypeTok{colour =} \KeywordTok{I}\NormalTok{(}\StringTok{"blue"}\NormalTok{))}
\end{Highlighting}
\end{Shaded}

\begin{figure}[H]
  \includegraphics[width=0.5\linewidth]{_figures/ggplot/unnamed-chunk-16-1}%
  \includegraphics[width=0.5\linewidth]{_figures/ggplot/unnamed-chunk-16-2}
\end{figure}

If you're used to \texttt{plot()} you may find \texttt{qplot()} to be a
useful crutch to get up and running quickly. However, while it's
possible to use \texttt{qplot()} to access all of the customizability of
ggplot2, I don't recommend it. If you find yourself making a more
complex graph, e.g.~using different aesthetics in different layers or
manually setting visual properties, use \texttt{ggplot()}, not
\texttt{qplot()}.

\providecommand{\setflag}{\newif \ifwhole \wholefalse}
\setflag
\ifwhole\else

% Typography and geometry ----------------------------------------------------
\documentclass[letterpaper]{scrbook}
\usepackage[inner=3cm,top=2.5cm,outer=3.5cm]{geometry}

\renewcommand\familydefault{bch}
\usepackage[utf8]{inputenc}
\usepackage{microtype}
\usepackage[small]{caption}
\usepackage[small]{titlesec}
\raggedbottom

% Graphics -------------------------------------------------------------------
\usepackage[pdftex]{graphicx}
\graphicspath{{_include/}}
\DeclareGraphicsExtensions{.png,.pdf}

% Code formatting ------------------------------------------------------------
\usepackage{fancyvrb}
\usepackage{courier}
\usepackage{listings}
\usepackage{color}
\usepackage{alltt}


\definecolor{comment}{rgb}{0.60, 0.60, 0.53}
\definecolor{background}{rgb}{0.97, 0.97, 1.00}
\definecolor{string}{rgb}{0.863, 0.066, 0.266}
\definecolor{number}{rgb}{0.0, 0.6, 0.6}
\definecolor{variable}{rgb}{0.00, 0.52, 0.70}
\lstset{
  basicstyle=\ttfamily,
  keywordstyle=\bfseries, 
  identifierstyle=,  
  commentstyle=\color{comment} \emph,
  stringstyle=\color{string},
  showstringspaces=false,
  columns = fullflexible,
  backgroundcolor=\color{background},
  mathescape = true,
  escapeinside=&&,
  fancyvrb
}
\newcommand{\code}[1]{\lstinline!#1!}



% Links ----------------------------------------------------------------------

\usepackage{hyperref}
\definecolor{slateblue}{rgb}{0.07,0.07,0.488}
\hypersetup{colorlinks=true,linkcolor=slateblue,anchorcolor=slateblue,citecolor=slateblue,filecolor=slateblue,urlcolor=slateblue,bookmarksnumbered=true,pdfview=FitB}
\usepackage{url}

% Tables ---------------------------------------------------------------------
\usepackage{longtable}
\usepackage{booktabs}

% Miscellaneous --------------------------------------------------------------
\usepackage{pdfsync}
\usepackage{appendix}

\usepackage[round,sort&compress,sectionbib]{natbib}
\bibliographystyle{plainnat}


\title{ggplot2}
\author{Hadley Wickham}

\begin{document}
\fi


\chapter{Toolbox}
\label{cha:toolbox}

\section{Introduction}\label{sec:introduction}

Graphical objects, or geoms for short, are a key feature of {\tt geom\_plot}.  Geoms create visual objects on the plot that allow you to see you data.  By choosing various types of geoms, you can recreate common plots.  For example, the {\tt geom\_point} geom will make a scatterplot, and the {\tt geom\_line} geom will create a line plot.  The advantage of the geom based system is that you can easily combine different geoms to create almost any plot that you can think of.  This section describes geom functions in detail, including what geoms are currently available in {\tt geom\_plot} and how you can go about creating your own.

Use a mixture of {\tt ggplot()} and {\tt qplot()} calls.  If you need a reminder on how to translate between the two, see Section~\ref{sec:qplot-ggplot}.

\section{Basics}\label{sub:basics}

These geoms are the basic geoms used to build up almost all of the other geoms.  These are useful for creating basic graphics, and when building your own geom function (see section \ref{sec:writing_your_own}).

\begin{itemize}
  \item {\tt geom\_point}: points
  \item {\tt geom\_line}, {\tt geom\_path}: paths and lines.  Lines are paths that have their x-axis values ordered in increasing value.
  \item {\tt geom\_polygon}: polygons
  \item {\tt geom\_bar}: bars
  \item {\tt geom\_text}: text
  \item {\tt geom\_tile}: tiles, rectangles which form a regular tessellation of the plane
\end{itemize}

\section{Displaying distributions}\label{sec:distributions}

There are quite a few geoms associated with displaying distributions:

\begin{itemize}
	\item {\tt geom\_boxplot}: box and whisker plot, for a continuous variable possibly conditioned by a categorical variable
	\item {\tt geom\_jitter}: a crude way of investigating densities
	\item {\tt geom\_quantile}: quantiles, for a continuous variable conditional on another continuous variable.
	\item {\tt geom\_density}: 
	\item {\tt geom\_histogram}: 
	\item {\tt geom\_2ddensity}: for displaying the density of points on the plot surface.
\end{itemize}

(Ask Heike about this)

We have a number of plots available to investigate distributions, depending on the number of variables, and whether we are interested in the conditional or joint distribution.

Single variable
+ cont: histogram, density plot, boxplot
+ cat:  barchart

Two variables: conditional
+ cat  | cat:  mosaic plot
+ cat  | cont: ?
+ cont | cont: quantiles
+ cat  | cont: boxplot

Two variables: joint
+ cat  * cat:  fluctuation diagram
+ cat  * cont: boxplots?
+ cont * cont: bagplot, geom\_2ddensity

Jittered points can be used for any joint distribution (or conditional if one or both variables are categorical)

\section{Dealing with overplotting}\label{sec:overplotting}

The simplest way to deal with overplotting is to bin the plot into small squares and count the number of points that lies in each square, much like a 2D histogram.  This count can then be visualised as the third variable on a plot.  However, breaking the plot into many small squares produces distracting visual artefacts.  Carr (reference) suggests using hexagons instead, and this is implemented with {\tt geom\_hexagon}, using the capabilities of the {\tt hexbin package}.

A continuous analogue of this is to compute a 2D density function and then visualise this as coloured tiles or contour lines.  This can be done with {\tt geom\_2ddensity}.

Another approach to dealing with overplotting is to add supplemental information to help guide the eye to the true shape of or pattern within the data:

\begin{itemize}
	\item {\tt geom\_smooth} add a smooth line showing the mean.
	\item {\tt geom\_quantile} add a smooth line showing any quantile you are interested in.
\end{itemize}

\section{``3d'' plots}

{\tt geom\_plot} currently does not support true 3D plots.  However, it does offer two tools for producing pseudo-3d plots, the imageplot and the contour plot.

\begin{itemize}
	\item {\tt geom\_tile}: map z variable to fill colour
	\item {\tt geom\_contour}: useful for smoother surfaces
	\item {\tt geom\_point}: can map abs(z) variable to size and sign(z) to colour
\end{itemize}

These are often better than ``true'' 3d for static plots anyway, because many perceptual cues necessary for accurate depth perception (eg. occlusion, parallax) are not present in static plots.  You can also manually project data points in a higher dimensional space by multiplying by a projection matrix.  However, correctly representing occlusion or generating correct perspective effects will require considerably more effort.  You may want to look at RGL (\url{http://www.rgl.com}) and rggobi (\url{http://www.ggobi.org/ggobi}) for other solutions.

\section{Revealing uncertainty}\label{sub:displaying_uncertainty}

If you have information about the uncertainty present in your data, possibly from a model or distributional assumptions, it is often useful to visualise.  There are two geoms that allow you to do this depending on whether you have point or functional confidence intervals:

\begin{itemize}
	\item {\tt geom\_errorbar}: for pointwise confidence intervals
	\item {\tt geom\_ribbon}: ribbons of variable width, useful for displaying confidence intervals around functions
\end{itemize}

There are two ways to display standard errors with {\tt ggplot}.  For point standard errors, you can use the {\tt errorbar} geom.  For continuous or functional standard errors, you can use the {\tt ribbon} grob.  We've have already seen an example of this: the {\tt ribbon} grob is used inside {\tt smooth} to display the standard errors of the smooth.  Because there are so many different ways to calculate standard errors, the calculation is up to you.  {\tt ggplot} only provides facilities for displaying them once you have them.

For both {\tt ribbon} and {\tt errobar} you can specify confidence internals using {\tt min} and {\tt lower} which specify the upper and lower edges of the confidence band.

Calculating with {\tt stat\_sum}, {\tt stat\_smooth} etc.

\section{Annotating a plot}\label{sub:annotating_a_plot}

Any of the basic geoms can be used to annotate a plot with additional output, for example, adding text with {\tt geom\_text}, or a point illustrating the mean with {\tt geom\_point}.  Additionally, there are several geoms whose use is almost entirely for annotation.  These are:

\begin{itemize}
	\item {\tt geom\_vline}, {\tt geom\_hline}: add vertical or horizontal lines to a plot
	\item {\tt geom\_abline}: add lines with arbitrary slope and intercept to a plot
\end{itemize}

See also \secref{sec:adding_annotation} for ways to add more general types of annotation using grid graphics.

\section{Plots for weighted data}\label{sec:weighted_data}

When you have aggregated data where each row in the dataset represents multiple observations, you need some way to take into account the weighting variable.  Since there are no variables appropriate for weighting in the diamonds data, we will use some data collected on Midwest states in the 2000 US census.  The data consists mainly of percentages (eg. percent white, percent below poverty line, percentage with college degree) and some information for each county (area, total population, population density).

There are few different things we might want to weight by: 

\begin{itemize}
	\item nothing, to look at county numbers
	\item total population, to work with absolute numbers
	\item area, to investigate geographic effects
\end{itemize}

\noindent The choice of a weighting variable profoundly effects what we are looking at in the plot and the conclusions that we will draw.  There are two aesthetic attributes that can be used to adjust for weights.  Firstly, for simple geoms like lines and points, you can make the size of the grob proportional to the number of points, using the {\tt size} aesthetic, as follows:

% decumar<<< 
% interweave({
% midwest <- read.csv("~/Documents/graphics/weighted/midwest.csv")
% qplot(percwhite, percbelowpoverty, data=midwest)
% qplot(percwhite, percbelowpoverty, data=midwest, size=poptotal)
% qplot(percwhite, percbelowpoverty, data=midwest, size=area)
% })
% |||
\begin{alltt}
> midwest <- read.csv("~/Documents/graphics/weighted/midwest.csv")
> qplot(percwhite, percbelowpoverty, data = midwest)
\includegraphics[scale=0.5]{4b8c600efdc1f31f77c7c3368cac11d2}

> qplot(percwhite, percbelowpoverty, data = midwest, size = poptotal)
\includegraphics[scale=0.5]{f3976daae1f4a7ef912ea259ca3dd8f5}

> qplot(percwhite, percbelowpoverty, data = midwest, size = area)
\includegraphics[scale=0.5]{440b95f469e5b96a08e077d4a177e603}

\end{alltt}
% >>>

For more complicated grobs which involve some statistical transformation, we specify weights with the {\tt weight} aesthetic.  These weights will be passed on to the statistical summary function.  Weights are supported for every case where it makes sense: smoothers, quantile regressions, box plots, histograms, and density plots.  You can't see this weighting variable directly, and it doesn't produce a legend, but it will change the results of the statistical summary.

The following example shows how weighting by population density effects the relationship between percent white and percent below the poverty line.

% decumar<<< 
% interweave({
% qplot(percwhite, percbelowpoverty, data=midwest, geom=c("point","smooth"), method=lm)
% qplot(percwhite, percbelowpoverty, data=midwest, size=popdensity, weight=popdensity,geom=c("point","smooth"), method=lm)
% })
% |||
\begin{alltt}
> qplot(percwhite, percbelowpoverty, data = midwest, geom = c("point", 
+     "smooth"), method = lm)
\includegraphics[scale=0.5]{54d5b5a95badb06ff2339bfd51826af0}

> qplot(percwhite, percbelowpoverty, data = midwest, size = popdensity, 
+     weight = popdensity, geom = c("point", "smooth"), method = lm)
\includegraphics[scale=0.5]{3353081c3ffa3fd62b69fb2c0aef9142}

\end{alltt}
% >>>

When we weight a histogram or density plot by total population, we change from looking at the distribution of the number of counties, to the distribution of the number of people.  This example shows the difference this makes for a histogram and density plot of the percentage below the poverty line.

% decumar<<< 
% interweave({
% qplot(percbelowpoverty, data=midwest, geom="histogram", binwidth=1)
% qplot(percbelowpoverty, data=midwest, geom="histogram", weight=poptotal, binwidth=1)
% })
% |||
\begin{alltt}
> qplot(percbelowpoverty, data = midwest, geom = "histogram", binwidth = 1)
\includegraphics[scale=0.5]{d8cef4ee59175bda59eaa2647a64b930}

> qplot(percbelowpoverty, data = midwest, geom = "histogram", weight = poptotal, 
+     binwidth = 1)
\includegraphics[scale=0.5]{affb33e87c76bc94bd3ea9eb2df0190c}

\end{alltt}
% >>>

\ifwhole
\else
  \bibliography{/Users/hadley/documents/phd/references}
  \end{document}
\fi


\part{The Grammar}

\providecommand{\setflag}{\newif \ifwhole \wholefalse}
\setflag
\ifwhole\else

% Typography and geometry ----------------------------------------------------
\documentclass[letterpaper]{scrbook}
\usepackage[inner=3cm,top=2.5cm,outer=3.5cm]{geometry}

\renewcommand\familydefault{bch}
\usepackage[utf8]{inputenc}
\usepackage{microtype}
\usepackage[small]{caption}
\usepackage[small]{titlesec}
\raggedbottom

% Graphics -------------------------------------------------------------------
\usepackage[pdftex]{graphicx}
\graphicspath{{_include/}}
\DeclareGraphicsExtensions{.png,.pdf}

% Code formatting ------------------------------------------------------------
\usepackage{fancyvrb}
\usepackage{courier}
\usepackage{listings}
\usepackage{color}
\usepackage{alltt}


\definecolor{comment}{rgb}{0.60, 0.60, 0.53}
\definecolor{background}{rgb}{0.97, 0.97, 1.00}
\definecolor{string}{rgb}{0.863, 0.066, 0.266}
\definecolor{number}{rgb}{0.0, 0.6, 0.6}
\definecolor{variable}{rgb}{0.00, 0.52, 0.70}
\lstset{
  basicstyle=\ttfamily,
  keywordstyle=\bfseries, 
  identifierstyle=,  
  commentstyle=\color{comment} \emph,
  stringstyle=\color{string},
  showstringspaces=false,
  columns = fullflexible,
  backgroundcolor=\color{background},
  mathescape = true,
  escapeinside=&&,
  fancyvrb
}
\newcommand{\code}[1]{\lstinline!#1!}



% Links ----------------------------------------------------------------------

\usepackage{hyperref}
\definecolor{slateblue}{rgb}{0.07,0.07,0.488}
\hypersetup{colorlinks=true,linkcolor=slateblue,anchorcolor=slateblue,citecolor=slateblue,filecolor=slateblue,urlcolor=slateblue,bookmarksnumbered=true,pdfview=FitB}
\usepackage{url}

% Tables ---------------------------------------------------------------------
\usepackage{longtable}
\usepackage{booktabs}

% Miscellaneous --------------------------------------------------------------
\usepackage{pdfsync}
\usepackage{appendix}

\usepackage[round,sort&compress,sectionbib]{natbib}
\bibliographystyle{plainnat}


\title{ggplot2}
\author{Hadley Wickham}

\begin{document}
\fi


\chapter{Mastering the grammar}
\label{cha:mastery}

% Introduction to the components of the grammar
% Introduction to the data structure
% Roadmap for next few chapters


\section{Introduction}\label{sec:introduction}

You can choose to use just {\tt qplot}, without any understanding of the underlying grammar, but you will not be able to use the full power of ggplot.  By learning more about the grammar, and the components that make it up, you will be able to create a wider range of plots, as well as being able to combine multiple sources of data, and customise to your heart's content.

This chapter describes the theoretical basis of ggplot2: the layered grammar of graphics, a based based on Wilkinson's grammar of graphics \citep{wilkinson:2006}.  The next chapters then describe parts of the grammar in more detail: layers (geoms and stats), scales, and positioning (coordinate systems, faceting and position adjustments),

You may want to skip this chapter in a first reading of the book, and then come back to it when you want a deeper understanding of how all the pieces fit together.

This chapter begins by describing in detail a simple plot is drawn.  We'll start with a simple scatterplot, and then grow progressively more complicated, by transforming the axes and adding a smooth line.

\section{Building a scatterplot}
\label{sec:building_a_plot}

When creating a plot we start with data.  Consider the mammal's sleep dataset illustrated in Table~\ref{tbl:sleep}.  This is included in \ggplot and is called \code{msleep}.  The data records the brain and body weights, sleep times, and a few other variables for 83 mammals.   You can find out more about the data by typing \code{?msleep} within R. 

% msleep <- read.csv("~/Documents/data/08-msleep/msleep.csv")
% animals <- sort(c("African elephant", "Pilot whale", "Human", "Horse", "Domestic cat"))
% msel <- subset(msleep, name %in% animals)


% source("latex.r")
% tabulate(msel[, c("name", "order", "brainwt", "bodywt", "awake", "sleep_rem")])

\begin{table}
  \begin{center}
  \begin{tabular}{llrrp{2cm}p{2cm}}
    \toprule
    Animal & Order & Brain weight & Body weight & Awake (hours) & Rem sleep (hours) \\
    \midrule
    Domestic cat     & Carnivora      & 0.0256 &    3.3 & 11.50 & 3.2\\
    Horse            & Perissodactyla & 0.6550 &  521.0 & 21.10 & 0.6\\
    Human            & Primates       & 1.3200 &   62.0 & 16.00 & 1.9\\
    African elephant & Proboscidea    & 5.7120 & 6654.0 & 20.70 & ---\\
    Pilot whale      & Cetacea        &    --- &  800.0 & 21.35 & 0.1\\
    \bottomrule
  \end{tabular}
  \end{center}
  \caption{Five mammals from the \code{msleep} data set.  Missing values are illustrated with a ---.  }
  \label{label}
\end{table}


Let's draw a scatterplot of \var{brainwt} vs. \var{bodywt}.  What exactly is a scatterplot?  One way to describe it is has a point for each observation, positioned horizontally according to the value of \var{brainwt}, and vertically according to \var{bodywt}.  The first step in making this plot is to create a new dataset which reflects the mapping of $x$-position to \var{brainwt}, and $y$-position to \var{bodywt} and colour to \var{diet}.  $x$-position, $y$-position and colour are all aesthetics, things that can be perceived on the graphic. We will also remove all other variables that do not appear in the plot.  This is shown in Table \ref{tbl:mapping}.

\begin{table}[ht]
	\begin{center}
	\begin{tabular}{r|r|r}
		$x$ & $y$ & $colour$\\
		\hline
		2 & 4 & a\\
		1 & 1 & a\\
		4 & 15 & b\\
		9 & 80 & b
	\end{tabular}
	\end{center}
	\caption{Sample rows from \code{msleep} with variables named according to the aesthetic that they use.}
	\label{tbl:mapping}
\end{table}

We can create many different types of plots using this data.  For example, if we were to draw lines instead of points we would get a line plot.  If we used bars, we'd get a bar plot.  Bars, lines and points are all examples of geometric objects.  

These values currently have no meaning to the computer.  We need to convert them from being in data units (e.g.\ kg) to physical units (e.g.\ pixels), that the computer knows how to display.  (The way that these physical units can been specified in R is described in Appendix~\ref{chp:specification}). Converting position is easy for this example which has linear scales and a Cartesian coordinate system - we just need a linear mapping from the range of the data to $[0, 1]$.  (We can use $[0, 1]$ instead of exact pixels because the drawing system that \ggplot uses, \code{grid}, will take care of that for us.)

\[ \frac{x - \min(x)}{\mbox{range}(x)}  \] 

It is the coordinate system that performs this mapping, and for other coordinate systems the mapping is more complex (and may depend on both x and y variables), but in all cases is mapped to range $[0, 1]$.

The process for mapping the colour is a little more complicated, and here the scale does all of the work.  A scale needs to know the range and the domain of the data.  The default scale in \ggplot maps colour to evenly space hues on a colour wheel.


\begin{table}[ht]
	\begin{center}
	\begin{tabular}{r|r|r}
		$x$ & $y$ & $colour$\\
		\hline
		25 & 11 & red\\
		0 & 0 & red \\
		75  & 53 & blue \\
		200 & 300 & blue
	\end{tabular}
	\end{center}
	\caption{Simple dataset with variables mapped into aesthetic space.}
	\label{tbl:scaled}
\end{table}

In general, there is another step that we've skipped in this simple example: a statistical transformation.  Here we are using the identity transformation, but there are many others that are useful, such as binning or aggregating.  Statistical transformations, or stats, are described in detail in Section~\ref{sub:stats}.

Finally, we need to render this data to create the graphical objects that are displayed on the screen.  To create a complete plot we need to combine graphical objects from three sources: the \emph{data}, represented by the point geom; the \emph{scales and coordinate system}, which generates axes and legends so that we can read values from the graph; and the \emph{plot annotations}, such as the background and plot title.  These components are shown in Figure~\ref{fig:simple-exploded}.  Combining and displaying these graphical objects produces the final plot, as in Figure~\ref{fig:simple}.

\begin{figure}[htbp]
	\centering
	\caption{Graphics objects produced by (from left to right): geometric objects, scales and coordinate system, plot annotations.}
	\label{fig:simple-exploded}
\end{figure}

\begin{figure}[htbp]
	\centering
	\caption{The final graphic, produced by combining the pieces in Figure~\ref{fig:simple-exploded}.}
	\label{fig:simple}
\end{figure}

\subsection{A more complicated plot}\label{sec:how_to_build_a_more_complicated_plot} 

Now that you are acquainted with drawing a simple plot, we will create a more complicated plot.  The big difference with this plot is that we'll use faceting.  Faceting is also known as conditioning, trellising and latticing, and produces small multiples showing different subsets of the data.  If we facet the previous plot by $D$ we will get a plot that looks like Figure~\ref{fig:complex}, where each value of $D$ is displayed in a different panel.

\begin{figure}[htbp]
	\centering
	\caption{A more complicated plot, which is faceted by variable $D$.  Here the faceting uses the same variable that is mapped to colour so that there is some redundancy in our visual representation.  This allows us to easily see how the data has been broken into panels.}
	\label{fig:complex}
\end{figure}

Faceting splits the original dataset into a dataset for each subset, so the data that underlies Figure~\ref{fig:complex} looks like Table \ref{tbl:complex}.

\begin{table}[ht]
	\centering
	\begin{tabular}{r|r|r|r}
		& $x$ & $y$ & $colour$\\
		\hline
		a & 2 & 4 & red\\
		a & 1 & 1 & red\\
		\hline \hline
		b & 4 & 15 & blue\\
		b & 9 & 80 & blue
	\end{tabular}

	\caption{Simple dataset faceted into subsets.}
	\label{tbl:complex}
\end{table}

The first steps of plot creation proceed as before, but new steps are necessary when we get to the scales.   Scaling actually occurs in three parts: transforming, training and mapping. 

\begin{itemize}
	\item  Scale transformation occurs before statistical transformation so that statistics are computed on the scale-transformed data.  This ensures that a plot of $log(x)$ vs $log(y)$ on linear scales looks the same as $x$ vs $y$ on log scales.  See Section~\ref{sub:transformation} for more details. Transformation is only necessary for non-linear scales, because all statistics are location-scale invariant.

	\item After the statistics are computed, each scale is trained on every faceted dataset (a plot can contain multiple datasets, e.g.\ raw data and predictions from a model).  The training operation combines the ranges of the individual datasets to get the range of the complete data.  If scales were applied locally, comparisons would only be meaningful within a facet.  This is shown in Table \ref{tbl:complex-incorrect}.

  % FIXME
	\item Finally the scales map the data values into aesthetic values.  This gives Table \ref{tbl:complex-mapping} which is essentially identical to Table \ref{tbl:mapping} apart from the structure of the datasets.  Given that we end up with an essentially identical structure you might wonder why we don't simply split up the final result.  There are several reasons for this.  It makes writing statistical transformation functions easier, as they only need to operate on a single facet of data, and some need to operate on a single subset, for example, calculating a percentage.  Also, in practice we may have a more complicated training scheme for the position scales so that different columns or rows can have different $x$ and $y$ scales.  
	
\end{itemize}

\begin{table}[ht]
	\centering
	\begin{tabular}{r|r|r|r}
		& $x$ & $y$ & $colour$\\
		\hline
		a & 200 & 300 & red\\
		a & 0 & 0 & red\\
		\hline \hline
		b & 0 & 0 & red\\
		b & 200 & 300 & red
	\end{tabular}

	\caption{Local scaling, where data are scaled independently within each facet. Note that each facet occupies the full range of positions, and only uses one colour.  Comparisons across facets are not necessarily meaningful.}
	\label{tbl:complex-incorrect}
\end{table}

\begin{table}[ht]
	\centering
	\begin{tabular}{r|r|r|r}
		& $x$ & $y$ & $colour$\\
		\hline
		a & 25 & 11 & red\\
		a & 0 & 0 & red\\
		\hline \hline
		b & 75 & 53 & blue\\
		b & 200 & 300 & blue
	\end{tabular}

	\caption{faceted data correctly mapped to aesthetics.  Note the similarity to Table \ref{tbl:scaled}.}
	\label{tbl:complex-mapping}
\end{table}

\subsection{Summary}\label{sec:the_grammar}

In the examples above, we have seen some of the components that make up a plot:

\begin{itemize}
  \item data and aesthetic mappings,
  \item geometric objects, 
  \item scales,
  \item and facet specification.
\end{itemize}

\noindent We have also touched on two other components: 

\begin{itemize}
  \item statistical transformations,
  \item and the coordinate system.
\end{itemize}

\noindent Together, the data, mappings, statistical transformation and geometric object form a layer.  A plot may have multiple layers, for example, when we overlay a scatterplot with a smoothed line.

\section{Components of the layered grammar}

To be precise, the layered grammar defines the components of a plot as:

\begin{itemize}
  \item A default dataset and set of mappings from variables to aesthetics.
  \item One or more layers, each composed of a geometric object, a statistical transformation, and a position adjustment, and optionally, a dataset and aesthetic mappings.
  \item One scale for each aesthetic mapping used.
  \item A coordinate system.
  \item The facet specification.
\end{itemize}


The layer component is particularly important as it determines the physical representation of the data, with the combination of stat and geom defining many familiar named graphics: the scatterplot, histogram, contourplot, and so.  In practice, many plots have (at least) three layers: the data, context for the data, and a statistical summary of the data.  For example, to visualise a  spatial point process, we might display the points themselves, a map giving some context to the locations of points, and contours of a 2d density estimate.

This grammar is useful for both the user and the developer of statistical graphics.  For the user, it makes it easier to iteratively update a plot, changing a single feature at a time.  The grammar is also useful because it suggests the high level aspects of a plot that \emph{can} be changed, giving us a framework to think about graphics, and hopefully shortening the distance from mind to paper.  It also encourages the use of graphics customised to a particular problem, rather than relying on generic named graphics.

For the developer, it makes it much easier to add new capabilities. You only need to add the one component that you need, and continue to use the all the other existing components.  For example, you can add a new statistical transformation, and continue to use the existing scales and geoms.  It is also useful for discovering new types of graphics, as the grammar effectively defines the parameter space of statistical graphics.

\subsection{Layers}

Layers are responsible for creating the objects that we perceive on the plot.  A layer is composed of four parts:  

\begin{itemize}
	\item data and aesthetic mapping,
	\item a statistical transformation (stat), 
	\item a geometric object (geom)
	\item and a position adjustment.
\end{itemize}

\noindent These parts are described in detail below.

Usually all the layers on a plot have something in common, which is typically that they are different views of the same data, e.g.\ a scatterplot with overlaid smoother.  


\subsubsection{Data and mapping}\label{sub:data_and_mapping} 

Data is obviously a critical part of the plot, but it is important to remember that it is independent from the other components: we can construct a graphic that can be applied to multiple datasets. Data is what turns an abstract graphic into a concrete graphic.

Along with the data, we need a specification of which variables are mapped to which aesthetics.  For example, we might map weight to x position, height to y position and age to size.  The details of the mapping are described by the scales, Section~\ref{sec:scales}.  Choosing a good mapping is crucial for generating a useful graphic, as described in Section~\ref{sec:strategy}.

\subsubsection{Statistical transformation}\label{sub:stats} 

A statistical transformation, or {\bf stat}, transforms the data, typically by summarising it in some manner.  For example, a useful stat is the smoother, which calculates the mean of y, conditional on x, subject to some restriction that ensures smoothness. Table \ref{tbl:statistics} lists some of the stats available in ggplot2.  To make sense in a graphic context a stat must be location-scale invariant: $\mbox{f}(x + a) = \mbox{f}(x) + a$ and $\mbox{f}(b \cdot x) = b \cdot \mbox{f}(x)$.  This ensures that the transformation is invariant under translation and scaling, which are common operations on a graphic.

A stat takes a dataset as input and returns a dataset as output, and so a stat can add new variables to the original dataset.  It is possible to map aesthetics to these new variables.  For example, one way to describe a histogram is as a binning of a continuous variable, plotted with bars whose height is proportional to the number of points in each bin, as described in Section \ref{sub:histogram}.  Another useful example is mapping the size of the lines in a contour plot to the height of the contour.

The actual statistical method used by a stat is conditional on the coordinate system.  For example, a smoother in polar coordinates should use circular regression, and in 3d should return a 2d surface rather than a 1d curve.  However, many statistical operations have not been derived for non-Cartesian coordinates and we so we generally fall back to Cartesian coordinates for calculation, which, while not strictly correct, will normally be a fairly close approximation.  

\begin{table}
	\begin{center}
	\begin{tabular}{l|l}
	Name & Description \\
	\hline
	bin & Divide continuous range into bins, and count number of points in each\\ 
	boxplot & Compute statistics necessary for boxplot\\
	contour & Calculate contour lines\\
	density & Compute 1d density estimate \\
	identity & Identity transformation, $f(x) = x$ \\
	jitter & Jitter values by adding small random value \\
	qq & Calculate values for quantile-quantile plot \\
	quantile & Quantile regression\\
	smooth & Smoothed conditional mean of $y$ given $x$ \\
	summary & Aggregate values of $y$ for given $x$ \\
	sortx & Sort values in order of ascending $x$\\
	unique & Remove duplicated observations\\
	\end{tabular}
	\end{center}
	\caption{Some statistical transformations provided by ggplot2.  The user is able to supplement this list in a straight forward manner.}
	\label{tbl:statistics}
\end{table}

\subsubsection{Geometric object}\label{sub:geometric-objects} 

Geometric objects, or {\bf geom}s for short, control the type of plot that you create.  For example, using a point geom will create a scatterplot, while using a line geom will create a line plot.  We can classify geoms by their dimensionality:

\begin{itemize}
	\item 0d: point, text
	\item 1d: path, line (ordered path)
	\item 2d: polygon, interval
\end{itemize}

Geometric objects are an abstract component and can be rendered in different ways. Figure~\ref{fig:interval} illustrates four possible renderings of the interval geom. 

\begin{figure}[htbp]
	\centering
	\caption{Four representations of an interval geom.  From left to right: as a bar, as a line, as a error bar, and (for continuous x) as a ribbon.}
	\label{fig:interval}
\end{figure}

Geoms are mostly general purpose, but do require certain outputs from a statistic.  For example, the boxplot geom requires the position of the upper and lower fences, upper and lower hinges, the middle bar and the outliers. Any statistic used with the boxplot needs to provide these values. 

Every geom has a default statistic, and every statistic a default geom.  For example, the bin statistic defaults to using the bar geom to produce a histogram.  Over-riding these defaults will still produce valid plots, but they may violate graphical conventions.

Each geom can only display certain aesthetics.  For example, a point geom has position, colour, and size aesthetics.  A bar geom has all those, plus height, width and fill colour.  Different parameterisations may be useful.  For example, instead of location and dimension, we could parameterise the bar with locations representing the four corners.  Parameterisations which involve dimension (e.g.\ height and width) only make sense for Cartesian coordinate systems.  For example, height of a bar geom in polar coordinates corresponds to radius of a segment.  For this reason location based parameterisations are used internally.  

\subsubsection{Position adjustment}

Sometimes we need to tweak the position of the geometric elements on the plot, when otherwise they would obscure each other.  This is most common in bar plots, where we stack or dodge (place side-by-side) the bar to avoid overlaps.  In scatterplots with few unique x and y values, we sometimes randomly jitter \citep{chambers:1983} the points to reduce overplotting.  

\subsection{Scales}\label{sec:scales}

A {\bf scale} controls the mapping from data to aesthetic attributes, and so we need one scale for each aesthetic property used in a layer.  Scales are common across layers to ensure a consistent mapping from data to aesthetics.  Some scales are illustrated in Figure~\ref{fig:scales}.

\begin{figure}[htbp]
	\centering
	\caption{Examples of four scales from ggplot2.  From left to right: continuous variable mapped to size and colour, discrete variable mapped to shape and colour.  The ordering of scales seems upside-down, but this matches the labelling of the $y$-axis: small values occur at the bottom.}
	\label{fig:scales}
\end{figure}

% sc <- ScaleSize$new()
% sc$train(1:10)
% pdf("2-scale-size.pdf", height=2, width=1); grid.draw(sc$guide_legend()); dev.off()
% sc <- ScaleColourContinuous$new()
% sc$train(1:10)
% pdf("2-scale-colour.pdf", height=2, width=1); grid.draw(sc$guide_legend()); dev.off()
% sc <- ScaleColourHue$new()
% sc$train(factor(letters[1:5]))
% pdf("2-scale-colour2.pdf", height=2, width=1); grid.draw(sc$guide_legend()); dev.off()
% sc <- ScaleShape$new()
% sc$train(factor(letters[1:5]))
% pdf("2-scale-shape.pdf", height=2, width=1); grid.draw(sc$guide_legend()); dev.off()

%(\citet{cleveland:1993a} uses scale instead of data and physical instead of aesthetic)

A scale is a function, and its inverse, along with a set of parameters.  For example, the colour gradient scale maps a segment of the real line to a path through a colour space.  The parameters of the function define whether the path is linear or curved, which colour space to use (eg. LUV or RGB), and the start and end colours.  

The inverse function is used to draw a guide so that you can read values from the graph.  Guides are either axes (for position scales) or legends (for everything else).  Most mappings have a unique inverse (i.e\. the mapping function is one-to-one), but many do not.  A unique inverse makes it possible to recover the original data, but this is not always desirable if we want to focus attention on a single aspect.

Scales typically map from a single variable to a single aesthetic, but there are exceptions.  For example, we can map one variable to hue and another to saturation, to create a single aesthetic, colour.  We can also create redundant mappings, mapping the same variable to multiple aesthetics.  This is particularly useful when producing a graphic that works in both colour and black and white. 

\subsection{Coordinate system}\label{sec:coordinate_systems}

A coordinate system, {\bf coord} for short, maps the position of objects onto the plane of the plot.  Position is often specified by two coordinates $(x, y)$, but could be any number of coordinates.  The Cartesian coordinate system is the most common coordinate system for two dimensions, while polar coordinates and various map projections are used less frequently.  For higher dimensions, we have parallel coordinates (a projective geometry), mosaic plots (a hierarchical coordinate system) and linear projections onto the plane.

Coordinate systems affect all position variables simultaneously and differ from scales in that they also change the appearance of the geometric objects.  For example, in polar coordinates, bar geoms look like segments of a circle.  Additionally, scaling is performed before statistical transformation, while coordinate transformations occur afterward.  The consequences of this are shown in Section \ref{sub:transformation}.

Coordinate systems control how the axes and grid lines are drawn.  Figure~\ref{fig:coord} illustrates three different types of coordinate systems.  Very little advice is available for drawing these for non-Cartesian coordinate systems, so a lot of work needs to be done to produce polished output.

% x1 <- c(1,10)
% y1 <- c(1, 5)
% old <- ggopt(grid.colour="black", grid.fill="white", border.colour ="black")
% p <- qplot(x1, y1, geom="blank")
% p
% ggsave(file="2-coord-cartesian.pdf", width=8, height=6)
% p + coord_polar()
% ggsave(file="2-coord-polar.pdf", width=6, height=6)
% p + coord_trans(y="log10")
% ggsave(file="2-coord-log.pdf", width=8, height=6)
% ggtheme(old)

\begin{figure}[htbp]
	\centering
	\caption{Examples of axes and grid lines for three coordinate systems: Cartesian, semi-log and polar. The polar coordinate system illustrates the difficulties associated with non-Cartesian coordinates: it is hard to draw the axes correctly!}
	\label{fig:coord}
\end{figure}

\subsection{Faceting}\label{sec:faceting}

There is also another thing that turns out to be sufficiently useful that we should include it in our general framework: faceting (also known as conditioned or trellis plots). This makes it easy to create small multiples of different subsets of an entire dataset. This is a powerful tool when investigating whether patterns hold across all conditions.  The faceting specification describes which variables should be used to split up the data, and how they should be arranged in a grid.



\section{Data structures}
\label{sec:data_structures}

These principles are encoded as data structures in a fairly straightforward way.

There are two ways to create these plot objects: all at once with \f{qplot}, as shown in the previous chapter, or piece-by-piece with \f{ggplot} and layer functions, as described in the next chapter.

One thing to note is that all ggplot2 objects (with the exception of the main plot object) are proto objects.  Proto is a package which implements the prototype-style of object-oriented programming.  There are some major differences between this and the typical S3 or S4 style of OO in R, but the good news is that you only need to worry about them if you want to develop your own extensions to ggplot2.  For everyday use, the proto objects are hidden behind a facade which makes them act like normal R objects.

{\tt str} to see full structure (it can be large!)

{\tt summary} briefly describes the structure of the plot

Data stored inside the plot - if you change the data outside of the plot, and then redraw a saved plot, it will not be updated.  Consequence of R copying semantics.

\ifwhole
\else
  \bibliography{/Users/hadley/documents/phd/references}
  \end{document}
\fi

% \section{What is a plot?}
% \label{sec:what_is_a_plot}
% 
% One way to think about the grammar of graphics is as a question: what is a plot?  The grammar answers this by describing a plot as a collection of independent components, each describing an independent part of the plot.  There are three basic things we need for a plot: one or more layers, scales to map variables from data space to visual space, and a coordinate system.  These are described below.
% 
% \begin{itemize}
%   \item One or more layers.  A layer is composed of data and a description of which data variables should be mapped to which aesthetic properties, a geometric object, and a statistical transformation:
%   
%   \begin{itemize}
%     \item Data is obviously the most important part, and it is what you provide.  This is what you are displaying visually to aid communication or analysis.  You also need to describe how variables in the dataset are mapped to visual properties.  For example, in Figure 2.X we mapped diamond price to y position, carat to y position and colour to colour.  Because the data and aesthetic mapping set is usually the same in most layers, these can also be set as defaults at the plot level.
%     
%     \item {\bf Geoms}, short for geometric objects, control the type of plot that you create.  For example, using a point geom will create a scatterplot, while using a line geom will create a line plot.
% 
%     \item {\bf Stats}, or statistical transformations, reduce or augment the data in a statistical manner.  For example, a useful stat is the smoother, which shows the mean of y, conditional on x.  Another common stat is the binner, which bins data in to bins.   Every geom has a default statistic, and every statistic a default geom.  For example, the bin statistic has defaults to using the bar geom to produce a histogram.
% 
%     \item {\bf Position adjustment}
%   \end{itemize}
% 
%   \item A scale for all the aesthetic properties.  {\bf Scales} control the mapping from data attributes to aesthetic attributes.  They also provide an inverse mapping in the form of a guide, an axis or legend, which facilitates reading the final graphic.  Aesthetic attributes are things like position, size, colour---anything that you can perceive.  The function that maps data to aesthetic attributes is a scale. It takes values in data space (continuous or categorical) and maps them into an aesthetic space (eg. colour, size, shape).  A scale also provides guides to convert back from the aesthetic attribute to the original data.  Guides are either axes (for position) or legends (for everything else)
% 
%   \item A coordinate system.  A {\bf coord}, or coordinate systems, maps the position of objects on to the plane of the plot.  Typically we will use the cartesian coordinate system, but sometimes others are useful.
% \end{itemize}
% 
% There is also another thing that turns out to be sufficiently useful that we should include it in our general framework: faceting (also known as conditioned or trellis plots). This allows us to easily create small multiples of different subsets of an entire dataset. This is a powerful tool when investigating whether patterns hold across all conditions.

\providecommand{\setflag}{\newif \ifwhole \wholefalse}
\setflag
\ifwhole\else

% Typography and geometry ----------------------------------------------------
\documentclass[letterpaper]{scrbook}
\usepackage[inner=3cm,top=2.5cm,outer=3.5cm]{geometry}

\renewcommand\familydefault{bch}
\usepackage[utf8]{inputenc}
\usepackage{microtype}
\usepackage[small]{caption}
\usepackage[small]{titlesec}
\raggedbottom

% Graphics -------------------------------------------------------------------
\usepackage[pdftex]{graphicx}
\graphicspath{{_include/}}
\DeclareGraphicsExtensions{.png,.pdf}

% Code formatting ------------------------------------------------------------
\usepackage{fancyvrb}
\usepackage{courier}
\usepackage{listings}
\usepackage{color}
\usepackage{alltt}


\definecolor{comment}{rgb}{0.60, 0.60, 0.53}
\definecolor{background}{rgb}{0.97, 0.97, 1.00}
\definecolor{string}{rgb}{0.863, 0.066, 0.266}
\definecolor{number}{rgb}{0.0, 0.6, 0.6}
\definecolor{variable}{rgb}{0.00, 0.52, 0.70}
\lstset{
  basicstyle=\ttfamily,
  keywordstyle=\bfseries, 
  identifierstyle=,  
  commentstyle=\color{comment} \emph,
  stringstyle=\color{string},
  showstringspaces=false,
  columns = fullflexible,
  backgroundcolor=\color{background},
  mathescape = true,
  escapeinside=&&,
  fancyvrb
}
\newcommand{\code}[1]{\lstinline!#1!}



% Links ----------------------------------------------------------------------

\usepackage{hyperref}
\definecolor{slateblue}{rgb}{0.07,0.07,0.488}
\hypersetup{colorlinks=true,linkcolor=slateblue,anchorcolor=slateblue,citecolor=slateblue,filecolor=slateblue,urlcolor=slateblue,bookmarksnumbered=true,pdfview=FitB}
\usepackage{url}

% Tables ---------------------------------------------------------------------
\usepackage{longtable}
\usepackage{booktabs}

% Miscellaneous --------------------------------------------------------------
\usepackage{pdfsync}
\usepackage{appendix}

\usepackage[round,sort&compress,sectionbib]{natbib}
\bibliographystyle{plainnat}


\title{ggplot2}
\author{Hadley Wickham}

\begin{document}
\fi


% decumar<<< 
% source("~/documents/ggplot/ggplot/load.r")
% library(xtable)
% doptions(width=8, height=4.8, lscale=0.5)
% ggopt(axis.color = "black")
% set.seed(1410)
% dsmall <- diamonds[sample(nrow(diamonds), 1000),]
% >>>

\chapter{Adding extra layers}

\section{Introduction}

To create 

A typical graphic has three layers:

\begin{itemize}
  \item The {\bf data}.  This layer appears on every graphic and 

  \item A statistical {\bf summary}.  For the purposes of communication and inference, any conclusions we make about the data will typically be summarised and supported by a quantitative model.  Displaying the predictions from this model in the context of the data is useful to both summarise the data, and to detect potential problems with the model.  This is normally the top layer.

  \item {\bf Context}.  The context for the data shouldn't interfere with the perception of the data, so it should normally be on the bottom layer and formatted so that it is minimally perceptible.  That is, if you concentrate on it, you can see it with ease, but it doesn't jump out at you when you are casually browsing the plot.
  
  For spatial data, the context will often be a map.  For temporal data, it might be significant events outside the data.  Context can also include labelling unusual points.
\end{itemize}

This section describes the main components of each layer (data, aesthetic mappings, geoms and statistics), lists the different geoms and stats available in ggplot2 and ...

\section{Creating a plot}
\label{sec:ggplot}

First need a plot object.  Have seen one way to create this with qplot.  But does more than just create a plot object - it also adds the layers etc, so you don't have total control over it. This means that {\tt qplot()} has some limitations:

\begin{itemize}
  \item Can't use different aesthetic mappings on different layers.  
  \item Can't use different parameters on different layers
  \item Can't change coordinate systems.  
  \item Can't change scales.  Can only have linear or log transformed axes, not any other function.
\end{itemize}

An alternative way, which most of the examples in this chapter will use, is to call the {\tt ggplot()} function.  The {\tt ggplot()} function has two optional arguments: {\bf data} and {\bf mapping}.  Data is self-explanatory: it is the data set that you want to visualise.  Mapping describes the mapping between the variables and the aesthetics of the plot.  It works the same way an in {\tt qplot()} but you need to wrap the pairs in a {\tt aes} call.  These arguments are just defaults, and can be omitted.  Described in more detail in the next section, on layers.

The result of this function is the same as the result of qplot: a ggplot object.  However, the plot created by ggplot can't be rendered:

\begin{alltt}
  ggplot(mtcars, aes(wt, mpg))
\end{alltt}

So far it doesn't actually have anything to draw.  You need to add some layers.

\section{Layers}\label{sec:layers}

% What is a layer?

You can create a layer with the {\tt layer()} function which has arguments geom, geom_params, stat, stat_params, position, data, and mapping.

However, most of the time, you will use a shortcut.  The shortcut is possible because associated with every geom is a default statistic and position, and with every statistic is a default geom.  This means that you only need to specify one of stat or geom to get a completely specified layer.

Geoms are described in more detail (including a list of all geoms available in ggplot) in Section~\ref{sec:geom}, and stats in Section~\ref{sec:stat}.

All layer functions start with {\tt geom\_} or {\tt stat\_} and are singular, eg. {\tt geom\_point}, {\tt stat\_bin}.  This convention is used through ggplot2: scales start with {\tt scale\_}, coordinate systems with {\tt coord\_}, and facet specifications with {\tt facet\_}.

All the layer functions have the same basic form:

\begin{alltt}
geom_XXX <- function(aesthetics, data, ..., geom, position) {}
stat_XXX <- function(aesthetics, data, ..., stat, position) {}
\end{alltt}

\noindent and they have common parameters:

\begin{itemize}
	\item The {\bf data} to use (optional).  If left out (as you will do most of the time), it will use the default data set that you specified when creating the plot.  

	\item Aesthetic {\bf mapping} (optional), specified by the {\tt aes()} function, described in Section~\ref{sec_aes}.  These are added on to the plot defaults.  If you want to remove want of the plot defaults, you'll need map the value to {\tt NULL}, e.g. {\tt aes(colour = NULL)}.
	
	\item For geoms, you can override the default {\tt stat}, and for stats the default {\tt geom}.  This should be a text string containing the name of the geom to use.  Using the default will give you a standard plot - if you want something more exotic, overriding the defaults gives you more freedom.  This is described in more detail in Section~\ref{sub:new_plot_types}.
	
	\item Any aesthetic that the geom recognises can also be specified as a parameter to the plot.  In this case the value should be an See Section\ref{sub:setting-mapping}

	\item Other parameters which differ from geom to geom.  These parameters control various settings of the geom, for example, bin width in the histogram, or bandwidth for a loess smoother.  Any aesthetic can also be used as a parameter, in which case it will be applied to all points.
\end{itemize}

All parameters to the layer are options.

The summary function can be helpful for inspecting the structure of a plot, without actually plotting it.

\begin{alltt}
p <- ggplot(data=mtcars, aes(mpg, wt))
summary(p)
p <- p + geom_point(p)
summary(p)
\end{alltt}

Note that the order of data and mapping is switched between {\tt ggplot()} and the layer functions.  This is for good reason - you almost always specify data for the plot, and almost always specify aesthetics (but not data) for the layers.  To keep your code readable, I suggest always explicitly naming other arguments.

Remember, you can always start with a plot created by {\tt qplot()} and then add on the layers and scales etc as you if you had started with {\tt ggplot()}.  If you do this, {\tt geom\_blank} can be useful for {\tt qplot()} because it's a geom that doesn't draw anything.  For data analysis, use whatever works for you - personally, I use whatever requires the least amount of typing.

\section{Data}
\label{sec:data}

Must be a data frame.
Is a copy not a reference.

If you use faceting on your plot, you must have a default dataset.  This is because faceting is a global operation (i.e. works on all layers) and it needs to have some base dataset to add in any missing columns.  See Section~\ref{sub:missing_faceting_columns} for more details.

\section{Aesthetic mapping}
\label{sec:aes}

To describe the way that variables in the data are mapped to things that we can perceive on the plot (the ``aesthetics''), we use the {\tt aes} function.  The {\tt aes} function takes a list of aesthetic-variable pairs as follows:

\begin{alltt}
aes(x = weight, y = height, colour = age)
\end{alltt}

Here we are mapping x-position to weight, y-position to height and colour to age.  Alternatively, the first two arguments can be left without names, in which case they are assumed to be for the x and y variables.  (This matches the way that {\tt qplot()} is normally used.)

\begin{alltt}
aes(weight, height, colour = age)
\end{alltt}

Things inside aes can be functions, but all variables must be contained inside the plot (or layer) data.  This is important because ggplot2 objects are entirely self-contained.  You can save one to disk and later plot it without needing anything else from that session.

The {\tt aes} function is usually used inside a layer, but it can also be added directly to the plot to override the previous (default) mappings.  

\begin{alltt}
(p <- qplot(mpg, wt, data=mtcars))
p + aes(wt, hp)
\end{alltt}

If you want to use the defaults you set up when creating the plot object, you just use the geom function like:

\begin{alltt}
p + geom_line()
\end{alltt}

This will create a new plot object with an addition layer that joins points with lines.  (To see how to control what sets of observations form a line, see Section~\ref{sec:grouping}) lines geom added to its list of geoms.   You can also add more aesthetics, or specify a different data set to use instead of the default.  

\begin{alltt}
p + geom_line(aes(colour=a, size=b), data=new.data.frame))
\end{alltt}

If there is a default aesthetic you want to unset, use {\tt NULL}:

\begin{alltt}
p + geom_line(p, aes(colour=NULL))
\end{alltt}

If the plot is facetted, and the new data set does not contain the faceting columns, then the data set will be duplicated for each value of the missing columns.  This has the effect of displaying the data in every facet.  This is explained in more detail, with examples, in chapter XXX, page X.

\subsection{Setting vs. mapping}
\label{sub:setting-mapping}

For every aesthetic the geom function understands, you can also supply that aesthetic as an option to the function.  Instead of mapping data to that aesthetic, this will change the default.  For example.

\begin{alltt}
p + geom_line(colour="red")  
\end{alltt}

\noindent will set the line colour to be red instead of black.  Other examples:

\begin{alltt}
p + geom_line(size=3)  
p + geom_line(size=3, colour="blue")  
\end{alltt}

\begin{table}
  \begin{tabular}{lp{2in}}
    \toprule
    Aesthetic & Value \\
    \midrule
    Colour and fill & A string in the format {\tt \#RRGGBB} or {\tt \#RRGGBBAA}, or an R colour name (see {\tt ?colours()} for a complete list) \\
    x, y, and z position & A value that makes sense in the data space \\ 
    Size & Numeric, measured in mm\\
    Line type &  \\
    Shape & As described in {\tt ?points()} (called {\tt pch}) \\
    \bottomrule
  \end{tabular}
  \caption{Acceptable values for setting aesthetics}
  \label{label}
\end{table}


\subsection{Grouping and weights}
\label{sub:grouping}

There are also two aesthetic attributes that can't be perceived directly.  The most important of these is the {\tt group} aesthetic, which divides the the data set into discrete components.   This is used in line and path plots to separate the data for different lines, and in the groups grob to divide the different groups. 

By default, the group aesthetic is set to the combination (interaction) of all discrete variables used in the plot.  Generally, this will create the correct separation of the data, but sometimes you will need to override it.  For example, if you want lines connecting observations with a discrete x scale you will need to manual specify the group aesthetic.  This section illustrates a few possible applications when this is useful.

The {\tt interaction()} function is particularly useful if there isn't a pre-existing variable that separates the groups you are interested in, but a combination of variables does.  

Examples.

The {\tt weight} aesthetic is also useful when you have weighted data.  All ggplot2 statistics know how to correctly deal with (WHAT TYPE OF WEIGHTS?) weights, which makes plotting your weighted data easy.  For a more comprehensive description of plotting weights, see ``Weight and see''.

\section{Translating between qplot and ggplot}
\label{sec:qplot-ggplot}


\begin{alltt}
qplot(x, y, data, shape=shape, colour = colour)
ggplot(data, aes(x, y, shape=shape, colour = colour)) + geom_point()
\end{alltt}

\begin{alltt}
qplot(x, y, data, shape=shape, colour = I("red"))
ggplot(data, aes(x, y, shape=shape)) + geom_point(colour="red")
\end{alltt}

The differences between setting and mapping are described in more detail in Section~\ref{sec:setting-mapping}.

\begin{alltt}
qplot(x, y, data, geom=c("line", "smooth"))
ggplot(data, aes(x, y)) + geom_line() + geom_smooth()
\end{alltt}

\begin{alltt}
qplot(x, y, data, geom=c("line", "smooth"), method="lm")
ggplot(data, aes(x, y)) + geom_line() + geom_smooth(method="lm")
\end{alltt}


\begin{alltt}
qplot(x, y, data, log="xy")
ggplot(data, aes(x, y)) + scale_x_log10() + scale_y_log10()
\end{alltt}

Section~\ref{sec:transformers} describes more possible transformations of the x and y scales.

\begin{alltt}
qplot(x, y, data, main="title", asp = 1)
ggplot(data, aes(x, y)) + opts(title = "title", aspect.ratio = 1)
\end{alltt}

Section~\ref{sec:plot_options} lists all possible plot options and their effects.



% \section{The pipeline}\label{sec:the_pipeline}
% 
% Another way of thinking about the grammar of graphics is as a pipeline which takes in raw data and outputs plots.  Each stage of the pipeline performs a transformation described by one of the components of the grammar.  For example, one of the first stages performs the statistical transformation described by the stat component.  
% 
% [Insert diagram of the pipelines here.]
% 
% You will notice there are two passes of the 
% 
% 
% \subsection{Differences from Wilkinson's grammar}
% 
% The pipelines of ggplot and the GoG differ slightly.  In ggplot, every stage takes a data frame as input and returns a data frame as output, apart from the final stage, which outputs grid grobs (graphical objects).  The GoG has three intermediate data formats: a varset (the same as a data frame), a graph and a graphic.
% 
% [Insert diagram of the pipelines here.]
% 
% The pipeline is a little more complicated when multiple data sources and geoms are involved.  The complication arises because all data must share the same scales and coordinate system, but may use different mappings, geoms and statistics.  All pipelines must share the same coordinate system, so this is where the connection occurs.  For this reason, ggplot has the layer, as discussed above, which combines the geom, stat, data and mapping.
% 
% [Insert more complicated diagram here]



\section{Geoms}
\label{sec:geom}

Geoms, or geometric elements, perform the actual rendering of the plot.

List of geoms, with aesthetics and parameters

They are also responsible for the appearance of the legend, described in more detail in Section~\ref{sec:legends_and_axes}.


\subsection{Interval geoms}
\label{sub:interval_geoms}


\section{Stat}
\label{sec:stat}

What do statistical transformations do?

What statistical transformations are available?

Naming convention.



\subsection{Generated aesthetics}
\label{sub:generated_aesthetics}

Each stat generate a number of output variables that can be used in aesthetic mappings.  These can be used with {\tt ..} around them.  That identifies that the variable is not present in the original dataset, but is computed by the stat.

\subsection{Summary statistic}
\label{sub:summary_statistic}

\section{Pulling it all together}
\label{sec:pulling_it_all_together}

\subsection{Creating new plot types}
\label{sub:new_plot_types}

By connecting geoms with different statistics, you can easily create new types of graphics.  

Violin plot example.

\subsection{Different data on different layers}
\label{sub:different_data_on_different_layers}

Example overlaying raw data and model predictions.


\ifwhole
\else
  \bibliography{/Users/hadley/documents/phd/references}
  \end{document}
\fi

\providecommand{\setflag}{\newif \ifwhole \wholefalse}
\setflag
\ifwhole\else

% Typography and geometry ----------------------------------------------------
\documentclass[letterpaper]{scrbook}
\usepackage[inner=3cm,top=2.5cm,outer=3.5cm]{geometry}

\renewcommand\familydefault{bch}
\usepackage[utf8]{inputenc}
\usepackage{microtype}
\usepackage[small]{caption}
\usepackage[small]{titlesec}
\raggedbottom

% Graphics -------------------------------------------------------------------
\usepackage[pdftex]{graphicx}
\graphicspath{{_include/}}
\DeclareGraphicsExtensions{.png,.pdf}

% Code formatting ------------------------------------------------------------
\usepackage{fancyvrb}
\usepackage{courier}
\usepackage{listings}
\usepackage{color}
\usepackage{alltt}


\definecolor{comment}{rgb}{0.60, 0.60, 0.53}
\definecolor{background}{rgb}{0.97, 0.97, 1.00}
\definecolor{string}{rgb}{0.863, 0.066, 0.266}
\definecolor{number}{rgb}{0.0, 0.6, 0.6}
\definecolor{variable}{rgb}{0.00, 0.52, 0.70}
\lstset{
  basicstyle=\ttfamily,
  keywordstyle=\bfseries, 
  identifierstyle=,  
  commentstyle=\color{comment} \emph,
  stringstyle=\color{string},
  showstringspaces=false,
  columns = fullflexible,
  backgroundcolor=\color{background},
  mathescape = true,
  escapeinside=&&,
  fancyvrb
}
\newcommand{\code}[1]{\lstinline!#1!}



% Links ----------------------------------------------------------------------

\usepackage{hyperref}
\definecolor{slateblue}{rgb}{0.07,0.07,0.488}
\hypersetup{colorlinks=true,linkcolor=slateblue,anchorcolor=slateblue,citecolor=slateblue,filecolor=slateblue,urlcolor=slateblue,bookmarksnumbered=true,pdfview=FitB}
\usepackage{url}

% Tables ---------------------------------------------------------------------
\usepackage{longtable}
\usepackage{booktabs}

% Miscellaneous --------------------------------------------------------------
\usepackage{pdfsync}
\usepackage{appendix}

\usepackage[round,sort&compress,sectionbib]{natbib}
\bibliographystyle{plainnat}


\title{ggplot2}
\author{Hadley Wickham}

\begin{document}
\fi

\chapter{Scales, axes and legends}

Scales control the mapping between data space and aesthetic space.  They convert data values into aesthetic attributes that can be a perceived: colour, shape, size, etc.  Each type of aesthetic attribute has a default scale, and may have other scales that provide different types of mappings.  For example, the default colour scale uses equally space hues, but other scales allow you to generate a gradient between two or three different colours.  This section describes the basic operation of a scale, the details of all the different scales and instructions on how to make your own.

By default, whenever you specify an aesthetic mapping (eg. {\tt colour=a}) a default scale is automatically added to the plot object based on the variable type (eg. \verb|scale_colour_discrete|).  However, if you want more control over the scale, then you'll need to add a scale manually.  Details for individual scales are provided by the documentation.  This chapter gives an overview of how legends work, and discusses broad issues that apply to many legends.

\begin{itemize}
	\item colour and fill: brewer, gradient, gradient2
	\item linetype
	\item shape
	\item size
	\item x, y, and z: 
\end{itemize}

Scales are added by default whenever you use {\tt qplot} or {\tt ggplot}.  If you want to modify them, you need to manually add a scale to the plot.  This will automatically override any defaults.

\section{How scales work}

\begin{itemize}
	\item  Scale transformation occurs before statistical transformation so that statistics are computed on the scale-transformed data.  This ensures that a plot of $log(x)$ vs $log(y)$ on linear scales looks the same as $x$ vs $y$ on log scales.  See Section~\ref{sec:trans} for more details. Transformation is only necessary for non-linear scales, because all statistics are location-scale invariant.

	\item After the statistics are computed, each scale is trained on every faceted dataset (a plot can contain multiple datasets, e.g.\ raw data and predictions from a model).  The training operation combines the ranges of the individual datasets to get the range of the complete data.  If scales were applied locally, comparisons would only be meaningful within a facet.

	\item Finally the scales map the data values into aesthetic values. Given that we end up with an essentially identical structure you might wonder why we don't simply split up the final result.  There are several reasons for this.  It makes writing statistical transformation functions easier, as they only need to operate on a single facet of data, and some need to operate on a single subset, for example, calculating a percentage.  Also, in practice we may have a more complicated training scheme for the position scales so that different columns or rows can have different $x$ and $y$ scales.  
	
\end{itemize}

\section{Usage}

Scales have a common naming scheme.  

If you want to use the default scale, but change some options the scale name is {\tt continuous} or {\tt discrete} according to the type of variable.  Having to specify the type of variable (continuous or discrete) seems like extra work: why can't ggplot2 figure that out by itself?  Well, it can't because you can create the scales independently of (and before) the plot.  Allows you to create scales independently of the plot.  Chapter~\ref{chp:strategy} shows some examples of these.

You can change the defaults using \verb|set_scale_default|.

\begin{itemize}
  \item {\bf name}:  the label which will appear on the axis or legend. You can supply text strings (using ``$\backslash$n'' for line breaks) or mathematical expressions (as described in \verb|?plotmath|):
  
  \begin{figure}[htbp]
    \centering
      \includegraphics[width=0.32\textwidth]{scales-name-1}
      \includegraphics[width=0.32\textwidth]{scales-name-2}
      \includegraphics[width=0.32\textwidth]{scales-name-3}
    \caption{caption}
    \label{fig:label}
  \end{figure}
  
  % decumar<<< 
  % interweave({
  % p <- qplot(tip, total_bill, data=tips, colour=tip/total_bill)
  % p + scale_colour_hue("Tip rate")
  % p + scale_colour_hue("The amount of the tip\ndivided by the total bill")
  % p + scale_colour_hue(expression(frac(tip, total_bill))
  % })
  % |||
  % >>>

  \item {\bf limits}: if not set, will be computed from the data.  Continuous scales take a numeric vector of length two.  Discrete scales take a character vector.  {\tt xlim} and {\tt ylim} shortcuts.  Limits should always be specified in the original data space.  If limits are set, no training of the data will be performed.  Particularly useful for zooming (limits smaller than range of data), and ensuring that limits are consistent across multiple plots (limits larger than range of some subsets of data)

  \item {\bf breaks} and {\bf labels}: control what appears where tick marks appear on the axis or what values appear on the legend.  If labels is specified, breaks must be specified too.  Together, these arguments give you control over 
  
  \item {\bf expand}: controls the amount of extra space added to the limits.  This is a numeric vector of length two: the first number is the multiplicative and the second is additive.  Default for continuous scale is {\tt c(0, 0.05)} and for discrete {\tt c(0.5, 0)}.  Set to {\tt c(0, 0)} to have no extra space.  This is added on top of any specified limits.
  
\end{itemize}

\section{Transformers}
\label{sec:trans}

Every continuous scale takes a {\tt trans} argument.  This argument allows to specify a non-linear transformation.  The transformation is carried out by a transformer, which describes the transformation, it's inverse and where to place labels. Table~\ref{tbl:common-trans} lists some of the more common transformers. A complete list is available in the documentation for {\tt transformation}.

\begin{table}
  \centering
  \begin{tabular}{lll}
    Name & Function $f(x)$ & Inverse $f^{-1}(x)$ \\
    \hline
    Identity    & $x$         & $x$      \\
    Reciprocal  & $x^{-1}$    & $x^{-1}$ \\
    Square root & $x^{1/2}$   & $x ^ 2$  \\
    Log         & $log(x)$    & $e ^ x$  \\
    Reverse     & $-x$        & $-x$     \\
  \end{tabular}
  \caption{List of common transformers.}
  \label{tbl:common-trans}
\end{table}

Because transformations are so commonly used to modify position scales, there is a shortcut for x, y, and z scales: \verb|scale_x_continuous(trans = "log10")| can be written as \verb|scale_x_log10()|.

You can also perform the transformation by hand.  For example instead of adding {\tt scale\_x\_log}, you could plot {\tt log(x)}.  These produce identical graphics except for one difference: the axis labels.  If you use a transformed scale, the axes will be labelled with the original values.  Figure~\ref{fig:trans} illustrates this difference.

\begin{figure}[htbp]
  \centering
    \includegraphics[width=0.49\linewidth]{trans-scale}
    \includegraphics[width=0.49\linewidth]{trans-data}
  \caption{A scatterplot of diamond price vs carat illustrating the difference between log transforming the scale (left) and log transforming the data (right).  The plots are identical, but the axis labels are different.}
  \label{fig:trans}
\end{figure}

Scale transformation occurs before the statistic is calculated.  Transformers are also used in \verb|coord_trans|, where the transformation occurs after the statistic has been calculated, and affect the shape of the grob.  \verb|coord_trans| is described in more detail in Section XXX.

\section{Special scales}
\label{sec:scale_special}

There are two special types of scales that work for all aesthetics.  They are the manual and identity scale.  

The identity scale is used when your data is already in a form that the plotting functions in R understand: size in mm, colour as \verb|"#RRGGBB"| etc. Because there are no labels associated with such data, the identity scale will not draw a legend unless you also provide labels and breaks, as described above. 

\begin{figure}[htbp]
  \centering
    \includegraphics[width=0.5\textwidth]{scale-identity}
  \caption{A plot of R colours in Luv space.  Points are coloured according to their colour, with }
  \label{fig:scale-identity}
\end{figure}

The manual scale is useful for creating your own discrete scales.  It has one important argument, \verb|values| in which you specify the values that the scale should produce.  If this vector is named, it will match the values of the output to the values of the input, otherwise it will match in order of the levels of the discrete variable.  If you use \verb|scale_manual| be careful to ensure that they are perceptually well founded.

% scale_product

\section{Legends and axes}
\label{sec:legends_and_axes}

In ggplot2, legends and axes are produced automatically based on the scales.  This section describes how legends are produced from the scales and the geoms that use them, as well as ways that you can customise them.

\begin{figure}[htbp]
  \centering
  \caption{Figure containing axis and legends, with pieces labelled.}
  \label{fig:label}
\end{figure}

\subsection{Customising appearance}

\begin{itemize}
  \item The {\tt breaks} and {\tt labels} arguments, described above, are particularly important because they control what tick marks appear on the axis and what keys appear on the legend.  If the breaks chosen by default are not appropriate (or you want to use more informative labels) setting these arguments will adjust the appearance of the legend keys and axis tick marks.  
  
  \item The theme settings {\tt axis.text}, {\tt axis.box}, ... control the visual appearance of the legend.  For more details on how to manipulate these settings, see the Section~.

  \item The internal grid lines are controlled by the breaks and minor breaks arguments.  By default minor grid lines are space evenly in the original data space - this gives the common behaviour of log-log plots where major grid lines are multiplicative and minor grid lines are additive.
  
  \item Position of legends.  Plot level option setting 
  
\end{itemize}

\subsection{Legend merging}

ggplot2 tries to minimise the number of legends that are displayed.  It does this by combining legends for the same variable.  Figure~\ref{fig:legend-merge} shows an example of this.  In order for legends to be merge, they must have the same title.  For this reason, if you change the title of one of the merged legends you'll need to change it for all of them.

\begin{figure}[htbp]
  \centering
  \caption{Colour legend, shape legend, colour + shape legend.}
  \label{fig:legend-merge}
\end{figure}

A legend will be drawn for each aesthetic attribute.  Every geom that uses that aesthetic will appear in the legend in some way.

\ifwhole
\else
  \bibliography{/Users/hadley/documents/phd/references}
  \end{document}
\fi
\providecommand{\setflag}{\newif \ifwhole \wholefalse}
\setflag
\ifwhole\else

% Typography and geometry ----------------------------------------------------
\documentclass[letterpaper]{scrbook}
\usepackage[inner=3cm,top=2.5cm,outer=3.5cm]{geometry}

\renewcommand\familydefault{bch}
\usepackage[utf8]{inputenc}
\usepackage{microtype}
\usepackage[small]{caption}
\usepackage[small]{titlesec}
\raggedbottom

% Graphics -------------------------------------------------------------------
\usepackage[pdftex]{graphicx}
\graphicspath{{_include/}}
\DeclareGraphicsExtensions{.png,.pdf}

% Code formatting ------------------------------------------------------------
\usepackage{fancyvrb}
\usepackage{courier}
\usepackage{listings}
\usepackage{color}
\usepackage{alltt}


\definecolor{comment}{rgb}{0.60, 0.60, 0.53}
\definecolor{background}{rgb}{0.97, 0.97, 1.00}
\definecolor{string}{rgb}{0.863, 0.066, 0.266}
\definecolor{number}{rgb}{0.0, 0.6, 0.6}
\definecolor{variable}{rgb}{0.00, 0.52, 0.70}
\lstset{
  basicstyle=\ttfamily,
  keywordstyle=\bfseries, 
  identifierstyle=,  
  commentstyle=\color{comment} \emph,
  stringstyle=\color{string},
  showstringspaces=false,
  columns = fullflexible,
  backgroundcolor=\color{background},
  mathescape = true,
  escapeinside=&&,
  fancyvrb
}
\newcommand{\code}[1]{\lstinline!#1!}



% Links ----------------------------------------------------------------------

\usepackage{hyperref}
\definecolor{slateblue}{rgb}{0.07,0.07,0.488}
\hypersetup{colorlinks=true,linkcolor=slateblue,anchorcolor=slateblue,citecolor=slateblue,filecolor=slateblue,urlcolor=slateblue,bookmarksnumbered=true,pdfview=FitB}
\usepackage{url}

% Tables ---------------------------------------------------------------------
\usepackage{longtable}
\usepackage{booktabs}

% Miscellaneous --------------------------------------------------------------
\usepackage{pdfsync}
\usepackage{appendix}

\usepackage[round,sort&compress,sectionbib]{natbib}
\bibliographystyle{plainnat}


\title{ggplot2}
\author{Hadley Wickham}

\begin{document}
\fi

% SET_DEFAULTS
%   GG-WIDTH: 4  GG-HEIGHT: 4
%   TEX-WIDTH: 0.5\linewidth COL: 2
%   INLINE: FALSE
%   CACHE: TRUE
% 

% END

\chapter{Positioning}
\label{cha:position}

\section{Introduction}

This chapter discusses position, particularly how facets are laid out on a page, and how coordinate systems within a panel work.  There are four components that control position.  You have already learned about two of them that work within a facet:

\begin{itemize}
  \item {\bf Position adjustments} adjust the position of overlapping objects within a layer, and were described in Section~\ref{sec:position}.  These are most useful for bar and other interval geoms, but can be useful in other situations.

  \item {\bf Position scales}, previously described in Section~\ref{sub:scale-position}, control how the values in the data are mapped to positions on the plot.  Common transformations are linear and log, but any other invertible function can also be used.
\end{itemize}

\noindent This chapter will describe the other two components and show you how all four components can be used together:

\begin{itemize}
  \item {\bf Faceting}, described in Section~\ref{sec:faceting}, is a mechanism for automatically laying out multiple plots on a page.  It splits the data into subsets, and then plots each subset into a different panel on the page.  Such plots are often called small multiples.  

  \item {\bf Coordinate systems}, described in Section~\ref{sec:coord}, control how the two independent position scales are combined to create a 2d coordinate system.  The most common coordinate system is Cartesian, but other coordinate systems can be useful in special circumstances.

\end{itemize}

\section{Faceting}
\label{sec:faceting}

You first encountered faceting in the introduction to \f{qplot}, Section~\ref{sec:qplot-faceting}, and you may already have been using it in your plots.  Faceting generates small multiples each showing a different subset of the data.  Small multiples are a powerful tool for exploratory data analysis: you can rapidly compare patterns in different parts of the data and see whether they are the same or different.  This section will discuss how you can fine-tune facets, particularly the way in which they interact with position scales. 

There are two types of faceting provided by \ggplot: \code{facet_grid} and \code{facet_wrap}.  Facet grid produces a 2d grid of panels defined by variables which form the rows and columns, while facet wrap produces a 1d ribbon of panels that is wrapped into 2d.  The grid layout is similar to the layout of \code{coplot} in base graphics, and the wrapped layout is similar to the layout of panels in \code{lattice}.  These differences are illustrated in Figure~\ref{fig:facets-sketch}.

\begin{figure}[htbp]
  \centering
    \includegraphics[width=0.5\linewidth]{position-facets}
  \caption{A sketch illustrating the difference between the two faceting systems.   \f{facet_grid} (left) is fundamentally 2d, being made up of two independent components.  \f{facet_wrap} (right) is 1d, but wrapped into 2d to save space.}
  \label{fig:facet-sketch}
\end{figure}

There are two basic arguments to the faceting systems: the variables to facet by, and whether position scales should be global or local to the facet.  The way these options are specified is a little different for the two systems, so they are described separately below.

% Faceting can be thought of a special type of coordinate system, one that is hierarchical.  At the top level we have a coordinate system created by the categorical variables that we are faceting by, and then within each of these regions another coordinate system generated by the x and y position of each graphic.

You can access either faceting system from \f{qplot}. A 2d faceting specification (e.g., \code{x ~ y}) will use \code{facet_grid}, while a 1d specification (e.g., \verb|~ x|) will use \code{facet_wrap}.

Faceted plots have the capability to fill up a lot of space, so for this chapter we will use a subset of the mpg dataset that has a manageable number of levels: three cylinders (4, 6, 8) and two types of drive train (4 and f).  This removes 29 vehicles from the original dataset.

% INTERWEAVE
%
% mpg2 <- subset(mpg, cyl != 5 & drv %in% c("4", "f"))
\input{_include/e801ae8d479ec9884e584f1f0e81f4ce.tex}
% END

\subsection{Facet grid}

The grid faceter lays out plots in a 2d grid.  When specifying a faceting formula, you specify which variables should appear in the columns and which should appear in the rows, as follows:  

\begin{itemize}
  \item \code{. ~ .}   The default. Neither rows nor columns are faceted, so you get a single panel.

  % INTERWEAVE
  % 
  % qplot(cty, hwy, data = mpg2) + facet_grid(. ~ .)
  \input{_include/d92a90e284f841d898d78f5acf01e80a.tex}  
  % END

  \item \code{. ~ a} A single row with multiple columns.  This is normally the most useful direction because computer screens are usually wider than they are long.  This direction of faceting facilitates comparisons of y position, because the vertical scales are aligned.

  % INTERWEAVE
  %   GG-WIDTH: 9 GG-HEIGHT: 3  TEX-WIDTH: \linewidth
  % 
  % qplot(cty, hwy, data = mpg2) + facet_grid(. ~ cyl)
  \input{_include/f925dc0e754745bf7d398766b6338823.tex}  
  % END
  
  \item \code{b ~ .} A single column with multiple rows.  This direction facilitates comparison of x position, because the horizontal scales are aligned, and so is particularly useful for comparing distributions.  Figure~\ref{fig:facet-hist} on page \pageref{fig:facet-hist} is a good example of this use.

  % INTERWEAVE
  % 
  % qplot(cty, data = mpg2, geom="histogram", binwidth = 2) +
  %   facet_grid(cyl ~ .)
  \input{_include/a18df8a9830ca8847d4991a16d12e6d3.tex}  
  % END

  \item \code{a ~ b}: Multiple rows and columns.  You'll usually want to put the variable with the greatest number of levels in the columns, to take advantage of the aspect ratio of your screen.

  % INTERWEAVE
  %   GG-WIDTH: 8 GG-HEIGHT: 3  TEX-WIDTH: \linewidth
  % 
  % qplot(cty, hwy, data = mpg2) + facet_grid(drv ~ cyl)
  \input{_include/2e721842c5b3dc4dd1f1d4b506211fa8.tex}  
  % END

  \item \code{. ~ a + b} or \code{a + b ~ .}  Multiple variables in the rows or columns (or both). This is unlikely to be useful unless the number of factor levels is small, you have a very wide screens or you want to produce a long, skinny poster.

  % INTERWEAVE
  %   GG-WIDTH: 12 GG-HEIGHT: 3  TEX-WIDTH: \linewidth
  % 
  % qplot(cty, hwy, data = mpg2) + facet_grid(. ~ cyl + drv)
  \input{_include/55c05e4252fa43d3cf644fa36a1009db.tex}  
  % END

\end{itemize}

Variables appearing together on the rows or columns are nested in the sense that only combinations that appear in the data will appear in the plot.  Variables that are specified on rows and columns will be crossed: all combinations will be shown, including those that didn't appear in the original dataset: this may result in empty panels.

\subsubsection{Margins}\label{sub:margins}

Faceting a plot is like creating a contingency table.  In contingency tables it is often useful to display marginal totals (totals over a row or column) as well as the individual cells.  It is also useful to be able to do this with graphics, and you can do so with the {\tt margins} argument.  This allows you to compare the conditional patterns with the marginal patterns.

You can either specify that all margins should be displayed, using {\tt margins = TRUE}, or by listing the names of the variables that you want margins for, {\tt margins = c("sex", "age")}.  You can also use \verb|"grand_row"| or \verb|"grand_col"| to produce grand row and grand column margins, respectively.  

Figure~\ref{fig:margins} shows what margins look like.  The first plot shows what the data looks like without margins, and the second shows all margins.  The margin column shows all drive trains, the margin row shows all cylinders and the bottom right plot (the grand total) shows the full dataset.  For this data we can see that as the number of cylinders increases, engine displacement increases and fuel economy decreases, and compared to front-wheel-drive vehicles, as a group four-wheel-drive vehicles have about the same displacement, but are less fuel efficient.  The figure was produced with the following code:

% FIGLISTING
%   FILETYPE: PNG
%   LABEL: margins
%   CAPTION: Graphical margins work like margins of a contingency table to
%   give unconditioned views of the data.  A plot faceted by number of 
%   cylinders and drive train (left) is supplemented with margins (right).
% 
% p <- qplot(displ, hwy, data = mpg2) +
%   geom_smooth(method = "lm", se = F)
% p + facet_grid(cyl ~ drv) 
% p + facet_grid(cyl ~ drv, margins = T)
\input{_include/ec3d7dc333b145baf6ca3e45acb06e6d.tex}
% END

Groups in the margins are controlled in the same way as groups in all other panels, defaulting to the interaction of all categorical variables present in the layer.  (See Section~\ref{sub:grouping} for a reminder.)  The following example shows what happens when we add a coloured smooth for each drive train. 

% INTERWEAVE
%   FILETYPE: PNG
% 
% qplot(displ, hwy, data = mpg2) + 
%   geom_smooth(aes(colour = drv), method = "lm", se = F) + 
%   facet_grid(cyl ~ drv, margins = T) 
\input{_include/4ef38da03e88283efecf2ce697e10b84.tex}
% END

Plots with many facets and margins may be more appropriate for printing than on screen display, as the higher resolution of print (600 dpi vs. 72 dpi) allows you to compare many more subsets.

\subsection{Facet wrap}
\label{sub:facet_wrap}

An alternative to the grid is a wrapped ribbon of plots.  Instead of having a 2d grid generated by the combination of two (or more) variables, \code{facet_wrap} makes a long ribbon of panels (generated by any number of variables) and wraps it into 2d.  This is useful if you have a single variable with many levels and want to arrange the plots in a more space efficient manner.  This is what trellising in lattice does.

Figure~\ref{fig:movies-wrap} shows the distribution of average movie ratings by decade. The main difference over time seems to be the increasing spread of ratings. This is probably an artefact of the number of votes: newer movies get more votes and so the average ratings are likely to be less extreme. The disadvantage of this style of faceting is that it is harder to compare some subsets that should be close together, as in this example where the plots for the 50's and 60's are particularly far apart because of the way the ribbon has been wrapped around. The figure was produced with the following code:

% FIGLISTING
%   COL: 1 GG-WIDTH: 10 TEX-WIDTH: \linewidth
%   LABEL: movies-wrap
%   CAPTION: Movie rating distribution by decade.
% 
% movies$decade <- round_any(movies$year, 10, floor)
% qplot(rating, ..density.., data=subset(movies, decade > 1890),
%   geom="histogram", binwidth = 0.5) + facet_wrap(~ decade, ncol = 6)
\input{_include/bd5052fe6b5cd25a74bd1173da5c2713.tex}
% END

The specification of faceting variables is of the form \code{ \~ a + b + c}. By default, \code{facet_wrap} will try and lay out the panels as close to a square as possible, with a slight bias towards wider rather than taller rectangles. You can override the default by setting \code{ncol}, \code{nrow} or both. See the documentation for more examples.

\subsection{Controlling scales}
\label{sub:controlling_scales}

For both types of faceting you can control whether the position scales are the same in all panels (fixed) or allowed to vary between panels (free).  This is controlled by the \code{scales} parameter:

\begin{itemize}
  \item \code{scales = "fixed"}: x and y scales are fixed across all panels.
  \item \code{scales = "free"}: x and y scales vary across panels.
  \item \code{scales = "free_x"}: the x scale is free, and the y scale is fixed.
  \item \code{scales = "free_y"}: the y scale is free, and the x scale is fixed.
\end{itemize}

\noindent Figure~\ref{fig:fixed-vs-free} illustrates the difference between the two extremes of fixed and free.

% FIGLSTING
%   LABEL: fixed-vs-free
%   CAPTION: Fixed scales (left) have the same scale for each facet, while 
%   free scales (right) have a different scale for each facet. 
% 
% p <- qplot(cty, hwy, data = mpg)
% p + facet_wrap(~ cyl)
% p + facet_wrap(~ cyl, scales = "free")

Fixed scales allow us to compare subsets on an equal basis, seeing where each fits into the overall pattern.  Free scales zoom in on the region that each subset occupies, allowing you to see more details. Free scales are particularly useful when we want to display multiple times series that were measured on different scales.  To do this, we first need to change from ``wide'' to ``long'' data, stacking the separate variables into a single column.  An example of this is shown in Figure~\ref{fig:time}, and the topic is discussed in more detail in Section~\ref{sec:melting}.  

% FIGLISTING
%   COL: 1 GG-WIDTH: 10 TEX-WIDTH: \linewidth
%   LABEL: time
%   CAPTION: Free scales are particularly useful when displaying multiple 
%   time series that are measured on different scales.
% 
% em <- melt(economics, id = "date")
% qplot(date, value, data = em, geom = "line", group = variable) + 
%   facet_grid(variable ~ ., scale = "free_y")
\input{_include/e09a6a0237f5a28f889a723932e81c09.tex}
% END

There is an additional constraint on the scales of \code{facet_grid}: all panels in a column must have the same x scale, and all panels in a row must have the same y scale.  This is because each column shares an x axis, and each row shares a y axis.

For \code{facet_grid} there is an additional parameter called \code{space}, which takes values \code{"free"} or \code{"fixed"}.  When the space can vary freely, each column (or row) will have width (or height) proportional to the range of the scale for that column (or row).  This makes the scaling equal across the whole plot: 1 cm on each panel maps to the same range of data.  (This is somewhat analogous to the ``sliced'' axis limits of lattice.)  For example, if panel a had range 2 and panel b had range 4, one-third of the space would be given to a, and two-thirds to b.  This is most useful for categorical scales, where we can assign space proportionally based on the number of levels in each facet, as illustrated by Figure~\ref{fig:discrete-free}.  The code to create this plot is shown below: note the use of \f{reorder} to arrange the models and manufacturers in order of city fuel usage.

% FIGLISTING
%   COL: 2 GG-HEIGHT: 8 TEX-WIDTH: 0.5\linewidth
%   LABEL: discrete-free
%   CAPTION: A dotplot showing the range of city gas mileage for each model
%   of car. (Left) Models ordered by average mpg, and (right) faceted by
%   manufacturer with \code{scales="free_y"} and \code{space = "free"}.  The 
%   {\tt strip.text.y} theme setting has been used to rotate the facet labels.
% 
% mpg3 <- within(mpg2, {
%   model <- reorder(model, cty)
%   manufacturer <- reorder(manufacturer, -cty)
% })
% models <- qplot(cty, model, data = mpg3)
%
% models
% models + facet_grid(manufacturer ~ ., scales = "free", space = "free") +  
%   opts(strip.text.y = theme_text())
\input{_include/6372918d59bea61377fc56df5649164a.tex}
% END

\subsection{Missing faceting variables}
\label{sub:missing_faceting_columns}

If you using faceting on a plot with multiple datasets, what happens when one of those datasets is missing the faceting variables? This situation commonly arises when you are adding contextual information that should be the same in all panels. For example, imagine you have spatial display of disease faceted by gender. What happens when you add a map layer that does not contain the gender variable?  Here \ggplot will do what you expect: it will display the map in every facet: missing faceting variables are treated like they have all values.

% % INTERWEAVE
% %   COL: 1 GG-WIDTH: 8 TEX-WIDTH: 0.8\linewidth
% % 
% % ctyhwy <- qplot(cty, hwy, data = mpg2) + facet_grid(. ~ cyl)
% % ctyhwy
% % 
% % # Extract best & worst
% % extreme_global <- subset(mpg2, rank(cty) < 5)
% % extreme <- subset(mpg2, cty < 12 | cty > 30)[c("cty", "hwy", "cyl")]
% % 
% % # Show extremes on all facets (remove cyl column)
% % ctyhwy + geom_point(data = extreme[, 1:2], colour = "red", size = 2)
% % # Show extremes for each facet (keep cyl column)
% % ctyhwy + geom_point(data = extreme, colour = "red", size = 2)
% \input{_include/05ca8013028096d11fa34a73dcafa7df.tex}
% % END

\subsection{Grouping vs. faceting}
\label{sub:group-vs-facet}

Faceting is an alternative to using aesthetics (like colour, shape or size) to differentiate groups. Both techniques have strengths and weaknesses, based around the relative positions of the subsets. 

With faceting, each group is quite far apart in its own panel, and there is no overlap between the groups.  This is good if the groups overlap a lot, but it does make small differences harder to see.  When using aesthetics to differentiate groups, the groups are close together and may overlap, but small differences are easier to see.  Figure~\ref{fig:group-vs-facet} illustrates these trade-offs.  With the scatterplots, it is possible to not realise the groups are overlapping when just colour is used to separate them, but with the regression lines they are too far apart to see that D, E and G are grouped together and J is farther away.  The code to produce these figures is shown below.

% FIGLISTING
%   LABEL: group-vs-facet
%   CAPTION: The differences between faceting vs. grouping, illustrated with
%   a log-log plot of carat vs. price with four selected colours.
%   FILETYPE: png
% 
% xmajor <- c(0.3, 0.5, 1,3, 5)
% xminor <- as.vector(outer(1:10, 10^c(-1, 0)))
% ymajor <- c(500, 1000, 5000, 10000)
% yminor <- as.vector(outer(1:10, 10^c(2,3,4)))
% dplot <- ggplot(subset(diamonds, color %in% c("D","E","G","J")), 
%   aes(carat, price, colour = color)) + 
%   scale_x_log10(breaks = xmajor, labels = xmajor, minor = xminor) + 
%   scale_y_log10(breaks = ymajor, labels = ymajor, minor = yminor) + 
%   scale_colour_hue(limits = levels(diamonds$color)) + 
%   opts(legend.position = "none")
%
% dplot + geom_point()
% dplot + geom_point() + facet_grid(. ~ color)
% 
% dplot + geom_smooth(method = lm, se = F, fullrange = T)
% dplot + geom_smooth(method = lm, se = F, fullrange = T) + 
%   facet_grid(. ~ color)
\input{_include/8f9272f9d0ff09afc1f2dc3d8247bd0a.tex}
% END

Faceting will also work with much larger number of groups, and because you can split in two dimensions, you can compare two variables simultaneously more easily than using two different aesthetics.  The other advantage of faceting is that the scales can vary across panels, which is useful if the subsets occupy very different ranges.


\subsection{Dodging vs. faceting}
\label{sub:dodge-vs-facet}

Faceting can achieve similar effects to dodging. Figure~\ref{fig:fvd-crossed} shows how dodging and faceting can create plots that look remarkably similar. The main difference is the labelling: the faceted plot has colour labelled above and cut below; and the dodged plot has colour below and cut is not explicitly labelled. In this example, the labels in the faceted plot need some adjustment to display in a readable way, see the code below for details.

% FIGLISTING
%   LABEL: fvd-crossed
%   COL: 1 GG-WIDTH: 10 TEX-WIDTH: 0.8\linewidth
%   CAPTION: Dodging (top) vs. faceting (bottom) for a completely crossed
%   pair of variables.
% 
% qplot(color, data=diamonds, geom="bar", fill=cut, position="dodge")
% qplot(cut, data = diamonds, geom = "bar", fill = cut) + 
%   facet_grid(. ~ color) + 
%   opts(axis.text.x = theme_text(angle = 90, hjust = 1, size = 8, 
%    colour = "grey50"))
\input{_include/6871ee3b22d05b1e0534210cd60c6d5c.tex}
% END

Apart from labelling, the main difference between dodging and faceting occurs when the two variables are nearly completely crossed, but there are some combinations that do not occur.  In this case, dodging becomes less useful because it only splits up the bars locally, and there are no labels. Faceting is more useful as we can control whether the splitting is local (\code{scales = "free_x"}, \code{space = "free"}) or global (\code{scales = "fixed"}). Figure~\ref{fig:fvd-nested} compares faceting and dodging for two nested variables from the \code{mpg} dataset, model and manufacturer, with the code shown below.

% FIGLISTING
%   LABEL: fvd-nested
%   COL: 1 GG-WIDTH: 10 TEX-WIDTH: 0.8\linewidth
%   CAPTION: For nested data, there is a clear advantage to faceting (top
%   and middle) compared to dodging (bottom), because it is possible to 
%   carefully control and label the facets.  For this example, the top
%   plot is not useful, but it will be useful in situations where the data
%   is almost crossed, i.e.\ where a single combination is missing.
% 
% mpg4 <- subset(mpg, manufacturer %in% c("audi", "volkswagen", "jeep"))
% base <- ggplot(mpg4, aes(fill = model)) + 
%   geom_bar(position = "dodge")
% 
% base + aes(x = model) + 
%   facet_grid(. ~ manufacturer) + 
%   opts(legend.position = "none")
% last_plot() +  
%   facet_grid(. ~ manufacturer, scales = "free_x", space = "free")
% base + aes(x = manufacturer)
\input{_include/3b5f96ffa2a5e967d9f26addf7ebb70c.tex}
% END

In summary, the choice between faceting and dodging depends on the relationship between the two variables:

\begin{itemize}
  \item Completely crossed: faceting and dodging are basically equivalent.

  \item Almost crossed: faceting with shared scales ensures that all combinations are visible, even if empty. This is particularly useful if missing combinations are non-structural missings.

  \item Nested: faceting with free scales and space allocates just enough space for each higher level group, and labels each item individually.
\end{itemize}

\subsection{Continuous variables}\label{sub:continuous_variables}

You can facet by continuous variables, but you will need to convert them into discrete categories first. There are three ways to do this:

\begin{itemize}
  \item Divide the data into n bins each of the same length: \code{cut_interval(x, n = 10)} to specify the number of bins, or \code{cut_interval(x, length = 1)} to specify the length of each interval.  Specifying the number of bins is easy, but may produce ranges that are not ``nice'' numbers.
  
  \item Divide the data into n bins each containing (approximately) the same number of points: \code{cut_number(x, n = 10)}.  This makes it easier to compare facets (they will all have the same number of points), but you need to note that the range of each bin is different.
    
\end{itemize}

\noindent The following code demonstrates each of the three possibilities, with the results shown in Figure~\ref{fig:discretising}.

% FIGLISTING
%   COL: 1 GG-HEIGHT: 2 GG-WIDTH: 8 TEX-WIDTH: 0.9\linewidth
%   LABEL: discretising
%   CAPTION: Three ways of breaking a continuous variable into discrete bins.
%   From top to bottom: bins of length one, six bins of equal length,
%   six bins containing equal numbers of points.
% 
% mpg2$disp_ww <- cut_interval(mpg2$displ, length = 1)
% mpg2$disp_wn <- cut_interval(mpg2$displ, n = 6)
% mpg2$disp_nn <- cut_number(mpg2$displ, n = 6)
%
% plot <- qplot(cty, hwy, data = mpg2) + labs(x = NULL, y = NULL)
% plot + facet_wrap(~ disp_ww, nrow = 1)
% plot + facet_wrap(~ disp_wn, nrow = 1)
% plot + facet_wrap(~ disp_nn, nrow = 1)
\input{_include/fb7fb59c23da7cb97ce30f5288840581.tex}
% END

Note that the faceting formula only works with variables in the dataset (not functions of the variables), so you will also need to create a new variable containing the discretised data.

\section{Coordinate systems}
\label{sec:coord}

Coordinate systems tie together the two position scales to produce a 2d location. Currently, \ggplot comes with six different coordinate systems, listed in Table~\ref{tbl:coord}.  All these coordinate systems are two dimensional, although one day I hope to add 3d graphics too. As with the other components in \ggplot, you generate the R name by joining \code{coord_} and the name of the coordinate system.  Most plots use the default Cartesian coordinate system, \f{coord_cartesian}, where the 2d position of an element is given by the combination of the x and y positions.  

Coordinate systems have two main jobs: 

\begin{itemize}
  \item Combine the two position aesthetics to produce a 2d position on the plot.  The position aesthetics are called \code{x} and \code{y}, but they might be better called position 1 and 2 because their meaning depends on the coordinate system used.  For example, with the polar coordinate system they become angle and radius (or radius and angle), and with maps they become latitude and longitude.
  
  \item In coordination with the faceter, coordinate systems draw axes and panel backgrounds.  While the scales control the values that appear on the axes, and how they map from data to position, it is the coordinate system which actually draws them.  This is because their appearance depends on the coordinate system: an angle axis looks quite different than an x axis. 

\end{itemize}

% source("latex.r")
% describe(Coord)

\begin{table}
  \begin{center}
  \begin{tabular}{ll}
    \toprule
    Name      & Description  \\
    \midrule
    cartesian & Cartesian coordinates                  \\
    equal     & Equal scale Cartesian coordinates      \\
    flip      & Flipped Cartesian coordinates          \\
    trans     & Transformed Cartesian coordinate system\\[1em]
    map       & Map projections                        \\
    polar     & Polar coordinates                      \\
    \bottomrule
    
  \end{tabular}
  \end{center}
  \caption{Coordinate systems available in ggplot.  \code{coord_equal}, \code{coord_flip} and \code{coord_trans} are all basically Cartesian in nature (i.e., the dimensions combine orthogonally), while \code{coord_map} and \code{coord_polar} are more complex.}
  \label{tbl:coord}
\end{table}

\subsection{Transformation}
\label{sub:coord-transformation}

Unlike transforming the data or transforming the scales, transformations carried out by the coordinate system change the appearance of the geoms: in polar coordinates a rectangle becomes a slice of a doughnut; in a map projection, the shortest path between two points will no longer be a straight line.  Figure~\ref{fig:coord-trans-examples} illustrates what happens to a line and a rectangle in a few different coordinate systems.

% FIGURE
%   LABEL: coord-trans-examples
%   GG-WIDTH: 3  GG-HEIGHT: 3 COL: 3
%   TEX-WIDTH: 0.25\linewidth
%   CAPTION: A set of examples illustrating what a line and rectangle
%   look like when displayed in a variety of coordinate systems.  From 
%   top left to bottom right: Cartesian, polar with x position mapped to
%   angle, polar with y position mapped to angle, flipped, transformed with
%   log in y direction, and equal scales.
% 
% rect <- data.frame(x = 50, y = 50)
% line <- data.frame(x = c(1, 200), y = c(100, 1))
% base <- ggplot(mapping = aes(x, y)) + 
%   geom_tile(data = rect, aes(width = 50, height = 50)) + 
%   geom_line(data = line)
% base
% base + coord_polar("x")
% base + coord_polar("y")
% base + coord_flip()
% base + coord_trans(y = "log10")
% base + coord_equal()
\input{_include/a3958c66682bc5eada30c9add00d824c.tex}
% END

This transformation takes part in two steps. Firstly, the parameterisation of each geom is changed to be purely location-based, rather than location and dimension based. For example, a bar can be represented as an x position (a location), a height and a width (two dimensions). But how do we interpret height and width in a non-Cartesian coordinate system, where a rectangle may not have constant height and width? We solve the problem by using a purely location-based representation, the location of the four corners of the rectangle, and then transforming these locations: we have converted a rectangle to a polygon. By doing this, we effectively convert all geoms to a combination of points, lines and polygons.

With all geoms in this consistent, location-based, representation, the next step is to transform each location into the new coordinate system. It is easy to transform points, because a point is still a point no matter what coordinate system you are in, but lines and polygons are harder, because a straight line may no longer be straight in the new coordinate system. To make the problem tractable we assume that all coordinate transformations are smooth, in the sense that all very short lines will still be very short straight lines in the new coordinate system. With this assumption in hand, we can transform lines and polygons by breaking them up into many small line segments and transforming each segment. This process is called munching. Figure~\ref{fig:coord-trans} illustrates this procedure. We start with a line parameterised by its two endpoints, then break it into multiple line segments, each with two endpoints. Those points are then translated into the new coordinate system, and connected. In the example, the number of line segments is too small, so you can see more easily how it works. For practical use, we use many more segments so that the result looks smooth.

% FIGURE
%   LABEL: coord-trans
%   COL: 3 TEX-WIDTH: 0.33\linewidth
%   CAPTION: How coordinate transformations work: converting a line in 
%   Cartesian coordinates to a line in polar coordinates.  The original x 
%   position is converted to radius, and the y position to angle.  
% 
% r_grid <- seq(0, 1, length = 15)
% theta_grid <- seq(0, 3 / 2 * pi, length = 15)
% extents <- data.frame(r = range(r_grid), theta = range(theta_grid))
% base <- ggplot(extents, aes(r, theta)) + opts(aspect.ratio = 1) +
%   scale_y_continuous(expression(theta))
%
% base + geom_point(colour = "red", size = 4) + geom_line()
% pts <- data.frame(r = r_grid, theta = theta_grid)
% base + geom_line() + geom_point(data = pts)
% base + geom_point(data = pts)
% 
% xlab <- scale_x_continuous(expression(x == r * sin(theta)))
% ylab <- scale_y_continuous(expression(x == r * cos(theta)))
% polar <- base %+% pts + aes(x = r * sin(theta), y = r * cos(theta)) + 
%   xlab + ylab
% polar + geom_point()
% polar + geom_point() + geom_path()
% polar + geom_point(data=extents, colour = "red", size = 4) + geom_path() 
\input{_include/786b63a311118e76f5aea3c8d87ee8da.tex}
% END

\subsection{Statistics}
\label{sub:statistics}

To be technically correct, the actual statistical method used by a stat should depend on the coordinate system.  For example, a smoother in polar coordinates should use circular regression, and in 3d should return a 2d surface rather than a 1d curve.  However, many statistical operations have not been derived for non-Cartesian coordinates and \ggplot falls back to Cartesian coordinates for calculation, which, while not strictly correct, will normally be a fairly close approximation.

\subsection{Cartesian coordinate systems}
\label{sub:cartesian}

The four Cartesian-based coordinate systems, \code{coord_cartesian}, \code{coord_equal}, \code{coord_flip} and \code{coord_trans}, share a number of common features.  They are still essentially Cartesian because the x and y positions map orthogonally to x and y positions on the plot.  

\paragraph{Setting limits.}  \code{coord_cartesian} has arguments \code{xlim} and \code{ylim}.  If you think back to the scales chapter, you might wonder why we need these.  Doesn't the limits argument of the scales already allow use to control what appears on the plot?  The key difference is how the limits work: when setting scale limits, any data outside the limits is thrown away; but when setting coordinate system limits we still use all the data, but we only display a small region of the plot.  Setting coordinate system limits is like looking at the plot under a magnifying glass.  Figures~\ref{fig:limits-smooth} and~\ref{fig:limits-bin} show an example of this.

% FIGLISTING
%   COL: 3 TEX-WIDTH: 0.33\linewidth
%   LABEL: limits-smooth
%   CAPTION: Setting limits on the coordinate system, vs setting limits
%   on the scales.  (Left) Entire dataset; (middle) x scale limits set to 
%   (325, 500); (right) coordinate system x limits set to (325, 500).  Scaling
%   the coordinate limits performs a visual zoom, while setting the scale
%   limits subsets the data and refits the smooth.
% 
% (p <- qplot(disp, wt, data=mtcars) + geom_smooth())
% p + scale_x_continuous(limits = c(325, 500))
% p + coord_cartesian(xlim = c(325, 500))
\input{_include/3a77065646d827ec4fc0c0ae99058a32.tex}
% END

% FIGLISTING
%   COL: 3 TEX-WIDTH: 0.33\linewidth
%   LABEL: limits-bin
%   CAPTION: Setting limits on the coordinate system, vs. setting limits
%   on the scales.  (Left) Entire dataset; (middle) x scale limits set to 
%   (0, 2); (right) coordinate x limits set to (0, 2).  Compare the size of 
%   the bins: when you set the scale limits, there are the same number of bins
%   but they each cover a smaller region of the data; when you set the
%   coordinate limits, there are fewer bins and they cover the same amount of
%   data as the original.
% 
% (d <- ggplot(diamonds, aes(carat, price)) + 
%   stat_bin2d(bins = 25, colour="grey70") + opts(legend.position = "none")) 
% d + scale_x_continuous(limits = c(0, 2))
% d + coord_cartesian(xlim = c(0, 2))
\input{_include/2e0ec07796b4f6abf97cc20ce64a8a8e.tex}
% END

\paragraph{Flipping the axes.}  Most statistics and geoms assume you are interested in y values conditional on x values (e.g., smooth, summary, boxplot, line): in most statistical models, the x values are assumed to be measured without error.  If you are interested in x conditional on y (or you just want to rotate the plot 90 degrees), you can use \code{coord_flip} to exchange the x and y axes.  Compare this with just exchanging the variables mapped to x and y, as shown in Figure~\ref{fig:coord-flip}.

% FIGLISTING
%   COL: 3 TEX-WIDTH: 0.33\linewidth
%   LABEL: coord-flip
%   CAPTION: (Left) A scatterplot and smoother with engine displacement on
%   x axis and city mpg on y axis.  (Middle) Exchanging \var{cty} and 
%   \var{displ} rotates the plot 90 degrees, but the smooth is fit to the
%   rotated data.  (Right) using \code{coord_flip} fits the smooth to the
%   original data, and then rotates the output, this is a smooth curve of x 
%   conditional on y.
% 
% qplot(displ, cty, data = mpg) + geom_smooth()
% qplot(cty, displ, data = mpg) + geom_smooth()
% qplot(cty, displ, data = mpg) + geom_smooth() + coord_flip()
\input{_include/3a73e288172b3cc27796b918ae17068e.tex}
% END

\paragraph{Transformations.}  Like limits, we can also transform the data in two places: at the scale level or at the coordinate system level. \code{coord_trans} has arguments \code{x} and \code{y} which should be strings naming the transformer (Table~\ref{tbl:common-trans}) to use for that axis. Transforming at the scale level occurs before statistics are computed and does not change the shape of the geom.  Transforming at the coordinate system level occurs after the statistics have been computed, and does affect the shape of the geom.  Using both together allows us to model the data on a transformed scale and then backtransform it for interpretation: a common pattern in analysis.  An example of this is shown in Figure~\ref{fig:backtrans}.

% FIGLISTING
%   FILETYPE: PNG
%   LABEL: backtrans
%   CAPTION: (Left) A scatterplot of carat vs. price on log base 10
%   transformed scales.  A linear regression summarises the trend: 
%   $\log(y) = a + b * \log(x)$.  (Right) The previous 
%   plot backtransformed (with {\tt coord\_trans(x = "pow10", y = "pow10")})
%   onto the original scales.  The linear trend line now becomes geometric,
%   $y = k * c^x$, and highlights the lack of expensive diamonds for larger 
%   carats.
% 
% qplot(carat, price, data = diamonds, log = "xy") + 
%   geom_smooth(method = "lm")
% last_plot() + coord_trans(x = "pow10", y = "pow10")
\input{_include/ce7caa7c9ec65d0eef501780c94184f5.tex}
% END

\paragraph{Equal scales.}  \code{coord_equal} ensures that the x and y axes have equal scales: i.e., 1 cm along the x axis represents the same range of data as 1 cm along the y axis.  By default it will assume that you want a one-to-one ratio, but you can change this with the \code{ratio} parameter.  The aspect ratio will also be set to ensure that the mapping is maintained regardless of the shape of the output device.  See the documentation of \f{coord_equal} for more details.

\subsection{Non-Cartesian coordinate systems}

There are two non-Cartesian coordinate systems: polar coordinates and map projections.  These coordinate systems are still somewhat experimental, and there are fewer standards for the layout of axes, so you may need to tweak them to meet your needs using the tools in Chapter~\ref{cha:grid}.

\paragraph{Polar coordinates.}  Using polar coordinates gives rise to pie charts and wind roses (from bar geoms), and radar charts (from line geoms).  Polar coordinates are often used for circular data, particularly time or direction, but the perceptual properties are not good because the angle is harder to perceive for small radii than it is for large radii.  The \code{theta} argument determines which position variable is mapped to angle (by default, x) and which to radius.  Figure~\ref{fig:polar} shows how by changing the coordinate system we can turn a bar chart into a pie chart or a bullseye chart.  The documentation includes other examples of polar charts.

% FIGLISTING
%   COL: 3 TEX-WIDTH: 0.33\linewidth
%   LABEL: polar
%   CAPTION: (Left) A stacked bar chart.  (Middle) The stacked bar chart 
%   in polar coordinates, with x position mapped to radius and y position
%   mapped to angle, \code{coord_polar(theta = "y")}).  This is more 
%   commonly known as a pie chart.  (Right) The stacked bar chart in polar
%   coordinates with the opposite mapping, \code{coord_polar(theta = "x")}.
%   This is sometimes called a bullseye chart.
% 
% # Stacked barchart
% (pie <- ggplot(mtcars, aes(x = factor(1), fill = factor(cyl))) +
%   geom_bar(width = 1))
% # Pie chart
% pie + coord_polar(theta = "y")
%
% # The bullseye chart
% pie + coord_polar()
\input{_include/a47a1bf3a0f152068e3777b13dbabc22.tex}
% END

\paragraph{Map projections.}  These are still rather experimental, and rely on the \code{mapproj} package \citep{mapproj}.  \f{coord_map} takes the same arguments as \f{mapproj} for controlling the projection.  See the documentation of \f{coord_map} for more examples, and consult a cartographer for the most appropriate projection for your data.

\ifwhole
\else
  \bibliography{/Users/hadley/documents/phd/references}
  \end{document}
\fi

\chapter{Themes}\label{cha:polishing}

\section{Introduction}

In this chapter you will learn how to use the ggplot2 theme system,
which allows you to exercise fine control over the non-data elements of
your plot. The theme system does not affect how the data is rendered by
geoms, or how it is transformed by scales. Themes don't change the
perceptual properties of the plot, but they do help you make the plot
aesthetically pleasing or match an existing style guide. Themes give you
control over things like fonts, ticks, panel strips, and backgrounds.
\index{Themes}

This separation of control into data and non-data parts is quite
different from base and lattice graphics. In base and lattice graphics,
most functions take a large number of arguments that specify both data
and non-data appearance, which makes the functions complicated and
harder to learn. ggplot2 takes a different approach: when creating the
plot you determine how the data is displayed, then \emph{after} it has
been created you can edit every detail of the rendering, using the
theming system.

The theming system is composed of four main components:

\begin{itemize}
\item
  Theme \textbf{elements} specify the non-data elements that you can
  control. For example, the \texttt{plot.title} element controls the
  appearance of the plot title; \texttt{axis.ticks.x}, the ticks on the
  x axis; \texttt{legend.key.height}, the height of the keys in the
  legend.
\item
  Each element is associated with an \textbf{element function}, which
  describes the visual properties of the element. For example,
  \texttt{element\_text()} sets the font size, colour and face of text
  elements like \texttt{plot.title}.
\item
  The \texttt{theme()} function which allows you to override the default
  theme elements by calling element functions, like
  \texttt{theme(plot.title\ =\ element\_text(colour\ =\ "red"))}.
\item
  Complete \textbf{themes}, like \texttt{theme\_grey()} set all of the
  theme elements to values designed to work together harmoniously.
\end{itemize}

For example, imagine you've made the following plot of your data.

\begin{Shaded}
\begin{Highlighting}[]
\NormalTok{base <-}\StringTok{ }\KeywordTok{ggplot}\NormalTok{(mpg, }\KeywordTok{aes}\NormalTok{(cty, hwy, }\DataTypeTok{color =} \KeywordTok{factor}\NormalTok{(cyl))) +}
\StringTok{  }\KeywordTok{geom_jitter}\NormalTok{() +}\StringTok{ }
\StringTok{  }\KeywordTok{geom_abline}\NormalTok{(}\DataTypeTok{colour =} \StringTok{"grey50"}\NormalTok{, }\DataTypeTok{size =} \DecValTok{2}\NormalTok{)}
\NormalTok{base}
\end{Highlighting}
\end{Shaded}

\begin{figure}[H]
  \centering
  \includegraphics[width=0.75\linewidth]{_figures/themes/motivation-1-1}
\end{figure}

It's served its purpose for you: you've learned that \texttt{cty} and
\texttt{hwy} are highly correlated, both are tightly coupled with
\texttt{cyl}, and that \texttt{hwy} is always greater than \texttt{cty}
(and the difference increases as \texttt{cty} increases). Now you want
to share the plot with others, perhaps by publishing it in a paper. That
requires some changes. First, you need to make sure the plot can stand
alone by:

\begin{itemize}
\tightlist
\item
  Improving the axes and legend labels.
\item
  Adding a title for the plot.
\item
  Tweaking the colour scale.
\end{itemize}

Fortunately you know how to do that already because you've read
\hyperref[cha:scales]{the scales chapter}:

\begin{Shaded}
\begin{Highlighting}[]
\NormalTok{labelled <-}\StringTok{ }\NormalTok{base +}
\StringTok{  }\KeywordTok{labs}\NormalTok{(}
    \DataTypeTok{x =} \StringTok{"City mileage/gallon"}\NormalTok{,}
    \DataTypeTok{y =} \StringTok{"Highway mileage/gallon"}\NormalTok{,}
    \DataTypeTok{colour =} \StringTok{"Cylinders"}\NormalTok{,}
    \DataTypeTok{title =} \StringTok{"Highway and city mileage are highly correlated"}
  \NormalTok{) +}
\StringTok{  }\KeywordTok{scale_colour_brewer}\NormalTok{(}\DataTypeTok{type =} \StringTok{"seq"}\NormalTok{, }\DataTypeTok{palette =} \StringTok{"Spectral"}\NormalTok{)}
\NormalTok{labelled}
\end{Highlighting}
\end{Shaded}

\begin{figure}[H]
  \centering
  \includegraphics[width=0.75\linewidth]{_figures/themes/motivation-2-1}
\end{figure}

Next, you need to make sure the plot matches the style guidelines of
your journal:

\begin{itemize}
\tightlist
\item
  The background should be white, not pale grey.
\item
  The legend should be placed inside the plot if there's room.
\item
  Major gridlines should be a pale grey and minor gridlines should be
  removed.
\item
  The plot title should be 12pt bold text.
\end{itemize}

In this chapter, you'll learn how to use the theming system to make
those changes, as shown below:

\begin{Shaded}
\begin{Highlighting}[]
\NormalTok{styled <-}\StringTok{ }\NormalTok{labelled +}
\StringTok{  }\KeywordTok{theme_bw}\NormalTok{() +}\StringTok{ }
\StringTok{  }\KeywordTok{theme}\NormalTok{(}
    \DataTypeTok{plot.title =} \KeywordTok{element_text}\NormalTok{(}\DataTypeTok{face =} \StringTok{"bold"}\NormalTok{, }\DataTypeTok{size =} \DecValTok{12}\NormalTok{),}
    \DataTypeTok{legend.background =} \KeywordTok{element_rect}\NormalTok{(}\DataTypeTok{fill =} \StringTok{"white"}\NormalTok{, }\DataTypeTok{size =} \DecValTok{4}\NormalTok{, }\DataTypeTok{colour =} \StringTok{"white"}\NormalTok{),}
    \DataTypeTok{legend.justification =} \KeywordTok{c}\NormalTok{(}\DecValTok{0}\NormalTok{, }\DecValTok{1}\NormalTok{),}
    \DataTypeTok{legend.position =} \KeywordTok{c}\NormalTok{(}\DecValTok{0}\NormalTok{, }\DecValTok{1}\NormalTok{),}
    \DataTypeTok{axis.ticks =} \KeywordTok{element_line}\NormalTok{(}\DataTypeTok{colour =} \StringTok{"grey70"}\NormalTok{, }\DataTypeTok{size =} \FloatTok{0.2}\NormalTok{),}
    \DataTypeTok{panel.grid.major =} \KeywordTok{element_line}\NormalTok{(}\DataTypeTok{colour =} \StringTok{"grey70"}\NormalTok{, }\DataTypeTok{size =} \FloatTok{0.2}\NormalTok{),}
    \DataTypeTok{panel.grid.minor =} \KeywordTok{element_blank}\NormalTok{()}
  \NormalTok{)}
\NormalTok{styled}
\end{Highlighting}
\end{Shaded}

\begin{figure}[H]
  \centering
  \includegraphics[width=0.75\linewidth]{_figures/themes/motivation-3-1}
\end{figure}

Finally, the journal wants the figure as a 600 dpi TIFF file. You'll
learn the fine details of \texttt{ggsave()} in
\hyperref[sec:saving]{saving your output}.

\section{Complete themes}\label{sec:themes}

ggplot2 comes with a number of built in themes. The most important is
\texttt{theme\_grey()}, the signature ggplot2 theme with a light grey
background and white gridlines. The theme is designed to put the data
forward while supporting comparisons, following the advice of (Tufte
2006; Brewer 1994; Carr 2002; Carr 1994; Carr and Sun 1999). We can
still see the gridlines to aid in the judgement of position (Cleveland
1993), but they have little visual impact and we can easily `tune' them
out. The grey background gives the plot a similar typographic colour to
the text, ensuring that the graphics fit in with the flow of a document
without jumping out with a bright white background. Finally, the grey
background creates a continuous field of colour which ensures that the
plot is perceived as a single visual entity. \index{Themes!built-in}
\indexf{theme\_grey}

There are seven other themes built in to ggplot2 1.1.0:

\begin{itemize}
\item
  \texttt{theme\_bw()}: a variation on \texttt{theme\_grey()} that uses
  a white background and thin grey grid lines. \indexf{theme\_bw}
\item
  \texttt{theme\_linedraw()}: A theme with only black lines of various
  widths on white backgrounds, reminiscent of a line drawing.
  \indexf{theme\_linedraw}
\item
  \texttt{theme\_light()}: similar to \texttt{theme\_linedraw()} but
  with light grey lines and axes, to direct more attention towards the
  data. \indexf{theme\_light}
\item
  \texttt{theme\_dark()}: the dark cousin of \texttt{theme\_light()},
  with similar line sizes but a dark background. Useful to make thin
  coloured lines pop out. \indexf{theme\_dark}
\item
  \texttt{theme\_minimal()}: A minimalistic theme with no background
  annotations. \indexf{theme\_minimal}
\item
  \texttt{theme\_classic()}: A classic-looking theme, with x and y axis
  lines and no gridlines. \indexf{theme\_classic}
\item
  \texttt{theme\_void()}: A completely empty theme. \indexf{theme\_void}
\end{itemize}

\begin{Shaded}
\begin{Highlighting}[]
\NormalTok{df <-}\StringTok{ }\KeywordTok{data.frame}\NormalTok{(}\DataTypeTok{x =} \DecValTok{1}\NormalTok{:}\DecValTok{3}\NormalTok{, }\DataTypeTok{y =} \DecValTok{1}\NormalTok{:}\DecValTok{3}\NormalTok{)}
\NormalTok{base <-}\StringTok{ }\KeywordTok{ggplot}\NormalTok{(df, }\KeywordTok{aes}\NormalTok{(x, y)) +}\StringTok{ }\KeywordTok{geom_point}\NormalTok{()}
\NormalTok{base +}\StringTok{ }\KeywordTok{theme_grey}\NormalTok{() +}\StringTok{ }\KeywordTok{ggtitle}\NormalTok{(}\StringTok{"theme_grey()"}\NormalTok{)}
\NormalTok{base +}\StringTok{ }\KeywordTok{theme_bw}\NormalTok{() +}\StringTok{ }\KeywordTok{ggtitle}\NormalTok{(}\StringTok{"theme_bw()"}\NormalTok{)}
\NormalTok{base +}\StringTok{ }\KeywordTok{theme_linedraw}\NormalTok{() +}\StringTok{ }\KeywordTok{ggtitle}\NormalTok{(}\StringTok{"theme_linedraw()"}\NormalTok{)}
\end{Highlighting}
\end{Shaded}

\begin{figure}[H]
  \includegraphics[width=0.333\linewidth]{_figures/themes/built-in-1}%
  \includegraphics[width=0.333\linewidth]{_figures/themes/built-in-2}%
  \includegraphics[width=0.333\linewidth]{_figures/themes/built-in-3}
\end{figure}

\begin{Shaded}
\begin{Highlighting}[]
\NormalTok{base +}\StringTok{ }\KeywordTok{theme_light}\NormalTok{() +}\StringTok{ }\KeywordTok{ggtitle}\NormalTok{(}\StringTok{"theme_light()"}\NormalTok{)}
\NormalTok{base +}\StringTok{ }\KeywordTok{theme_dark}\NormalTok{() +}\StringTok{ }\KeywordTok{ggtitle}\NormalTok{(}\StringTok{"theme_dark()"}\NormalTok{)}
\NormalTok{base +}\StringTok{ }\KeywordTok{theme_minimal}\NormalTok{()  +}\StringTok{ }\KeywordTok{ggtitle}\NormalTok{(}\StringTok{"theme_minimal()"}\NormalTok{)}
\end{Highlighting}
\end{Shaded}

\begin{figure}[H]
  \includegraphics[width=0.333\linewidth]{_figures/themes/unnamed-chunk-1-1}%
  \includegraphics[width=0.333\linewidth]{_figures/themes/unnamed-chunk-1-2}%
  \includegraphics[width=0.333\linewidth]{_figures/themes/unnamed-chunk-1-3}
\end{figure}

\begin{Shaded}
\begin{Highlighting}[]
\NormalTok{base +}\StringTok{ }\KeywordTok{theme_classic}\NormalTok{() +}\StringTok{ }\KeywordTok{ggtitle}\NormalTok{(}\StringTok{"theme_classic()"}\NormalTok{)}
\NormalTok{base +}\StringTok{ }\KeywordTok{theme_void}\NormalTok{() +}\StringTok{ }\KeywordTok{ggtitle}\NormalTok{(}\StringTok{"theme_void()"}\NormalTok{)}
\end{Highlighting}
\end{Shaded}

\begin{figure}[H]
  \includegraphics[width=0.333\linewidth]{_figures/themes/unnamed-chunk-2-1}%
  \includegraphics[width=0.333\linewidth]{_figures/themes/unnamed-chunk-2-2}
\end{figure}

All themes have a \texttt{base\_size} parameter which controls the base
font size. The base font size is the size that the axis titles use: the
plot title is usually bigger (1.2x), and the tick and strip labels are
smaller (0.8x). If you want to control these sizes separately, you'll
need to modify the individual elements as described below.

As well as applying themes a plot at a time, you can change the default
theme with \texttt{theme\_set()}. For example, if you really hate the
default grey background, run \texttt{theme\_set(theme\_bw())} to use a
white background for all plots. \indexf{theme\_set}

You're not limited to the themes built-in to ggplot2. Other packages,
like ggthemes by Jeffrey Arnold, add even more. Here's a few of my
favourites from ggthemes: \index{ggtheme}

\begin{Shaded}
\begin{Highlighting}[]
\KeywordTok{library}\NormalTok{(ggthemes)}
\NormalTok{base +}\StringTok{ }\KeywordTok{theme_tufte}\NormalTok{() +}\StringTok{ }\KeywordTok{ggtitle}\NormalTok{(}\StringTok{"theme_tufte()"}\NormalTok{)}
\NormalTok{base +}\StringTok{ }\KeywordTok{theme_solarized}\NormalTok{() +}\StringTok{ }\KeywordTok{ggtitle}\NormalTok{(}\StringTok{"theme_solarized()"}\NormalTok{)}
\NormalTok{base +}\StringTok{ }\KeywordTok{theme_excel}\NormalTok{() +}\StringTok{ }\KeywordTok{ggtitle}\NormalTok{(}\StringTok{"theme_excel()"}\NormalTok{) }\CommentTok{# ;)}
\end{Highlighting}
\end{Shaded}

\begin{figure}[H]
  \includegraphics[width=0.333\linewidth]{_figures/themes/ggtheme-1}%
  \includegraphics[width=0.333\linewidth]{_figures/themes/ggtheme-2}%
  \includegraphics[width=0.333\linewidth]{_figures/themes/ggtheme-3}
\end{figure}

The complete themes are a great place to start but don't give you a lot
of control. To modify individual elements, you need to use
\texttt{theme()} to override the default setting for an element with an
element function.

\subsection{Exercises}

\begin{enumerate}
\def\labelenumi{\arabic{enumi}.}
\item
  Try out all the themes in ggthemes. Which do you like the best?
\item
  What aspects of the default theme do you like? What don't you like?\\
  What would you change?
\item
  Look at the plots in your favourite scientific journal. What theme do
  they most resemble? What are the main differences?
\end{enumerate}

\section{Modifying theme components}

To modify an individual theme component you use code like
\texttt{plot\ +\ theme(element.name\ =\ element\_function())}. In this
section you'll learn about the basic element functions, and then in the
next section, you'll see all the elements that you can modify.
\indexf{theme}

There are four basic types of built-in element functions: text, lines,
rectangles, and blank. Each element function has a set of parameters
that control the appearance:

\begin{itemize}
\item
  \texttt{element\_text()} draws labels and headings. You can control
  the font \texttt{family}, \texttt{face}, \texttt{colour},
  \texttt{size} (in points), \texttt{hjust}, \texttt{vjust},
  \texttt{angle} (in degrees) and \texttt{lineheight} (as ratio of
  \texttt{fontcase}). More details on the parameters can be found in
  \texttt{vignette("ggplot2-specs")}. Setting the font face is
  particularly challenging. \index{Themes!labels} \indexf{element\_text}

\begin{Shaded}
\begin{Highlighting}[]
\NormalTok{base_t <-}\StringTok{ }\NormalTok{base +}\StringTok{ }\KeywordTok{labs}\NormalTok{(}\DataTypeTok{title =} \StringTok{"This is a ggplot"}\NormalTok{) +}\StringTok{ }\KeywordTok{xlab}\NormalTok{(}\OtherTok{NULL}\NormalTok{) +}\StringTok{ }\KeywordTok{ylab}\NormalTok{(}\OtherTok{NULL}\NormalTok{)}
\NormalTok{base_t +}\StringTok{ }\KeywordTok{theme}\NormalTok{(}\DataTypeTok{plot.title =} \KeywordTok{element_text}\NormalTok{(}\DataTypeTok{size =} \DecValTok{16}\NormalTok{))}
\NormalTok{base_t +}\StringTok{ }\KeywordTok{theme}\NormalTok{(}\DataTypeTok{plot.title =} \KeywordTok{element_text}\NormalTok{(}\DataTypeTok{face =} \StringTok{"bold"}\NormalTok{, }\DataTypeTok{colour =} \StringTok{"red"}\NormalTok{))}
\NormalTok{base_t +}\StringTok{ }\KeywordTok{theme}\NormalTok{(}\DataTypeTok{plot.title =} \KeywordTok{element_text}\NormalTok{(}\DataTypeTok{hjust =} \DecValTok{1}\NormalTok{))}
\end{Highlighting}
\end{Shaded}

  \begin{figure}[H]
    \includegraphics[width=0.333\linewidth]{_figures/themes/element_text-1}%
    \includegraphics[width=0.333\linewidth]{_figures/themes/element_text-2}%
    \includegraphics[width=0.333\linewidth]{_figures/themes/element_text-3}
  \end{figure}

  You can control the margins around the text with the \texttt{margin}
  argument and \texttt{margin()} function. \texttt{margin()} has four
  arguments: the amount of space (in points) to add to the top, right,
  bottom and left sides of the text. Any elements not specified default
  to 0.

\begin{Shaded}
\begin{Highlighting}[]
\CommentTok{# The margins here look asymmetric because there are also plot margins}
\NormalTok{base_t +}\StringTok{ }\KeywordTok{theme}\NormalTok{(}\DataTypeTok{plot.title =} \KeywordTok{element_text}\NormalTok{(}\DataTypeTok{margin =} \KeywordTok{margin}\NormalTok{()))}
\NormalTok{base_t +}\StringTok{ }\KeywordTok{theme}\NormalTok{(}\DataTypeTok{plot.title =} \KeywordTok{element_text}\NormalTok{(}\DataTypeTok{margin =} \KeywordTok{margin}\NormalTok{(}\DataTypeTok{t =} \DecValTok{10}\NormalTok{, }\DataTypeTok{b =} \DecValTok{10}\NormalTok{)))}
\NormalTok{base_t +}\StringTok{ }\KeywordTok{theme}\NormalTok{(}\DataTypeTok{axis.title.y =} \KeywordTok{element_text}\NormalTok{(}\DataTypeTok{margin =} \KeywordTok{margin}\NormalTok{(}\DataTypeTok{r =} \DecValTok{10}\NormalTok{)))}
\end{Highlighting}
\end{Shaded}

  \begin{figure}[H]
    \includegraphics[width=0.333\linewidth]{_figures/themes/element_text-margin-1}%
    \includegraphics[width=0.333\linewidth]{_figures/themes/element_text-margin-2}%
    \includegraphics[width=0.333\linewidth]{_figures/themes/element_text-margin-3}
  \end{figure}
\item
  \texttt{element\_line()} draws lines parameterised by \texttt{colour},
  \texttt{size} and \texttt{linetype}: \indexf{element\_line}
  \index{Themes!lines}

\begin{Shaded}
\begin{Highlighting}[]
\NormalTok{base +}\StringTok{ }\KeywordTok{theme}\NormalTok{(}\DataTypeTok{panel.grid.major =} \KeywordTok{element_line}\NormalTok{(}\DataTypeTok{colour =} \StringTok{"black"}\NormalTok{))}
\NormalTok{base +}\StringTok{ }\KeywordTok{theme}\NormalTok{(}\DataTypeTok{panel.grid.major =} \KeywordTok{element_line}\NormalTok{(}\DataTypeTok{size =} \DecValTok{2}\NormalTok{))}
\NormalTok{base +}\StringTok{ }\KeywordTok{theme}\NormalTok{(}\DataTypeTok{panel.grid.major =} \KeywordTok{element_line}\NormalTok{(}\DataTypeTok{linetype =} \StringTok{"dotted"}\NormalTok{))}
\end{Highlighting}
\end{Shaded}

  \begin{figure}[H]
    \includegraphics[width=0.333\linewidth]{_figures/themes/element_line-1}%
    \includegraphics[width=0.333\linewidth]{_figures/themes/element_line-2}%
    \includegraphics[width=0.333\linewidth]{_figures/themes/element_line-3}
  \end{figure}
\item
  \texttt{element\_rect()} draws rectangles, mostly used for
  backgrounds, parameterised by \texttt{fill} colour and border
  \texttt{colour}, \texttt{size} and \texttt{linetype}.\\
   \index{Background} \index{Themes!background} \indexf{theme\_rect}

\begin{Shaded}
\begin{Highlighting}[]
\NormalTok{base +}\StringTok{ }\KeywordTok{theme}\NormalTok{(}\DataTypeTok{plot.background =} \KeywordTok{element_rect}\NormalTok{(}\DataTypeTok{fill =} \StringTok{"grey80"}\NormalTok{, }\DataTypeTok{colour =} \OtherTok{NA}\NormalTok{))}
\NormalTok{base +}\StringTok{ }\KeywordTok{theme}\NormalTok{(}\DataTypeTok{plot.background =} \KeywordTok{element_rect}\NormalTok{(}\DataTypeTok{colour =} \StringTok{"red"}\NormalTok{, }\DataTypeTok{size =} \DecValTok{2}\NormalTok{))}
\NormalTok{base +}\StringTok{ }\KeywordTok{theme}\NormalTok{(}\DataTypeTok{panel.background =} \KeywordTok{element_rect}\NormalTok{(}\DataTypeTok{fill =} \StringTok{"linen"}\NormalTok{))}
\end{Highlighting}
\end{Shaded}

  \begin{figure}[H]
    \includegraphics[width=0.333\linewidth]{_figures/themes/element_rect-1}%
    \includegraphics[width=0.333\linewidth]{_figures/themes/element_rect-2}%
    \includegraphics[width=0.333\linewidth]{_figures/themes/element_rect-3}
  \end{figure}
\item
  \texttt{element\_blank()} draws nothing. Use this if you don't want
  anything drawn, and no space allocated for that element. The following
  example uses \texttt{element\_blank()} to progressively suppress the
  appearance of elements we're not interested in. Notice how the plot
  automatically reclaims the space previously used by these elements: if
  you don't want this to happen (perhaps because they need to line up
  with other plots on the page), use
  \texttt{colour\ =\ NA,\ fill\ =\ NA} to create invisible elements that
  still take up space. \indexf{element\_blank}

\begin{Shaded}
\begin{Highlighting}[]
\NormalTok{base}
\KeywordTok{last_plot}\NormalTok{() +}\StringTok{ }\KeywordTok{theme}\NormalTok{(}\DataTypeTok{panel.grid.minor =} \KeywordTok{element_blank}\NormalTok{())}
\KeywordTok{last_plot}\NormalTok{() +}\StringTok{ }\KeywordTok{theme}\NormalTok{(}\DataTypeTok{panel.grid.major =} \KeywordTok{element_blank}\NormalTok{())}
\end{Highlighting}
\end{Shaded}

  \begin{figure}[H]
    \includegraphics[width=0.333\linewidth]{_figures/themes/element_blank-1}%
    \includegraphics[width=0.333\linewidth]{_figures/themes/element_blank-2}%
    \includegraphics[width=0.333\linewidth]{_figures/themes/element_blank-3}
  \end{figure}

\begin{Shaded}
\begin{Highlighting}[]
\KeywordTok{last_plot}\NormalTok{() +}\StringTok{ }\KeywordTok{theme}\NormalTok{(}\DataTypeTok{panel.background =} \KeywordTok{element_blank}\NormalTok{())}
\KeywordTok{last_plot}\NormalTok{() +}\StringTok{ }\KeywordTok{theme}\NormalTok{(}
  \DataTypeTok{axis.title.x =} \KeywordTok{element_blank}\NormalTok{(), }
  \DataTypeTok{axis.title.y =} \KeywordTok{element_blank}\NormalTok{()}
\NormalTok{)}
\KeywordTok{last_plot}\NormalTok{() +}\StringTok{ }\KeywordTok{theme}\NormalTok{(}\DataTypeTok{axis.line =} \KeywordTok{element_line}\NormalTok{(}\DataTypeTok{colour =} \StringTok{"grey50"}\NormalTok{))}
\end{Highlighting}
\end{Shaded}

  \begin{figure}[H]
    \includegraphics[width=0.333\linewidth]{_figures/themes/element_blank-2-1}%
    \includegraphics[width=0.333\linewidth]{_figures/themes/element_blank-2-2}%
    \includegraphics[width=0.333\linewidth]{_figures/themes/element_blank-2-3}
  \end{figure}
\item
  A few other settings take grid units. Create them with
  \texttt{unit(1,\ "cm")} or \texttt{unit(0.25,\ "in")}.
\end{itemize}

To modify theme elements for all future plots, use
\texttt{theme\_update()}. It returns the previous theme settings, so you
can easily restore the original parameters once you're done.
\index{Themes!updating} \indexf{theme\_set}

\begin{Shaded}
\begin{Highlighting}[]
\NormalTok{old_theme <-}\StringTok{ }\KeywordTok{theme_update}\NormalTok{(}
  \DataTypeTok{plot.background =} \KeywordTok{element_rect}\NormalTok{(}\DataTypeTok{fill =} \StringTok{"lightblue3"}\NormalTok{, }\DataTypeTok{colour =} \OtherTok{NA}\NormalTok{),}
  \DataTypeTok{panel.background =} \KeywordTok{element_rect}\NormalTok{(}\DataTypeTok{fill =} \StringTok{"lightblue"}\NormalTok{, }\DataTypeTok{colour =} \OtherTok{NA}\NormalTok{),}
  \DataTypeTok{axis.text =} \KeywordTok{element_text}\NormalTok{(}\DataTypeTok{colour =} \StringTok{"linen"}\NormalTok{),}
  \DataTypeTok{axis.title =} \KeywordTok{element_text}\NormalTok{(}\DataTypeTok{colour =} \StringTok{"linen"}\NormalTok{)}
\NormalTok{)}
\NormalTok{base}
\KeywordTok{theme_set}\NormalTok{(old_theme)}
\NormalTok{base}
\end{Highlighting}
\end{Shaded}

\begin{figure}[H]
  \centering
  \includegraphics[width=0.333\linewidth]{_figures/themes/theme-update-1}%
  \includegraphics[width=0.333\linewidth]{_figures/themes/theme-update-2}
\end{figure}

\section{Theme elements}\label{sec:theme-elements}

There are around 40 unique elements that control the appearance of the
plot. They can be roughly grouped into five categories: plot, axis,
legend, panel and facet. The following sections describe each in turn.
\index{Themes!elements}

\subsection{Plot elements}

\index{Themes!plot}

Some elements affect the plot as a whole:

\begin{longtable}[c]{@{}lll@{}}
\toprule
Element & Setter & Description\tabularnewline
\midrule
\endhead
plot.background & \texttt{element\_rect()} & plot
background\tabularnewline
plot.title & \texttt{element\_text()} & plot title\tabularnewline
plot.margin & \texttt{margin()} & margins around plot\tabularnewline
\bottomrule
\end{longtable}

\texttt{plot.background} draws a rectangle that underlies everything
else on the plot. By default, ggplot2 uses a white background which
ensures that the plot is usable wherever it might end up (e.g.~even if
you save as a png and put on a slide with a black background). When
exporting plots to use in other systems, you might want to make the
background transparent with \texttt{fill\ =\ NA}. Similarly, if you're
embedding a plot in a system that already has margins you might want to
eliminate the built-in margins. Note that a small margin is still
necessary if you want to draw a border around the plot.

\begin{Shaded}
\begin{Highlighting}[]
\NormalTok{base +}\StringTok{ }\KeywordTok{theme}\NormalTok{(}\DataTypeTok{plot.background =} \KeywordTok{element_rect}\NormalTok{(}\DataTypeTok{colour =} \StringTok{"grey50"}\NormalTok{, }\DataTypeTok{size =} \DecValTok{2}\NormalTok{))}
\NormalTok{base +}\StringTok{ }\KeywordTok{theme}\NormalTok{(}
  \DataTypeTok{plot.background =} \KeywordTok{element_rect}\NormalTok{(}\DataTypeTok{colour =} \StringTok{"grey50"}\NormalTok{, }\DataTypeTok{size =} \DecValTok{2}\NormalTok{),}
  \DataTypeTok{plot.margin =} \KeywordTok{margin}\NormalTok{(}\DecValTok{2}\NormalTok{, }\DecValTok{2}\NormalTok{, }\DecValTok{2}\NormalTok{, }\DecValTok{2}\NormalTok{)}
\NormalTok{)}
\NormalTok{base +}\StringTok{ }\KeywordTok{theme}\NormalTok{(}\DataTypeTok{plot.background =} \KeywordTok{element_rect}\NormalTok{(}\DataTypeTok{fill =} \StringTok{"lightblue"}\NormalTok{))}
\end{Highlighting}
\end{Shaded}

\begin{figure}[H]
  \includegraphics[width=0.333\linewidth]{_figures/themes/plot-1}%
  \includegraphics[width=0.333\linewidth]{_figures/themes/plot-2}%
  \includegraphics[width=0.333\linewidth]{_figures/themes/plot-3}
\end{figure}

\subsection{Axis elements}\label{sub:theme-axis}

\index{Themes!axis} \index{Axis!styling}

The axis elements control the apperance of the axes:

\begin{longtable}[c]{@{}lll@{}}
\toprule
Element & Setter & Description\tabularnewline
\midrule
\endhead
axis.line & \texttt{element\_line()} & line parallel to axis (hidden in
default themes)\tabularnewline
axis.text & \texttt{element\_text()} & tick labels\tabularnewline
axis.text.x & \texttt{element\_text()} & x-axis tick
labels\tabularnewline
axis.text.y & \texttt{element\_text()} & y-axis tick
labels\tabularnewline
axis.title & \texttt{element\_text()} & axis titles\tabularnewline
axis.title.x & \texttt{element\_text()} & x-axis title\tabularnewline
axis.title.y & \texttt{element\_text()} & y-axis title\tabularnewline
axis.ticks & \texttt{element\_line()} & axis tick marks\tabularnewline
axis.ticks.length & \texttt{unit()} & length of tick
marks\tabularnewline
\bottomrule
\end{longtable}

Note that \texttt{axis.text} (and \texttt{axis.title}) comes in three
forms: \texttt{axis.text}, \texttt{axis.text.x}, and
\texttt{axis.text.y}. Use the first form if you want to modify the
properties of both axes at once: any properties that you don't
explicitly set in \texttt{axis.text.x} and \texttt{axis.text.y} will be
inherited from \texttt{axis.text}.

\begin{Shaded}
\begin{Highlighting}[]
\NormalTok{df <-}\StringTok{ }\KeywordTok{data.frame}\NormalTok{(}\DataTypeTok{x =} \DecValTok{1}\NormalTok{:}\DecValTok{3}\NormalTok{, }\DataTypeTok{y =} \DecValTok{1}\NormalTok{:}\DecValTok{3}\NormalTok{)}
\NormalTok{base <-}\StringTok{ }\KeywordTok{ggplot}\NormalTok{(df, }\KeywordTok{aes}\NormalTok{(x, y)) +}\StringTok{ }\KeywordTok{geom_point}\NormalTok{()}

\CommentTok{# Accentuate the axes}
\NormalTok{base +}\StringTok{ }\KeywordTok{theme}\NormalTok{(}\DataTypeTok{axis.line =} \KeywordTok{element_line}\NormalTok{(}\DataTypeTok{colour =} \StringTok{"grey50"}\NormalTok{, }\DataTypeTok{size =} \DecValTok{1}\NormalTok{))}
\CommentTok{# Style both x and y axis labels}
\NormalTok{base +}\StringTok{ }\KeywordTok{theme}\NormalTok{(}\DataTypeTok{axis.text =} \KeywordTok{element_text}\NormalTok{(}\DataTypeTok{color =} \StringTok{"blue"}\NormalTok{, }\DataTypeTok{size =} \DecValTok{12}\NormalTok{))}
\CommentTok{# Useful for long labels}
\NormalTok{base +}\StringTok{ }\KeywordTok{theme}\NormalTok{(}\DataTypeTok{axis.text.x =} \KeywordTok{element_text}\NormalTok{(}\DataTypeTok{angle =} \NormalTok{-}\DecValTok{90}\NormalTok{, }\DataTypeTok{vjust =} \FloatTok{0.5}\NormalTok{))}
\end{Highlighting}
\end{Shaded}

\begin{figure}[H]
  \includegraphics[width=0.333\linewidth]{_figures/themes/axis-1}%
  \includegraphics[width=0.333\linewidth]{_figures/themes/axis-2}%
  \includegraphics[width=0.333\linewidth]{_figures/themes/axis-3}
\end{figure}

The most common adjustment is to rotate the x-axis labels to avoid long
overlapping labels. If you do this, note negative angles tend to look
best and you should set \texttt{hjust\ =\ 0} and \texttt{vjust\ =\ 1}:

\begin{Shaded}
\begin{Highlighting}[]
\NormalTok{df <-}\StringTok{ }\KeywordTok{data.frame}\NormalTok{(}
  \DataTypeTok{x =} \KeywordTok{c}\NormalTok{(}\StringTok{"label"}\NormalTok{, }\StringTok{"a long label"}\NormalTok{, }\StringTok{"an even longer label"}\NormalTok{), }
  \DataTypeTok{y =} \DecValTok{1}\NormalTok{:}\DecValTok{3}
\NormalTok{)}
\NormalTok{base <-}\StringTok{ }\KeywordTok{ggplot}\NormalTok{(df, }\KeywordTok{aes}\NormalTok{(x, y)) +}\StringTok{ }\KeywordTok{geom_point}\NormalTok{()}
\NormalTok{base}
\NormalTok{base +}\StringTok{ }
\StringTok{  }\KeywordTok{theme}\NormalTok{(}\DataTypeTok{axis.text.x =} \KeywordTok{element_text}\NormalTok{(}\DataTypeTok{angle =} \NormalTok{-}\DecValTok{30}\NormalTok{, }\DataTypeTok{vjust =} \DecValTok{1}\NormalTok{, }\DataTypeTok{hjust =} \DecValTok{0}\NormalTok{)) +}\StringTok{ }
\StringTok{  }\KeywordTok{xlab}\NormalTok{(}\OtherTok{NULL}\NormalTok{) +}\StringTok{ }
\StringTok{  }\KeywordTok{ylab}\NormalTok{(}\OtherTok{NULL}\NormalTok{)}
\end{Highlighting}
\end{Shaded}

\begin{figure}[H]
  \includegraphics[width=0.5\linewidth]{_figures/themes/axis-labels-1}%
  \includegraphics[width=0.5\linewidth]{_figures/themes/axis-labels-2}
\end{figure}

\subsection{Legend elements}

\index{Themes!legend} \index{Legend!styling}

The legend elements control the apperance of all legends. You can also
modify the appearance of individual legends by modifying the same
elements in \texttt{guide\_legend()} or \texttt{guide\_colourbar()}.

\begin{longtable}[c]{@{}lll@{}}
\toprule
Element & Setter & Description\tabularnewline
\midrule
\endhead
legend.background & \texttt{element\_rect()} & legend
background\tabularnewline
legend.key & \texttt{element\_rect()} & background of legend
keys\tabularnewline
legend.key.size & \texttt{unit()} & legend key size\tabularnewline
legend.key.height & \texttt{unit()} & legend key height\tabularnewline
legend.key.width & \texttt{unit()} & legend key width\tabularnewline
legend.margin & \texttt{unit()} & legend margin\tabularnewline
legend.text & \texttt{element\_text()} & legend labels\tabularnewline
legend.text.align & 0--1 & legend label alignment (0 = right, 1 =
left)\tabularnewline
legend.title & \texttt{element\_text()} & legend name\tabularnewline
legend.title.align & 0--1 & legend name alignment (0 = right, 1 =
left)\tabularnewline
\bottomrule
\end{longtable}

These options are illustrated below:

\begin{Shaded}
\begin{Highlighting}[]
\NormalTok{df <-}\StringTok{ }\KeywordTok{data.frame}\NormalTok{(}\DataTypeTok{x =} \DecValTok{1}\NormalTok{:}\DecValTok{4}\NormalTok{, }\DataTypeTok{y =} \DecValTok{1}\NormalTok{:}\DecValTok{4}\NormalTok{, }\DataTypeTok{z =} \KeywordTok{rep}\NormalTok{(}\KeywordTok{c}\NormalTok{(}\StringTok{"a"}\NormalTok{, }\StringTok{"b"}\NormalTok{), }\DataTypeTok{each =} \DecValTok{2}\NormalTok{))}
\NormalTok{base <-}\StringTok{ }\KeywordTok{ggplot}\NormalTok{(df, }\KeywordTok{aes}\NormalTok{(x, y, }\DataTypeTok{colour =} \NormalTok{z)) +}\StringTok{ }\KeywordTok{geom_point}\NormalTok{()}

\NormalTok{base +}\StringTok{ }\KeywordTok{theme}\NormalTok{(}
  \DataTypeTok{legend.background =} \KeywordTok{element_rect}\NormalTok{(}
    \DataTypeTok{fill =} \StringTok{"lemonchiffon"}\NormalTok{, }
    \DataTypeTok{colour =} \StringTok{"grey50"}\NormalTok{, }
    \DataTypeTok{size =} \DecValTok{1}
  \NormalTok{)}
\NormalTok{)}
\NormalTok{base +}\StringTok{ }\KeywordTok{theme}\NormalTok{(}
  \DataTypeTok{legend.key =} \KeywordTok{element_rect}\NormalTok{(}\DataTypeTok{color =} \StringTok{"grey50"}\NormalTok{),}
  \DataTypeTok{legend.key.width =} \KeywordTok{unit}\NormalTok{(}\FloatTok{0.9}\NormalTok{, }\StringTok{"cm"}\NormalTok{),}
  \DataTypeTok{legend.key.height =} \KeywordTok{unit}\NormalTok{(}\FloatTok{0.75}\NormalTok{, }\StringTok{"cm"}\NormalTok{)}
\NormalTok{)}
\NormalTok{base +}\StringTok{ }\KeywordTok{theme}\NormalTok{(}
  \DataTypeTok{legend.text =} \KeywordTok{element_text}\NormalTok{(}\DataTypeTok{size =} \DecValTok{15}\NormalTok{),}
  \DataTypeTok{legend.title =} \KeywordTok{element_text}\NormalTok{(}\DataTypeTok{size =} \DecValTok{15}\NormalTok{, }\DataTypeTok{face =} \StringTok{"bold"}\NormalTok{)}
\NormalTok{)}
\end{Highlighting}
\end{Shaded}

\begin{figure}[H]
  \includegraphics[width=0.333\linewidth]{_figures/themes/legend-1}%
  \includegraphics[width=0.333\linewidth]{_figures/themes/legend-2}%
  \includegraphics[width=0.333\linewidth]{_figures/themes/legend-3}
\end{figure}

There are four other properties that control how legends are laid out in
the context of the plot (\texttt{legend.position},
\texttt{legend.direction}, \texttt{legend.justification},
\texttt{legend.box}). They are described in
\hyperref[sub:legend-layout]{legend layout}.

\subsection{Panel elements}

\index{Themes!panel} \index{Aspect ratio}

Panel elements control the appearance of the plotting panels:

\begin{longtable}[c]{@{}lll@{}}
\toprule
Element & Setter & Description\tabularnewline
\midrule
\endhead
panel.background & \texttt{element\_rect()} & panel background (under
data)\tabularnewline
panel.border & \texttt{element\_rect()} & panel border (over
data)\tabularnewline
panel.grid.major & \texttt{element\_line()} & major grid
lines\tabularnewline
panel.grid.major.x & \texttt{element\_line()} & vertical major grid
lines\tabularnewline
panel.grid.major.y & \texttt{element\_line()} & horizontal major grid
lines\tabularnewline
panel.grid.minor & \texttt{element\_line()} & minor grid
lines\tabularnewline
panel.grid.minor.x & \texttt{element\_line()} & vertical minor grid
lines\tabularnewline
panel.grid.minor.y & \texttt{element\_line()} & horizontal minor grid
lines\tabularnewline
aspect.ratio & numeric & plot aspect ratio\tabularnewline
\bottomrule
\end{longtable}

The main difference between \texttt{panel.background} and
\texttt{panel.border} is that the background is drawn underneath the
data, and the border is drawn on top of it. For that reason, you'll
always need to assign \texttt{fill\ =\ NA} when overriding
\texttt{panel.border}.

\begin{Shaded}
\begin{Highlighting}[]
\NormalTok{base <-}\StringTok{ }\KeywordTok{ggplot}\NormalTok{(df, }\KeywordTok{aes}\NormalTok{(x, y)) +}\StringTok{ }\KeywordTok{geom_point}\NormalTok{()}
\CommentTok{# Modify background}
\NormalTok{base +}\StringTok{ }\KeywordTok{theme}\NormalTok{(}\DataTypeTok{panel.background =} \KeywordTok{element_rect}\NormalTok{(}\DataTypeTok{fill =} \StringTok{"lightblue"}\NormalTok{))}

\CommentTok{# Tweak major grid lines}
\NormalTok{base +}\StringTok{ }\KeywordTok{theme}\NormalTok{(}
  \DataTypeTok{panel.grid.major =} \KeywordTok{element_line}\NormalTok{(}\DataTypeTok{color =} \StringTok{"gray60"}\NormalTok{, }\DataTypeTok{size =} \FloatTok{0.8}\NormalTok{)}
\NormalTok{)}
\CommentTok{# Just in one direction  }
\NormalTok{base +}\StringTok{ }\KeywordTok{theme}\NormalTok{(}
  \DataTypeTok{panel.grid.major.x =} \KeywordTok{element_line}\NormalTok{(}\DataTypeTok{color =} \StringTok{"gray60"}\NormalTok{, }\DataTypeTok{size =} \FloatTok{0.8}\NormalTok{)}
\NormalTok{)}
\end{Highlighting}
\end{Shaded}

\begin{figure}[H]
  \includegraphics[width=0.333\linewidth]{_figures/themes/panel-1}%
  \includegraphics[width=0.333\linewidth]{_figures/themes/panel-2}%
  \includegraphics[width=0.333\linewidth]{_figures/themes/panel-3}
\end{figure}

Note that aspect ratio controls the aspect ratio of the \emph{panel},
not the overall plot:

\begin{Shaded}
\begin{Highlighting}[]
\NormalTok{base2 <-}\StringTok{ }\NormalTok{base +}\StringTok{ }\KeywordTok{theme}\NormalTok{(}\DataTypeTok{plot.background =} \KeywordTok{element_rect}\NormalTok{(}\DataTypeTok{colour =} \StringTok{"grey50"}\NormalTok{))}
\CommentTok{# Wide screen}
\NormalTok{base2 +}\StringTok{ }\KeywordTok{theme}\NormalTok{(}\DataTypeTok{aspect.ratio =} \DecValTok{9} \NormalTok{/}\StringTok{ }\DecValTok{16}\NormalTok{)}
\CommentTok{# Long and skiny}
\NormalTok{base2 +}\StringTok{ }\KeywordTok{theme}\NormalTok{(}\DataTypeTok{aspect.ratio =} \DecValTok{2} \NormalTok{/}\StringTok{ }\DecValTok{1}\NormalTok{)}
\CommentTok{# Square}
\NormalTok{base2 +}\StringTok{ }\KeywordTok{theme}\NormalTok{(}\DataTypeTok{aspect.ratio =} \DecValTok{1}\NormalTok{)}
\end{Highlighting}
\end{Shaded}

\begin{figure}[H]
  \includegraphics[width=0.333\linewidth]{_figures/themes/aspect-ratio-1}%
  \includegraphics[width=0.333\linewidth]{_figures/themes/aspect-ratio-2}%
  \includegraphics[width=0.333\linewidth]{_figures/themes/aspect-ratio-3}
\end{figure}

\subsection{Facetting elements}

\index{Themes!facets} \index{Facetting!styling}

The following theme elements are associated with faceted ggplots:

\begin{longtable}[c]{@{}lll@{}}
\toprule
Element & Setter & Description\tabularnewline
\midrule
\endhead
strip.background & \texttt{element\_rect()} & background of panel
strips\tabularnewline
strip.text & \texttt{element\_text()} & strip text\tabularnewline
strip.text.x & \texttt{element\_text()} & horizontal strip
text\tabularnewline
strip.text.y & \texttt{element\_text()} & vertical strip
text\tabularnewline
panel.margin & \texttt{unit()} & margin between facets\tabularnewline
panel.margin.x & \texttt{unit()} & margin between facets
(vertical)\tabularnewline
panel.margin.y & \texttt{unit()} & margin between facets
(horizontal)\tabularnewline
\bottomrule
\end{longtable}

Element \texttt{strip.text.x} affects both \texttt{facet\_wrap()} or
\texttt{facet\_grid()}; \texttt{strip.text.y} only affects
\texttt{facet\_grid()}.

\begin{Shaded}
\begin{Highlighting}[]
\NormalTok{df <-}\StringTok{ }\KeywordTok{data.frame}\NormalTok{(}\DataTypeTok{x =} \DecValTok{1}\NormalTok{:}\DecValTok{4}\NormalTok{, }\DataTypeTok{y =} \DecValTok{1}\NormalTok{:}\DecValTok{4}\NormalTok{, }\DataTypeTok{z =} \KeywordTok{c}\NormalTok{(}\StringTok{"a"}\NormalTok{, }\StringTok{"a"}\NormalTok{, }\StringTok{"b"}\NormalTok{, }\StringTok{"b"}\NormalTok{))}
\NormalTok{base_f <-}\StringTok{ }\KeywordTok{ggplot}\NormalTok{(df, }\KeywordTok{aes}\NormalTok{(x, y)) +}\StringTok{ }\KeywordTok{geom_point}\NormalTok{() +}\StringTok{ }\KeywordTok{facet_wrap}\NormalTok{(~z)}

\NormalTok{base_f}
\NormalTok{base_f +}\StringTok{ }\KeywordTok{theme}\NormalTok{(}\DataTypeTok{panel.margin =} \KeywordTok{unit}\NormalTok{(}\FloatTok{0.5}\NormalTok{, }\StringTok{"in"}\NormalTok{))}
\NormalTok{base_f +}\StringTok{ }\KeywordTok{theme}\NormalTok{(}
  \DataTypeTok{strip.background =} \KeywordTok{element_rect}\NormalTok{(}\DataTypeTok{fill =} \StringTok{"grey20"}\NormalTok{, }\DataTypeTok{color =} \StringTok{"grey80"}\NormalTok{, }\DataTypeTok{size =} \DecValTok{1}\NormalTok{),}
  \DataTypeTok{strip.text =} \KeywordTok{element_text}\NormalTok{(}\DataTypeTok{colour =} \StringTok{"white"}\NormalTok{)}
\NormalTok{)}
\end{Highlighting}
\end{Shaded}

\begin{figure}[H]
  \includegraphics[width=0.333\linewidth]{_figures/themes/facetting-1}%
  \includegraphics[width=0.333\linewidth]{_figures/themes/facetting-2}%
  \includegraphics[width=0.333\linewidth]{_figures/themes/facetting-3}
\end{figure}

\subsection{Exercises}

\begin{enumerate}
\def\labelenumi{\arabic{enumi}.}
\item
  Create the ugliest plot possible! (Contributed by Andrew D. Steen,
  University of Tennessee - Knoxville)
\item
  \texttt{theme\_dark()} makes the inside of the plot dark, but not the
  outside. Change the plot background to black, and then update the text
  settings so you can still read the labels.
\item
  Make an elegant theme that uses ``linen'' as the background colour and
  a serif font for the text.
\item
  Systematically explore the effects of \texttt{hjust} when you have a
  multiline title. Why doesn't \texttt{vjust} do anything?
\end{enumerate}

\hyperdef{}{sec:saving}{\section{Saving your output}\label{sec:saving}}

When saving a plot to use in another program, you have two basic choices
of output: raster or vector: \index{Exporting} \index{Saving output}

\begin{itemize}
\item
  Vector graphics describe a plot as sequence of operations: draw a line
  from \((x_1, y_1)\) to \((x_2, y_2)\), draw a circle at \((x_3, x_4)\)
  with radius \(r\). This means that they are effectively `infinitely'
  zoomable; there is no loss of detail. The most useful vector graphic
  formats are pdf and svg.
\item
  Raster graphics are stored as an array of pixel colours and have a
  fixed optimal viewing size. The most useful raster graphic format is
  png.
\end{itemize}

Figure \ref{fig:vector-raster} illustrates the basic differences in
these formats for a circle. A good description is available at
\url{http://tinyurl.com/rstrvctr}.

\begin{figure}[htbp]
  \centering
    \includegraphics[width= 0.5\linewidth]{diagrams/vector-raster}
  \caption{The schematic difference between raster (left) and vector (right) graphics. }
  \label{fig:vector-raster}
\end{figure}

Unless there is a compelling reason not to, use vector graphics: they
look better in more places. There are two main reasons to use raster
graphics:

\begin{itemize}
\item
  You have a plot (e.g.~a scatterplot) with thousands of graphical
  objects (i.e.~points). A vector version will be large and slow to
  render.
\item
  You want to embed the graphic in MS Office. MS has poor support for
  vector graphics (except for their own DrawingXML format which is not
  currently easy to make from R), so raster graphics are easier.
\end{itemize}

There are two ways to save output from ggplot2. You can use the standard
R approach where you open a graphics device, generate the plot, then
close the device: \indexf{pdf}

\begin{Shaded}
\begin{Highlighting}[]
\KeywordTok{pdf}\NormalTok{(}\StringTok{"output.pdf"}\NormalTok{, }\DataTypeTok{width =} \DecValTok{6}\NormalTok{, }\DataTypeTok{height =} \DecValTok{6}\NormalTok{)}
\KeywordTok{ggplot}\NormalTok{(mpg, }\KeywordTok{aes}\NormalTok{(displ, cty)) +}\StringTok{ }\KeywordTok{geom_point}\NormalTok{()}
\KeywordTok{dev.off}\NormalTok{()}
\end{Highlighting}
\end{Shaded}

This works for all packages, but is verbose. ggplot2 provides a
convenient shorthand with \texttt{ggsave()}:

\begin{Shaded}
\begin{Highlighting}[]
\KeywordTok{ggplot}\NormalTok{(mpg, }\KeywordTok{aes}\NormalTok{(displ, cty)) +}\StringTok{ }\KeywordTok{geom_point}\NormalTok{()}
\KeywordTok{ggsave}\NormalTok{(}\StringTok{"output.pdf"}\NormalTok{)}
\end{Highlighting}
\end{Shaded}

\texttt{ggsave()} is optimised for interactive use: you can use it after
you've drawn a plot. It has the following important arguments:
\indexf{ggsave}

\begin{itemize}
\item
  The first argument, \texttt{path}, specifies the path where the image
  should be saved. The file extension will be used to automatically
  select the correct graphics device. \texttt{ggsave()} can produce
  \texttt{.eps}, \texttt{.pdf}, \texttt{.svg}, \texttt{.wmf},
  \texttt{.png}, \texttt{.jpg}, \texttt{.bmp}, and \texttt{.tiff}.
\item
  \texttt{width} and \texttt{height} control the output size, specified
  in inches. If left blank, they'll use the size of the on-screen
  graphics device.
\item
  For raster graphics (i.e. \texttt{.png}, \texttt{.jpg}), the
  \texttt{dpi} argument controls the resolution of the plot. It defaults
  to 300, which is appropriate for most printers, but you may want to
  use 600 for particularly high-resolution output, or 96 for on-screen
  (e.g., web) display.
\end{itemize}

See \texttt{?ggsave} for more details.

\section*{References}
\addcontentsline{toc}{section}{References}

\hyperdef{}{ref-brewer:1994}{\label{ref-brewer:1994}}
Brewer, Cynthia A. 1994. ``Color Use Guidelines for Mapping and
Visualization.'' In \emph{Visualization in Modern Cartography}, edited
by A.M. MacEachren and D.R.F. Taylor, 123--47. Elsevier Science.

\hyperdef{}{ref-carr:1994}{\label{ref-carr:1994}}
Carr, Dan. 1994. ``Using Gray in Plots.'' \emph{ASA Statistical
Computing and Graphics Newsletter} 2 (5): 11--14.
\url{http://www.galaxy.gmu.edu/~dcarr/lib/v5n2.pdf}.

\hyperdef{}{ref-carr:2002}{\label{ref-carr:2002}}
---------. 2002. ``Graphical Displays.'' In \emph{Encyclopedia of
Environmetrics}, edited by Abdel H. El-Shaarawi and Walter W. Piegorsch,
2:933--60. John Wiley \& Sons.
\url{http://www.galaxy.gmu.edu/~dcarr/lib/EnvironmentalGraphics.pdf}.

\hyperdef{}{ref-carr:1999}{\label{ref-carr:1999}}
Carr, Dan, and Ru Sun. 1999. ``Using Layering and Perceptual Grouping in
Statistical Graphics.'' \emph{ASA Statistical Computing and Graphics
Newsletter} 10 (1): 25--31.

\hyperdef{}{ref-cleveland:1993a}{\label{ref-cleveland:1993a}}
Cleveland, William. 1993. ``A Model for Studying Display Methods of
Statistical Graphics.'' \emph{Journal of Computational and Graphical
Statistics} 2: 323--64. \url{http://stat.bell-labs.com/doc/93.4.ps}.

\hyperdef{}{ref-tufte:2006}{\label{ref-tufte:2006}}
Tufte, Edward R. 2006. \emph{Beautiful Evidence}. Graphics Press.


\part{Data analysis}

\hypertarget{cha:data}{%
\chapter{Data analysis}\label{cha:data}}

\hypertarget{introduction}{%
\section{Introduction}\label{introduction}}

So far, every example in this book has started with a nice dataset
that's easy to plot. That's great for learning (because you don't want
to struggle with data handling while you're learning visualisation), but
in real life, datasets hardly ever come in exactly the right structure.
To use ggplot2 in practice, you'll need to learn some data wrangling
skills. Indeed, in my experience, visualisation is often the easiest
part of the data analysis process: once you have the right data, in the
right format, aggregated in the right way, the right visualisation is
often obvious.

The goal of this part of the book is to show you how to integrate
ggplot2 with other tools needed for a complete data analysis:

\begin{itemize}
\item
  In this chapter, you'll learn the principles of tidy data (Wickham
  2014), which help you organise your data in a way that makes it easy
  to visualise with ggplot2, manipulate with dplyr and model with the
  many modelling packages. The principles of tidy data are supported by
  the \textbf{tidyr} package, which helps you tidy messy datasets.
\item
  Most visualisations require some data transformation whether it's
  creating a new variable from existing variables, or performing simple
  aggregations so you can see the forest for the trees.
  \protect\hyperlink{cha:dplyr}{dplyr} will show you how to do this with
  the \textbf{dplyr} package.
\item
  If you're using R, you're almost certainly using it for its fantastic
  modelling capabilities. While there's an R package for almost every
  type of model that you can think of, the results of these models can
  be hard to visualise. In \protect\hyperlink{cha:modelling}{modelling},
  you'll learn about the \textbf{broom} package, by David Robinson, to
  convert models into tidy datasets so you can easily visualise them
  with ggplot2.
\end{itemize}

Tidy data is the foundation for data manipulation and visualising
models. In the following sections, you'll learn the definition of tidy
data, and the tools you need to make messy data tidy. The chapter
concludes with two case studies that show how to apply the tools in
sequence to work with real(istic) data.

\hypertarget{sec:tidy-data}{%
\section{Tidy data}\label{sec:tidy-data}}

The principle behind tidy data is simple: storing your data in a
consistent way makes it easier to work with it. Tidy data is a mapping
between the statistical structure of a data frame (variables and
observations) and the physical structure (columns and rows). Tidy data
follows two main principles: \index{Tidy data}
\index{Data!best form for ggplot2}

\begin{enumerate}
\def\labelenumi{\arabic{enumi}.}
\tightlist
\item
  Variables go in columns.
\item
  Observations go in rows.
\end{enumerate}

Tidy data is particularly important for ggplot2 because the job of
ggplot2 is to map variables to visual properties: if your data isn't
tidy, you'll have a hard time visualising it.

Sometimes you'll find a dataset that you have no idea how to plot.
That's normally because it's not tidy. For example, take this data frame
that contains monthly employment data for the United States:

\begin{Shaded}
\begin{Highlighting}[]
\NormalTok{ec2}
\CommentTok{#> # A tibble: 12 x 11}
\CommentTok{#>   month `2006` `2007` `2008` `2009` `2010` `2011` `2012` `2013`}
\CommentTok{#>   <dbl>  <dbl>  <dbl>  <dbl>  <dbl>  <dbl>  <dbl>  <dbl>  <dbl>}
\CommentTok{#> 1     1    8.6    8.3    9     10.7   20     21.6   21     16.2}
\CommentTok{#> 2     2    9.1    8.5    8.7   11.7   19.9   21.1   19.8   17.5}
\CommentTok{#> 3     3    8.7    9.1    8.7   12.3   20.4   21.5   19.2   17.7}
\CommentTok{#> 4     4    8.4    8.6    9.4   13.1   22.1   20.9   19.1   17.1}
\CommentTok{#> 5     5    8.5    8.2    7.9   14.2   22.3   21.6   19.9   17  }
\CommentTok{#> 6     6    7.3    7.7    9     17.2   25.2   22.3   20.1   16.6}
\CommentTok{#> # ... with 6 more rows, and 2 more variables: `2014` <dbl>,}
\CommentTok{#> #   `2015` <dbl>}
\end{Highlighting}
\end{Shaded}

(If it looks familiar, it's because it's derived from the
\texttt{economics} dataset that we used earlier in the book.)

Imagine you want to plot a time series showing how unemployment has
changed over the last 10 years. Can you picture the ggplot2 command
you'd need to do it? What if you wanted to focus on the seasonal
component of unemployment by putting months on the x-axis and drawing
one line for each year? It's difficult to see how to create those plots
because the data is not tidy. There are three variables, month, year and
unemployment rate, but each variable is stored in a different way:

\begin{itemize}
\tightlist
\item
  \texttt{month} is stored in a column.
\item
  \texttt{year} is spread across the column names.
\item
  \texttt{rate} is the value of each cell.
\end{itemize}

To make it possible to plot this data we first need to tidy it. There
are two important pairs of tools:

\begin{itemize}
\tightlist
\item
  Spread \& gather.
\item
  Separate \& unite.
\end{itemize}

\hypertarget{sec:spread-gather}{%
\section{Spread and gather}\label{sec:spread-gather}}

Take a look at the two tables below:

\begin{longtable}[]{@{}llr@{}}
\toprule
x & y & z\tabularnewline
\midrule
\endhead
a & A & 1\tabularnewline
b & D & 5\tabularnewline
c & A & 4\tabularnewline
c & B & 10\tabularnewline
d & C & 9\tabularnewline
\bottomrule
\end{longtable}

\begin{longtable}[]{@{}lrrrr@{}}
\toprule
x & A & B & C & D\tabularnewline
\midrule
\endhead
a & 1 & NA & NA & NA\tabularnewline
b & NA & NA & NA & 5\tabularnewline
c & 4 & 10 & NA & NA\tabularnewline
d & NA & NA & 9 & NA\tabularnewline
\bottomrule
\end{longtable}

If you study them for a little while, you'll notice that they contain
the same data in different forms. I call the first form \textbf{indexed}
data, because you look up a value using an index (the values of the
\texttt{x} and \texttt{y} variables). I call the second form
\textbf{Cartesian} data, because you find a value by looking at
intersection of a row and a column. We can't tell if these datasets are
tidy or not. Either form could be tidy depending on what the values
``A'', ``B'', ``C'', ``D'' mean.

(Also note the missing values: missing values that are explicit in one
form may be implicit in the other. An \texttt{NA} is the presence of an
absense; but sometimes a missing value is the absense of a presence.)

Tidying your data will often require translating Cartesian → indexed
forms, called \textbf{gathering}, and less commonly, indexed →
Cartesian, called \textbf{spreading}. The tidyr package provides the
\texttt{spread()} and \texttt{gather()} functions to perform these
operations, as described below.

(You can imagine generalising these ideas to higher dimensions. However,
data is almost always stored in 2d (rows \& columns), so these
generalisations are fun to think about, but not that practical. I
explore the idea more in Wickham (2007).

\hypertarget{gather}{%
\subsection{Gather}\label{gather}}

\texttt{gather()} has four main arguments: \indexf{gather}

\begin{itemize}
\item
  \texttt{data}: the dataset to translate.
\item
  \texttt{key} \& \texttt{value}: the key is the name of the variable
  that will be created from the column names, and the value is the name
  of the variable that will be created from the cell values.
\item
  \texttt{...}: which variables to gather. You can specify individually,
  \texttt{A,\ B,\ C,\ D}, or as a range \texttt{A:D}. Alternatively, you
  can specify which columns are \emph{not} to be gathered with
  \texttt{-}: \texttt{-E,\ -F}.
\end{itemize}

To tidy the economics dataset shown above, you first need to identify
the variables: \texttt{year}, \texttt{month} and \texttt{rate}.
\texttt{month} is already in a column, but \texttt{year} and
\texttt{rate} are in Cartesian form, and we want them in indexed form,
so we need to use \texttt{gather()}. In this example, the key is
\texttt{year}, the value is \texttt{unemp} and we want to select columns
from \texttt{2006} to \texttt{2015}:

\begin{Shaded}
\begin{Highlighting}[]
\KeywordTok{gather}\NormalTok{(ec2, }\DataTypeTok{key =}\NormalTok{ year, }\DataTypeTok{value =}\NormalTok{ unemp, }\StringTok{`}\DataTypeTok{2006}\StringTok{`}\OperatorTok{:}\StringTok{`}\DataTypeTok{2015}\StringTok{`}\NormalTok{)}
\CommentTok{#> # A tibble: 120 x 3}
\CommentTok{#>   month year  unemp}
\CommentTok{#>   <dbl> <chr> <dbl>}
\CommentTok{#> 1     1 2006    8.6}
\CommentTok{#> 2     2 2006    9.1}
\CommentTok{#> 3     3 2006    8.7}
\CommentTok{#> 4     4 2006    8.4}
\CommentTok{#> 5     5 2006    8.5}
\CommentTok{#> 6     6 2006    7.3}
\CommentTok{#> # ... with 114 more rows}
\end{Highlighting}
\end{Shaded}

Note that the columns have names that are not standard variable names in
R (they don't start with a letter). This means that we need to surround
them in backticks, i.e. \texttt{\textasciigrave{}2006\textasciigrave{}}
to refer to them.

Alternatively, we could gather all columns except \texttt{month}:

\begin{Shaded}
\begin{Highlighting}[]
\KeywordTok{gather}\NormalTok{(ec2, }\DataTypeTok{key =}\NormalTok{ year, }\DataTypeTok{value =}\NormalTok{ unemp, }\OperatorTok{-}\NormalTok{month)}
\CommentTok{#> # A tibble: 120 x 3}
\CommentTok{#>   month year  unemp}
\CommentTok{#>   <dbl> <chr> <dbl>}
\CommentTok{#> 1     1 2006    8.6}
\CommentTok{#> 2     2 2006    9.1}
\CommentTok{#> 3     3 2006    8.7}
\CommentTok{#> 4     4 2006    8.4}
\CommentTok{#> 5     5 2006    8.5}
\CommentTok{#> 6     6 2006    7.3}
\CommentTok{#> # ... with 114 more rows}
\end{Highlighting}
\end{Shaded}

To be most useful, we can provide two extra arguments:

\begin{Shaded}
\begin{Highlighting}[]
\NormalTok{economics_}\DecValTok{2}\NormalTok{ <-}\StringTok{ }\KeywordTok{gather}\NormalTok{(ec2, year, rate, }\StringTok{`}\DataTypeTok{2006}\StringTok{`}\OperatorTok{:}\StringTok{`}\DataTypeTok{2015}\StringTok{`}\NormalTok{, }
  \DataTypeTok{convert =} \OtherTok{TRUE}\NormalTok{, }\DataTypeTok{na.rm =} \OtherTok{TRUE}\NormalTok{)}
\NormalTok{economics_}\DecValTok{2}
\CommentTok{#> # A tibble: 112 x 3}
\CommentTok{#>   month  year  rate}
\CommentTok{#>   <dbl> <int> <dbl>}
\CommentTok{#> 1     1  2006   8.6}
\CommentTok{#> 2     2  2006   9.1}
\CommentTok{#> 3     3  2006   8.7}
\CommentTok{#> 4     4  2006   8.4}
\CommentTok{#> 5     5  2006   8.5}
\CommentTok{#> 6     6  2006   7.3}
\CommentTok{#> # ... with 106 more rows}
\end{Highlighting}
\end{Shaded}

We use \texttt{convert\ =\ TRUE} to automatically convert the years from
character strings to numbers, and \texttt{na.rm\ =\ TRUE} to remove the
months with no data. (In some sense the data isn't actually missing
because it represents dates that haven't occurred yet.)

When the data is in this form, it's easy to visualise in many different
ways. For example, we can choose to emphasise either long term trend or
seasonal variations:

\begin{Shaded}
\begin{Highlighting}[]
\KeywordTok{ggplot}\NormalTok{(economics_}\DecValTok{2}\NormalTok{, }\KeywordTok{aes}\NormalTok{(year }\OperatorTok{+}\StringTok{ }\NormalTok{(month }\OperatorTok{-}\StringTok{ }\DecValTok{1}\NormalTok{) }\OperatorTok{/}\StringTok{ }\DecValTok{12}\NormalTok{, rate)) }\OperatorTok{+}
\StringTok{  }\KeywordTok{geom_line}\NormalTok{()}

\KeywordTok{ggplot}\NormalTok{(economics_}\DecValTok{2}\NormalTok{, }\KeywordTok{aes}\NormalTok{(month, rate, }\DataTypeTok{group =}\NormalTok{ year)) }\OperatorTok{+}
\StringTok{  }\KeywordTok{geom_line}\NormalTok{(}\KeywordTok{aes}\NormalTok{(}\DataTypeTok{colour =}\NormalTok{ year), }\DataTypeTok{size =} \DecValTok{1}\NormalTok{)}
\end{Highlighting}
\end{Shaded}

\begin{figure}[H]
  \includegraphics[width=0.5\linewidth]{_figures/tidy_data/ec2-plots-1}%
  \includegraphics[width=0.5\linewidth]{_figures/tidy_data/ec2-plots-2}
\end{figure}

\hypertarget{spread}{%
\subsection{Spread}\label{spread}}

\texttt{spread()} is the opposite of \texttt{gather()}. You use it when
you have a pair of columns that are in indexed form, instead of
Cartesian form. For example, the following example dataset contains
three variables (\texttt{day}, \texttt{rain} and \texttt{temp}), but
\texttt{rain} and \texttt{temp} are stored in indexed form.
\indexf{spread}

\begin{Shaded}
\begin{Highlighting}[]
\NormalTok{weather <-}\StringTok{ }\NormalTok{dplyr}\OperatorTok{::}\KeywordTok{data_frame}\NormalTok{(}
  \DataTypeTok{day =} \KeywordTok{rep}\NormalTok{(}\DecValTok{1}\OperatorTok{:}\DecValTok{3}\NormalTok{, }\DecValTok{2}\NormalTok{),}
  \DataTypeTok{obs =} \KeywordTok{rep}\NormalTok{(}\KeywordTok{c}\NormalTok{(}\StringTok{"temp"}\NormalTok{, }\StringTok{"rain"}\NormalTok{), }\DataTypeTok{each =} \DecValTok{3}\NormalTok{),}
  \DataTypeTok{val =} \KeywordTok{c}\NormalTok{(}\KeywordTok{c}\NormalTok{(}\DecValTok{23}\NormalTok{, }\DecValTok{22}\NormalTok{, }\DecValTok{20}\NormalTok{), }\KeywordTok{c}\NormalTok{(}\DecValTok{0}\NormalTok{, }\DecValTok{0}\NormalTok{, }\DecValTok{5}\NormalTok{))}
\NormalTok{)}
\CommentTok{#> Warning: `data_frame()` is deprecated, use `tibble()`.}
\CommentTok{#> This warning is displayed once per session.}
\NormalTok{weather}
\CommentTok{#> # A tibble: 6 x 3}
\CommentTok{#>     day obs     val}
\CommentTok{#>   <int> <chr> <dbl>}
\CommentTok{#> 1     1 temp     23}
\CommentTok{#> 2     2 temp     22}
\CommentTok{#> 3     3 temp     20}
\CommentTok{#> 4     1 rain      0}
\CommentTok{#> 5     2 rain      0}
\CommentTok{#> 6     3 rain      5}
\end{Highlighting}
\end{Shaded}

Spread allows us to turn this messy indexed form into a tidy Cartesian
form. It shares many of the arguments with \texttt{gather()}. You'll
need to supply the \texttt{data} to translate, as well as the name of
the \texttt{key} column which gives the variable names, and the
\texttt{value} column which contains the cell values. Here the key is
\texttt{obs} and the value is \texttt{val}:

\begin{Shaded}
\begin{Highlighting}[]
\KeywordTok{spread}\NormalTok{(weather, }\DataTypeTok{key =}\NormalTok{ obs, }\DataTypeTok{value =}\NormalTok{ val)}
\CommentTok{#> # A tibble: 3 x 3}
\CommentTok{#>     day  rain  temp}
\CommentTok{#>   <int> <dbl> <dbl>}
\CommentTok{#> 1     1     0    23}
\CommentTok{#> 2     2     0    22}
\CommentTok{#> 3     3     5    20}
\end{Highlighting}
\end{Shaded}

\hypertarget{exercises}{%
\subsection{Exercises}\label{exercises}}

\begin{enumerate}
\def\labelenumi{\arabic{enumi}.}
\item
  How can you translate each of the initial example datasets into the
  other form?
\item
  How can you convert back and forth between the \texttt{economics} and
  \texttt{economics\_long} datasets built into ggplot2?
\item
  Install the EDAWR package from \url{https://github.com/rstudio/EDAWR}.
  Tidy the \texttt{storms}, \texttt{population} and \texttt{tb}
  datasets.
\end{enumerate}

\hypertarget{sec:separate-unite}{%
\section{Separate and unite}\label{sec:separate-unite}}

Spread and gather help when the variables are in the wrong place in the
dataset. Separate and unite help when multiple variables are crammed
into one column, or spread across multiple columns. \indexf{separate}
\indexf{unite}

For example, the following dataset stores some information about the
response to a medical treatment. There are three variables (time,
treatment and value), but time and treatment are jammed in one variable
together:

\begin{Shaded}
\begin{Highlighting}[]
\NormalTok{trt <-}\StringTok{ }\NormalTok{dplyr}\OperatorTok{::}\KeywordTok{data_frame}\NormalTok{(}
  \DataTypeTok{var =} \KeywordTok{paste0}\NormalTok{(}\KeywordTok{rep}\NormalTok{(}\KeywordTok{c}\NormalTok{(}\StringTok{"beg"}\NormalTok{, }\StringTok{"end"}\NormalTok{), }\DataTypeTok{each =} \DecValTok{3}\NormalTok{), }\StringTok{"_"}\NormalTok{, }\KeywordTok{rep}\NormalTok{(}\KeywordTok{c}\NormalTok{(}\StringTok{"a"}\NormalTok{, }\StringTok{"b"}\NormalTok{, }\StringTok{"c"}\NormalTok{))),}
  \DataTypeTok{val =} \KeywordTok{c}\NormalTok{(}\DecValTok{1}\NormalTok{, }\DecValTok{4}\NormalTok{, }\DecValTok{2}\NormalTok{, }\DecValTok{10}\NormalTok{, }\DecValTok{5}\NormalTok{, }\DecValTok{11}\NormalTok{)}
\NormalTok{)}
\NormalTok{trt}
\CommentTok{#> # A tibble: 6 x 2}
\CommentTok{#>   var     val}
\CommentTok{#>   <chr> <dbl>}
\CommentTok{#> 1 beg_a     1}
\CommentTok{#> 2 beg_b     4}
\CommentTok{#> 3 beg_c     2}
\CommentTok{#> 4 end_a    10}
\CommentTok{#> 5 end_b     5}
\CommentTok{#> 6 end_c    11}
\end{Highlighting}
\end{Shaded}

The \texttt{separate()} function makes it easy to tease apart multiple
variables stored in one column. It takes four arguments:

\begin{itemize}
\item
  \texttt{data}: the data frame to modify.
\item
  \texttt{col}: the name of the variable to split into pieces.
\item
  \texttt{into}: a character vector giving the names of the new
  variables.
\item
  \texttt{sep}: a description of how to split the variable apart. This
  can either be a regular expression, e.g. \texttt{\_} to split by
  underscores, or \texttt{{[}\^{}a-z{]}} to split by any non-letter, or
  an integer giving a position.
\end{itemize}

In this case, we want to split by the \texttt{\_} character:

\begin{Shaded}
\begin{Highlighting}[]
\KeywordTok{separate}\NormalTok{(trt, var, }\KeywordTok{c}\NormalTok{(}\StringTok{"time"}\NormalTok{, }\StringTok{"treatment"}\NormalTok{), }\StringTok{"_"}\NormalTok{)}
\CommentTok{#> # A tibble: 6 x 3}
\CommentTok{#>   time  treatment   val}
\CommentTok{#>   <chr> <chr>     <dbl>}
\CommentTok{#> 1 beg   a             1}
\CommentTok{#> 2 beg   b             4}
\CommentTok{#> 3 beg   c             2}
\CommentTok{#> 4 end   a            10}
\CommentTok{#> 5 end   b             5}
\CommentTok{#> 6 end   c            11}
\end{Highlighting}
\end{Shaded}

(If the variables are combined in a more complex form, have a look at
\texttt{extract()}. Alternatively, you might need to create columns
individually yourself using other calculations. A useful tool for this
is \texttt{mutate()} which you'll learn about in the next chapter.)

\texttt{unite()} is the inverse of \texttt{separate()} - it joins
together multiple columns into one column. This is less common, but it's
useful to know about as the inverse of \texttt{separate()}.

\hypertarget{exercises-1}{%
\subsection{Exercises}\label{exercises-1}}

\begin{enumerate}
\def\labelenumi{\arabic{enumi}.}
\item
  Install the EDAWR package from \url{https://github.com/rstudio/EDAWR}.
  Tidy the \texttt{who} dataset.
\item
  Work through the demos included in the tidyr package
  (\texttt{demo(package\ =\ "tidyr")})
\end{enumerate}

\hypertarget{sec:tidy-case-study}{%
\section{Case studies}\label{sec:tidy-case-study}}

For most real datasets, you'll need to use more than one tidying verb.
There many be multiple ways to get there, but as long as each step makes
the data tidier, you'll eventually get to the tidy dataset. That said,
you typically apply the functions in the same order: \texttt{gather()},
\texttt{separate()} and \texttt{spread()} (although you might not use
all three).

\hypertarget{blood-pressure}{%
\subsection{Blood pressure}\label{blood-pressure}}

The first step when tidying a new dataset is always to identify the
variables. Take the following simulated medical data. There are seven
variables in this dataset: name, age, start date, week, systolic \&
diastolic blood pressure. Can you see how they're stored?

\begin{Shaded}
\begin{Highlighting}[]
\CommentTok{# Adapted from example by Barry Rowlingson, }
\CommentTok{# http://barryrowlingson.github.io/hadleyverse/}
\NormalTok{bpd <-}\StringTok{ }\NormalTok{readr}\OperatorTok{::}\KeywordTok{read_table}\NormalTok{(}
\StringTok{"name age      start  week1  week2  week3}
\StringTok{Anne  35 2014-03-27 100/80 100/75 120/90}
\StringTok{ Ben  41 2014-03-09 110/65 100/65 135/70}
\StringTok{Carl  33 2014-04-02 125/80   <NA>   <NA>}
\StringTok{"}\NormalTok{, }\DataTypeTok{na =} \StringTok{"<NA>"}\NormalTok{)}
\end{Highlighting}
\end{Shaded}

The first step is to convert from Cartesian to indexed form:

\begin{Shaded}
\begin{Highlighting}[]
\NormalTok{bpd_}\DecValTok{1}\NormalTok{ <-}\StringTok{ }\KeywordTok{gather}\NormalTok{(bpd, week, bp, week1}\OperatorTok{:}\NormalTok{week3)}
\NormalTok{bpd_}\DecValTok{1}
\CommentTok{#> # A tibble: 9 x 5}
\CommentTok{#>   name    age start      week  bp    }
\CommentTok{#>   <chr> <dbl> <date>     <chr> <chr> }
\CommentTok{#> 1 Anne     35 2014-03-27 week1 100/80}
\CommentTok{#> 2 Ben      41 2014-03-09 week1 110/65}
\CommentTok{#> 3 Carl     33 2014-04-02 week1 125/80}
\CommentTok{#> 4 Anne     35 2014-03-27 week2 100/75}
\CommentTok{#> 5 Ben      41 2014-03-09 week2 100/65}
\CommentTok{#> 6 Carl     33 2014-04-02 week2 <NA>  }
\CommentTok{#> # ... with 3 more rows}
\end{Highlighting}
\end{Shaded}

This is tidier, but we have two variables combined together in the
\texttt{bp} variable. This is a common way of writing down the blood
pressure, but analysis is easier if we break it into two variables.
That's the job of separate:

\begin{Shaded}
\begin{Highlighting}[]
\NormalTok{bpd_}\DecValTok{2}\NormalTok{ <-}\StringTok{ }\KeywordTok{separate}\NormalTok{(bpd_}\DecValTok{1}\NormalTok{, bp, }\KeywordTok{c}\NormalTok{(}\StringTok{"sys"}\NormalTok{, }\StringTok{"dia"}\NormalTok{), }\StringTok{"/"}\NormalTok{)}
\NormalTok{bpd_}\DecValTok{2}
\CommentTok{#> # A tibble: 9 x 6}
\CommentTok{#>   name    age start      week  sys   dia  }
\CommentTok{#>   <chr> <dbl> <date>     <chr> <chr> <chr>}
\CommentTok{#> 1 Anne     35 2014-03-27 week1 100   80   }
\CommentTok{#> 2 Ben      41 2014-03-09 week1 110   65   }
\CommentTok{#> 3 Carl     33 2014-04-02 week1 125   80   }
\CommentTok{#> 4 Anne     35 2014-03-27 week2 100   75   }
\CommentTok{#> 5 Ben      41 2014-03-09 week2 100   65   }
\CommentTok{#> 6 Carl     33 2014-04-02 week2 <NA>  <NA> }
\CommentTok{#> # ... with 3 more rows}
\end{Highlighting}
\end{Shaded}

This dataset is now tidy, but we could do a little more to make it
easier to use. The following code uses \texttt{extract()} to pull the
week number out into its own variable (using regular expressions is
beyond the scope of the book, but
\texttt{\textbackslash{}\textbackslash{}d} stands for any digit). I also
use \texttt{arrange()} (which you'll learn about in the next chapter) to
order the rows to keep the records for each person together.

\begin{Shaded}
\begin{Highlighting}[]
\NormalTok{bpd_}\DecValTok{3}\NormalTok{ <-}\StringTok{ }\KeywordTok{extract}\NormalTok{(bpd_}\DecValTok{2}\NormalTok{, week, }\StringTok{"week"}\NormalTok{, }\StringTok{"(}\CharTok{\textbackslash{}\textbackslash{}}\StringTok{d)"}\NormalTok{, }\DataTypeTok{convert =} \OtherTok{TRUE}\NormalTok{)}
\NormalTok{bpd_}\DecValTok{4}\NormalTok{ <-}\StringTok{ }\NormalTok{dplyr}\OperatorTok{::}\KeywordTok{arrange}\NormalTok{(bpd_}\DecValTok{3}\NormalTok{, name, week)}
\NormalTok{bpd_}\DecValTok{4}
\CommentTok{#> # A tibble: 9 x 6}
\CommentTok{#>   name    age start       week sys   dia  }
\CommentTok{#>   <chr> <dbl> <date>     <int> <chr> <chr>}
\CommentTok{#> 1 Anne     35 2014-03-27     1 100   80   }
\CommentTok{#> 2 Anne     35 2014-03-27     2 100   75   }
\CommentTok{#> 3 Anne     35 2014-03-27     3 120   90   }
\CommentTok{#> 4 Ben      41 2014-03-09     1 110   65   }
\CommentTok{#> 5 Ben      41 2014-03-09     2 100   65   }
\CommentTok{#> 6 Ben      41 2014-03-09     3 135   70   }
\CommentTok{#> # ... with 3 more rows}
\end{Highlighting}
\end{Shaded}

You might notice that there's some repetition in this dataset: if you
know the name, then you also know the age and start date. This reflects
a third condition of tidyness that I don't discuss here: each data frame
should contain one and only one data set. Here there are really two
datasets: information about each person that doesn't change over time,
and their weekly blood pressure measurements. You can learn more about
this sort of messiness in the resources mentioned at the end of the
chapter.

\hypertarget{test-scores}{%
\subsection{Test scores}\label{test-scores}}

Imagine you're interested in the effect of an intervention on test
scores. You've collected the following data. What are the variables?

\begin{Shaded}
\begin{Highlighting}[]
\CommentTok{# Adapted from http://stackoverflow.com/questions/29775461}
\NormalTok{scores <-}\StringTok{ }\NormalTok{dplyr}\OperatorTok{::}\KeywordTok{data_frame}\NormalTok{(}
  \DataTypeTok{person =} \KeywordTok{rep}\NormalTok{(}\KeywordTok{c}\NormalTok{(}\StringTok{"Greg"}\NormalTok{, }\StringTok{"Sally"}\NormalTok{, }\StringTok{"Sue"}\NormalTok{), }\DataTypeTok{each =} \DecValTok{2}\NormalTok{),}
  \DataTypeTok{time   =} \KeywordTok{rep}\NormalTok{(}\KeywordTok{c}\NormalTok{(}\StringTok{"pre"}\NormalTok{, }\StringTok{"post"}\NormalTok{), }\DecValTok{3}\NormalTok{),}
  \DataTypeTok{test1  =} \KeywordTok{round}\NormalTok{(}\KeywordTok{rnorm}\NormalTok{(}\DecValTok{6}\NormalTok{, }\DataTypeTok{mean =} \DecValTok{80}\NormalTok{, }\DataTypeTok{sd =} \DecValTok{4}\NormalTok{), }\DecValTok{0}\NormalTok{),}
  \DataTypeTok{test2  =} \KeywordTok{round}\NormalTok{(}\KeywordTok{jitter}\NormalTok{(test1, }\DecValTok{15}\NormalTok{), }\DecValTok{0}\NormalTok{)}
\NormalTok{)}
\NormalTok{scores}
\CommentTok{#> # A tibble: 6 x 4}
\CommentTok{#>   person time  test1 test2}
\CommentTok{#>   <chr>  <chr> <dbl> <dbl>}
\CommentTok{#> 1 Greg   pre      74    71}
\CommentTok{#> 2 Greg   post     81    80}
\CommentTok{#> 3 Sally  pre      70    69}
\CommentTok{#> 4 Sally  post     80    78}
\CommentTok{#> 5 Sue    pre      82    81}
\CommentTok{#> 6 Sue    post     85    82}
\end{Highlighting}
\end{Shaded}

I think the variables are person, test, pre-test score and post-test
score. As usual, we start by converting columns in Cartesian form
(\texttt{test1} and \texttt{test2}) to indexed form (\texttt{test} and
\texttt{score}):

\begin{Shaded}
\begin{Highlighting}[]
\NormalTok{scores_}\DecValTok{1}\NormalTok{ <-}\StringTok{ }\KeywordTok{gather}\NormalTok{(scores, test, score, test1}\OperatorTok{:}\NormalTok{test2)}
\NormalTok{scores_}\DecValTok{1}
\CommentTok{#> # A tibble: 12 x 4}
\CommentTok{#>   person time  test  score}
\CommentTok{#>   <chr>  <chr> <chr> <dbl>}
\CommentTok{#> 1 Greg   pre   test1    74}
\CommentTok{#> 2 Greg   post  test1    81}
\CommentTok{#> 3 Sally  pre   test1    70}
\CommentTok{#> 4 Sally  post  test1    80}
\CommentTok{#> 5 Sue    pre   test1    82}
\CommentTok{#> 6 Sue    post  test1    85}
\CommentTok{#> # ... with 6 more rows}
\end{Highlighting}
\end{Shaded}

Now we need to do the opposite: \texttt{pre} and \texttt{post} should be
variables, not values, so we need to spread \texttt{time} and
\texttt{score}:

\begin{Shaded}
\begin{Highlighting}[]
\NormalTok{scores_}\DecValTok{2}\NormalTok{ <-}\StringTok{ }\KeywordTok{spread}\NormalTok{(scores_}\DecValTok{1}\NormalTok{, time, score)}
\NormalTok{scores_}\DecValTok{2}
\CommentTok{#> # A tibble: 6 x 4}
\CommentTok{#>   person test   post   pre}
\CommentTok{#>   <chr>  <chr> <dbl> <dbl>}
\CommentTok{#> 1 Greg   test1    81    74}
\CommentTok{#> 2 Greg   test2    80    71}
\CommentTok{#> 3 Sally  test1    80    70}
\CommentTok{#> 4 Sally  test2    78    69}
\CommentTok{#> 5 Sue    test1    85    82}
\CommentTok{#> 6 Sue    test2    82    81}
\end{Highlighting}
\end{Shaded}

A good indication that we have made a tidy dataset is that it's now easy
to calculate the statistic of interest: the difference between pre- and
post-intervention scores:

\begin{Shaded}
\begin{Highlighting}[]
\NormalTok{scores_}\DecValTok{3}\NormalTok{ <-}\StringTok{ }\KeywordTok{mutate}\NormalTok{(scores_}\DecValTok{2}\NormalTok{, }\DataTypeTok{diff =}\NormalTok{ post }\OperatorTok{-}\StringTok{ }\NormalTok{pre)}
\NormalTok{scores_}\DecValTok{3}
\CommentTok{#> # A tibble: 6 x 5}
\CommentTok{#>   person test   post   pre  diff}
\CommentTok{#>   <chr>  <chr> <dbl> <dbl> <dbl>}
\CommentTok{#> 1 Greg   test1    81    74     7}
\CommentTok{#> 2 Greg   test2    80    71     9}
\CommentTok{#> 3 Sally  test1    80    70    10}
\CommentTok{#> 4 Sally  test2    78    69     9}
\CommentTok{#> 5 Sue    test1    85    82     3}
\CommentTok{#> 6 Sue    test2    82    81     1}
\end{Highlighting}
\end{Shaded}

And it's similarly easy to plot:

\begin{Shaded}
\begin{Highlighting}[]
\KeywordTok{ggplot}\NormalTok{(scores_}\DecValTok{3}\NormalTok{, }\KeywordTok{aes}\NormalTok{(person, diff, }\DataTypeTok{color =}\NormalTok{ test)) }\OperatorTok{+}
\StringTok{  }\KeywordTok{geom_hline}\NormalTok{(}\DataTypeTok{size =} \DecValTok{2}\NormalTok{, }\DataTypeTok{colour =} \StringTok{"white"}\NormalTok{, }\DataTypeTok{yintercept =} \DecValTok{0}\NormalTok{) }\OperatorTok{+}
\StringTok{  }\KeywordTok{geom_point}\NormalTok{() }\OperatorTok{+}
\StringTok{  }\KeywordTok{geom_path}\NormalTok{(}\KeywordTok{aes}\NormalTok{(}\DataTypeTok{group =}\NormalTok{ person), }\DataTypeTok{colour =} \StringTok{"grey50"}\NormalTok{, }
    \DataTypeTok{arrow =} \KeywordTok{arrow}\NormalTok{(}\DataTypeTok{length =} \KeywordTok{unit}\NormalTok{(}\FloatTok{0.25}\NormalTok{, }\StringTok{"cm"}\NormalTok{)))}
\end{Highlighting}
\end{Shaded}

\begin{figure}[H]
  \includegraphics[width=0.5\linewidth]{_figures/tidy_data/scores4-1}
\end{figure}

(Again, you'll learn about \texttt{mutate()} in the next chapter.)

\hypertarget{learning-more}{%
\section{Learning more}\label{learning-more}}

Data tidying is a big topic and this chapter only scratches the surface.
I recommend the following references which go into considerably more
depth on this topic:

\begin{itemize}
\item
  The tidyr documentation. I've described the most important arguments,
  but most functions have other arguments that help deal with less
  common situations. If you're struggling, make sure to read the
  documentation to see if there's an argument that might help you.
\item
  ``\href{http://www.jstatsoft.org/v59/i10/}{Tidy data}'', an article in
  the \emph{Journal of Statistical Software}. It describes the ideas of
  tidy data in more depth and shows other types of messy data.
  Unfortunately the paper was written before tidyr existed, so to see
  how to use tidyr instead of reshape2, consult the
  \href{http://cran.r-project.org/web/packages/tidyr/vignettes/tidy-data.html}{tidyr
  vignette}.
\item
  The \href{http://rstudio.com/cheatsheets}{data wrangling cheatsheet}
  by RStudio, includes the most common tidyr verbs in a form designed to
  jog your memory when you're stuck.
\end{itemize}

\hypertarget{references}{%
\section*{References}\label{references}}
\addcontentsline{toc}{section}{References}

\hypertarget{refs}{}
\leavevmode\hypertarget{ref-wickham:2007b}{}%
Wickham, Hadley. 2007. ``Reshaping Data with the Reshape Package.''
\emph{Journal of Statistical Software} 21 (12).
\url{http://www.jstatsoft.org/v21/i12/paper}.

\leavevmode\hypertarget{ref-tidy-data}{}%
---------. 2014. ``Tidy Data.'' \emph{The Journal of Statistical
Software} 59. \url{http://www.jstatsoft.org/v59/i10/}.

\chapter{Data transformation}\label{cha:dplyr}

\section{Introduction}

Tidy data is important, but it's not the end of the road. Often you
won't have quite the right variables, or your data might need a little
aggregation before you visualise it. This chapter will show you how to
solve these problems (and more!) with the \textbf{dplyr} package.
\index{Data!manipulating} \index{dplyr}
\index{Grammar!of data manipulation}

The goal of dplyr is to provide verbs (functions) that help you solve
the most common 95\% of data manipulation problems. dplyr is similar to
ggplot2, but instead of providing a grammar of graphics, it provides a
grammar of data manipulation. Like ggplot2, dplyr helps you not just by
giving you functions, but it also helps you think about data
manipulation. In particular, dplyr helps by constraining you: instead of
struggling to think about which of the thousands of functions that might
help, you can just pick from a handful that are design to be very likely
to be helpful. In this chapter you'll learn four of the most important
dplyr verbs:

\begin{itemize}
\tightlist
\item
  \texttt{filter()}
\item
  \texttt{mutate()}
\item
  \texttt{group\_by()} \& \texttt{summarise()}
\end{itemize}

These verbs are easy to learn because they all work the same way: they
take a data frame as the first argument, and return a modified data
frame. The other arguments control the details of the transformation,
and are always interpreted in the context of the data frame so you can
refer to variables directly. I'll also explain each in the same way:
I'll show you a motivating example using the \texttt{diamonds} data,
give you more details about how the function works, and finish up with
some exercises for you to practice your skills with.

You'll also learn how to create data transformation pipelines using
\texttt{\%\textgreater{}\%}. \texttt{\%\textgreater{}\%} plays a similar
role to \texttt{+} in ggplot2: it allows you to solve complex problems
by combining small pieces that are easily understood in isolation.

This chapter only scratches the surface of dplyr's capabilities but it
should be enough to help you with visualisation problems. You can learn
more by using the resources discussed at the end of the chapter.

\section{Filter observations}

It's common to only want to explore one part of a dataset. A great data
analysis strategy is to start with just one observation unit (one
person, one city, etc), and understand how it works before attempting to
generalise the conclusion to others. This is a great technique if you
ever feel overwhelmed by an analysis: zoom down to a small subset,
master it, and then zoom back out, to apply your conclusions to the full
dataset. \indexf{filter}

Filtering is also useful for extracting outliers. Generally, you don't
want to just throw outliers away, as they're often highly revealing, but
it's useful to think about partitioning the data into the common and the
unusual. You summarise the common to look at the broad trends; you
examine the outliers individually to see if you can figure out what's
going on.

For example, look at this plot that shows how the x and y dimensions of
the diamonds are related:

\begin{Shaded}
\begin{Highlighting}[]
\KeywordTok{ggplot}\NormalTok{(diamonds, }\KeywordTok{aes}\NormalTok{(x, y)) +}\StringTok{ }
\StringTok{  }\KeywordTok{geom_bin2d}\NormalTok{()}
\end{Highlighting}
\end{Shaded}

\begin{figure}[H]
  \centering
  \includegraphics[width=0.65\linewidth]{_figures/data-manip/diamonds-x-y-1}
\end{figure}

There are around 50,000 points in this dataset: most of them lie along
the diagonal, but there are a handful of outliers. One clear set of
incorrect values are those diamonds with zero dimensions. We can use
\texttt{filter()} to pull them out:

\begin{Shaded}
\begin{Highlighting}[]
\KeywordTok{filter}\NormalTok{(diamonds, x ==}\StringTok{ }\DecValTok{0} \NormalTok{|}\StringTok{ }\NormalTok{y ==}\StringTok{ }\DecValTok{0}\NormalTok{)}
\CommentTok{#> Source: local data frame [8 x 10]}
\CommentTok{#> }
\CommentTok{#>    carat       cut  color clarity depth table price     x     y}
\CommentTok{#>    (dbl)    (fctr) (fctr)  (fctr) (dbl) (dbl) (int) (dbl) (dbl)}
\CommentTok{#> 1   1.07     Ideal      F     SI2  61.6    56  4954     0  6.62}
\CommentTok{#> 2   1.00 Very Good      H     VS2  63.3    53  5139     0  0.00}
\CommentTok{#> 3   1.14      Fair      G     VS1  57.5    67  6381     0  0.00}
\CommentTok{#> 4   1.56     Ideal      G     VS2  62.2    54 12800     0  0.00}
\CommentTok{#> 5   1.20   Premium      D    VVS1  62.1    59 15686     0  0.00}
\CommentTok{#> 6   2.25   Premium      H     SI2  62.8    59 18034     0  0.00}
\CommentTok{#> ..   ...       ...    ...     ...   ...   ...   ...   ...   ...}
\CommentTok{#>        z}
\CommentTok{#>    (dbl)}
\CommentTok{#> 1      0}
\CommentTok{#> 2      0}
\CommentTok{#> 3      0}
\CommentTok{#> 4      0}
\CommentTok{#> 5      0}
\CommentTok{#> 6      0}
\CommentTok{#> ..   ...}
\end{Highlighting}
\end{Shaded}

This is equivalent to the base R code
\texttt{diamonds{[}diamonds\$x\ ==\ 0\ \textbar{}\ diamonds\$y\ ==\ 0,\ {]}},
but is more concise because \texttt{filter()} knows to look for the bare
\texttt{x} in the data frame.

(If you've used \texttt{subset()} before, you'll notice that it has very
similar behaviour. The biggest difference is that \texttt{subset()} can
select both observations and variables, where in dplyr,
\texttt{filter()} works exclusively with observations and
\texttt{select()} with variables. There are some other subtle
differences, but the main advantage to using \texttt{filter()} is that
it behaves identically to the other dplyr verbs and it tends to be a bit
faster than \texttt{subset()}.)

In a real analysis, you'd look at the outliers in more detail to see if
you can find the root cause of the data quality problem. In this case,
we're just going to throw them out and focus on what remains. To save
some typing, we may provide multiple arguments to \texttt{filter()}
which combines them.

\begin{Shaded}
\begin{Highlighting}[]
\NormalTok{diamonds_ok <-}\StringTok{ }\KeywordTok{filter}\NormalTok{(diamonds, x >}\StringTok{ }\DecValTok{0}\NormalTok{, y >}\StringTok{ }\DecValTok{0}\NormalTok{, y <}\StringTok{ }\DecValTok{20}\NormalTok{)}
\KeywordTok{ggplot}\NormalTok{(diamonds_ok, }\KeywordTok{aes}\NormalTok{(x, y)) +}
\StringTok{  }\KeywordTok{geom_bin2d}\NormalTok{() +}
\StringTok{  }\KeywordTok{geom_abline}\NormalTok{(}\DataTypeTok{slope =} \DecValTok{1}\NormalTok{, }\DataTypeTok{colour =} \StringTok{"white"}\NormalTok{, }\DataTypeTok{size =} \DecValTok{1}\NormalTok{, }\DataTypeTok{alpha =} \FloatTok{0.5}\NormalTok{)}
\end{Highlighting}
\end{Shaded}

\begin{figure}[H]
  \centering
  \includegraphics[width=0.65\linewidth]{_figures/data-manip/diamonds-ok-1}
\end{figure}

This plot is now more informative - we can see a very strong
relationship between \texttt{x} and \texttt{y}. I've added the reference
line to make it clear that for most diamonds, \texttt{x} and \texttt{y}
are very similar. However, this plot still has problems:

\begin{itemize}
\item
  The plot is mostly empty, because most of the data lies along the
  diagonal.
\item
  There are some clear bivariate outliers, but it's hard to select them
  with a simple filter.
\end{itemize}

We'll solve both of these problem in the next section by adding a new
variable that's a transformation of x and y. But before we continue on
to that, let's talk more about the details of \texttt{filter()}.

\subsection{Useful tools}

The first argument to \texttt{filter()} is a data frame. The second and
subsequent arguments must be logical vectors: \texttt{filter()} selects
every row where all the logical expressions are \texttt{TRUE}. The
logical vectors must always be the same length as the data frame: if
not, you'll get an error. Typically you create the logical vector with
the comparison operators:

\begin{itemize}
\tightlist
\item
  \texttt{x\ ==\ y}: x and y are equal.
\item
  \texttt{x\ !=\ y}: x and y are not equal.
\item
  \texttt{x\ \%in\%\ c("a",\ "b",\ "c")}: \texttt{x} is one of the
  values in the right hand side.
\item
  \texttt{x\ \textgreater{}\ y}, \texttt{x\ \textgreater{}=\ y},
  \texttt{x\ \textless{}\ y}, \texttt{x\ \textless{}=\ y}: greater than,
  greater than or equal to, less than, less than or equal to.
\end{itemize}

And combine them with logical operators:

\begin{itemize}
\tightlist
\item
  \texttt{!x} (pronounced ``not x''), flips \texttt{TRUE} and
  \texttt{FALSE} so it keeps all the values where \texttt{x} is
  \texttt{FALSE}.
\item
  \texttt{x\ \&\ y}: \texttt{TRUE} if both \texttt{x} and \texttt{y} are
  \texttt{TRUE}.
\item
  \texttt{x\ \textbar{}\ y}: \texttt{TRUE} if either \texttt{x} or
  \texttt{y} (or both) are \texttt{TRUE}.
\item
  \texttt{xor(x,\ y)}: \texttt{TRUE} if either \texttt{x} or \texttt{y}
  are \texttt{TRUE}, but not both (exclusive or).
\end{itemize}

Most real queries involve some combination of both:

\begin{itemize}
\tightlist
\item
  Price less than \$500: \texttt{price\ \textless{}\ 500}
\item
  Size between 1 and 2 carats:
  \texttt{carat\ \textgreater{}=\ 1\ \&\ carat\ \textless{}\ 2}
\item
  Cut is ideal or premium:
  \texttt{cut\ ==\ "Premium"\ \textbar{}\ cut\ ==\ "Ideal"}, or
  \texttt{cut\ \%in\%\ c("Premium",\ "Ideal")} (note that R is case
  sensitive)
\item
  Worst colour, cut and clarity:
  \texttt{cut\ ==\ "Fair"\ \&\ color\ ==\ "J"\ \&\ clarity\ ==\ "SI2"}
\end{itemize}

You can also use functions in the filtering expression:

\begin{itemize}
\tightlist
\item
  Size is between 1 and 2 carats: \texttt{floor(carat)\ ==\ 1}
\item
  An average dimension greater than 3:
  \texttt{(x\ +\ y\ +\ z)\ /\ 3\ \textgreater{}\ 3}
\end{itemize}

This is useful for simple expressions, but as things get more
complicated it's better to create a new variable first so you can check
that you've done the computation correctly before doing the subsetting.
You'll learn how to do that in the next section.

The rules for \texttt{NA} are a bit trickier, so I'll explain them next.

\subsection{Missing values}

\texttt{NA}, R's missing value indicator, can be frustrating to work
with. R's underlying philosophy is to force you to recognise that you
have missing values, and make a deliberate choice to deal with them:
missing values never silently go missing. This is a pain because you
almost always want to just get rid of them, but it's a good principle to
force you to think about the correct option. \indexc{NA}
\index{Missing values}

The most important thing to understand about missing values is that they
are infectious: with few exceptions, the result of any operation that
includes a missing value will be a missing value. This happens because
\texttt{NA} represents an unknown value, and there are few operations
that turn an unknown value into a known value.

\begin{Shaded}
\begin{Highlighting}[]
\NormalTok{x <-}\StringTok{ }\KeywordTok{c}\NormalTok{(}\DecValTok{1}\NormalTok{, }\OtherTok{NA}\NormalTok{, }\DecValTok{2}\NormalTok{)}
\NormalTok{x ==}\StringTok{ }\DecValTok{1}
\CommentTok{#> [1]  TRUE    NA FALSE}
\NormalTok{x >}\StringTok{ }\DecValTok{2}
\CommentTok{#> [1] FALSE    NA FALSE}
\NormalTok{x +}\StringTok{ }\DecValTok{10}
\CommentTok{#> [1] 11 NA 12}
\end{Highlighting}
\end{Shaded}

When you first learn R, you might be tempted to find missing values
using \texttt{==}:

\begin{Shaded}
\begin{Highlighting}[]
\NormalTok{x ==}\StringTok{ }\OtherTok{NA}
\CommentTok{#> [1] NA NA NA}
\NormalTok{x !=}\StringTok{ }\OtherTok{NA}
\CommentTok{#> [1] NA NA NA}
\end{Highlighting}
\end{Shaded}

But that doesn't work! A little thought reveals why: there's no reason
why two unknown values should be the same. Instead, use
\texttt{is.na(X)} to determine if a value is missing: \indexf{is.na}

\begin{Shaded}
\begin{Highlighting}[]
\KeywordTok{is.na}\NormalTok{(x)}
\CommentTok{#> [1] FALSE  TRUE FALSE}
\end{Highlighting}
\end{Shaded}

\texttt{filter()} only includes observations where all arguments are
\texttt{TRUE}, so \texttt{NA} values are automatically dropped. If you
want to include missing values, be explicit:
\texttt{x\ \textgreater{}\ 10\ \textbar{}\ is.na(x)}. In other parts of
R, you'll sometimes need to convert missing values into \texttt{FALSE}.
You can do that with \texttt{x\ \textgreater{}\ 10\ \&\ !is.na(x)}

\subsection{Exercises}

\begin{enumerate}
\def\labelenumi{\arabic{enumi}.}
\item
  Practice your filtering skills by:

  \begin{itemize}
  \tightlist
  \item
    Finding all the diamonds with equal x and y dimensions.
  \item
    A depth between 55 and 70.
  \item
    A carat smaller than the median carat.
  \item
    Cost more than \$10,000 per carat
  \item
    Are of good or better quality
  \end{itemize}
\item
  Fill in the question marks in this table:

  \begin{longtable}[c]{@{}llll@{}}
  \toprule
  Expression & \texttt{TRUE} & \texttt{FALSE} &
  \texttt{NA}\tabularnewline
  \midrule
  \endhead
  \texttt{x} & • &\tabularnewline
  ? & & •\tabularnewline
  \texttt{is.na(x)} & & & •\tabularnewline
  \texttt{!is.na(x)} & ? & ? & ?\tabularnewline
  ? & • & & •\tabularnewline
  ? & & • & •\tabularnewline
  \bottomrule
  \end{longtable}
\item
  Repeat the analysis of outlying values with \texttt{z}. Compared to
  \texttt{x} and \texttt{y}, how would you characterise the relationship
  of \texttt{x} and \texttt{z}, or \texttt{y} and \texttt{z}?
\item
  Install the \textbf{ggplot2movies} package and look at the movies that
  have a missing budget. How are they different from the movies with a
  budget? (Hint: try a frequency polygon plus
  \texttt{colour\ =\ is.na(budget)}.)
\item
  What is \texttt{NA\ \&\ FALSE} and \texttt{NA\ \textbar{}\ TRUE}? Why?
  Why doesn't \texttt{NA\ *\ 0} equal zero? What number times zero does
  not equal 0? What do you expect \texttt{NA\ \^{}\ 0} to equal? Why?
\end{enumerate}

\section{Create new variables}\label{mutate}

To better explore the relationship between \texttt{x} and \texttt{y},
it's useful to ``rotate'' the plot so that the data is flat, not
diagonal. We can do that by creating two new variables: one that
represents the difference between \texttt{x} and \texttt{y} (which in
this context represents the symmetry of the diamond) and one that
represents its size (the length of the diagonal). \indexf{mutate}
\index{Data!creating new variables}

To create new variables use \texttt{mutate()}. Like \texttt{filter()} it
takes a data frame as its first argument and returns a data frame. Its
second and subsequent arguments are named expressions that generate new
variables. Like \texttt{filter()} you can refer to variables just by
their name, you don't need to also include the name of the dataset.

\begin{Shaded}
\begin{Highlighting}[]
\NormalTok{diamonds_ok2 <-}\StringTok{ }\KeywordTok{mutate}\NormalTok{(diamonds_ok,}
  \DataTypeTok{sym =} \NormalTok{x -}\StringTok{ }\NormalTok{y,}
  \DataTypeTok{size =} \KeywordTok{sqrt}\NormalTok{(x ^}\StringTok{ }\DecValTok{2} \NormalTok{+}\StringTok{ }\NormalTok{y ^}\StringTok{ }\DecValTok{2}\NormalTok{)}
\NormalTok{)}
\NormalTok{diamonds_ok2}
\CommentTok{#> Source: local data frame [53,930 x 12]}
\CommentTok{#> }
\CommentTok{#>    carat       cut  color clarity depth table price     x     y}
\CommentTok{#>    (dbl)    (fctr) (fctr)  (fctr) (dbl) (dbl) (int) (dbl) (dbl)}
\CommentTok{#> 1   0.23     Ideal      E     SI2  61.5    55   326  3.95  3.98}
\CommentTok{#> 2   0.21   Premium      E     SI1  59.8    61   326  3.89  3.84}
\CommentTok{#> 3   0.23      Good      E     VS1  56.9    65   327  4.05  4.07}
\CommentTok{#> 4   0.29   Premium      I     VS2  62.4    58   334  4.20  4.23}
\CommentTok{#> 5   0.31      Good      J     SI2  63.3    58   335  4.34  4.35}
\CommentTok{#> 6   0.24 Very Good      J    VVS2  62.8    57   336  3.94  3.96}
\CommentTok{#> ..   ...       ...    ...     ...   ...   ...   ...   ...   ...}
\CommentTok{#>        z}
\CommentTok{#>    (dbl)}
\CommentTok{#> 1   2.43}
\CommentTok{#> 2   2.31}
\CommentTok{#> 3   2.31}
\CommentTok{#> 4   2.63}
\CommentTok{#> 5   2.75}
\CommentTok{#> 6   2.48}
\CommentTok{#> ..   ...}
\CommentTok{#> Variables not shown: sym (dbl), size (dbl)}

\KeywordTok{ggplot}\NormalTok{(diamonds_ok2, }\KeywordTok{aes}\NormalTok{(size, sym)) +}\StringTok{ }
\StringTok{  }\KeywordTok{stat_bin2d}\NormalTok{()}
\end{Highlighting}
\end{Shaded}

\begin{figure}[H]
  \centering
  \includegraphics[width=0.65\linewidth]{_figures/data-manip/mutate1-1}
\end{figure}

This plot has two advantages: we can more easily see the pattern
followed by most diamonds, and we can easily select outliers. Here, it
doesn't seem important whether the outliers are positive (i.e.
\texttt{x} is bigger than \texttt{y}) or negative (i.e. \texttt{y} is
bigger \texttt{x}). So we can use the absolute value of the symmetry
variable to pull out the outliers. Based on the plot, and a little
experimentation, I came up with a threshold of 0.20. We'll check out the
results with a histogram.

\begin{Shaded}
\begin{Highlighting}[]
\KeywordTok{ggplot}\NormalTok{(diamonds_ok2, }\KeywordTok{aes}\NormalTok{(}\KeywordTok{abs}\NormalTok{(sym))) +}\StringTok{ }
\StringTok{  }\KeywordTok{geom_histogram}\NormalTok{(}\DataTypeTok{binwidth =} \FloatTok{0.10}\NormalTok{)}

\NormalTok{diamonds_ok3 <-}\StringTok{ }\KeywordTok{filter}\NormalTok{(diamonds_ok2, }\KeywordTok{abs}\NormalTok{(sym) <}\StringTok{ }\FloatTok{0.20}\NormalTok{)}
\KeywordTok{ggplot}\NormalTok{(diamonds_ok3, }\KeywordTok{aes}\NormalTok{(}\KeywordTok{abs}\NormalTok{(sym))) +}\StringTok{ }
\StringTok{  }\KeywordTok{geom_histogram}\NormalTok{(}\DataTypeTok{binwidth =} \FloatTok{0.01}\NormalTok{)}
\end{Highlighting}
\end{Shaded}

\begin{figure}[H]
  \includegraphics[width=0.5\linewidth]{_figures/data-manip/sym-hist-1}%
  \includegraphics[width=0.5\linewidth]{_figures/data-manip/sym-hist-2}
\end{figure}

That's an interesting histogram! While most diamonds are close to being
symmetric there are very few that are perfectly symmetric (i.e.
\texttt{x\ ==\ y}).

\subsection{Useful tools}

Typically, transformations will be suggested by your domain knowledge.
However, there are a few transformations that are useful in a
surprisingly wide range of circumstances.

\begin{itemize}
\item
  Log-transformations are often useful. They turn multiplicative
  relationships into additive relationships; they compress data that
  varies over orders of magnitude; they convert power relationships to
  linear relationship. See examples at
  \url{http://stats.stackexchange.com/questions/27951}
\item
  Relative difference: If you're interested in the relative difference
  between two variables, use \texttt{log(x\ /\ y)}. It's better than
  \texttt{x\ /\ y} because it's symmetric: if x \textless{} y,
  \texttt{x\ /\ y} takes values {[}0, 1), but if x \textgreater{} y,
  \texttt{x\ /\ y} takes values (1, Inf). See Törnqvist, Vartia, and
  Vartia (1985) for more details. \indexf{log}
\item
  Sometimes integrating or differentiating might make the data more
  interpretable: if you have distance and time, would speed or
  acceleration be more useful? (or vice versa). (Note that integration
  makes data more smooth; differentiation makes it less smooth.)
\item
  Partition a number into magnitude and direction with \texttt{abs(x)}
  and \texttt{sign(x)}.
\end{itemize}

There are also a few useful ways to transform pairs of variables:

\begin{itemize}
\item
  Partitioning into overall size and difference is often useful, as seen
  above.
\item
  If you see a strong trend, use a model to partition it into pattern
  and residuals is often useful. You'll learn more about that in the
  next chapter.
\item
  Sometimes it's useful to change positions to polar coordinates (or
  vice versa): distance (\texttt{sqrt(x\^{}2\ +\ y\^{}2)}) and angle
  (\texttt{atan2(y,\ x)}).
\end{itemize}

\subsection{Exercises}

\begin{enumerate}
\def\labelenumi{\arabic{enumi}.}
\item
  Practice your variable creation skills by creating the following new
  variables:

  \begin{itemize}
  \tightlist
  \item
    The approximate volume of the diamond (using x, y, and z).
  \item
    The approximate density of the diamond.
  \item
    The price per carat.
  \item
    Log transformation of carat and price.
  \end{itemize}
\item
  How can you improve the data density of
  \texttt{ggplot(diamonds,\ aes(x,\ z))\ +\ stat\_bin2d()}. What
  transformation makes it easier to extract outliers?
\item
  The depth variable is just the width of the diamond (average of
  \texttt{x} and \texttt{y}) divided by its height (\texttt{z})
  multiplied by 100 and round to the nearest integer. Compute the depth
  yourself and compare it to the existing depth variable. Summarise your
  findings with a plot.
\item
  Compare the distribution of symmetry for diamonds with \(x > y\) vs.
  \(x < y\).
\end{enumerate}

\section{Group-wise summaries}\label{sec:summarise}

Many insightful visualisations require that you reduce the full dataset
down to a meaningful summary. ggplot2 provides a number of geoms that
will do summaries for you. But it's often useful to do summaries by
hand: that gives you more flexibility and you can use the summaries for
other purposes. \indexf{group\_by} \indexf{summarise}

dplyr does summaries in two steps:

\begin{enumerate}
\def\labelenumi{\arabic{enumi}.}
\tightlist
\item
  Define the grouping variables with \texttt{group\_by()}.
\item
  Describe how to summarise each group with a single row with
  \texttt{summarise()}
\end{enumerate}

For example, to look at the average price per clarity, we first group by
clarity, then summarise:

\begin{Shaded}
\begin{Highlighting}[]
\NormalTok{by_clarity <-}\StringTok{ }\KeywordTok{group_by}\NormalTok{(diamonds, clarity)}
\NormalTok{sum_clarity <-}\StringTok{ }\KeywordTok{summarise}\NormalTok{(by_clarity, }\DataTypeTok{price =} \KeywordTok{mean}\NormalTok{(price))}
\NormalTok{sum_clarity}
\CommentTok{#> Source: local data frame [8 x 2]}
\CommentTok{#> }
\CommentTok{#>    clarity price}
\CommentTok{#>     (fctr) (dbl)}
\CommentTok{#> 1       I1  3924}
\CommentTok{#> 2      SI2  5063}
\CommentTok{#> 3      SI1  3996}
\CommentTok{#> 4      VS2  3925}
\CommentTok{#> 5      VS1  3839}
\CommentTok{#> 6     VVS2  3284}
\CommentTok{#> ..     ...   ...}

\KeywordTok{ggplot}\NormalTok{(sum_clarity, }\KeywordTok{aes}\NormalTok{(clarity, price)) +}\StringTok{ }
\StringTok{  }\KeywordTok{geom_line}\NormalTok{(}\KeywordTok{aes}\NormalTok{(}\DataTypeTok{group =} \DecValTok{1}\NormalTok{), }\DataTypeTok{colour =} \StringTok{"grey80"}\NormalTok{) +}
\StringTok{  }\KeywordTok{geom_point}\NormalTok{(}\DataTypeTok{size =} \DecValTok{2}\NormalTok{)}
\end{Highlighting}
\end{Shaded}

\begin{figure}[H]
  \centering
  \includegraphics[width=0.65\linewidth]{_figures/data-manip/price-by-clarity-1}
\end{figure}

You might be surprised by this pattern: why do diamonds with better
clarity have lower prices? We'll see why this is the case and what to do
about it in \hyperref[sub:trend]{removing trend}.

Supply additional variables to \texttt{group\_by()} to create groups
based on more than one variable. The next example shows how we can
compute (by hand) a frequency polygon that shows how cut and depth
interact. The special summary function \texttt{n()} counts the number of
observations in each group.

\begin{Shaded}
\begin{Highlighting}[]
\NormalTok{cut_depth <-}\StringTok{ }\KeywordTok{summarise}\NormalTok{(}\KeywordTok{group_by}\NormalTok{(diamonds, cut, depth), }\DataTypeTok{n =} \KeywordTok{n}\NormalTok{())}
\NormalTok{cut_depth <-}\StringTok{ }\KeywordTok{filter}\NormalTok{(cut_depth, depth >}\StringTok{ }\DecValTok{55}\NormalTok{, depth <}\StringTok{ }\DecValTok{70}\NormalTok{)}
\NormalTok{cut_depth}
\CommentTok{#> Source: local data frame [455 x 3]}
\CommentTok{#> Groups: cut [5]}
\CommentTok{#> }
\CommentTok{#>       cut depth     n}
\CommentTok{#>    (fctr) (dbl) (int)}
\CommentTok{#> 1    Fair  55.1     3}
\CommentTok{#> 2    Fair  55.2     6}
\CommentTok{#> 3    Fair  55.3     5}
\CommentTok{#> 4    Fair  55.4     2}
\CommentTok{#> 5    Fair  55.5     3}
\CommentTok{#> 6    Fair  55.6     4}
\CommentTok{#> ..    ...   ...   ...}

\KeywordTok{ggplot}\NormalTok{(cut_depth, }\KeywordTok{aes}\NormalTok{(depth, n, }\DataTypeTok{colour =} \NormalTok{cut)) +}\StringTok{ }
\StringTok{  }\KeywordTok{geom_line}\NormalTok{()}
\end{Highlighting}
\end{Shaded}

\begin{figure}[H]
  \centering
  \includegraphics[width=0.65\linewidth]{_figures/data-manip/freqpoly-by-hand-1}
\end{figure}

We can use a grouped \texttt{mutate()} to convert counts to proportions,
so it's easier to compare across the cuts. \texttt{summarise()} strips
one level of grouping off, so \texttt{cut\_depth} will be grouped by
cut.

\begin{Shaded}
\begin{Highlighting}[]
\NormalTok{cut_depth <-}\StringTok{ }\KeywordTok{mutate}\NormalTok{(cut_depth, }\DataTypeTok{prop =} \NormalTok{n /}\StringTok{ }\KeywordTok{sum}\NormalTok{(n))}
\KeywordTok{ggplot}\NormalTok{(cut_depth, }\KeywordTok{aes}\NormalTok{(depth, prop, }\DataTypeTok{colour =} \NormalTok{cut)) +}\StringTok{ }
\StringTok{  }\KeywordTok{geom_line}\NormalTok{()}
\end{Highlighting}
\end{Shaded}

\begin{figure}[H]
  \centering
  \includegraphics[width=0.65\linewidth]{_figures/data-manip/freqpoly-scaled-1}
\end{figure}

\subsection{Useful tools}

\texttt{summarise()} needs to be used with functions that take a vector
of \(n\) values and always return a single value. Those functions
include:

\begin{itemize}
\tightlist
\item
  Counts: \texttt{n()}, \texttt{n\_distinct(x)}.
\item
  Middle: \texttt{mean(x)}, \texttt{median(x)}.
\item
  Spread: \texttt{sd(x)}, \texttt{mad(x)}, \texttt{IQR(x)}.
\item
  Extremes: \texttt{quartile(x)}, \texttt{min(x)}, \texttt{max(x)}.
\item
  Positions: \texttt{first(x)}, \texttt{last(x)}, \texttt{nth(x,\ 2)}.
\end{itemize}

Another extremely useful technique is to use \texttt{sum()} or
\texttt{mean()} with a logical vector. When a logical vector is treated
as numeric, \texttt{TRUE} becomes 1 and \texttt{FALSE} becomes 0. This
means that \texttt{sum()} tells you the number of \texttt{TRUE}s, and
\texttt{mean()} tells you the proportion of \texttt{TRUE}s. For example,
the following code counts the number of diamonds with carat greater than
or equal to 4, and the proportion of diamonds that cost less than
\$1000.

\begin{Shaded}
\begin{Highlighting}[]
\KeywordTok{summarise}\NormalTok{(diamonds, }
  \DataTypeTok{n_big =} \KeywordTok{sum}\NormalTok{(carat >=}\StringTok{ }\DecValTok{4}\NormalTok{), }
  \DataTypeTok{prop_cheap =} \KeywordTok{mean}\NormalTok{(price <}\StringTok{ }\DecValTok{1000}\NormalTok{)}
\NormalTok{)}
\CommentTok{#> Source: local data frame [1 x 2]}
\CommentTok{#> }
\CommentTok{#>   n_big prop_cheap}
\CommentTok{#>   (int)      (dbl)}
\CommentTok{#> 1     6      0.269}
\end{Highlighting}
\end{Shaded}

Most summary functions have a \texttt{na.rm} argument:
\texttt{na.rm\ =\ TRUE} tells the summary function to remove any missing
values prior to summiarisation. This is a convenient shortcut: rather
than removing the missing values then summarising, you can do it in one
step.

\subsection{Statistical considerations}

When summarising with the mean or median, it's always a good idea to
include a count and a measure of spread. This helps you calibrate your
assessments - if you don't include them you're likely to think that the
data is less variable than it really is, and potentially draw
unwarranted conclusions.

The following example extends our previous summary of the average price
by clarity to also include the number of observations in each group, and
the upper and lower quartiles. It suggests the mean might be a bad
summary for this data - the distributions of price are so highly skewed
that the mean is higher than the upper quartile for some of the groups!

\begin{Shaded}
\begin{Highlighting}[]
\NormalTok{by_clarity <-}\StringTok{ }\NormalTok{diamonds %>%}
\StringTok{  }\KeywordTok{group_by}\NormalTok{(clarity) %>%}
\StringTok{  }\KeywordTok{summarise}\NormalTok{(}
    \DataTypeTok{n =} \KeywordTok{n}\NormalTok{(), }
    \DataTypeTok{mean =} \KeywordTok{mean}\NormalTok{(price), }
    \DataTypeTok{lq =} \KeywordTok{quantile}\NormalTok{(price, }\FloatTok{0.25}\NormalTok{), }
    \DataTypeTok{uq =} \KeywordTok{quantile}\NormalTok{(price, }\FloatTok{0.75}\NormalTok{)}
  \NormalTok{)}
\NormalTok{by_clarity}
\CommentTok{#> Source: local data frame [8 x 5]}
\CommentTok{#> }
\CommentTok{#>    clarity     n  mean    lq    uq}
\CommentTok{#>     (fctr) (int) (dbl) (dbl) (dbl)}
\CommentTok{#> 1       I1   741  3924  2080  5161}
\CommentTok{#> 2      SI2  9194  5063  2264  5777}
\CommentTok{#> 3      SI1 13065  3996  1089  5250}
\CommentTok{#> 4      VS2 12258  3925   900  6024}
\CommentTok{#> 5      VS1  8171  3839   876  6023}
\CommentTok{#> 6     VVS2  5066  3284   794  3638}
\CommentTok{#> ..     ...   ...   ...   ...   ...}
\KeywordTok{ggplot}\NormalTok{(by_clarity, }\KeywordTok{aes}\NormalTok{(clarity, mean)) +}\StringTok{ }
\StringTok{  }\KeywordTok{geom_linerange}\NormalTok{(}\KeywordTok{aes}\NormalTok{(}\DataTypeTok{ymin =} \NormalTok{lq, }\DataTypeTok{ymax =} \NormalTok{uq)) +}\StringTok{ }
\StringTok{  }\KeywordTok{geom_line}\NormalTok{(}\KeywordTok{aes}\NormalTok{(}\DataTypeTok{group =} \DecValTok{1}\NormalTok{), }\DataTypeTok{colour =} \StringTok{"grey50"}\NormalTok{) +}
\StringTok{  }\KeywordTok{geom_point}\NormalTok{(}\KeywordTok{aes}\NormalTok{(}\DataTypeTok{size =} \NormalTok{n))}
\end{Highlighting}
\end{Shaded}

\begin{figure}[H]
  \centering
  \includegraphics[width=0.65\linewidth]{_figures/data-manip/unnamed-chunk-2-1}
\end{figure}

Another example of this comes from baseball. Let's take the MLB batting
data from the Lahman package and calculate the batting average: the
number of hits divided by the number of at bats. Who's the best batter
according to this metric?

\begin{Shaded}
\begin{Highlighting}[]
\KeywordTok{data}\NormalTok{(Batting, }\DataTypeTok{package =} \StringTok{"Lahman"}\NormalTok{)}
\NormalTok{batters <-}\StringTok{ }\KeywordTok{filter}\NormalTok{(Batting, AB >}\StringTok{ }\DecValTok{0}\NormalTok{)}
\NormalTok{per_player <-}\StringTok{ }\KeywordTok{group_by}\NormalTok{(batters, playerID)}
\NormalTok{ba <-}\StringTok{ }\KeywordTok{summarise}\NormalTok{(per_player, }
  \DataTypeTok{ba =} \KeywordTok{sum}\NormalTok{(H, }\DataTypeTok{na.rm =} \OtherTok{TRUE}\NormalTok{) /}\StringTok{ }\KeywordTok{sum}\NormalTok{(AB, }\DataTypeTok{na.rm =} \OtherTok{TRUE}\NormalTok{)}
\NormalTok{)}
\KeywordTok{ggplot}\NormalTok{(ba, }\KeywordTok{aes}\NormalTok{(ba)) +}\StringTok{ }
\StringTok{  }\KeywordTok{geom_histogram}\NormalTok{(}\DataTypeTok{binwidth =} \FloatTok{0.01}\NormalTok{)}
\end{Highlighting}
\end{Shaded}

\begin{figure}[H]
  \centering
  \includegraphics[width=0.65\linewidth]{_figures/data-manip/unnamed-chunk-3-1}
\end{figure}

Wow, there are a lot of players who can hit the ball every single time!
Would you want them on your fantasy baseball team? Let's double check
they're really that good by calibrating also showing the total number of
at bats:

\begin{Shaded}
\begin{Highlighting}[]
\NormalTok{ba <-}\StringTok{ }\KeywordTok{summarise}\NormalTok{(per_player, }
  \DataTypeTok{ba =} \KeywordTok{sum}\NormalTok{(H, }\DataTypeTok{na.rm =} \OtherTok{TRUE}\NormalTok{) /}\StringTok{ }\KeywordTok{sum}\NormalTok{(AB, }\DataTypeTok{na.rm =} \OtherTok{TRUE}\NormalTok{),}
  \DataTypeTok{ab =} \KeywordTok{sum}\NormalTok{(AB, }\DataTypeTok{na.rm =} \OtherTok{TRUE}\NormalTok{)}
\NormalTok{)}
\KeywordTok{ggplot}\NormalTok{(ba, }\KeywordTok{aes}\NormalTok{(ab, ba)) +}\StringTok{ }
\StringTok{  }\KeywordTok{geom_bin2d}\NormalTok{(}\DataTypeTok{bins =} \DecValTok{100}\NormalTok{) +}\StringTok{ }
\StringTok{  }\KeywordTok{geom_smooth}\NormalTok{()}
\end{Highlighting}
\end{Shaded}

\begin{figure}[H]
  \centering
  \includegraphics[width=0.65\linewidth]{_figures/data-manip/unnamed-chunk-4-1}
\end{figure}

The highest batting averages occur for the players with the smallest
number of at bats - it's not hard to hit the ball every time if you've
only had two pitches. We can make the pattern a little more clear by
getting rid of the players with less than 10 at bats.

\begin{Shaded}
\begin{Highlighting}[]
\KeywordTok{ggplot}\NormalTok{(}\KeywordTok{filter}\NormalTok{(ba, ab >=}\StringTok{ }\DecValTok{10}\NormalTok{), }\KeywordTok{aes}\NormalTok{(ab, ba)) +}\StringTok{ }
\StringTok{  }\KeywordTok{geom_bin2d}\NormalTok{() +}\StringTok{ }
\StringTok{  }\KeywordTok{geom_smooth}\NormalTok{()}
\end{Highlighting}
\end{Shaded}

\begin{figure}[H]
  \centering
  \includegraphics[width=0.65\linewidth]{_figures/data-manip/unnamed-chunk-5-1}
\end{figure}

You'll often see a similar pattern whenever you plot number of
observations vs.~an average. Be aware!

\subsection{Exercises}

\begin{enumerate}
\def\labelenumi{\arabic{enumi}.}
\item
  For each year in the \texttt{ggplot2movies::movies} data determine the
  percent of movies with missing budgets. Visualise the result.
\item
  How does the average length of a movie change over time? Display your
  answer with a plot, including some display of uncertainty.
\item
  For each combination of diamond quality (e.g.~cut, colour and
  clarity), count the number of diamonds, the average price and the
  average size. Visualise the results.
\item
  Compute a histogram of carat by ``hand'' using a binwidth of 0.1.
  Display the results with \texttt{geom\_bar(stat\ =\ "identity")}.
  (Hint: you might need to create a new variable first).
\item
  In the baseball example, the batting average seems to increase as the
  number of at bats increases. Why?
\end{enumerate}

\section{Transformation pipelines}

Most real analyses require you to string together multiple
\texttt{mutate()}s, \texttt{filter()}s, \texttt{group\_by()}s , and
\texttt{summarise()}s. For example, above, we created a frequency
polygon by hand with a combination of all four verbs: \indexc{\%>\%}

\begin{Shaded}
\begin{Highlighting}[]
\CommentTok{# By using intermediate values}
\NormalTok{cut_depth <-}\StringTok{ }\KeywordTok{group_by}\NormalTok{(diamonds, cut, depth)}
\NormalTok{cut_depth <-}\StringTok{ }\KeywordTok{summarise}\NormalTok{(cut_depth, }\DataTypeTok{n =} \KeywordTok{n}\NormalTok{())}
\NormalTok{cut_depth <-}\StringTok{ }\KeywordTok{filter}\NormalTok{(cut_depth, depth >}\StringTok{ }\DecValTok{55}\NormalTok{, depth <}\StringTok{ }\DecValTok{70}\NormalTok{)}
\NormalTok{cut_depth <-}\StringTok{ }\KeywordTok{mutate}\NormalTok{(cut_depth, }\DataTypeTok{prop =} \NormalTok{n /}\StringTok{ }\KeywordTok{sum}\NormalTok{(n))}
\end{Highlighting}
\end{Shaded}

This sequence of operations is a bit painful because we repeated the
name of the data frame many times. An alternative is just to do it with
one sequence of function calls:

\begin{Shaded}
\begin{Highlighting}[]
\CommentTok{# By "composing" functions}
\KeywordTok{mutate}\NormalTok{(}
  \KeywordTok{filter}\NormalTok{(}
    \KeywordTok{summarise}\NormalTok{(}
      \KeywordTok{group_by}\NormalTok{(}
        \NormalTok{diamonds, }
        \NormalTok{cut, }
        \NormalTok{depth}
      \NormalTok{), }
      \DataTypeTok{n =} \KeywordTok{n}\NormalTok{()}
    \NormalTok{), }
    \NormalTok{depth >}\StringTok{ }\DecValTok{55}\NormalTok{, }
    \NormalTok{depth <}\StringTok{ }\DecValTok{70}
  \NormalTok{), }
  \DataTypeTok{prop =} \NormalTok{n /}\StringTok{ }\KeywordTok{sum}\NormalTok{(n)}
\NormalTok{)}
\end{Highlighting}
\end{Shaded}

But this is also hard to read because the sequence of operations is
inside out, and the arguments to each function can be quite far apart.
dplyr provides an alternative approach with the \textbf{pipe},
\texttt{\%\textgreater{}\%}. With the pipe, we can write the above
sequence of operations as:

\begin{Shaded}
\begin{Highlighting}[]
\NormalTok{cut_depth <-}\StringTok{ }\NormalTok{diamonds %>%}\StringTok{ }
\StringTok{  }\KeywordTok{group_by}\NormalTok{(cut, depth) %>%}\StringTok{ }
\StringTok{  }\KeywordTok{summarise}\NormalTok{(}\DataTypeTok{n =} \KeywordTok{n}\NormalTok{()) %>%}\StringTok{ }
\StringTok{  }\KeywordTok{filter}\NormalTok{(depth >}\StringTok{ }\DecValTok{55}\NormalTok{, depth <}\StringTok{ }\DecValTok{70}\NormalTok{) %>%}\StringTok{ }
\StringTok{  }\KeywordTok{mutate}\NormalTok{(}\DataTypeTok{prop =} \NormalTok{n /}\StringTok{ }\KeywordTok{sum}\NormalTok{(n))}
\end{Highlighting}
\end{Shaded}

This makes it easier to understand what's going on as we can read it
almost like an English sentence: first group, then summarise, then
filter, then mutate. In fact, the best way to pronounce
\texttt{\%\textgreater{}\%} when reading a sequence of code is as
``then''. \texttt{\%\textgreater{}\%} comes from the magrittr package,
by Stefan Milton Bache. It provides a number of other tools that dplyr
doesn't expose by default, so I highly recommend that you check out the
\href{https://github.com/smbache/magrittr}{magrittr website}.
\index{magrittr}

\texttt{\%\textgreater{}\%} works by taking the thing on the left hand
side (LHS) and supplying it as the first argument to the function on the
right hand side (RHS). Each of these pairs of calls is equivalent:

\begin{Shaded}
\begin{Highlighting}[]
\KeywordTok{f}\NormalTok{(x, y)}
\CommentTok{# is the same as}
\NormalTok{x %>%}\StringTok{ }\KeywordTok{f}\NormalTok{(y)}

\KeywordTok{g}\NormalTok{(}\KeywordTok{f}\NormalTok{(x, y), z)}
\CommentTok{# is the same as}
\NormalTok{x %>%}\StringTok{ }\KeywordTok{f}\NormalTok{(y) %>%}\StringTok{ }\KeywordTok{g}\NormalTok{(z)}
\end{Highlighting}
\end{Shaded}

\subsection{Exercises}

\begin{enumerate}
\def\labelenumi{\arabic{enumi}.}
\item
  Translate each of the examples in this chapter to use the pipe.
\item
  What does the following pipe do?

\begin{Shaded}
\begin{Highlighting}[]
\KeywordTok{library}\NormalTok{(magrittr)}
\NormalTok{x <-}\StringTok{ }\KeywordTok{runif}\NormalTok{(}\DecValTok{100}\NormalTok{)}
\NormalTok{x %>%}
\StringTok{  }\KeywordTok{subtract}\NormalTok{(}\KeywordTok{mean}\NormalTok{(.)) %>%}
\StringTok{  }\KeywordTok{raise_to_power}\NormalTok{(}\DecValTok{2}\NormalTok{) %>%}
\StringTok{  }\KeywordTok{mean}\NormalTok{() %>%}
\StringTok{  }\KeywordTok{sqrt}\NormalTok{()}
\end{Highlighting}
\end{Shaded}
\item
  Which player in the \texttt{Batting} dataset has had the most
  consistently good performance over the course of their career?
\end{enumerate}

\section{Learning more}

dplyr provides a number of other verbs that are less useful for
visualisation, but important to know about in general:

\begin{itemize}
\item
  \texttt{arrange()} orders observations according to variable(s). This
  is most useful when you're looking at the data from the console. It
  can also be useful for visualisations if you want to control which
  points are plotted on top.
\item
  \texttt{select()} picks variables based on their names. Useful when
  you have many variables and want to focus on just a few for analysis.
\item
  \texttt{rename()} allows you to change the name of variables.
\item
  Grouped mutates and filters are also useful, but more advanced. See
  \texttt{vignette("window-functions",\ package\ =\ "dplyr")} for more
  details.
\item
  There are a number of verbs designed to work with two tables of data
  at a time. These include SQL joins (like the base \texttt{merge()}
  function) and set operations. Learn more about them in
  \texttt{vignette("two-table",\ package\ =\ "dplyr")}.
\item
  dplyr can work directly with data stored in a database - you use the
  same R code as you do for local data and dplyr generates SQL to send
  to the database. See
  \texttt{vignette("databases",\ package\ =\ "dplyr")} for the details.
\end{itemize}

Finally, RStudio provides a handy dplyr cheatsheet that will help jog
your memory when you're wondering which function to use. Get it from
\url{http://rstudio.com/cheatsheets}.

\section*{References}
\addcontentsline{toc}{section}{References}

\hyperdef{}{ref-tornqvist:1985}{\label{ref-tornqvist:1985}}
Törnqvist, Leo, Pentti Vartia, and Yrjö O Vartia. 1985. ``How Should
Relative Changes Be Measured?'' \emph{The American Statistician} 39 (1):
43--46.

\chapter{Modelling for visualisation}\label{cha:modelling}

\section{Introduction}

Modelling is an essential tool for visualisation. There are two
particularly strong connections between modelling and visualisation that
I want to explore in this chapter: \index{Modelling}

\begin{itemize}
\item
  Using models as a tool to remove obvious patterns in your plots. This
  is useful because strong patterns mask subtler effects. Often the
  strongest effects are already known and expected, and removing them
  allows you to see surprises more easily.
\item
  Other times you have a lot of data, too much to show on a handful of
  plots. Models can be a powerful tool for summarising data so that you
  get a higher level view.
\end{itemize}

In this chapter, I'm going to focus on the use of linear models to
acheive these goals. Linear models are a basic, but powerful, tool of
statistics, and I recommend that everyone serious about visualisation
learns at least the basics of how to use them. To this end, I highly
recommend two books by Julian J. Faraway:

\begin{itemize}
\tightlist
\item
  Linear Models with R \url{http://amzn.com/1439887330}
\item
  Extending the Linear Model with R \url{http://amzn.com/158488424X}
\end{itemize}

These books cover some of the theory of linear models, but are pragmatic
and focussed on how to actually use linear models (and their extensions)
in R. \index{Linear models}

There are many other modelling tools, which I don't have the space to
show. If you understand how linear models can help improve your
visualisations, you should be able to translate the basic idea to other
families of models. This chapter just scratches the surface of what you
can do. But hopefully it reinforces how visualisation can combine with
modelling to help you build a powerful data analysis toolbox. For more
ideas, check out Wickham, Cook, and Hofmann (2015).

This chapter only scratches the surface of the intersection between
visualisation and modelling. In my opinion, mastering the combination of
visualisations and models is key to being an effective data scientist.
Unfortunately most books (like this one!) only focus on either
visualisation or modelling, but not both. There's a lot of interesting
work to be done.

\section{Removing trend}\label{sub:trend}

So far our analysis of the diamonds data has been plagued by the
powerful relationship between size and price. It makes it very difficult
to see the impact of cut, colour and clarity because higher quality
diamonds tend to be smaller, and hence cheaper. This challenge is often
called confounding. We can use a linear model to remove the effect of
size on price. Instead of looking at the raw price, we can look at the
relative price: how valuable is this diamond relative to the average
diamond of the same size. \index{Removing trend}

To get started, we'll focus on diamonds of size two carats or less (96\%
of the dataset). This avoids some incidental problems that you can
explore in the exercises if you're interested. We'll also create two new
variables: log price and log carat. These variables are useful because
they produce a plot with a strong linear trend.

\begin{Shaded}
\begin{Highlighting}[]
\NormalTok{diamonds2 <-}\StringTok{ }\NormalTok{diamonds %>%}\StringTok{ }
\StringTok{  }\KeywordTok{filter}\NormalTok{(carat <=}\StringTok{ }\DecValTok{2}\NormalTok{) %>%}
\StringTok{  }\KeywordTok{mutate}\NormalTok{(}
    \DataTypeTok{lcarat =} \KeywordTok{log2}\NormalTok{(carat),}
    \DataTypeTok{lprice =} \KeywordTok{log2}\NormalTok{(price)}
  \NormalTok{)}
\NormalTok{diamonds2}
\CommentTok{#> Source: local data frame [52,051 x 12]}
\CommentTok{#> }
\CommentTok{#>    carat       cut  color clarity depth table price     x     y}
\CommentTok{#>    (dbl)    (fctr) (fctr)  (fctr) (dbl) (dbl) (int) (dbl) (dbl)}
\CommentTok{#> 1   0.23     Ideal      E     SI2  61.5    55   326  3.95  3.98}
\CommentTok{#> 2   0.21   Premium      E     SI1  59.8    61   326  3.89  3.84}
\CommentTok{#> 3   0.23      Good      E     VS1  56.9    65   327  4.05  4.07}
\CommentTok{#> 4   0.29   Premium      I     VS2  62.4    58   334  4.20  4.23}
\CommentTok{#> 5   0.31      Good      J     SI2  63.3    58   335  4.34  4.35}
\CommentTok{#> 6   0.24 Very Good      J    VVS2  62.8    57   336  3.94  3.96}
\CommentTok{#> ..   ...       ...    ...     ...   ...   ...   ...   ...   ...}
\CommentTok{#>        z}
\CommentTok{#>    (dbl)}
\CommentTok{#> 1   2.43}
\CommentTok{#> 2   2.31}
\CommentTok{#> 3   2.31}
\CommentTok{#> 4   2.63}
\CommentTok{#> 5   2.75}
\CommentTok{#> 6   2.48}
\CommentTok{#> ..   ...}
\CommentTok{#> Variables not shown: lcarat (dbl), lprice (dbl)}

\KeywordTok{ggplot}\NormalTok{(diamonds2, }\KeywordTok{aes}\NormalTok{(lcarat, lprice)) +}\StringTok{ }
\StringTok{  }\KeywordTok{geom_bin2d}\NormalTok{() +}\StringTok{ }
\StringTok{  }\KeywordTok{geom_smooth}\NormalTok{(}\DataTypeTok{method =} \StringTok{"lm"}\NormalTok{, }\DataTypeTok{se =} \OtherTok{FALSE}\NormalTok{, }\DataTypeTok{size =} \DecValTok{2}\NormalTok{, }\DataTypeTok{colour =} \StringTok{"yellow"}\NormalTok{)}
\end{Highlighting}
\end{Shaded}

\begin{figure}[H]
  \centering
  \includegraphics[width=0.75\linewidth]{_figures/modelling/unnamed-chunk-1-1}
\end{figure}

In the graphic we used \texttt{geom\_smooth()} to overlay the line of
best fit to the data. We can replicate this outside of ggplot2 by
fitting a linear model with \texttt{lm()}. This allows us to find out
the slope and intercept of the line: \indexf{lm} \indexf{coef}

\begin{Shaded}
\begin{Highlighting}[]
\NormalTok{mod <-}\StringTok{ }\KeywordTok{lm}\NormalTok{(lprice ~}\StringTok{ }\NormalTok{lcarat, }\DataTypeTok{data =} \NormalTok{diamonds2)}
\KeywordTok{coef}\NormalTok{(}\KeywordTok{summary}\NormalTok{(mod))}
\CommentTok{#>             Estimate Std. Error t value Pr(>|t|)}
\CommentTok{#> (Intercept)     12.2    0.00211    5789        0}
\CommentTok{#> lcarat           1.7    0.00208     816        0}
\end{Highlighting}
\end{Shaded}

If you're familiar with linear models, you might want to interpret those
coefficients: \(\log_2(price) = 12.2 + 1.7 \cdot \log_2(carat)\), which
implies \(price = 4900 \cdot carat ^ {1.7}\). Interpreting those
coefficients certainly is useful, but even if you don't understand them,
the model can still be useful. We can use it to subtract the trend away
by looking at the residuals: the price of each diamond minus its
predicted price, based on weight alone. Geometrically, the residuals are
the vertical distance between each point and the line of best fit. They
tell us the price relative to the ``average'' diamond of that size.
\indexf{resid}

\begin{Shaded}
\begin{Highlighting}[]
\NormalTok{diamonds2 <-}\StringTok{ }\NormalTok{diamonds2 %>%}\StringTok{ }\KeywordTok{mutate}\NormalTok{(}\DataTypeTok{rel_price =} \KeywordTok{resid}\NormalTok{(mod))}
\KeywordTok{ggplot}\NormalTok{(diamonds2, }\KeywordTok{aes}\NormalTok{(carat, rel_price)) +}\StringTok{ }
\StringTok{  }\KeywordTok{geom_bin2d}\NormalTok{()}
\end{Highlighting}
\end{Shaded}

\begin{figure}[H]
  \centering
  \includegraphics[width=0.75\linewidth]{_figures/modelling/unnamed-chunk-3-1}
\end{figure}

A relative price of zero means that the diamond was at the average
price; positive means that it's more expensive than expected (based on
its size), and negative means that it's cheaper than expected.

Interpreting the values precisely is a little tricky here because we've
log-transformed price. The residuals give the absolute difference
(\(x - expected\)), but here we have
\(\log_2(price) - \log_2(expected price)\), or equivalently
\(\log_2(price / expected price)\). If we ``back-transform'' to the
original scale by applying the opposite transformation (\(2 ^ x\)) we
get \(price / expected price\). This makes the values more
interpretable, at the cost of the nice symmetry property of the logged
values (i.e.~both relatively cheaper and relatively more expensive
diamonds have the same range). We can make a little table to help
interpret the values:

\begin{Shaded}
\begin{Highlighting}[]
\NormalTok{xgrid <-}\StringTok{ }\KeywordTok{seq}\NormalTok{(-}\DecValTok{2}\NormalTok{, }\DecValTok{1}\NormalTok{, }\DataTypeTok{by =} \DecValTok{1}\NormalTok{/}\DecValTok{3}\NormalTok{)}
\KeywordTok{data.frame}\NormalTok{(}\DataTypeTok{logx =} \NormalTok{xgrid, }\DataTypeTok{x =} \KeywordTok{round}\NormalTok{(}\DecValTok{2} \NormalTok{^}\StringTok{ }\NormalTok{xgrid, }\DecValTok{2}\NormalTok{))}
\CommentTok{#>      logx    x}
\CommentTok{#> 1  -2.000 0.25}
\CommentTok{#> 2  -1.667 0.31}
\CommentTok{#> 3  -1.333 0.40}
\CommentTok{#> 4  -1.000 0.50}
\CommentTok{#> 5  -0.667 0.63}
\CommentTok{#> 6  -0.333 0.79}
\CommentTok{#> 7   0.000 1.00}
\CommentTok{#> 8   0.333 1.26}
\CommentTok{#> 9   0.667 1.59}
\CommentTok{#> 10  1.000 2.00}
\end{Highlighting}
\end{Shaded}

This table illustrates why we used \texttt{log2()} rather than
\texttt{log()}: a change of 1 unit on the logged scale, corresponding to
a doubling on the original scale. For example, a \texttt{rel\_price} of
-1 means that it's half of the expected price; a relative price of 1
means that it's twice the expected price. \index{Log!transform}

Let's use both price and relative price to see how colour and cut affect
the value of a diamond. We'll compute the average price and average
relative price for each combination of colour and cut:

\begin{Shaded}
\begin{Highlighting}[]
\NormalTok{color_cut <-}\StringTok{ }\NormalTok{diamonds2 %>%}\StringTok{ }
\StringTok{  }\KeywordTok{group_by}\NormalTok{(color, cut) %>%}
\StringTok{  }\KeywordTok{summarise}\NormalTok{(}
    \DataTypeTok{price =} \KeywordTok{mean}\NormalTok{(price), }
    \DataTypeTok{rel_price =} \KeywordTok{mean}\NormalTok{(rel_price)}
  \NormalTok{)}
\NormalTok{color_cut}
\CommentTok{#> Source: local data frame [35 x 4]}
\CommentTok{#> Groups: color [?]}
\CommentTok{#> }
\CommentTok{#>     color       cut price rel_price}
\CommentTok{#>    (fctr)    (fctr) (dbl)     (dbl)}
\CommentTok{#> 1       D      Fair  3939   -0.0755}
\CommentTok{#> 2       D      Good  3309   -0.0472}
\CommentTok{#> 3       D Very Good  3368    0.1038}
\CommentTok{#> 4       D   Premium  3513    0.1093}
\CommentTok{#> 5       D     Ideal  2595    0.2173}
\CommentTok{#> 6       E      Fair  3516   -0.1720}
\CommentTok{#> ..    ...       ...   ...       ...}
\end{Highlighting}
\end{Shaded}

If we look at price, it's hard to see how the quality of the diamond
affects the price. The lowest quality diamonds (fair cut with colour J)
have the highest average value! This is because those diamonds also tend
to be larger: size and quality are confounded.

\begin{Shaded}
\begin{Highlighting}[]
\KeywordTok{ggplot}\NormalTok{(color_cut, }\KeywordTok{aes}\NormalTok{(color, price)) +}\StringTok{ }
\StringTok{  }\KeywordTok{geom_line}\NormalTok{(}\KeywordTok{aes}\NormalTok{(}\DataTypeTok{group =} \NormalTok{cut), }\DataTypeTok{colour =} \StringTok{"grey80"}\NormalTok{) +}
\StringTok{  }\KeywordTok{geom_point}\NormalTok{(}\KeywordTok{aes}\NormalTok{(}\DataTypeTok{colour =} \NormalTok{cut))}
\end{Highlighting}
\end{Shaded}

\begin{figure}[H]
  \centering
  \includegraphics[width=0.75\linewidth]{_figures/modelling/unnamed-chunk-6-1}
\end{figure}

If however, we plot the relative price, you see the pattern that you
expect: as the quality of the diamonds decreases, the relative price
decreases. The worst quality diamond is 0.61x (\(2 ^ {-0.7}\)) the price
of an ``average'' diamond.

\begin{Shaded}
\begin{Highlighting}[]
\KeywordTok{ggplot}\NormalTok{(color_cut, }\KeywordTok{aes}\NormalTok{(color, rel_price)) +}\StringTok{ }
\StringTok{  }\KeywordTok{geom_line}\NormalTok{(}\KeywordTok{aes}\NormalTok{(}\DataTypeTok{group =} \NormalTok{cut), }\DataTypeTok{colour =} \StringTok{"grey80"}\NormalTok{) +}
\StringTok{  }\KeywordTok{geom_point}\NormalTok{(}\KeywordTok{aes}\NormalTok{(}\DataTypeTok{colour =} \NormalTok{cut))}
\end{Highlighting}
\end{Shaded}

\begin{figure}[H]
  \centering
  \includegraphics[width=0.75\linewidth]{_figures/modelling/unnamed-chunk-7-1}
\end{figure}

This technique can be employed in a wide range of situations. Wherever
you can explicitly model a strong pattern that you see in a plot, it's
worthwhile to use a model to remove that strong pattern so that you can
see what interesting trends remain.

\subsection{Exercises}

\begin{enumerate}
\def\labelenumi{\arabic{enumi}.}
\item
  What happens if you repeat the above analysis with all diamonds? (Not
  just all diamonds with two or fewer carats). What does the strange
  geometry of \texttt{log(carat)} vs relative price represent? What does
  the diagonal line without any points represent?
\item
  I made an unsupported assertion that lower-quality diamonds tend to be
  larger. Support my claim with a plot.
\item
  Can you create a plot that simultaneously shows the effect of colour,
  cut, and clarity on relative price? If there's too much information to
  show on one plot, think about how you might create a sequence of plots
  to convey the same message.
\item
  How do depth and table relate to the relative price?
\end{enumerate}

\section{Texas housing data}

We'll continue to explore the connection between modelling and
visualisation with the \texttt{txhousing} dataset:

\begin{Shaded}
\begin{Highlighting}[]
\NormalTok{txhousing}
\CommentTok{#> Source: local data frame [8,034 x 9]}
\CommentTok{#> }
\CommentTok{#>       city  year month sales   volume median listings inventory}
\CommentTok{#>      (chr) (int) (int) (dbl)    (dbl)  (dbl)    (dbl)     (dbl)}
\CommentTok{#> 1  Abilene  2000     1    72  5380000  71400      701       6.3}
\CommentTok{#> 2  Abilene  2000     2    98  6505000  58700      746       6.6}
\CommentTok{#> 3  Abilene  2000     3   130  9285000  58100      784       6.8}
\CommentTok{#> 4  Abilene  2000     4    98  9730000  68600      785       6.9}
\CommentTok{#> 5  Abilene  2000     5   141 10590000  67300      794       6.8}
\CommentTok{#> 6  Abilene  2000     6   156 13910000  66900      780       6.6}
\CommentTok{#> ..     ...   ...   ...   ...      ...    ...      ...       ...}
\CommentTok{#>     date}
\CommentTok{#>    (dbl)}
\CommentTok{#> 1   2000}
\CommentTok{#> 2   2000}
\CommentTok{#> 3   2000}
\CommentTok{#> 4   2000}
\CommentTok{#> 5   2000}
\CommentTok{#> 6   2000}
\CommentTok{#> ..   ...}
\end{Highlighting}
\end{Shaded}

This data was collected by the Real Estate Center at Texas A\&M
University, \url{http://recenter.tamu.edu/Data/hs/}. The data contains
information about 46 Texas cities, recording the number of house sales
(\texttt{sales}), the total volume of sales (\texttt{volume}), the
\texttt{average} and \texttt{median} sale prices, the number of houses
listed for sale (\texttt{listings}) and the number of months inventory
(\texttt{inventory}). Data is recorded monthly from Jan 2000 to Apr
2015, 187 entries for each city.
\index{Data!txhousing@\texttt{txhousing}}

We're going to explore how sales have varied over time for each city as
it shows some interesting trends and poses some interesting challenges.
Let's start with an overview: a time series of sales for each city:
\index{Data!longitudinal}

\begin{Shaded}
\begin{Highlighting}[]
\KeywordTok{ggplot}\NormalTok{(txhousing, }\KeywordTok{aes}\NormalTok{(date, sales)) +}\StringTok{ }
\StringTok{  }\KeywordTok{geom_line}\NormalTok{(}\KeywordTok{aes}\NormalTok{(}\DataTypeTok{group =} \NormalTok{city), }\DataTypeTok{alpha =} \DecValTok{1}\NormalTok{/}\DecValTok{2}\NormalTok{)}
\end{Highlighting}
\end{Shaded}

\begin{figure}[H]
  \includegraphics[width=1\linewidth]{_figures/modelling/unnamed-chunk-9-1}
\end{figure}

Two factors make it hard to see the long-term trend in this plot:

\begin{enumerate}
\def\labelenumi{\arabic{enumi}.}
\item
  The range of sales varies over multiple orders of magnitude. The
  biggest city, Houston, averages over \textasciitilde{}4000 sales per
  month; the smallest city, San Marcos, only averages
  \textasciitilde{}20 sales per month.
\item
  There is a strong seasonal trend: sales are much higher in the summer
  than in the winter.
\end{enumerate}

We can fix the first problem by plotting log sales:

\begin{Shaded}
\begin{Highlighting}[]
\KeywordTok{ggplot}\NormalTok{(txhousing, }\KeywordTok{aes}\NormalTok{(date, }\KeywordTok{log}\NormalTok{(sales))) +}\StringTok{ }
\StringTok{  }\KeywordTok{geom_line}\NormalTok{(}\KeywordTok{aes}\NormalTok{(}\DataTypeTok{group =} \NormalTok{city), }\DataTypeTok{alpha =} \DecValTok{1}\NormalTok{/}\DecValTok{2}\NormalTok{)}
\end{Highlighting}
\end{Shaded}

\begin{figure}[H]
  \includegraphics[width=1\linewidth]{_figures/modelling/unnamed-chunk-10-1}
\end{figure}

We can fix the second problem using the same technique we used for
removing the trend in the diamonds data: we'll fit a linear model and
look at the residuals. This time we'll use a categorical predictor to
remove the month effect. First we check that the technique works by
applying it to a single city. It's always a good idea to start simple so
that if something goes wrong you can more easily pinpoint the problem.

\begin{Shaded}
\begin{Highlighting}[]
\NormalTok{abilene <-}\StringTok{ }\NormalTok{txhousing %>%}\StringTok{ }\KeywordTok{filter}\NormalTok{(city ==}\StringTok{ "Abilene"}\NormalTok{)}
\KeywordTok{ggplot}\NormalTok{(abilene, }\KeywordTok{aes}\NormalTok{(date, }\KeywordTok{log}\NormalTok{(sales))) +}\StringTok{ }
\StringTok{  }\KeywordTok{geom_line}\NormalTok{()}

\NormalTok{mod <-}\StringTok{ }\KeywordTok{lm}\NormalTok{(}\KeywordTok{log}\NormalTok{(sales) ~}\StringTok{ }\KeywordTok{factor}\NormalTok{(month), }\DataTypeTok{data =} \NormalTok{abilene)}
\NormalTok{abilene$rel_sales <-}\StringTok{ }\KeywordTok{resid}\NormalTok{(mod)}
\KeywordTok{ggplot}\NormalTok{(abilene, }\KeywordTok{aes}\NormalTok{(date, rel_sales)) +}\StringTok{ }
\StringTok{  }\KeywordTok{geom_line}\NormalTok{()}
\end{Highlighting}
\end{Shaded}

\begin{figure}[H]
  \includegraphics[width=0.5\linewidth]{_figures/modelling/unnamed-chunk-11-1}%
  \includegraphics[width=0.5\linewidth]{_figures/modelling/unnamed-chunk-11-2}
\end{figure}

We can apply this transformation to every city with \texttt{group\_by()}
and \texttt{mutate()}. Note the use of \texttt{na.action\ =\ na.exclude}
argument to \texttt{lm()}. Counterintuitively this ensures that missing
values in the input are matched with missing values in the output
predictions and residuals. Without this argument, missing values are
just dropped, and the residuals don't line up with the inputs.

\begin{Shaded}
\begin{Highlighting}[]
\NormalTok{txhousing <-}\StringTok{ }\NormalTok{txhousing %>%}\StringTok{ }
\StringTok{  }\KeywordTok{group_by}\NormalTok{(city) %>%}\StringTok{ }
\StringTok{  }\KeywordTok{mutate}\NormalTok{(}\DataTypeTok{rel_sales =} \KeywordTok{resid}\NormalTok{(}\KeywordTok{lm}\NormalTok{(}\KeywordTok{log}\NormalTok{(sales) ~}\StringTok{ }\KeywordTok{factor}\NormalTok{(month), }
    \DataTypeTok{na.action =} \NormalTok{na.exclude))}
  \NormalTok{)}
\end{Highlighting}
\end{Shaded}

With this data in hand, we can re-plot the data. Now that we have
log-transformed the data and removed the strong seasonal effects we can
see there is a strong common pattern: a consistent increase from
2000-2007, a drop until 2010 (with quite some noise), and then a gradual
rebound. To make that more clear, I included a summary line that shows
the mean relative sales across all cities.

\begin{Shaded}
\begin{Highlighting}[]
\KeywordTok{ggplot}\NormalTok{(txhousing, }\KeywordTok{aes}\NormalTok{(date, rel_sales)) +}
\StringTok{  }\KeywordTok{geom_line}\NormalTok{(}\KeywordTok{aes}\NormalTok{(}\DataTypeTok{group =} \NormalTok{city), }\DataTypeTok{alpha =} \DecValTok{1}\NormalTok{/}\DecValTok{5}\NormalTok{) +}\StringTok{ }
\StringTok{  }\KeywordTok{geom_line}\NormalTok{(}\DataTypeTok{stat =} \StringTok{"summary"}\NormalTok{, }\DataTypeTok{fun.y =} \StringTok{"mean"}\NormalTok{, }\DataTypeTok{colour =} \StringTok{"red"}\NormalTok{)}
\end{Highlighting}
\end{Shaded}

\begin{figure}[H]
  \includegraphics[width=1\linewidth]{_figures/modelling/unnamed-chunk-13-1}
\end{figure}

(Note that removing the seasonal effect also removed the intercept - we
see the trend for each city relative to its average number of sales.)

\subsection{Exercises}

\begin{enumerate}
\def\labelenumi{\arabic{enumi}.}
\item
  The final plot shows a lot of short-term noise in the overall trend.
  How could you smooth this further to focus on long-term changes?
\item
  If you look closely (e.g. \texttt{+\ xlim(2008,\ 2012)}) at the
  long-term trend you'll notice a weird pattern in 2009-2011. It looks
  like there was a big dip in 2010. Is this dip ``real''? (i.e.~can you
  spot it in the original data)
\item
  What other variables in the TX housing data show strong seasonal
  effects? Does this technique help to remove them?
\item
  Not all the cities in this data set have complete time series. Use
  your dplyr skills to figure out how much data each city is missing.
  Display the results with a visualisation.
\item
  Replicate the computation that \texttt{stat\_summary()} did with dplyr
  so you can plot the data ``by hand''.
\end{enumerate}

\section{Visualising models}\label{sub:modelvis}

The previous examples used the linear model just as a tool for removing
trend: we fit the model and immediately threw it away. We didn't care
about the model itself, just what it could do for us. But the models
themselves contain useful information and if we keep them around, there
are many new problems that we can solve:

\begin{itemize}
\item
  We might be interested in cities where the model didn't fit well: a
  poorly fitting model suggests that there isn't much of a seasonal
  pattern, which contradicts our implicit hypothesis that all cities
  share a similar pattern.
\item
  The coefficients themselves might be interesting. In this case,
  looking at the coefficients will show us how the seasonal pattern
  varies between cities.
\item
  We may want to dive into the details of the model itself, and see
  exactly what it says about each observation. For this data, it might
  help us find suspicious data points that might reflect data entry
  errors.
\end{itemize}

To take advantage of this data, we need to store the models. We can do
this using a new dplyr verb: \texttt{do()}. It allows us to store the
result of arbitrary computation in a column. Here we'll use it to store
that linear model: \indexf{do}

\begin{Shaded}
\begin{Highlighting}[]
\NormalTok{models <-}\StringTok{ }\NormalTok{txhousing %>%}\StringTok{ }
\StringTok{  }\KeywordTok{group_by}\NormalTok{(city) %>%}
\StringTok{  }\KeywordTok{do}\NormalTok{(}\DataTypeTok{mod =} \KeywordTok{lm}\NormalTok{(}
    \KeywordTok{log2}\NormalTok{(sales) ~}\StringTok{ }\KeywordTok{factor}\NormalTok{(month), }
    \DataTypeTok{data =} \NormalTok{., }
    \DataTypeTok{na.action =} \NormalTok{na.exclude}
  \NormalTok{))}
\NormalTok{models}
\CommentTok{#> Source: local data frame [46 x 2]}
\CommentTok{#> Groups: <by row>}
\CommentTok{#> }
\CommentTok{#>         city     mod}
\CommentTok{#>        (chr)  (list)}
\CommentTok{#> 1    Abilene <S3:lm>}
\CommentTok{#> 2   Amarillo <S3:lm>}
\CommentTok{#> 3  Arlington <S3:lm>}
\CommentTok{#> 4     Austin <S3:lm>}
\CommentTok{#> 5   Bay Area <S3:lm>}
\CommentTok{#> 6   Beaumont <S3:lm>}
\CommentTok{#> ..       ...     ...}
\end{Highlighting}
\end{Shaded}

There are two important things to note in this code:

\begin{itemize}
\item
  \texttt{do()} creates a new column called \texttt{mod.} This is a
  special type of column: instead of containing an atomic vector (a
  logical, integer, numeric, or character) like usual, it's a list.
  Lists are R's most flexible data structure and can hold anything,
  including linear models.
\item
  \texttt{.} is a special pronoun used by \texttt{do()}. It refers to
  the ``current'' data frame. In this example, \texttt{do()} fits the
  model 46 times (once for each city), each time replacing \texttt{.}
  with the data for one city. \indexc{.}
\end{itemize}

If you're an experienced modeller, you might wonder why I didn't fit one
model to all cities simultaneously. That's a great next step, but it's
often useful to start off simple. Once we have a model that works for
each city individually, you can figure out how to generalise it to fit
all cities simultaneously.

To visualise these models, we'll turn them into tidy data frames. We'll
do that with the \textbf{broom} package by David Robinson. \index{broom}
\index{Tidy models} \index{Model data}

\begin{Shaded}
\begin{Highlighting}[]
\KeywordTok{library}\NormalTok{(broom)}
\end{Highlighting}
\end{Shaded}

Broom provides three key verbs, each corresponding to one of the
challenges outlined above:

\begin{itemize}
\item
  \texttt{glance()} extracts \textbf{model}-level summaries with one row
  of data for each model. It contains summary statistics like the
  \(R^2\) and degrees of freedom.
\item
  \texttt{tidy()} extracts \textbf{coefficient}-level summaries with one
  row of data for each coefficient in each model. It contains
  information about individual coefficients like their estimate and
  standard error.
\item
  \texttt{augment()} extracts \textbf{observation}-level summaries with
  one row of data for each observation in each model. It includes
  variables like the residual and influence metrics useful for
  diagnosing outliers.
\end{itemize}

We'll learn more about each of these functions in the following three
sections.

\section{Model-level summaries}

We'll begin by looking at how well the model fit to each city with
\texttt{glance()}: \indexf{glance}

\begin{Shaded}
\begin{Highlighting}[]
\NormalTok{model_sum <-}\StringTok{ }\NormalTok{models %>%}\StringTok{ }\KeywordTok{glance}\NormalTok{(mod)}
\NormalTok{model_sum}
\CommentTok{#> Source: local data frame [46 x 12]}
\CommentTok{#> Groups: city [46]}
\CommentTok{#> }
\CommentTok{#>         city r.squared adj.r.squared sigma statistic  p.value    df}
\CommentTok{#>        (chr)     (dbl)         (dbl) (dbl)     (dbl)    (dbl) (int)}
\CommentTok{#> 1    Abilene     0.530         0.500 0.282      17.9 1.50e-23    12}
\CommentTok{#> 2   Amarillo     0.449         0.415 0.302      13.0 7.41e-18    12}
\CommentTok{#> 3  Arlington     0.513         0.483 0.267      16.8 2.75e-22    12}
\CommentTok{#> 4     Austin     0.487         0.455 0.310      15.1 2.04e-20    12}
\CommentTok{#> 5   Bay Area     0.555         0.527 0.265      19.9 1.45e-25    12}
\CommentTok{#> 6   Beaumont     0.430         0.395 0.275      12.0 1.18e-16    12}
\CommentTok{#> ..       ...       ...           ...   ...       ...      ...   ...}
\CommentTok{#> Variables not shown: logLik (dbl), AIC (dbl), BIC (dbl), deviance}
\CommentTok{#>   (dbl), df.residual (int)}
\end{Highlighting}
\end{Shaded}

This creates a variable with one row for each city, and variables that
either summarise complexity (e.g. \texttt{df}) or fit (e.g.
\texttt{r.squared}, \texttt{p.value}, \texttt{AIC}). Since all the
models we fit have the same complexity (12 terms: one for each month),
we'll focus on the model fit summaries. \(R^2\) is a reasonable place to
start because it's well known. We can use a dot plot to see the
variation across cities:

\begin{Shaded}
\begin{Highlighting}[]
\KeywordTok{ggplot}\NormalTok{(model_sum, }\KeywordTok{aes}\NormalTok{(r.squared, }\KeywordTok{reorder}\NormalTok{(city, r.squared))) +}\StringTok{ }
\StringTok{  }\KeywordTok{geom_point}\NormalTok{()}
\end{Highlighting}
\end{Shaded}

\begin{figure}[H]
  \centering
  \includegraphics[width=0.65\linewidth]{_figures/modelling/unnamed-chunk-17-1}
\end{figure}

It's hard to picture exactly what those values of \(R^2\) mean, so it's
helpful to pick out a few exemplars. The following code extracts and
plots out the three cities with the highest and lowest \(R^2\):

\begin{Shaded}
\begin{Highlighting}[]
\NormalTok{top3 <-}\StringTok{ }\KeywordTok{c}\NormalTok{(}\StringTok{"Bryan-College Station"}\NormalTok{, }\StringTok{"Lubbock"}\NormalTok{, }\StringTok{"NE Tarrant County"}\NormalTok{)}
\NormalTok{bottom3 <-}\StringTok{ }\KeywordTok{c}\NormalTok{(}\StringTok{"McAllen"}\NormalTok{, }\StringTok{"Brownsville"}\NormalTok{, }\StringTok{"Harlingen"}\NormalTok{)}
\NormalTok{extreme <-}\StringTok{ }\NormalTok{txhousing %>%}\StringTok{ }\KeywordTok{ungroup}\NormalTok{() %>%}
\StringTok{  }\KeywordTok{filter}\NormalTok{(city %in%}\StringTok{ }\KeywordTok{c}\NormalTok{(top3, bottom3), !}\KeywordTok{is.na}\NormalTok{(sales)) %>%}
\StringTok{  }\KeywordTok{mutate}\NormalTok{(}\DataTypeTok{city =} \KeywordTok{factor}\NormalTok{(city, }\KeywordTok{c}\NormalTok{(top3, bottom3)))}

\KeywordTok{ggplot}\NormalTok{(extreme, }\KeywordTok{aes}\NormalTok{(month, }\KeywordTok{log}\NormalTok{(sales))) +}\StringTok{ }
\StringTok{  }\KeywordTok{geom_line}\NormalTok{(}\KeywordTok{aes}\NormalTok{(}\DataTypeTok{group =} \NormalTok{year)) +}\StringTok{ }
\StringTok{  }\KeywordTok{facet_wrap}\NormalTok{(~city)}
\end{Highlighting}
\end{Shaded}

\begin{figure}[H]
  \centering
  \includegraphics[width=0.65\linewidth]{_figures/modelling/unnamed-chunk-18-1}
\end{figure}

The cities with low \(R^2\) have weaker seasonal patterns and more
variation between years. The data for Harlingen seems particularly
noisy.

\subsection{Exercises}

\begin{enumerate}
\def\labelenumi{\arabic{enumi}.}
\item
  Do your conclusions change if you use a different measurement of model
  fit like AIC or deviance? Why/why not?
\item
  One possible hypothesis that explains why McAllen, Harlingen and
  Brownsville have lower \(R^2\) is that they're smaller towns so there
  are fewer sales and more noise. Confirm or refute this hypothesis.
\item
  McAllen, Harlingen and Brownsville seem to have much more year-to-year
  variation than Bryan-College Station, Lubbock, and NE Tarrant County.
  How does the model change if you also include a linear trend for year?
  (i.e. \texttt{log(sales)\ \textasciitilde{}\ factor(month)\ +\ year}).
\item
  Create a faceted plot that shows the seasonal patterns for all
  cities.\\
  Order the facets by the \(R^2\) for the city.
\end{enumerate}

\section{Coefficient-level summaries}

The model fit summaries suggest that there are some important
differences in seasonality between the different cities. Let's dive into
those differences by using \texttt{tidy()} to extract detail about each
individual coefficient: \indexf{tidy}

\begin{Shaded}
\begin{Highlighting}[]
\NormalTok{coefs <-}\StringTok{ }\NormalTok{models %>%}\StringTok{ }\KeywordTok{tidy}\NormalTok{(mod)}
\NormalTok{coefs}
\CommentTok{#> Source: local data frame [552 x 6]}
\CommentTok{#> Groups: city [46]}
\CommentTok{#> }
\CommentTok{#>       city           term estimate std.error statistic   p.value}
\CommentTok{#>      (chr)          (chr)    (dbl)     (dbl)     (dbl)     (dbl)}
\CommentTok{#> 1  Abilene    (Intercept)    6.542    0.0704     92.88 7.90e-151}
\CommentTok{#> 2  Abilene factor(month)2    0.354    0.0996      3.55  4.91e-04}
\CommentTok{#> 3  Abilene factor(month)3    0.675    0.0996      6.77  1.83e-10}
\CommentTok{#> 4  Abilene factor(month)4    0.749    0.0996      7.52  2.76e-12}
\CommentTok{#> 5  Abilene factor(month)5    0.916    0.0996      9.20  1.06e-16}
\CommentTok{#> 6  Abilene factor(month)6    1.002    0.0996     10.06  4.37e-19}
\CommentTok{#> ..     ...            ...      ...       ...       ...       ...}
\end{Highlighting}
\end{Shaded}

We're more interested in the month effect, so we'll do a little extra
tidying to only look at the month coefficients, and then to extract the
month value into a numeric variable:

\begin{Shaded}
\begin{Highlighting}[]
\NormalTok{months <-}\StringTok{ }\NormalTok{coefs %>%}
\StringTok{  }\KeywordTok{filter}\NormalTok{(}\KeywordTok{grepl}\NormalTok{(}\StringTok{"factor"}\NormalTok{, term)) %>%}
\StringTok{  }\NormalTok{tidyr::}\KeywordTok{extract}\NormalTok{(term, }\StringTok{"month"}\NormalTok{, }\StringTok{"(}\CharTok{\textbackslash{}\textbackslash{}}\StringTok{d+)"}\NormalTok{, }\DataTypeTok{convert =} \OtherTok{TRUE}\NormalTok{)}
\NormalTok{months}
\CommentTok{#> Source: local data frame [506 x 6]}
\CommentTok{#> }
\CommentTok{#>       city month estimate std.error statistic  p.value}
\CommentTok{#>      (chr) (int)    (dbl)     (dbl)     (dbl)    (dbl)}
\CommentTok{#> 1  Abilene     2    0.354    0.0996      3.55 4.91e-04}
\CommentTok{#> 2  Abilene     3    0.675    0.0996      6.77 1.83e-10}
\CommentTok{#> 3  Abilene     4    0.749    0.0996      7.52 2.76e-12}
\CommentTok{#> 4  Abilene     5    0.916    0.0996      9.20 1.06e-16}
\CommentTok{#> 5  Abilene     6    1.002    0.0996     10.06 4.37e-19}
\CommentTok{#> 6  Abilene     7    0.954    0.0996      9.58 9.81e-18}
\CommentTok{#> ..     ...   ...      ...       ...       ...      ...}
\end{Highlighting}
\end{Shaded}

This is a common pattern. You need to use your data tidying skills at
many points in an analysis. Once you have the correct tidy dataset,
creating the plot is usually easy. Here we'll put month on the x-axis,
estimate on the y-axis, and draw one line for each city. I've
back-transformed to make the coefficients more interpretable: these are
now ratios of sales compared to January.

\begin{Shaded}
\begin{Highlighting}[]
\KeywordTok{ggplot}\NormalTok{(months, }\KeywordTok{aes}\NormalTok{(month, }\DecValTok{2} \NormalTok{^}\StringTok{ }\NormalTok{estimate)) +}
\StringTok{  }\KeywordTok{geom_line}\NormalTok{(}\KeywordTok{aes}\NormalTok{(}\DataTypeTok{group =} \NormalTok{city))}
\end{Highlighting}
\end{Shaded}

\begin{figure}[H]
  \centering
  \includegraphics[width=0.65\linewidth]{_figures/modelling/unnamed-chunk-21-1}
\end{figure}

The pattern seems similar across the cities. The main difference is the
strength of the seasonal effect. Let's pull that out and plot it:

\begin{Shaded}
\begin{Highlighting}[]
\NormalTok{coef_sum <-}\StringTok{ }\NormalTok{months %>%}
\StringTok{  }\KeywordTok{group_by}\NormalTok{(city) %>%}
\StringTok{  }\KeywordTok{summarise}\NormalTok{(}\DataTypeTok{max =} \KeywordTok{max}\NormalTok{(estimate))}
\KeywordTok{ggplot}\NormalTok{(coef_sum, }\KeywordTok{aes}\NormalTok{(}\DecValTok{2} \NormalTok{^}\StringTok{ }\NormalTok{max, }\KeywordTok{reorder}\NormalTok{(city, max))) +}\StringTok{ }
\StringTok{  }\KeywordTok{geom_point}\NormalTok{()}
\end{Highlighting}
\end{Shaded}

\begin{figure}[H]
  \centering
  \includegraphics[width=0.65\linewidth]{_figures/modelling/unnamed-chunk-22-1}
\end{figure}

The cities with the strongest seasonal effect are College Station and
San Marcos (both college towns) and Galveston and South Padre Island
(beach cities). It makes sense that these cities would have very strong
seasonal effects.

\subsection{Exercises}

\begin{enumerate}
\def\labelenumi{\arabic{enumi}.}
\item
  Pull out the three cities with highest and lowest seasonal effect.
  Plot their coefficients.
\item
  How does strength of seasonal effect relate to the \(R^2\) for the
  model? Answer with a plot.
\item
  You should be extra cautious when your results agree with your prior
  beliefs. How can you confirm or refute my hypothesis about the causes
  of strong seasonal patterns?
\item
  Group the diamonds data by cut, clarity and colour. Fit a linear model
  \texttt{log(price)\ \textasciitilde{}\ log(carat)}. What does the
  intercept tell you? What does the slope tell you? How do the slope and
  intercept vary across the groups? Answer with a plot.
\end{enumerate}

\section{Observation data}

Observation-level data, which include residual diagnostics, is most
useful in the traditional model fitting scenario, because it can helps
you find ``high-leverage'' points, point that have a big influence on
the final model. It's also useful in conjunction with visualisation,
particularly because it provides an alternative way to access the
residuals.

Extracting observation-level data is the job of the \texttt{augment()}
function. This adds one row for each observation. It includes the
variables used in the original model, the residuals, and a number of
common influence statistics (see \texttt{?augment.lm} for more details):
\indexf{augment}

\begin{Shaded}
\begin{Highlighting}[]
\NormalTok{obs_sum <-}\StringTok{ }\NormalTok{models %>%}\StringTok{ }\KeywordTok{augment}\NormalTok{(mod)}
\NormalTok{obs_sum}
\CommentTok{#> Source: local data frame [8,034 x 10]}
\CommentTok{#> Groups: city [46]}
\CommentTok{#> }
\CommentTok{#>       city log2.sales. factor.month. .fitted .se.fit .resid   .hat}
\CommentTok{#>      (chr)       (dbl)        (fctr)   (dbl)   (dbl)  (dbl)  (dbl)}
\CommentTok{#> 1  Abilene        6.17             1    6.54  0.0704 -0.372 0.0625}
\CommentTok{#> 2  Abilene        6.61             2    6.90  0.0704 -0.281 0.0625}
\CommentTok{#> 3  Abilene        7.02             3    7.22  0.0704 -0.194 0.0625}
\CommentTok{#> 4  Abilene        6.61             4    7.29  0.0704 -0.676 0.0625}
\CommentTok{#> 5  Abilene        7.14             5    7.46  0.0704 -0.319 0.0625}
\CommentTok{#> 6  Abilene        7.29             6    7.54  0.0704 -0.259 0.0625}
\CommentTok{#> ..     ...         ...           ...     ...     ...    ...    ...}
\CommentTok{#> Variables not shown: .sigma (dbl), .cooksd (dbl), .std.resid (dbl)}
\end{Highlighting}
\end{Shaded}

For example, it might be interesting to look at the distribution of
standardised residuals. (These are residuals standardised to have a
variance of one in each model, making them more comparable). We're
looking for unusual values that might need deeper exploration:

\begin{Shaded}
\begin{Highlighting}[]
\KeywordTok{ggplot}\NormalTok{(obs_sum, }\KeywordTok{aes}\NormalTok{(.std.resid)) +}\StringTok{ }
\StringTok{  }\KeywordTok{geom_histogram}\NormalTok{(}\DataTypeTok{binwidth =} \FloatTok{0.1}\NormalTok{)}
\KeywordTok{ggplot}\NormalTok{(obs_sum, }\KeywordTok{aes}\NormalTok{(}\KeywordTok{abs}\NormalTok{(.std.resid))) +}\StringTok{ }
\StringTok{  }\KeywordTok{geom_histogram}\NormalTok{(}\DataTypeTok{binwidth =} \FloatTok{0.1}\NormalTok{)}
\end{Highlighting}
\end{Shaded}

\begin{figure}[H]
  \includegraphics[width=0.5\linewidth]{_figures/modelling/unnamed-chunk-25-1}%
  \includegraphics[width=0.5\linewidth]{_figures/modelling/unnamed-chunk-25-2}
\end{figure}

A threshold of 2 seems like a reasonable threshold to explore
individually:

\begin{Shaded}
\begin{Highlighting}[]
\NormalTok{obs_sum %>%}\StringTok{ }
\StringTok{  }\KeywordTok{filter}\NormalTok{(}\KeywordTok{abs}\NormalTok{(.std.resid) >}\StringTok{ }\DecValTok{2}\NormalTok{) %>%}
\StringTok{  }\KeywordTok{group_by}\NormalTok{(city) %>%}
\StringTok{  }\KeywordTok{summarise}\NormalTok{(}\DataTypeTok{n =} \KeywordTok{n}\NormalTok{(), }\DataTypeTok{avg =} \KeywordTok{mean}\NormalTok{(}\KeywordTok{abs}\NormalTok{(.std.resid))) %>%}
\StringTok{  }\KeywordTok{arrange}\NormalTok{(}\KeywordTok{desc}\NormalTok{(n))}
\CommentTok{#> Source: local data frame [43 x 3]}
\CommentTok{#> }
\CommentTok{#>               city     n   avg}
\CommentTok{#>              (chr) (int) (dbl)}
\CommentTok{#> 1        Texarkana    12  2.43}
\CommentTok{#> 2        Harlingen    11  2.73}
\CommentTok{#> 3             Waco    11  2.96}
\CommentTok{#> 4         Victoria    10  2.49}
\CommentTok{#> 5  Brazoria County     9  2.31}
\CommentTok{#> 6      Brownsville     9  2.48}
\CommentTok{#> ..             ...   ...   ...}
\end{Highlighting}
\end{Shaded}

In a real analysis, you'd want to look into these cities in more detail.

\subsection{Exercises}

\begin{enumerate}
\def\labelenumi{\arabic{enumi}.}
\item
  A common diagnotic plot is fitted values (\texttt{.fitted})
  vs.~residuals (\texttt{.resid}). Do you see any patterns? What if you
  include the city or month on the same plot?
\item
  Create a time series of log(sales) for each city. Highlight points
  that have a standardised residual of greater than 2.
\end{enumerate}

\section*{References}
\addcontentsline{toc}{section}{References}

\hyperdef{}{ref-model-vis-paper}{\label{ref-model-vis-paper}}
Wickham, Hadley, Dianne Cook, and Heike Hofmann. 2015. ``Visualizing
Statistical Models: Removing the Blindfold.'' \emph{Statistical Analysis
and Data Mining: The ASA Data Science Journal} 8 (4): 203--25.

\chapter{Programming with ggplot2}\label{cha:programming}

\section{Introduction}

A major requirement of a good data analysis is flexibility. If your data
changes, or you discover something that makes you rethink your basic
assumptions, you need to be able to easily change many plots at once.
The main inhibitor of flexibility is code duplication. If you have the
same plotting statement repeated over and over again, you'll have to
make the same change in many different places. Often just the thought of
making all those changes is exhausting! This chapter will help you
overcome that problem by showing you how to program with ggplot2.
\index{Programming}

To make your code more flexible, you need to reduce duplicated code by
writing functions. When you notice you're doing the same thing over and
over again, think about how you might generalise it and turn it into a
function. If you're not that familiar with how functions work in R, you
might want to brush up your knowledge at
\url{http://adv-r.had.co.nz/Functions.html}.

In this chapter I'll show how to write functions that create:

\begin{itemize}
\tightlist
\item
  A single ggplot2 component.
\item
  Multiple ggplot2 components.
\item
  A complete plot.
\end{itemize}

And then I'll finish off with a brief illustration of how you can apply
functional programming techniques to ggplot2 objects.

You might also find the
\href{https://github.com/wilkelab/cowplot}{cowplot} and
\href{https://github.com/jrnold/ggthemes}{ggthemes} packages helpful. As
well as providing reusable components that help you directly, you can
also read the source code of the packages to figure out how they work.

\section{Single components}

Each component of a ggplot plot is an object. Most of the time you
create the component and immediately add it to a plot, but you don't
have to. Instead, you can save any component to a variable (giving it a
name), and then add it to multiple plots:

\begin{Shaded}
\begin{Highlighting}[]
\NormalTok{bestfit <-}\StringTok{ }\KeywordTok{geom_smooth}\NormalTok{(}
  \DataTypeTok{method =} \StringTok{"lm"}\NormalTok{, }
  \DataTypeTok{se =} \OtherTok{FALSE}\NormalTok{, }
  \DataTypeTok{colour =} \KeywordTok{alpha}\NormalTok{(}\StringTok{"steelblue"}\NormalTok{, }\FloatTok{0.5}\NormalTok{), }
  \DataTypeTok{size =} \DecValTok{2}
\NormalTok{)}
\KeywordTok{ggplot}\NormalTok{(mpg, }\KeywordTok{aes}\NormalTok{(cty, hwy)) +}\StringTok{ }
\StringTok{  }\KeywordTok{geom_point}\NormalTok{() +}\StringTok{ }
\StringTok{  }\NormalTok{bestfit}
\KeywordTok{ggplot}\NormalTok{(mpg, }\KeywordTok{aes}\NormalTok{(displ, hwy)) +}\StringTok{ }
\StringTok{  }\KeywordTok{geom_point}\NormalTok{() +}\StringTok{ }
\StringTok{  }\NormalTok{bestfit}
\end{Highlighting}
\end{Shaded}

\begin{figure}[H]
  \centering
  \includegraphics[width=0.375\linewidth]{_figures/programming/layer9-1}%
  \includegraphics[width=0.375\linewidth]{_figures/programming/layer9-2}
\end{figure}

That's a great way to reduce simple types of duplication (it's much
better than copying-and-pasting!), but requires that the component be
exactly the same each time. If you need more flexibility, you can wrap
these reusable snippets in a function. For example, we could extend our
\texttt{bestfit} object to a more general function for adding lines of
best fit to a plot. The following code creates a \texttt{geom\_lm()}
with three parameters: the model \texttt{formula}, the line
\texttt{colour} and the line \texttt{size}:

\begin{Shaded}
\begin{Highlighting}[]
\NormalTok{geom_lm <-}\StringTok{ }\NormalTok{function(}\DataTypeTok{formula =} \NormalTok{y ~}\StringTok{ }\NormalTok{x, }\DataTypeTok{colour =} \KeywordTok{alpha}\NormalTok{(}\StringTok{"steelblue"}\NormalTok{, }\FloatTok{0.5}\NormalTok{), }
                    \DataTypeTok{size =} \DecValTok{2}\NormalTok{, ...)  \{}
  \KeywordTok{geom_smooth}\NormalTok{(}\DataTypeTok{formula =} \NormalTok{formula, }\DataTypeTok{se =} \OtherTok{FALSE}\NormalTok{, }\DataTypeTok{method =} \StringTok{"lm"}\NormalTok{, }\DataTypeTok{colour =} \NormalTok{colour,}
    \DataTypeTok{size =} \NormalTok{size, ...)}
\NormalTok{\}}
\KeywordTok{ggplot}\NormalTok{(mpg, }\KeywordTok{aes}\NormalTok{(displ, }\DecValTok{1} \NormalTok{/}\StringTok{ }\NormalTok{hwy)) +}\StringTok{ }
\StringTok{  }\KeywordTok{geom_point}\NormalTok{() +}\StringTok{ }
\StringTok{  }\KeywordTok{geom_lm}\NormalTok{()}
\KeywordTok{ggplot}\NormalTok{(mpg, }\KeywordTok{aes}\NormalTok{(displ, }\DecValTok{1} \NormalTok{/}\StringTok{ }\NormalTok{hwy)) +}\StringTok{ }
\StringTok{  }\KeywordTok{geom_point}\NormalTok{() +}\StringTok{ }
\StringTok{  }\KeywordTok{geom_lm}\NormalTok{(y ~}\StringTok{ }\KeywordTok{poly}\NormalTok{(x, }\DecValTok{2}\NormalTok{), }\DataTypeTok{size =} \DecValTok{1}\NormalTok{, }\DataTypeTok{colour =} \StringTok{"red"}\NormalTok{)}
\end{Highlighting}
\end{Shaded}

\begin{figure}[H]
  \centering
  \includegraphics[width=0.375\linewidth]{_figures/programming/geom-lm-1}%
  \includegraphics[width=0.375\linewidth]{_figures/programming/geom-lm-2}
\end{figure}

Pay close attention to the use of ``\texttt{...}''. When included in the
function definition ``\texttt{...}'' allows a function to accept
arbitrary additional arguments. Inside the function, you can then use
``\texttt{...}'' to pass those arguments on to another function. Here we
pass ``\texttt{...}'' onto \texttt{geom\_smooth()} so the user can still
modify all the other arguments we haven't explicitly overridden. When
you write your own component functions, it's a good idea to always use
``\texttt{...}'' in this way. \indexc{...}

Finally, note that you can only \emph{add} components to a plot; you
can't modify or remove existing objects.

\subsection{Exercises}

\begin{enumerate}
\def\labelenumi{\arabic{enumi}.}
\item
  Create an object that represents a pink histogram with 100 bins.
\item
  Create an object that represents a fill scale with the Blues
  ColorBrewer palette.
\item
  Read the source code for \texttt{theme\_grey()}. What are its
  arguments? How does it work?
\item
  Create \texttt{scale\_colour\_wesanderson()}. It should have a
  parameter to pick the palette from the wesanderson package, and create
  either a continuous or discrete scale.
\end{enumerate}

\section{Multiple components}

It's not always possible to achieve your goals with a single component.
Fortunately, ggplot2 has a convenient way of adding multiple components
to a plot in one step with a list. The following function adds two
layers: one to show the mean, and one to show its standard error:

\begin{Shaded}
\begin{Highlighting}[]
\NormalTok{geom_mean <-}\StringTok{ }\NormalTok{function() \{}
  \KeywordTok{list}\NormalTok{(}
    \KeywordTok{stat_summary}\NormalTok{(}\DataTypeTok{fun.y =} \StringTok{"mean"}\NormalTok{, }\DataTypeTok{geom =} \StringTok{"bar"}\NormalTok{, }\DataTypeTok{fill =} \StringTok{"grey70"}\NormalTok{),}
    \KeywordTok{stat_summary}\NormalTok{(}\DataTypeTok{fun.data =} \StringTok{"mean_cl_normal"}\NormalTok{, }\DataTypeTok{geom =} \StringTok{"errorbar"}\NormalTok{, }\DataTypeTok{width =} \FloatTok{0.4}\NormalTok{)}
  \NormalTok{)}
\NormalTok{\}}
\KeywordTok{ggplot}\NormalTok{(mpg, }\KeywordTok{aes}\NormalTok{(class, cty)) +}\StringTok{ }\KeywordTok{geom_mean}\NormalTok{()}
\KeywordTok{ggplot}\NormalTok{(mpg, }\KeywordTok{aes}\NormalTok{(drv, cty)) +}\StringTok{ }\KeywordTok{geom_mean}\NormalTok{()}
\end{Highlighting}
\end{Shaded}

\begin{figure}[H]
  \centering
  \includegraphics[width=0.375\linewidth]{_figures/programming/geom-mean-1-1}%
  \includegraphics[width=0.375\linewidth]{_figures/programming/geom-mean-1-2}
\end{figure}

If the list contains any \texttt{NULL} elements, they're ignored. This
makes it easy to conditionally add components:

\begin{Shaded}
\begin{Highlighting}[]
\NormalTok{geom_mean <-}\StringTok{ }\NormalTok{function(}\DataTypeTok{se =} \OtherTok{TRUE}\NormalTok{) \{}
  \KeywordTok{list}\NormalTok{(}
    \KeywordTok{stat_summary}\NormalTok{(}\DataTypeTok{fun.y =} \StringTok{"mean"}\NormalTok{, }\DataTypeTok{geom =} \StringTok{"bar"}\NormalTok{, }\DataTypeTok{fill =} \StringTok{"grey70"}\NormalTok{),}
    \NormalTok{if (se) }
      \KeywordTok{stat_summary}\NormalTok{(}\DataTypeTok{fun.data =} \StringTok{"mean_cl_normal"}\NormalTok{, }\DataTypeTok{geom =} \StringTok{"errorbar"}\NormalTok{, }\DataTypeTok{width =} \FloatTok{0.4}\NormalTok{)}
  \NormalTok{)}
\NormalTok{\}}
\KeywordTok{ggplot}\NormalTok{(mpg, }\KeywordTok{aes}\NormalTok{(drv, cty)) +}\StringTok{ }\KeywordTok{geom_mean}\NormalTok{()}
\KeywordTok{ggplot}\NormalTok{(mpg, }\KeywordTok{aes}\NormalTok{(drv, cty)) +}\StringTok{ }\KeywordTok{geom_mean}\NormalTok{(}\DataTypeTok{se =} \OtherTok{FALSE}\NormalTok{)}
\end{Highlighting}
\end{Shaded}

\begin{figure}[H]
  \centering
  \includegraphics[width=0.375\linewidth]{_figures/programming/geom-mean-2-1}%
  \includegraphics[width=0.375\linewidth]{_figures/programming/geom-mean-2-2}
\end{figure}

\subsection{Plot components}

You're not just limited to adding layers in this way. You can also
include any of the following object types in the list:

\begin{itemize}
\item
  A data.frame, which will override the default dataset associated with
  the plot. (If you add a data frame by itself, you'll need to use
  \texttt{\%+\%}, but this is not necessary if the data frame is in a
  list.)
\item
  An \texttt{aes()} object, which will be combined with the existing
  default aesthetic mapping.
\item
  Scales, which override existing scales, with a warning if they've
  already been set by the user.
\item
  Coordinate systems and facetting specification, which override the
  existing settings.
\item
  Theme components, which override the specified components.
\end{itemize}

\subsection{Annotation}

It's often useful to add standard annotations to a plot. In this case,
your function will also set the data in the layer function, rather than
inheriting it from the plot. There are two other options that you should
set when you do this. These ensure that the layer is self-contained:
\index{Annotation!functions}

\begin{itemize}
\item
  \texttt{inherit.aes\ =\ FALSE} prevents the layer from inheriting
  aesthetics from the parent plot. This ensures your annotation works
  regardless of what else is on the plot. \indexc{inherit.aes}
\item
  \texttt{show.legend\ =\ FALSE} ensures that your annotation won't
  appear in the legend. \indexc{show.legend}
\end{itemize}

One example of this technique is the \texttt{borders()} function built
into ggplot2. It's designed to add map borders from one of the datasets
in the maps package: \indexf{borders}

\begin{Shaded}
\begin{Highlighting}[]
\NormalTok{borders <-}\StringTok{ }\NormalTok{function(}\DataTypeTok{database =} \StringTok{"world"}\NormalTok{, }\DataTypeTok{regions =} \StringTok{"."}\NormalTok{, }\DataTypeTok{fill =} \OtherTok{NA}\NormalTok{, }
                    \DataTypeTok{colour =} \StringTok{"grey50"}\NormalTok{, ...) \{}
  \NormalTok{df <-}\StringTok{ }\KeywordTok{map_data}\NormalTok{(database, regions)}
  \KeywordTok{geom_polygon}\NormalTok{(}
    \KeywordTok{aes_}\NormalTok{(~lat, ~long, }\DataTypeTok{group =} \NormalTok{~group), }
    \DataTypeTok{data =} \NormalTok{df, }\DataTypeTok{fill =} \NormalTok{fill, }\DataTypeTok{colour =} \NormalTok{colour, ..., }
    \DataTypeTok{inherit.aes =} \OtherTok{FALSE}\NormalTok{, }\DataTypeTok{show.legend =} \OtherTok{FALSE}
  \NormalTok{)}
\NormalTok{\}}
\end{Highlighting}
\end{Shaded}

\subsection{Additional arguments}

If you want to pass additional arguments to the components in your
function, \texttt{...} is no good: there's no way to direct different
arguments to different components. Instead, you'll need to think about
how you want your function to work, balancing the benefits of having one
function that does it all vs.~the cost of having a complex function
that's harder to understand. \indexc{...}

To get you started, here's one approach using \texttt{modifyList()} and
\texttt{do.call()}: \indexf{modifyList} \indexf{do.call}

\begin{Shaded}
\begin{Highlighting}[]
\NormalTok{geom_mean <-}\StringTok{ }\NormalTok{function(..., }\DataTypeTok{bar.params =} \KeywordTok{list}\NormalTok{(), }\DataTypeTok{errorbar.params =} \KeywordTok{list}\NormalTok{()) \{}
  \NormalTok{params <-}\StringTok{ }\KeywordTok{list}\NormalTok{(...)}
  \NormalTok{bar.params <-}\StringTok{ }\KeywordTok{modifyList}\NormalTok{(params, bar.params)}
  \NormalTok{errorbar.params  <-}\StringTok{ }\KeywordTok{modifyList}\NormalTok{(params, errorbar.params)}
  
  \NormalTok{bar <-}\StringTok{ }\KeywordTok{do.call}\NormalTok{(}\StringTok{"stat_summary"}\NormalTok{, }\KeywordTok{modifyList}\NormalTok{(}
    \KeywordTok{list}\NormalTok{(}\DataTypeTok{fun.y =} \StringTok{"mean"}\NormalTok{, }\DataTypeTok{geom =} \StringTok{"bar"}\NormalTok{, }\DataTypeTok{fill =} \StringTok{"grey70"}\NormalTok{),}
    \NormalTok{bar.params)}
  \NormalTok{)}
  \NormalTok{errorbar <-}\StringTok{ }\KeywordTok{do.call}\NormalTok{(}\StringTok{"stat_summary"}\NormalTok{, }\KeywordTok{modifyList}\NormalTok{(}
    \KeywordTok{list}\NormalTok{(}\DataTypeTok{fun.data =} \StringTok{"mean_cl_normal"}\NormalTok{, }\DataTypeTok{geom =} \StringTok{"errorbar"}\NormalTok{, }\DataTypeTok{width =} \FloatTok{0.4}\NormalTok{),}
    \NormalTok{errorbar.params)}
  \NormalTok{)}

  \KeywordTok{list}\NormalTok{(bar, errorbar)}
\NormalTok{\}}

\KeywordTok{ggplot}\NormalTok{(mpg, }\KeywordTok{aes}\NormalTok{(class, cty)) +}\StringTok{ }
\StringTok{  }\KeywordTok{geom_mean}\NormalTok{(}
    \DataTypeTok{colour =} \StringTok{"steelblue"}\NormalTok{,}
    \DataTypeTok{errorbar.params =} \KeywordTok{list}\NormalTok{(}\DataTypeTok{width =} \FloatTok{0.5}\NormalTok{, }\DataTypeTok{size =} \DecValTok{1}\NormalTok{)}
  \NormalTok{)}
\KeywordTok{ggplot}\NormalTok{(mpg, }\KeywordTok{aes}\NormalTok{(class, cty)) +}\StringTok{ }
\StringTok{  }\KeywordTok{geom_mean}\NormalTok{(}
    \DataTypeTok{bar.params =} \KeywordTok{list}\NormalTok{(}\DataTypeTok{fill =} \StringTok{"steelblue"}\NormalTok{),}
    \DataTypeTok{errorbar.params =} \KeywordTok{list}\NormalTok{(}\DataTypeTok{colour =} \StringTok{"blue"}\NormalTok{)}
  \NormalTok{)}
\end{Highlighting}
\end{Shaded}

\begin{figure}[H]
  \centering
  \includegraphics[width=0.375\linewidth]{_figures/programming/unnamed-chunk-3-1}%
  \includegraphics[width=0.375\linewidth]{_figures/programming/unnamed-chunk-3-2}
\end{figure}

If you need more complex behaviour, it might be easier to create a
custom geom or stat. You can learn about that in the extending ggplot2
vignette included with the package. Read it by running
\texttt{vignette("extending-ggplot2")}.

\subsection{Exercises}

\begin{enumerate}
\def\labelenumi{\arabic{enumi}.}
\item
  To make the best use of space, many examples in this book hide the
  axes labels and legend. I've just copied-and-pasted the same code into
  multiple places, but it would make more sense to create a reusable
  function. What would that function look like?
\item
  Extend the \texttt{borders()} function to also add
  \texttt{coord\_quickmap()} to the plot.
\item
  Look through your own code. What combinations of geoms or scales do
  you use all the time? How could you extract the pattern into a
  reusable function?
\end{enumerate}

\section{Plot functions}\label{sec:functions}

Creating small reusable components is most in line with the ggplot2
spirit: you can recombine them flexibly to create whatever plot you
want. But sometimes you're creating the same plot over and over again,
and you don't need that flexibility. Instead of creating components, you
might want to write a function that takes data and parameters and
returns a complete plot. \index{Plot functions}

For example, you could wrap up the complete code needed to make a
piechart:

\begin{Shaded}
\begin{Highlighting}[]
\NormalTok{piechart <-}\StringTok{ }\NormalTok{function(data, mapping) \{}
  \KeywordTok{ggplot}\NormalTok{(data, mapping) +}
\StringTok{    }\KeywordTok{geom_bar}\NormalTok{(}\DataTypeTok{width =} \DecValTok{1}\NormalTok{) +}\StringTok{ }
\StringTok{    }\KeywordTok{coord_polar}\NormalTok{(}\DataTypeTok{theta =} \StringTok{"y"}\NormalTok{) +}\StringTok{ }
\StringTok{    }\KeywordTok{xlab}\NormalTok{(}\OtherTok{NULL}\NormalTok{) +}\StringTok{ }
\StringTok{    }\KeywordTok{ylab}\NormalTok{(}\OtherTok{NULL}\NormalTok{)}
\NormalTok{\}}
\KeywordTok{piechart}\NormalTok{(mpg, }\KeywordTok{aes}\NormalTok{(}\KeywordTok{factor}\NormalTok{(}\DecValTok{1}\NormalTok{), }\DataTypeTok{fill =} \NormalTok{class))}
\end{Highlighting}
\end{Shaded}

\begin{figure}[H]
  \centering
  \includegraphics[width=0.5\linewidth]{_figures/programming/unnamed-chunk-4-1}
\end{figure}

This is much less flexible than the component based approach, but
equally, it's much more concise. Note that I was careful to return the
plot object, rather than printing it. That makes it possible add on
other ggplot2 components.

You can take a similar approach to drawing parallel coordinates plots
(PCPs). PCPs require a transformation of the data, so I recommend
writing two functions: one that does the transformation and one that
generates the plot. Keeping these two pieces separate makes life much
easier if you later want to reuse the same transformation for a
different visualisation. \index{Parallel coordinate plots}

\begin{Shaded}
\begin{Highlighting}[]
\NormalTok{pcp_data <-}\StringTok{ }\NormalTok{function(df) \{}
  \NormalTok{is_numeric <-}\StringTok{ }\KeywordTok{vapply}\NormalTok{(df, is.numeric, }\KeywordTok{logical}\NormalTok{(}\DecValTok{1}\NormalTok{))}

  \CommentTok{# Rescale numeric columns}
  \NormalTok{rescale01 <-}\StringTok{ }\NormalTok{function(x) \{}
    \NormalTok{rng <-}\StringTok{ }\KeywordTok{range}\NormalTok{(x, }\DataTypeTok{na.rm =} \OtherTok{TRUE}\NormalTok{)}
    \NormalTok{(x -}\StringTok{ }\NormalTok{rng[}\DecValTok{1}\NormalTok{]) /}\StringTok{ }\NormalTok{(rng[}\DecValTok{2}\NormalTok{] -}\StringTok{ }\NormalTok{rng[}\DecValTok{1}\NormalTok{])}
  \NormalTok{\}}
  \NormalTok{df[is_numeric] <-}\StringTok{ }\KeywordTok{lapply}\NormalTok{(df[is_numeric], rescale01)}
  
  \CommentTok{# Add row identifier}
  \NormalTok{df$.row <-}\StringTok{ }\KeywordTok{rownames}\NormalTok{(df)}
  
  \CommentTok{# Treat numerics as value (aka measure) variables}
  \CommentTok{# gather_ is the standard-evaluation version of gather, and}
  \CommentTok{# is usually easier to program with.}
  \NormalTok{tidyr::}\KeywordTok{gather_}\NormalTok{(df, }\StringTok{"variable"}\NormalTok{, }\StringTok{"value"}\NormalTok{, }\KeywordTok{names}\NormalTok{(df)[is_numeric])}
\NormalTok{\}}
\NormalTok{pcp <-}\StringTok{ }\NormalTok{function(df, ...) \{}
  \NormalTok{df <-}\StringTok{ }\KeywordTok{pcp_data}\NormalTok{(df)}
  \KeywordTok{ggplot}\NormalTok{(df, }\KeywordTok{aes}\NormalTok{(variable, value, }\DataTypeTok{group =} \NormalTok{.row)) +}\StringTok{ }\KeywordTok{geom_line}\NormalTok{(...)}
\NormalTok{\}}
\KeywordTok{pcp}\NormalTok{(mpg)}
\KeywordTok{pcp}\NormalTok{(mpg, }\KeywordTok{aes}\NormalTok{(}\DataTypeTok{colour =} \NormalTok{drv))}
\end{Highlighting}
\end{Shaded}

\begin{figure}[H]
  \includegraphics[width=0.5\linewidth]{_figures/programming/pcp_data-1}%
  \includegraphics[width=0.5\linewidth]{_figures/programming/pcp_data-2}
\end{figure}

A complete exploration of this idea is \texttt{qplot()}, which provides
a fairly deep wrapper around the most common \texttt{ggplot()} options.
I recommend studying the source code if you want to see how far these
basic techniques can take you. \indexf{qplot}

\subsection{Indirectly referring to variables}

The \texttt{piechart()} function above is a little unappealing because
it requires the user to know the exact \texttt{aes()} specification that
generates a pie chart. It would be more convenient if the user could
simply specify the name of the variable to plot. To do that you'll need
to learn a bit more about how \texttt{aes()} works.

\texttt{aes()} uses non-standard evaluation: rather than looking at the
values of its arguments, it looks at their expressions. This makes it
difficult to work with programmatically as there's no way to store the
name of a variable in an object and then refer to it later:

\begin{Shaded}
\begin{Highlighting}[]
\NormalTok{x_var <-}\StringTok{ "displ"}
\KeywordTok{aes}\NormalTok{(x_var)}
\CommentTok{#> * x -> x_var}
\end{Highlighting}
\end{Shaded}

Instead we need to use \texttt{aes\_()}, which uses regular evaluation.
There are two basic ways to create a mapping with \texttt{aes\_()}:
\indexf{aes\_}

\begin{itemize}
\item
  Using a \emph{quoted call}, created by \texttt{quote()},
  \texttt{substitute()}, \texttt{as.name()}, or \texttt{parse()}.
  \indexf{quote} \indexf{substitute} \indexf{parse} \indexf{as.name}

\begin{Shaded}
\begin{Highlighting}[]
\KeywordTok{aes_}\NormalTok{(}\KeywordTok{quote}\NormalTok{(displ))}
\CommentTok{#> * x -> displ}
\KeywordTok{aes_}\NormalTok{(}\KeywordTok{as.name}\NormalTok{(x_var))}
\CommentTok{#> * x -> displ}
\KeywordTok{aes_}\NormalTok{(}\KeywordTok{parse}\NormalTok{(}\DataTypeTok{text =} \NormalTok{x_var)[[}\DecValTok{1}\NormalTok{]])}
\CommentTok{#> * x -> displ}

\NormalTok{f <-}\StringTok{ }\NormalTok{function(x_var) \{}
  \KeywordTok{aes_}\NormalTok{(}\KeywordTok{substitute}\NormalTok{(x_var))}
\NormalTok{\}}
\KeywordTok{f}\NormalTok{(displ)}
\CommentTok{#> * x -> displ}
\end{Highlighting}
\end{Shaded}

  The difference between \texttt{as.name()} and \texttt{parse()} is
  subtle. If \texttt{x\_var} is ``a + b'', \texttt{as.name()} will turn
  it into a variable called \texttt{`a\ +\ b`}, \texttt{parse()} will
  turn it into the function call \texttt{a\ +\ b}. (If this is
  confusing, \url{http://adv-r.had.co.nz/Expressions.html} might help).
\item
  Using a formula, created with \texttt{\textasciitilde{}}.
  \indexc{\textasciitilde}

\begin{Shaded}
\begin{Highlighting}[]
\KeywordTok{aes_}\NormalTok{(~displ)}
\CommentTok{#> * x -> displ}
\end{Highlighting}
\end{Shaded}
\end{itemize}

\texttt{aes\_()} gives us three options for how a user can supply
variables: as a string, as a formula, or as a bare expression. These
three options are illustrated below

\begin{Shaded}
\begin{Highlighting}[]
\NormalTok{piechart1 <-}\StringTok{ }\NormalTok{function(data, var, ...) \{}
  \KeywordTok{piechart}\NormalTok{(data, }\KeywordTok{aes_}\NormalTok{(~}\KeywordTok{factor}\NormalTok{(}\DecValTok{1}\NormalTok{), }\DataTypeTok{fill =} \KeywordTok{as.name}\NormalTok{(var)))}
\NormalTok{\}}
\KeywordTok{piechart1}\NormalTok{(mpg, }\StringTok{"class"}\NormalTok{) +}\StringTok{ }\KeywordTok{theme}\NormalTok{(}\DataTypeTok{legend.position =} \StringTok{"none"}\NormalTok{)}

\NormalTok{piechart2 <-}\StringTok{ }\NormalTok{function(data, var, ...) \{}
  \KeywordTok{piechart}\NormalTok{(data, }\KeywordTok{aes_}\NormalTok{(~}\KeywordTok{factor}\NormalTok{(}\DecValTok{1}\NormalTok{), }\DataTypeTok{fill =} \NormalTok{var))}
\NormalTok{\}}
\KeywordTok{piechart2}\NormalTok{(mpg, ~class) +}\StringTok{ }\KeywordTok{theme}\NormalTok{(}\DataTypeTok{legend.position =} \StringTok{"none"}\NormalTok{)}

\NormalTok{piechart3 <-}\StringTok{ }\NormalTok{function(data, var, ...) \{}
  \KeywordTok{piechart}\NormalTok{(data, }\KeywordTok{aes_}\NormalTok{(~}\KeywordTok{factor}\NormalTok{(}\DecValTok{1}\NormalTok{), }\DataTypeTok{fill =} \KeywordTok{substitute}\NormalTok{(var)))}
\NormalTok{\}}
\KeywordTok{piechart3}\NormalTok{(mpg, class) +}\StringTok{ }\KeywordTok{theme}\NormalTok{(}\DataTypeTok{legend.position =} \StringTok{"none"}\NormalTok{)}
\end{Highlighting}
\end{Shaded}

\begin{figure}[H]
  \includegraphics[width=0.333\linewidth]{_figures/programming/unnamed-chunk-8-1}%
  \includegraphics[width=0.333\linewidth]{_figures/programming/unnamed-chunk-8-2}%
  \includegraphics[width=0.333\linewidth]{_figures/programming/unnamed-chunk-8-3}
\end{figure}

There's another advantage to \texttt{aes\_()} over \texttt{aes()} if
you're writing ggplot2 plots inside a package: using
\texttt{aes\_(\textasciitilde{}x,\ \textasciitilde{}y)} instead of
\texttt{aes(x,\ y)} avoids the global variables NOTE in
\texttt{R\ CMD\ check}. \index{Global variables}

\subsection{The plot environment}

As you create more sophisticated plotting functions, you'll need to
understand a bit more about ggplot2's scoping rules. ggplot2 was written
well before I understood the full intricacies of non-standard
evaluation, so it has a rather simple scoping system. If a variable is
not found in the \texttt{data}, it is looked for in \emph{the} plot
environment. There is only one environment for a plot (not one for each
layer), and it is the environment in which \texttt{ggplot()} is called
from (i.e.~the \texttt{parent.frame()}). \index{Environments}
\indexf{parent.frame}

This means that the following function won't work because \texttt{n} is
not stored in an environment accessible when the expressions in
\texttt{aes()} are evaluated.

\begin{Shaded}
\begin{Highlighting}[]
\NormalTok{f <-}\StringTok{ }\NormalTok{function() \{}
  \NormalTok{n <-}\StringTok{ }\DecValTok{10}
  \KeywordTok{geom_line}\NormalTok{(}\KeywordTok{aes}\NormalTok{(x /}\StringTok{ }\NormalTok{n)) }
\NormalTok{\}}
\NormalTok{df <-}\StringTok{ }\KeywordTok{data.frame}\NormalTok{(}\DataTypeTok{x =} \DecValTok{1}\NormalTok{:}\DecValTok{3}\NormalTok{, }\DataTypeTok{y =} \DecValTok{1}\NormalTok{:}\DecValTok{3}\NormalTok{)}
\KeywordTok{ggplot}\NormalTok{(df, }\KeywordTok{aes}\NormalTok{(x, y)) +}\StringTok{ }\KeywordTok{f}\NormalTok{()}
\CommentTok{#> Error in x/n: non-numeric argument to binary operator}
\end{Highlighting}
\end{Shaded}

Note that this is only a problem with the \texttt{mapping} argument. All
other arguments are evaluated immediately so their values (not a
reference to a name) are stored in the plot object. This means the
following function will work:

\begin{Shaded}
\begin{Highlighting}[]
\NormalTok{f <-}\StringTok{ }\NormalTok{function() \{}
  \NormalTok{colour <-}\StringTok{ "blue"}
  \KeywordTok{geom_line}\NormalTok{(}\DataTypeTok{colour =} \NormalTok{colour) }
\NormalTok{\}}
\KeywordTok{ggplot}\NormalTok{(df, }\KeywordTok{aes}\NormalTok{(x, y)) +}\StringTok{ }\KeywordTok{f}\NormalTok{()}
\end{Highlighting}
\end{Shaded}

If you need to use a different environment for the plot, you can specify
it with the \texttt{environment} argument to \texttt{ggplot()}. You'll
need to do this if you're creating a plot function that takes user
provided data. See \texttt{qplot()} for an example.

\subsection{Exercises}

\begin{enumerate}
\def\labelenumi{\arabic{enumi}.}
\item
  Create a \texttt{distribution()} function specially designed for
  visualising continuous distributions. Allow the user to supply a
  dataset and the name of a variable to visualise. Let them choose
  between histograms, frequency polygons, and density plots. What other
  arguments might you want to include?
\item
  What additional arguments should \texttt{pcp()} take? What are the
  downsides of how \texttt{...} is used in the current code?
\item
  Advanced: why doesn't this code work? How can you fix it?

\begin{Shaded}
\begin{Highlighting}[]
\NormalTok{f <-}\StringTok{ }\NormalTok{function() \{}
  \NormalTok{levs <-}\StringTok{ }\KeywordTok{c}\NormalTok{(}\StringTok{"2seater"}\NormalTok{, }\StringTok{"compact"}\NormalTok{, }\StringTok{"midsize"}\NormalTok{, }\StringTok{"minivan"}\NormalTok{, }\StringTok{"pickup"}\NormalTok{, }
    \StringTok{"subcompact"}\NormalTok{, }\StringTok{"suv"}\NormalTok{)}
  \KeywordTok{piechart3}\NormalTok{(mpg, }\KeywordTok{factor}\NormalTok{(class, }\DataTypeTok{levels =} \NormalTok{levs))}
\NormalTok{\}}
\KeywordTok{f}\NormalTok{()}
\CommentTok{#> Error in factor(class, levels = levs): object 'levs' not found}
\end{Highlighting}
\end{Shaded}
\end{enumerate}

\section{Functional programming}

Since ggplot2 objects are just regular R objects, you can put them in a
list. This means you can apply all of R's great functional programming
tools. For example, if you wanted to add different geoms to the same
base plot, you could put them in a list and use \texttt{lapply()}.
\index{Functional programming} \indexf{lapply}

\begin{Shaded}
\begin{Highlighting}[]
\NormalTok{geoms <-}\StringTok{ }\KeywordTok{list}\NormalTok{(}
  \KeywordTok{geom_point}\NormalTok{(),}
  \KeywordTok{geom_boxplot}\NormalTok{(}\KeywordTok{aes}\NormalTok{(}\DataTypeTok{group =} \KeywordTok{cut_width}\NormalTok{(displ, }\DecValTok{1}\NormalTok{))),}
  \KeywordTok{list}\NormalTok{(}\KeywordTok{geom_point}\NormalTok{(), }\KeywordTok{geom_smooth}\NormalTok{())}
\NormalTok{)}

\NormalTok{p <-}\StringTok{ }\KeywordTok{ggplot}\NormalTok{(mpg, }\KeywordTok{aes}\NormalTok{(displ, hwy))}
\KeywordTok{lapply}\NormalTok{(geoms, function(g) p +}\StringTok{ }\NormalTok{g)}
\CommentTok{#> [[1]]}
\CommentTok{#> }
\CommentTok{#> [[2]]}
\CommentTok{#> }
\CommentTok{#> [[3]]}
\end{Highlighting}
\end{Shaded}

\begin{figure}[H]
  \includegraphics[width=0.333\linewidth]{_figures/programming/unnamed-chunk-12-1}%
  \includegraphics[width=0.333\linewidth]{_figures/programming/unnamed-chunk-12-2}%
  \includegraphics[width=0.333\linewidth]{_figures/programming/unnamed-chunk-12-3}
\end{figure}

If you're not familiar with functional programming, read through
\url{http://adv-r.had.co.nz/Functional-programming.html} and think about
how you might apply the techniques to your duplicated plotting code.

\subsection{Exercises}

\begin{enumerate}
\def\labelenumi{\arabic{enumi}.}
\item
  How could you add a \texttt{geom\_point()} layer to each element of
  the following list?

\begin{Shaded}
\begin{Highlighting}[]
\NormalTok{plots <-}\StringTok{ }\KeywordTok{list}\NormalTok{(}
  \KeywordTok{ggplot}\NormalTok{(mpg, }\KeywordTok{aes}\NormalTok{(displ, hwy)),}
  \KeywordTok{ggplot}\NormalTok{(diamonds, }\KeywordTok{aes}\NormalTok{(carat, price)),}
  \KeywordTok{ggplot}\NormalTok{(faithfuld, }\KeywordTok{aes}\NormalTok{(waiting, eruptions, }\DataTypeTok{size =} \NormalTok{density))}
\NormalTok{)}
\end{Highlighting}
\end{Shaded}
\item
  What does the following function do? What's a better name for it?

\begin{Shaded}
\begin{Highlighting}[]
\NormalTok{mystery <-}\StringTok{ }\NormalTok{function(...) \{}
  \KeywordTok{Reduce}\NormalTok{(}\StringTok{`}\DataTypeTok{+}\StringTok{`}\NormalTok{, }\KeywordTok{list}\NormalTok{(...), }\DataTypeTok{accumulate =} \OtherTok{TRUE}\NormalTok{)}
\NormalTok{\}}

\KeywordTok{mystery}\NormalTok{(}
  \KeywordTok{ggplot}\NormalTok{(mpg, }\KeywordTok{aes}\NormalTok{(displ, hwy)) +}\StringTok{ }\KeywordTok{geom_point}\NormalTok{(), }
  \KeywordTok{geom_smooth}\NormalTok{(), }
  \KeywordTok{xlab}\NormalTok{(}\OtherTok{NULL}\NormalTok{), }
  \KeywordTok{ylab}\NormalTok{(}\OtherTok{NULL}\NormalTok{)}
\NormalTok{)}
\end{Highlighting}
\end{Shaded}
\end{enumerate}


\backmatter

\let\hyperlink=\oldhyperlink % Restore old hyperlink behaviour
\cleardoublepage
\markboth{Index}{Index}
\addcontentsline{toc}{chapter}{Index}
\printindex

\addcontentsline{toc}{chapter}{Code index}
\printindex[code]

\end{document}
