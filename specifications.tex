\providecommand{\setflag}{\newif \ifwhole \wholefalse}
\setflag
\ifwhole\else

% Typography and geometry ----------------------------------------------------
\documentclass[letterpaper]{scrbook}
\usepackage[inner=3cm,top=2.5cm,outer=3.5cm]{geometry}

\renewcommand\familydefault{bch}
\usepackage[utf8]{inputenc}
\usepackage{microtype}
\usepackage[small]{caption}
\usepackage[small]{titlesec}
\raggedbottom

% Graphics -------------------------------------------------------------------
\usepackage[pdftex]{graphicx}
\graphicspath{{_include/}}
\DeclareGraphicsExtensions{.png,.pdf}

% Code formatting ------------------------------------------------------------
\usepackage{fancyvrb}
\usepackage{courier}
\usepackage{listings}
\usepackage{color}
\usepackage{alltt}


\definecolor{comment}{rgb}{0.60, 0.60, 0.53}
\definecolor{background}{rgb}{0.97, 0.97, 1.00}
\definecolor{string}{rgb}{0.863, 0.066, 0.266}
\definecolor{number}{rgb}{0.0, 0.6, 0.6}
\definecolor{variable}{rgb}{0.00, 0.52, 0.70}
\lstset{
  basicstyle=\ttfamily,
  keywordstyle=\bfseries, 
  identifierstyle=,  
  commentstyle=\color{comment} \emph,
  stringstyle=\color{string},
  showstringspaces=false,
  columns = fullflexible,
  backgroundcolor=\color{background},
  mathescape = true,
  escapeinside=&&,
  fancyvrb
}
\newcommand{\code}[1]{\lstinline!#1!}



% Links ----------------------------------------------------------------------

\usepackage{hyperref}
\definecolor{slateblue}{rgb}{0.07,0.07,0.488}
\hypersetup{colorlinks=true,linkcolor=slateblue,anchorcolor=slateblue,citecolor=slateblue,filecolor=slateblue,urlcolor=slateblue,bookmarksnumbered=true,pdfview=FitB}
\usepackage{url}

% Tables ---------------------------------------------------------------------
\usepackage{longtable}
\usepackage{booktabs}

% Miscellaneous --------------------------------------------------------------
\usepackage{pdfsync}
\usepackage{appendix}

\usepackage[round,sort&compress,sectionbib]{natbib}
\bibliographystyle{plainnat}


\title{ggplot2}
\author{Hadley Wickham}

\begin{document}
\fi


\chapter{Aesthetic specifications}
\label{cha:aesthetic_specifications}

\section{Introduction}


\section{Colour}
\label{sec:colour_spec}

Colors can be specified in several different ways. The simplest way is with a character string giving the color name (e.g., \code{"red"}). A list of the possible colors can be obtained with the function \code{colors}. Alternatively, colors can be specified directly in terms of their RGB components with a string of the form \code{"#RRGGBB"} where each of the pairs \code{RR}, \code{GG}, \code{BB} consist of two hexadecimal digits giving a value in the range \code{00} to \code{FF}. Colors can also be specified by giving an index into a small table of colors, the \code{palette}. This provides compatibility with S. Index \code{0} corresponds to the background color. (Because apparently some people have been assuming it, it is also possible to specify integers as character strings, e.g. \code{"3"}.)

Additionally, \code{"transparent"} or (integer) \code{NA} is transparent, useful for filled areas (such as the background!), and just invisible for things like lines or text. Semi-transparent colors are available for use on devices that support them.

The functions \code{rgb}, \code{hsv}, \code{hcl}, \code{gray} and \code{rainbow} provide additional ways of generating colors.


\section{Line type}
\label{sec:line_type_spec}

Line types can either be specified by giving an index into a small built-in table of line types (1 = solid, 2 = dashed, etc, see \code{lty} above) or directly as the lengths of on/off stretches of line. This is done with a string of an even number (up to eight) of characters, namely non-zero (hexadecimal) digits which give the lengths in consecutive positions in the string. For example, the string \code{"33"} specifies three units on followed by three off and \code{"3313"} specifies three units on followed by three off followed by one on and finally three off. The \code{units} here are (on most devices) proportional to \code{lwd}, and with \code{lwd = 1} are in pixels or points or 1/96 inch.

The five standard dash-dot line types (\code{lty = 2:6}) correspond to \code{c("44", "13", "1343", "73", "2262")}.

Note that \code{NA} is not a valid value for \code{lty}.

\section{Shape}
\label{sec:shape_spec}

Shapes take four types of values:

\begin{itemize}
  \item Integer in $[0, 25]$.  These are illustrated in Figure~\ref{fig:shape}.

  \item A single character.  Can be a unicode charater, which 

  \item \code{.} is handled specially. In most devices, it draws the smallest shape that is visible (i.e. about one pixel).
  
  \item \code{NA}.  Point omitted.  
\end{itemize}

\begin{figure}[htbp]
  \centering
    \includegraphics[width=0.5 \textwidth]{shape-specification}
  \caption{R plotting symbols.  Colour is black, and fill is blue.  Symbol 25 has been omitted from this plot, it is symbol 24 rotated 180 degrees.}
  \label{fig:shape}
\end{figure}

While all symbols have a foreground colour, symbols 19--25 also take a background colour (fill).  


\section{Justification}
\label{sec:justification_spec}

The justification of the text relative to its (x, y) location. If there are two values, the first value specifies horizontal justification and the second value specifies vertical justification. Possible string values are: \code{"left"}, \code{"right"}, \code{"centre"}, \code{"center"}, \code{"bottom"}, and \code{"top"}. For numeric values, 0 means left alignment and 1 means right alignment.

\ifwhole
\else
  \bibliography{/Users/hadley/documents/phd/references}
  \end{document}
\fi