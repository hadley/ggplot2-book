\providecommand{\setflag}{\newif \ifwhole \wholefalse}
\setflag
\ifwhole\else

% Typography and geometry ----------------------------------------------------
\documentclass[letterpaper]{scrbook}
\usepackage[inner=3cm,top=2.5cm,outer=3.5cm]{geometry}

\renewcommand\familydefault{bch}
\usepackage[utf8]{inputenc}
\usepackage{microtype}
\usepackage[small]{caption}
\usepackage[small]{titlesec}
\raggedbottom

% Graphics -------------------------------------------------------------------
\usepackage[pdftex]{graphicx}
\graphicspath{{_include/}}
\DeclareGraphicsExtensions{.png,.pdf}

% Code formatting ------------------------------------------------------------
\usepackage{fancyvrb}
\usepackage{courier}
\usepackage{listings}
\usepackage{color}
\usepackage{alltt}


\definecolor{comment}{rgb}{0.60, 0.60, 0.53}
\definecolor{background}{rgb}{0.97, 0.97, 1.00}
\definecolor{string}{rgb}{0.863, 0.066, 0.266}
\definecolor{number}{rgb}{0.0, 0.6, 0.6}
\definecolor{variable}{rgb}{0.00, 0.52, 0.70}
\lstset{
  basicstyle=\ttfamily,
  keywordstyle=\bfseries, 
  identifierstyle=,  
  commentstyle=\color{comment} \emph,
  stringstyle=\color{string},
  showstringspaces=false,
  columns = fullflexible,
  backgroundcolor=\color{background},
  mathescape = true,
  escapeinside=&&,
  fancyvrb
}
\newcommand{\code}[1]{\lstinline!#1!}



% Links ----------------------------------------------------------------------

\usepackage{hyperref}
\definecolor{slateblue}{rgb}{0.07,0.07,0.488}
\hypersetup{colorlinks=true,linkcolor=slateblue,anchorcolor=slateblue,citecolor=slateblue,filecolor=slateblue,urlcolor=slateblue,bookmarksnumbered=true,pdfview=FitB}
\usepackage{url}

% Tables ---------------------------------------------------------------------
\usepackage{longtable}
\usepackage{booktabs}

% Miscellaneous --------------------------------------------------------------
\usepackage{pdfsync}
\usepackage{appendix}

\usepackage[round,sort&compress,sectionbib]{natbib}
\bibliographystyle{plainnat}


\title{ggplot2}
\author{Hadley Wickham}

\begin{document}
\fi


\chapter{Mastering the grammar}
\label{cha:mastery}

% Introduction to the components of the grammar
% Introduction to the data structure
% Roadmap for next few chapters

\section{Introduction}\label{sec:introduction}

You can choose to use just {\tt qplot}, without any understanding of the underlying grammar, but you will not be able to use the full power of ggplot.  By learning more about the grammar, and the components that make it up, you will be able to create a wider range of plots, as well as being able to combine multiple sources of data, and customise to your heart's content.

This chapter describes the theoretical basis of ggplot2: the layered grammar of graphics, a based based on Wilkinson's grammar of graphics \citep{wilkinson:2006}.  The next chapters then describe parts of the grammar in more details: layers (geoms and stats), scales, and positioning (coordinate systems, faceting and position adjustments),

Pull more from layered grammar paper, but needs to be rather different.

\section{Building a plot}
\label{sec:building_a_plot}

Consider the mammal's sleep dataset illustrated in Table~\ref{tbl:sleep}

\begin{alltt}
  qplot(bodywt, brainwt, data=msleep)
  qplot(bodywt, brainwt, data=msleep, log="xy")
  qplot(bodywt, brainwt, data=msleep, log="xy", facets = . ~ sleepyhead)
\end{alltt}

Pick out 5 animals to use throughout in tables.

\section{What is a plot?}
\label{sec:what_is_a_plot}

One way to think about the grammar of graphics is as a question: what is a plot?  The grammar answers this by describing a plot as a collection of independent components, each describing an independent part of the plot.  There are three basic things we need for a plot: one or more layers, scales to map variables from data space to visual space, and a coordinate system.  These are described below.

\begin{itemize}
  \item One or more layers.  A layer is composed of data and a description of which data variables should be mapped to which aesthetic properties, a geometric object, and a statistical transformation:
  
  \begin{itemize}
  	\item Data is obviously the most important part, and it is what you provide.  This is what you are displaying visually to aid communication or analysis.  You also need to describe how variables in the dataset are mapped to visual properties.  For example, in Figure 2.X we mapped diamond price to y position, carat to y position and colour to colour.  Because the data and aesthetic mapping set is usually the same in most layers, these can also be set as defaults at the plot level.
  	
  	\item {\bf Geoms}, short for geometric objects, control the type of plot that you create.  For example, using a point geom will create a scatterplot, while using a line geom will create a line plot.

  	\item {\bf Stats}, or statistical transformations, reduce or augment the data in a statistical manner.  For example, a useful stat is the smoother, which shows the mean of y, conditional on x.  Another common stat is the binner, which bins data in to bins.   Every geom has a default statistic, and every statistic a default geom.  For example, the bin statistic has defaults to using the bar geom to produce a histogram.

  	\item {\bf Position adjustment}
  \end{itemize}

  \item A scale for all the aesthetic properties.  {\bf Scales} control the mapping from data attributes to aesthetic attributes.  They also provide an inverse mapping in the form of a guide, an axis or legend, which facilitates reading the final graphic.  Aesthetic attributes are things like position, size, colour---anything that you can perceive.  The function that maps data to aesthetic attributes is a scale. It takes values in data space (continuous or categorical) and maps them into an aesthetic space (eg. colour, size, shape).  A scale also provides guides to convert back from the aesthetic attribute to the original data.  Guides are either axes (for position) or legends (for everything else)

  \item A coordinate system.  A {\bf coord}, or coordinate systems, maps the position of objects on to the plane of the plot.  Typically we will use the cartesian coordinate system, but sometimes others are useful.
\end{itemize}

There is also another thing that turns out to be sufficiently useful that we should include it in our general framework: faceting (also known as conditioned or trellis plots). This allows us to easily create small multiples of different subsets of an entire dataset. This is a powerful tool when investigating whether patterns hold across all conditions.


\section{Data structures}
\label{sec:data_structures}

These principles are encoded as data structures in a fairly straightforward way.

One thing to note is that all ggplot2 objects (with the exception of the main plot object) are proto objects.  Proto is a package which implements the prototype-style of object-oriented programming.  There are some major differences between this and the typical S3 or S4 style of OO in R, but the good news is that you only need to worry about them if you want to develop your own extensions to ggplot2.  For everyday use, the proto objects are hidden behind a facade which makes them act like normal R objects.

{\tt str} to see full structure (it can be large!)

{\tt summary} briefly describes the structure of the plot

Data stored inside the plot - if you change the data outside of the plot, and then redraw a saved plot, it will not be updated.  Consequence of R copying semantics.

\ifwhole
\else
  \bibliography{/Users/hadley/documents/phd/references}
  \end{document}
\fi