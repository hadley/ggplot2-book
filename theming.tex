\providecommand{\setflag}{\newif \ifwhole \wholefalse}
\setflag
\ifwhole\else

% Typography and geometry ----------------------------------------------------
\documentclass[letterpaper]{scrbook}
\usepackage[inner=3cm,top=2.5cm,outer=3.5cm]{geometry}

\renewcommand\familydefault{bch}
\usepackage[utf8]{inputenc}
\usepackage{microtype}
\usepackage[small]{caption}
\usepackage[small]{titlesec}
\raggedbottom

% Graphics -------------------------------------------------------------------
\usepackage[pdftex]{graphicx}
\graphicspath{{_include/}}
\DeclareGraphicsExtensions{.png,.pdf}

% Code formatting ------------------------------------------------------------
\usepackage{fancyvrb}
\usepackage{courier}
\usepackage{listings}
\usepackage{color}
\usepackage{alltt}


\definecolor{comment}{rgb}{0.60, 0.60, 0.53}
\definecolor{background}{rgb}{0.97, 0.97, 1.00}
\definecolor{string}{rgb}{0.863, 0.066, 0.266}
\definecolor{number}{rgb}{0.0, 0.6, 0.6}
\definecolor{variable}{rgb}{0.00, 0.52, 0.70}
\lstset{
  basicstyle=\ttfamily,
  keywordstyle=\bfseries, 
  identifierstyle=,  
  commentstyle=\color{comment} \emph,
  stringstyle=\color{string},
  showstringspaces=false,
  columns = fullflexible,
  backgroundcolor=\color{background},
  mathescape = true,
  escapeinside=&&,
  fancyvrb
}
\newcommand{\code}[1]{\lstinline!#1!}



% Links ----------------------------------------------------------------------

\usepackage{hyperref}
\definecolor{slateblue}{rgb}{0.07,0.07,0.488}
\hypersetup{colorlinks=true,linkcolor=slateblue,anchorcolor=slateblue,citecolor=slateblue,filecolor=slateblue,urlcolor=slateblue,bookmarksnumbered=true,pdfview=FitB}
\usepackage{url}

% Tables ---------------------------------------------------------------------
\usepackage{longtable}
\usepackage{booktabs}

% Miscellaneous --------------------------------------------------------------
\usepackage{pdfsync}
\usepackage{appendix}

\usepackage[round,sort&compress,sectionbib]{natbib}
\bibliographystyle{plainnat}


\title{ggplot2}
\author{Hadley Wickham}

\begin{document}
\fi


\chapter{Theming}


\section{Settings}\label{sec:options}

ggplot provides convenient access to a select set of commonly used options, described in Table \ref{tbl:options}.  These options are described in more depth, with many examples, in the online documentation {\tt ?ggopt}.
There are two ways to set options:

\begin{itemize}
  \item Globally (for all plots), using {\tt ggopt}.  For example {\tt ggopt(grid.fill = "white")}.   Global options are applied when a plot is rendered, not when it is created.  This lets you experiment with plot structure and appearance independently.

  \item Locally (for one plot), using \$ or {\tt update}.  For example, {\tt p\$grid.fill <- "white"}, or {\tt update(p, grid.fill = "white")}.  The first modifies the object in place, while the second creates a modified copy. Local options override the more general global options.
\end{itemize}     


\begin{table}
\begin{tabular}{lll}
Option & Valid values & Details \\
\hline
background.fill    & any colour & Background fill colour behind entire plot\\
background.colour  & any colour & \\
grid.colour        & any colour & Colour of grid lines within panel \\
grid.fill          & any colour & Panel background colour \\
strip.gp           & gpar object & Graphical parameters used for strip. Colour and fill of most interest \\
strip.text.gp      & any colour & Graphical parameters used for strip text. \\
strip.text         & a function & Function should accept two arguments, variable and value and return a character vector of length one. Function which determines how strip labels are formatted \\
legend.position    & left, right, top, bottom, none, c(x, y) & Position of legend.  Can use numeric vector of length 2 to set position of centre of length, units are npc relative to whole plot viewport. \\
aspect.ratio       & a number or NULL & Aspect ratio of plot. \\
\hline
\end{tabular}
  \caption{ggplot options}
  \label{tbl:options}
\end{table}

\begin{table}
  \begin{center}
  \begin{tabular}{ll}
  Option & Description \\
  \hline
  fill, col & Background and foreground colours.  You can supply a named colour (see {\tt ?colors}) or a hex string of the form \#RRGGBBAA or \#RRGGBB. \\
  lty & Line type: solid, dotted, dashed etc. \\
  lwd & Width of line, in points. \\
  fontfamily, fontface & Refer to R-news article \\
  fontsize & Font size, in points.  
    
  \end{tabular}
  \end{center}
  \caption{gpar specifications.  See {\tt ?gpar} for more details}
  \label{tbl:gpar}
\end{table}

The following example demonstrates some of the possibilities.

% decumar<<< 
% dsmall <- diamonds[sample(1:nrow(diamonds), 1000), ]
% doptions(height=3, width=4)
% interweave({
% # Always a good idea to save existing options before making
% # radical changes
% old <- ggopt(grid.colour="grey50", grid.fill="white")
% qplot(carat, price, data=dsmall)
% (p <- qplot(cut, clarity, data=dsmall, geom="jitter"))
% # Only changes the current plot
% p$grid.colour <- "darkgreen"
% p
% # Restore old settings:
% ggtheme(old)
% p
% })
% |||
\begin{alltt}
> old <- ggopt(grid.colour = "grey50", grid.fill = "white")
> qplot(carat, price, data = dsmall)
\includegraphics[scale=1]{7f8d370a14179d509879f35839b54ea9}

> (p <- qplot(cut, clarity, data = dsmall, geom = "jitter"))
\includegraphics[scale=1]{5830a37fc4252db571b25f373c8f141f}

> p$grid.colour <- "darkgreen"
> p
\includegraphics[scale=1]{656e171fb34b88b28f6789aa9ad535e6}

> ggtheme(old)
> p
\includegraphics[scale=1]{656e171fb34b88b28f6789aa9ad535e6}

\end{alltt}
% >>>

\section{Themes}\label{sub:themes}

Like lattice, it is also possible to create a theme which encapsulates multiple options.  A theme is a very simple structure, just a list of multiple options, so it's  easy to create your own.  One theme is included by default: {\tt theme\_bw}, which sets up a white background with black grid lines.  You can use a theme with {\tt ggtheme(theme)}, or for a single plot {\tt update(plot, theme)}.

% decumar<<< 
% interweave({
% ggtheme(theme_bw)
% qplot(carat, price, data=dsmall)
% })
% |||
\begin{alltt}
> ggtheme(theme_bw)
> qplot(carat, price, data = dsmall)
\includegraphics[scale=1]{7f8d370a14179d509879f35839b54ea9}

\end{alltt}
% >>>


\ifwhole
\else
  \bibliography{/Users/hadley/documents/phd/references}
  \end{document}
\fi
