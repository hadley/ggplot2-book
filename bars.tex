% \subsection{Barcharts for categorical data}\label{sub:bar_plots}
% 
% This section needs to be rethought.  How do people typically use this sort of data?
% 
% Barplots are typically used for displaying categorical data in the form of contingency tables.  Lets generate some data of that form using the {\tt mtcars} dataset:
% 
% % decumar<<< 
% % interweave({
% % counts <- as.data.frame(table(mtcars[,c("cyl","vs","am")]))
% % counts
% % })
% % |||
\begin{alltt}
> counts <- as.data.frame(table(mtcars[, c("cyl", "vs", "am")]))
> counts
   cyl vs am Freq
1    4  0  0    0
2    6  0  0    0
3    8  0  0   12
4    4  1  0    3
5    6  1  0    4
6    8  1  0    0
7    4  0  1    1
8    6  0  1    3
9    8  0  1    2
10   4  1  1    7
11   6  1  1    0
12   8  1  1    0

\end{alltt}
% % >>>
% 
% We can then plot as follows:
% 
% % decumar<<< 
% % interweave({
% % qplot(cyl, Freq, data=counts, geom="bar", avoid="stack")
% % })
% % |||
\begin{alltt}
> qplot(cyl, Freq, data = counts, geom = "bar", avoid = "stack")
\end{alltt}
\includegraphics[scale=0.5]{./include/da1bb49ccde6dedd810609192db55996-001.pdf}
\begin{alltt}

\end{alltt}
% % >>>
% 
% We use the {\tt avoid} argument to tell the bar plot how to avoid have multiple bars on top of each other.  Bars can either be stacked ({\tt avoid="stack"}) or displayed side-by-side ({\tt stack="dodge"}).  
% 
% For bar plots, there is only one aesthetic attribute that you can control, the fill colour, {\tt fill}, because all other attributes are constrained by the shape of the bar.
% 
% If your data is not already in contingency table form, you can use a shortcut.  By colouring based on another aesthetic you can see how this works.  Each observation is simply a 
% 
% % decumar<<< 
% % interweave({
% % qplot(factor(cyl), 1, data=mtcars, geom="bar", avoid="stack")
% % qplot(factor(cyl), 1, data=mtcars, geom="bar", avoid="stack", fill=mpg)
% % qplot(factor(cyl), 1, data=mtcars, geom="bar", avoid="stack", fill=mpg, sort=TRUE)
% % })
% % |||
\begin{alltt}
> qplot(factor(cyl), 1, data = mtcars, geom = "bar", avoid = "stack")
\end{alltt}
\includegraphics[scale=0.5]{./include/86083f1ebbf9459b811783fd30237130-001.pdf}
\begin{alltt}

> qplot(factor(cyl), 1, data = mtcars, geom = "bar", avoid = "stack", 
+     fill = mpg)
\end{alltt}
\includegraphics[scale=0.5]{./include/3aab961b76c4a1548862cf1f1d031b3c-001.pdf}
\begin{alltt}

> qplot(factor(cyl), 1, data = mtcars, geom = "bar", avoid = "stack", 
+     fill = mpg, sort = TRUE)
\end{alltt}
\includegraphics[scale=0.5]{./include/c47d48aeb9b6194434c40f021e90b822-001.pdf}
\begin{alltt}

\end{alltt}
% % >>>